\section{Introduction}

%% With the emergence of data-intensive applications such as digital
%% film, medical imaging and geographic information systems, the
%% performance of next-generation systems will often depend on their
%% ability to process huge volumes of data.  For example, each frame of a
%% digital film requires approximately 2 megabytes, implying that a
%% fully-edited 90-minute video demands about 300 gigabytes of data for
%% the imagery alone~\cite{ibm-video}.  Industrial Light and Magic
%% reports that, in 2003, their processing pipeline output 13.7 million
%% frames and their internal network processed 9 petabytes of
%% data~\cite{ilm-interview}.  The U.S. Geological Survey had archived
%% over 13 million frames of photographic data by the end of 2004, and
%% estimates that 5 years is needed to digitize 8.6 million additional
%% images~\cite{usgs}.  In all of these situations, the data is highly
%% compressed to reduce storage costs.  At the same time, extensive
%% post-processing is often required for adding captions, watermarking,
%% resizing, compositing, adjusting colors, converting formats, and so
%% on.  As such processing logically operates on the uncompressed format,
%% the usual practice is to decompress and re-compress the data whenever
%% it needs to be modified.

%YouTube manages about 45 terabytes of video data~\cite{wsj-youtube},
%with 65,000 videos uploaded daily~\cite{youtube}.  Microsoft
%TerraServer holds upwards of 22 terabytes of image data, serving 69
%gigabytes per day~\cite{terraserver}.

In order to accelerate the process of editing compressed data,
researchers have identified specific transformations that can be
mapped into the compressed domain---that is, they can operate directly
on the compressed data format rather than on the uncompressed
format~\cite{chang95survey,mandal95survey,smith95survey,wee02survey}.
In addition to avoiding the cost of the decompression and
re-compression, such techniques greatly reduce the total volume of
data processed, thereby offering large savings in both execution time
and memory footprint.  However, existing techniques for operating
directly on compressed data are largely limited to lossy compression
formats such as
JPEG~\cite{dugad01,feng03,mukherjee02,shen96b,shen96,shen98,smith96b}
and MPEG~\cite{smith98,dorai00,nang00,vasudev98,wee02survey}.  While
these formats are used pervasively in the distribution of image and
video content, they are rarely used during the production of such
content.  Instead, professional artists and filmmakers rely on
lossless compression formats (BMP, PNG, Apple Animation) to avoid
accumulating artifacts during the editing process.  Given the
computational intensity of professional video editing, there is a
large demand for new techniques that could accelerate operations on
lossless formats.

In this paper, we present a technique for translating a specific class
of computations to operate directly on losslessly-compressed data.  We
consider compression formats that are based on LZ77, a compression
algorithm that is utilized by ZIP and fully encapsulates common
formats such as Apple Animation, Microsoft RLE, and Targa.  Our
transformation applies to a restricted class of programs, termed {\it
  stream programs}~\cite{streamitcc}, that operate on continuous
streams of data.  The transformation is most efficient when each
element of the stream is transformed in a uniform way (e.g., adjusting
the brightness of each pixel).  However, it also applies to cases in
which multiple items are processed at once (e.g., averaging pixels) or
in which multiple streams are split or combined (e.g., compositing
frames).  The precise coverage of our transformation is defined in
Section 2.

%% However, existing techniques for operating directly on compressed data
%% have two limitations.  First, they focus on lossy compression formats
%% (e.g., JPEG, MPEG) rather than lossless compression formats; lossless
%% compression is the new standard for professional video editing and
%% movie production.  Second, they rely on specialized and ad-hoc
%% techniques for translating individual operations into the compressed
%% domain.  For example, for DCT-based spatial compression formats (JPEG,
%% Motion-JPEG), researchers have developed separate algorithms for
%% resizing~\cite{dugad01,mukherjee02}, edge
%% detection~\cite{shen96b,shen96}, image segmentation~\cite{feng03},
%% shearing and rotating inner blocks~\cite{shen98}, and arbitrary linear
%% combinations of pixels~\cite{smith96b}.  Techniques extending to
%% DCT-based temporal compression (MPEG) include
%% captioning~\cite{nang00}, reversal~\cite{vasudev98}, distortion
%% detection~\cite{dorai00}, transcoding~\cite{smith98}, and
%% others~\cite{wee02survey}.  For run-length encoded images, algorithms
%% have been devised for efficient transpose and
%% rotation~\cite{misra99,shoji95}.  A compressed audio format has been
%% invented that allows direct modification of pitch and playback
%% speed~\cite{levine98}.  While these techniques are powerful, they
%% remain inaccessible to most application programmers because they
%% demand intricate manipulation of the underlying compression format.
%% It is difficult for non-experts to compose existing compressed-domain
%% operations into a complete program, let alone translate a new and
%% unique operation into the compressed domain.

%% This paper presents a technique for automatically mapping complete
%% user-defined programs into the compressed domain.  The technique
%% applies to stream programs: a restricted but practical class of
%% applications that perform regular processing over long data sequences.
%% Stream programming captures the essential functionality needed by
%% image, video, and signal processing applications while exposing the
%% flow of data to the compiler.  Our formulation is based on LZ77, a
%% lossless compression algorithm utilized by ZIP, and naturally applies
%% to formats such as Apple Animation, Microsoft RLE, and Targa (which
%% are special cases of LZ77).  
%% %Lossless compression is widely used in
%% %computer animation and digital video editing in order to avoid
%% %accumulating compression artifacts.  
%% By providing an automatic mapping into the compressed domain, our
%% technique enables a large class of transformations to be customized by
%% the user and directly applied to the compressed data.

The key idea behind our technique can be understood in simple terms.
In LZ77, compression is achieved by indicating that a given part of
the data stream is a repeat of a previous part of the stream.  If a
program is transforming each element of the stream in the same way,
then any repetitions in the input will necessarily be present in the
output as well.  Thus, while new data sequences need to be processed
as usual, any repeats of those sequences do not need to be transformed
again.  Rather, {\it the repetitions in the input stream can be
  directly copied to the output stream}, thereby referencing the
previously-computed values.  This preserves the compression in the
stream while avoiding the cost of decompression, re-compression, and
computing on the uncompressed data.  

%In this paper, we extend this simple idea to a broad class of
%programs: those which input and output multiple data items at a time,
%and those which split, combine, and reorder the data in the stream.

In this paper, we extend this simple idea to encompass a broad class
of programs that can be expressed in the StreamIt programming
language~\cite{streamitcc}.  We have implemented a subset of our
general technique in the StreamIt compiler.  The end result is a
fully-automatic system in which the user writes programs that operate
on uncompressed data, and our compiler emits an optimized program that
operates directly on compressed data.  Our compiler generates plugins
for two popular video editing tools (MEncoder and Blender), allowing
the optimized transformations to be used as part of a standard video
editing process.

Using a suite of 12 videos (screencasts, animations, and stock
footage) in Apple Animation format, our transformation offers a
speedup roughly proportional to the compression factor.  For
transformations that adjust a single video (brightness, contrast,
color inversion), speedups range from 2.5x to 471x, with a median of
17x.  For transformations that combine two videos (overlays and
mattes), speedups range from 1.1x to 32x, with a median of 6.6x.  We
believe this is the first demonstration of compressed-domain
techniques for losslessly compressed video content.

%% In the general case, compressed processing techniques may need to
%% partially decompress the input data to support the behavior of certain
%% programs.  Even if no decompression is performed, the output may
%% benefit from an additional re-compression step if new redundancy is
%% introduced during the processing (for example, increasing image
%% brightness can whiteout parts of the image).  This effect turns out to
%% be minor in the case of our experiments.  For pixel transformations,
%% output sizes are within 0.1\% of input sizes and often (more than half
%% the time) are within 5\% of a full re-compression.  For video
%% compositing, output files maintain a sizable compression ratio of 8.8x
%% (median) while full re-compression results in a ratio of 13x (median).

%% In the rest of this paper, we describe additional background material
%% (Section 2), our transformation into the compressed domain (Section 3)
%% and our experimental evaluation (Section 4).  We close with related
%% work (Section 5) and conclusions (Section 6).

To summarize, this paper makes the following contributions:
\begin{itemize}

\item An algorithm for mapping an arbitrary stream program to operate
  directly on lossless LZ77-compressed data.  In addition to
  transforming a single stream, programs may interleave and
  de-interleave multiple streams while maintaining compression
  (Sections 2-3).

\item An analysis of popular lossless compression formats and the
  opportunities for direct processing on each (Section 4).

\item An experimental evaluation in the StreamIt compiler,
  demonstrating that automatic translation to the compressed domain
  can speedup realistic operations in popular video editing tools.
  Across our benchmarks, the median speedup is 15x (Section 5).

\end{itemize}

The paper concludes with related work (Section 6) and conclusions
(Section 7).
