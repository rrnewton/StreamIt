With multicore architectures becoming more prominent in computing
systems, there is a growing need to adapt programs to use these
available resources.  Stream programming is a programming 
paradigm that can yield high-performance programs by effectively
mapping computations across many cores.  Such programs are naturally
expressible as graphs of independently executing filters communicating 
over data channels.  

Even with such inherent parallelism in this paradigm, data parallelism
may be inhibited in filters due largely to how they are written.  
Most commonly, filters may be "`stateful"', meaning they retain mutable 
state between filter execution.  Such filters must be executed in a 
serial fashion and cannot be replicated and run in parallel in order
to maintain correctness.  A common category of filter state is that of 
induction state.  Such state keeps track of how often the filter has 
been invoked, allowing it to perform some special action at certain 
iterations.  

In this work, we attempt to eliminate such throughput bottlenecks on 
data parallelism that may be caused by induction state usage.  This 
process uses an internal induction variable and provides slight 
modifications to filter fission, a transformation that increases the 
granularity of stream graphs to improve parallelism in the stream program.

The proposed modifications to filter fission are implemented in the 
StreamIt language.  StreamIt Users can improve data parallelism on 
filters that use induction variables by modifying common induction 
variable patterns to use the added StreamIt language features. 
