\section{Compiler Infrastructure}
\label{sec:compiler}

\begin{table}[t]
\begin{center}
\scriptsize
\begin{tabular}{|l|l|} \hline
{\bf Phase} & {\bf Function} \\
\hline \hline
Front-end & Parse StreamIt code and create AST. \\ &\\
\hline
SIR Conversion & Converts the AST to the StreamIt IR (SIR). \\ &\\
\hline
Graph Expansion & Expands all parameterized structures in the \\
                & stream graph. \\
\hline
Scheduling & Calculates initialization and steady-state \\
           & execution orderings for filter firings. \\
\hline
Partitioning & Performs fission and fusion transformations \\
             & for coarsening and parallelism. \\
\hline
Layout & Assign filters of the partitioned graph to cores \\
\hline
Filter optimizations & Apply traditional optimizations such as
constant \\
& propagation, loop unrolling, scalarization and batching. \\
\hline
Code generation & Generates code for synchronization, computation, and \\
& communication. \\
\hline
\end{tabular}
\caption{\protect\small Phases of the StreamIt compiler's
  coarse-grained task, data, and software pipeline parallelism backend~\cite{gordon-asplos06}.
\label{tab:phases}}
\end{center}
\end{table}

We have modified the StreamIt compiler
infrastructure~\cite{streamitweb} to include full support for the
iteration keyword and to correctly handle the desugared filters in
passes and transformation that are affected by the presense of state.
This section provides an overview of the compiler and the changes we
implemented.

The overall flow of the StreamIt compiler is given in
Table~\ref{tab:phases}.  Begininning our discussion from the first
phase, we made modifications to the parser to correctly parse the new
\iter keyword.  An expression class representing \iter was added to the
intermediate representation.  The type of this expression is always of
the base integer type in the language.  We added \iter to the IR so uses of
it can be type-checked by StreamIt's type checker.

The desugaring step (see \S\ref{sec:desugar}) is performed after Graph
Expansion and before Scheduling.  The modifications to the Fission
transformation described in \S\ref{sec:fission} are included and
called by the partitioning algorithm.  Since the Scheduling phase is
called before partitioning, the initialization and steady-state
multiplicities for each filter is available for use by the modified
fission transformation.

The compiler includes a function that determines if a filter is
stateful or stateless. Originally, this function looks for a filter
field that may be assigned a non-constant value, and the field may
also be read.  If such a field exists the filter is classified as stateful.  We
do not want desugared iteration filters to be classified as stateful,
so we modified the function to ignore the {\it iter} field added by
the desugaring, thus classifying filters that include state that is
only the generated {\it iter} field as stateless. 

The only filter optimization that needed alteration was batching.  The
batching optimization increases the steady-state multiplicities of all
filters to amortize the cost of synchronization at the expense of
latency~\cite{gordon-asplos06}.  Since this optimization increases the
steady-state multiplicities by a calculated factor $m$, it must now
scale the fields of all fissed iteration filters as described in
\S\ref{sec:ssmod}.  Since the fields are compiler generated, it is
simple to find them by name and scale them appropriately.

The StreamIt compiler infrastructure includes multiple paths and
backends.  For the evaluation in the next section, we employ the
coarse-grained task, data, and software pipelining path as described
in~\cite{gordon-asplos06}.  This approach aggressively fuses
filters, careful not to introduce state and obscure data
parallelism.  For our modifications, desugared iteration filters without
additional state are classified as stateless and fused with other stateless
filters during this coarsening step.  

A heuristic algorithm considers each task-parallel slice
of the graph, introducing data parallelism when available such that a
slice will occupy all cores of the chip.  Coarse-grained software
pipelining is introduced by adding appropriate buffering and a prolog
schedule such that within the steady state, there are no
dependences between operators.  Finally, a heuristic mapping algorithm
first groups data parallel filters together into sets, and then
schedules the sets so that each occupies all cores. Stateful filters
are greedily bin-packed to take advantage of software pipeline
parallelism.   The next section gives the experiment results.