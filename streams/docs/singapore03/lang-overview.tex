% \begin{figure}
% \vspace{-6pt}
% \begin{minipage}{3in}
% \psfig{figure=fir-streamit-left.eps,width=162.35pt}
% \end{minipage}
% \hspace{0.2in}
% \begin{minipage}{3in}
% \psfig{figure=fir-streamit-right.eps,width=102.35pt}
% \vspace{0.55in}
% \caption{\protect\small An FIR filter in StreamIt.
% \protect\label{fig:firstreamit}}
% \end{minipage}
% \vspace{-12pt}
% \end{figure}

\section{Language Overview}
\label{sec:overview}

StreamIt includes stream-specific abstractions and representations
that are designed to improve programmer productivity for the domain of
programs described above.  In this paper, we present StreamIt in legal
Java syntax\footnote{However, for the sake of brevity, the code
fragments in this document are sometimes lacking modifiers or methods
that would be needed to make them strictly legal Java.}.  Using Java
has many advantages, including programmer familiarity, availability of
compiler frameworks and a robust language specification.  However, the
resulting syntax can be cumbersome, and in the future we plan to
develop a cleaner and more abstract syntax that is designed
specifically for stream programs.

\begin{figure}[htpb]
\begin{minipage}{3in}
\psfig{figure=fir-streamit.eps,width=61.42mm}
\caption{\protect\small An FIR filter in StreamIt.~~~~~
\protect\label{fig:firstreamit}}
\end{minipage}
\begin{minipage}{3in}
\centering
\vspace{10pt}
\psfig{figure=basic-pipeline.eps,width=1.8in}

(a) A Pipeline. \\
\vspace{10pt}
\psfig{figure=basic-splitjoin.eps,width=1.8in}

(b) A SplitJoin. \\
\vspace{10pt}
\psfig{figure=basic-feedback.eps,width=1.8in}

(c) A FeedbackLoop. \\
\caption{\protect\small Stream structures supported by StreamIt.
\protect\label{fig:structuresp}
}
\end{minipage}
\vspace{-8pt}
\end{figure}

\subsection{Filters}

The basic unit of computation in StreamIt is the Filter.  An example of
a Filter is the {\tt FIRFilter}, shown in Figure~\ref{fig:firstreamit}.  The central
aspect of a filter is the {\tt work} function, which describes the
filter's most fine grained execution step in the steady state.  Within
the {\tt work} function, a filter can communicate with neighboring
blocks using the {\tt input} and {\tt output} channels, which are FIFO
queues declared as fields in the Filter base class.  These high-volume
channels support the three intuitive operations: 1) {\bf {\tt pop()}}
removes an item from the end of the channel and returns its value, 2)
{\bf {\tt peek($i$)}} returns the value of the item $i$ spaces from the
end of the channel without removing it, and 3) {\bf {\tt push($x$)}}
writes $x$ to the front of the channel.  The argument $x$ is passed by
value; if it is an object, a separate copy is enqueued on the channel.
% \begin{enumerate}
% \item {\tt pop()} removes an item from the end of the channel and
% returns its value.
% \item {\tt peek($i$)} returns the value of the item $i$ spaces from
% the end of the channel without removing it.
% \item {\tt push($x$)} writes $x$ to the front of the channel.  The
% argument $x$ is passed by value; if it is an object, a separate copy
% is enqueued on the channel.
% \end{enumerate}

A major restriction of StreamIt 1.0 is that it requires filters to have
static input and output rates.  That is, the number of items peeked,
popped, and pushed by each filter must be constant from one invocation
of the {\tt work} function to the next.  In fact, as described below,
the input and output rates must be declared in the filter's {\tt init}
function.  If a filter violates the declared rates, StreamIt throws a
runtime error and the subsequent behavior of the program is
undefined. We plan to support dynamically changing rates in a future
version of StreamIt.

Each Filter also contains an {\tt init} function, which is called at
initialization time.  The {\tt init} function serves two purposes.
Firstly, it is for the user to establish the initial state of the
filter.  For example, the FIRFilter records {\tt weights}, the
coefficients that it should use for filtering.  A filter can also push,
pop, and peek items from within the {\tt init} function if it needs to
set up some initial state on its channels, although this usually is not
necessary.  A user should instantiate a filter by using its constructor,
and the {\tt init} function will be called implicitly with the same
arguments that were passed to the constructor.
%{\footnote{This design
%might seem unnatural, but it is necessary to allow inlining (Section
%\ref{sec:connecting}) within a Java-based syntax.}}.

The second purpose of the {\tt init} function is to specify the
filter's I/O types and data rates to the StreamIt compiler.  The types
are specified with calls to {\tt setInput} and {\tt setOutput}, while
the rates are specified with calls to {\tt setPush}, {\tt setPop}, and
{\tt setPeek}.  The {\tt setPeek} call can be ommitted if the peek
count is the same as the pop count.

%\begin{figure}[t]
%\begin{minipage}{3in}
%\psfig{figure=fir-proc.eps,width=67.5mm}
%\vspace{-2pt}
%\caption{\protect\small An optimized FIR filter in a procedural
%   language.  A complicated loop nest is required to avoid mod
%   functions and to use memory efficiently, and the structure of the
%  loops depends on the data rates (e.g., BLOCK\_SIZE) within the
%   stream.  An actual implementation might inline the calls to {\tt step}.
%\protect\label{fig:firprocedural}}
%\end{minipage}
%\hspace{0.3in}
%\vspace{-2pt}
%\begin{minipage}{3in}
%\psfig{figure=fir-object.eps,width=61.43mm}
%\caption{\protect\small An FIR filter in an object oriented language.
%A ``pull model'' is used by each filter object to retrieve a chunk of
%data from its source, and straight-line code connects one filter to
%another.
%\protect\label{fig:firobject}}
%\end{minipage}
%\end{figure}


\subsection{Connecting Filters}
\label{sec:connecting}

\begin{figure}[t]
\begin{minipage}{3in}
% \psfig{figure=fft-streamit.eps,width=165.88pt}
% \vspace{-18pt}
% \caption{\protect\small A Fast Fourier Transform (FFT) in StreamIt.
% \protect\label{fig:fft}}
% \end{minipage}
\psfig{figure=echo.eps,width=58.92mm}
\vspace{-8pt}
\caption{\protect\small An echo effect in StreamIt.
\protect\label{fig:echo}}
\end{minipage}
\hspace{0.2in}
\begin{minipage}{3in}
\psfig{figure=fib-streamit.eps,width=62.5mm}
\vspace{8pt}
\caption{\protect\small A FeedbackLoop version of Fibonnacci.
\protect\label{fig:feed}}
\end{minipage}
\vspace{-2pt}
\end{figure}

StreamIt provides three constructs for composing filters into a
communicating network: Pipeline, SplitJoin, and FeedbackLoop (see
Figure~\ref{fig:structuresp}).  Each structure specifies a pre-defined
way of connecting filters into a single-input, single-output block,
which we will henceforth refer to as a ``stream''.  That is, a stream
is any instance of a Filter, Pipeline, SplitJoin, or FeedbackLoop.
% Generally, a Pipeline is for building a sequence of streams, a
% SplitJoin is for running streams in parallel, and a FeedbackLoop is
%for introducing a loop in the stream graph.  
Every StreamIt program is a hierarchical composition of these stream
structures.

The {\bf Pipeline} construct is for building a sequence of streams.
Like a Filter, a Pipeline has an {\tt init} function that is called upon
its instantiation.  Within {\tt init}, component streams are added to
the Pipeline via successive calls to {\tt add}.  For example, in the
{\tt AudioEcho} in Figure~\ref{fig:echo}, the {\tt init} function adds
four streams to the Pipeline: an {\tt AudioSource}, an {\tt EchoEffect},
an {\tt Adder}, and a {\tt Speaker}.  This sequence of statements
automatically connects these four streams in the order specified.  Thus,
there is no {\tt work} function in a Pipeline, as the component streams
fully specify the behavior.  The channel types and data rates are also
implicit from the connections.

Each of the stream constructs can either be executed on its own, or
embedded in an enclosing stream structure.  The {\tt AudioEcho} can
execute independently, since the first component consumes no items and
the last component produces no items.  However, the {\tt EchoEffect}
must be used as a component, since the first stream inputs items and the
last stream outputs items.  When a stream is embedded in another
construct, the first and last components of the stream are implicitly
connected to the stream's neighbors in the parent construct.

% The input and output channels of the Pipeline itself are connected to
% the first and last component streams, respectively.  Thus, a Pipeline
% can be embedded in another stream construct, and the channels of the
% first and last components are implicitly connected to the Pipeline's
% neighbors in the parent construct.  Alternatively, if a Pipeline
% consumes no items from its input and produces no items to its output (as
% is the case with {\tt AudioEcho}), then the Pipeline can be run as an
% independent unit--a toplevel StreamIt program.  Any Pipeline, SplitJoin,
% or FeedbackLoop that has consumes no input and produces no output is a
% valid toplevel program in StreamIt.

% Of course, there are a number of semantic restrictions on the
% construction of Pipelines.  For each pair of streams that is
% connected, the output type of the first stream must match the input
% type of the next.  Also, there must be non-zero production and
% consumption rates along the inner connections of a Pipeline, such that
% data flows through the entire pipe.  We omit a full discussion of
% semantic checking in StreamIt due to lack of space.

The {\bf SplitJoin} construct is used to specify independent parallel
streams that diverge from a common {\it splitter} and merge into a
common {\it joiner}.  As in a Pipeline, the components of a SplitJoin
are specified with successive calls to {\tt add} from the {\tt init}
function.  For example, the {\tt EchoEffect} in Figure~\ref{fig:echo}
adds two streams that run in parallel, each of which is a {\tt Delay}
filter.

The splitter specifies how items from the input of the SplitJoin are
distributed to the parallel components.  For simplicity, we allow only
compiler-defined splitters, of which there are three types: 1) {\it
Duplicate}, which replicates each data item and sends a copy to each
parallel stream, 2) {\it RoundRobin($i_1$, $i_2$, $\dots$, $i_k)$},
which sends the first $i_1$ data items to the stream that was added
first, the next $i_2$ data items to the stream that was added second,
and so on, and 3) {\it Null}, which means that none of the parallel
components require any input, and there are no input items to split.  If
the weights are ommitted from a RoundRobin, then they are assumed to be
equal to one for each stream.  Note that RoundRobin can function as an
exclusive selector if one or more of the weights are zero.
% \begin{enumerate}
% \item {\tt Duplicate}, which replicates each data item and sends a copy to each
% parallel stream.
% \item {\tt RoundRobin($i_1$, $i_2$, $\dots$, $i_k)$},
% which sends the first $i_1$ data items to parallel stream 1, the next
% $i_2$ data items to parallel stream 2, and so on.  If the weights are
% ommitted, then they are assumed to be equal to one for each stream.  Note
% that RoundRobin can function as an exclusive selector if one or more of the
% weights are zero.
% \item {\tt Null}, which means that none of the parallel components require
% any input, and there are no input items to split.
% \end{enumerate}

Likewise, the joiner is used to indicate how the outputs of the parallel
streams should be interleaved on the output channel of the SplitJoin.
There are two kinds of joiners: 1) {\it RoundRobin}, whose function is
analogous to a RoundRobin splitter, and 2) {\it Null}, which means that
none of the parallel components produce any output, and there are no
output items to join.
% \begin{enumerate}
% \item {\tt RoundRobin($i_1$, $i_2$, $\dots$, $i_k)$}, whose function is analogous to a RoundRobin splitter.
% \item {\tt Null}, which means that none of the parallel components produce
% any output, and there are no output items to join.
% \end{enumerate}
The splitter and joiner types are specified with calls to {\tt
setSplitter} and {\tt setJoiner}, respectively.  The {\tt EchoEffect}
uses a Duplicate splitter so that each item appears both directly and as
an echo; it uses a RoundRobin joiner to interleave the immediate signals
with the delayed ones.  In {\tt AudioEcho}, an {\tt Adder} is used to
combine each pair of interleaved signals.

The {\bf FeedbackLoop} construct provides a way to create cycles in the
stream graph.  The {\tt Fibonacci} stream in Figure \ref{fig:feed}
illustrates the use of this construct.  Each FeedbackLoop contains: 1) a
body stream, which is the block around which a backwards ``feedback
path'' is being created, 2) a loop stream, which can perform some
computation along the feedback path, 3) a splitter, which distributes
data between the feedback path and the output channel at the bottom of
the loop, and 4) a joiner, which merges items between the feedback path
and the input channel at the top of the loop.  These components are
specified from within the {\tt init} function via calls to {\tt
setBody}, {\tt setLoop}, {\tt setSplitter}, and {\tt setJoiner},
respectively.
%
% Each FeedbackLoop contains: 1) a
% joiner, which is at the top of the loop and merges the outside input
% stream with the ``feedback path'' from downstream, 2) a body stream,
% which comprises the forward path of the loop, 3) a splitter, which
% distributes data between the output channel and the feedback path, and
% 4) a loop stream, which can perform some computation along the feedback
% path.  These components are specified from within the {\tt init}
% function via calls to {\tt setJoiner}, {\tt setBody}, {\tt setSplitter},
% and {\tt setLoop}, respectively.  
%
The splitters and joiners can be any of those for SplitJoin, except for
Null.  The call to {\tt setLoop} can be ommitted if no computation is
performed along the feedback path.

The FeedbackLoop has a special semantics when the stream is first
starting to run.  Since there are no items on the feedback path at
first, the stream instead inputs items from an {\tt initPath} function
defined by the FeedbackLoop; given an index $i$, {\tt initPath}
provides the $i$'th initial input for the feedback joiner.  With a
call to {\tt setDelay} from within the {\tt init} function, the user
can specify how many items should be calculated with {\tt initPath}
before the joiner looks for data items from the feedback channel.

%Evident in the {\tt Fibonnacci} example of Figure \ref{fig:feed} is
%another feature of the StreamIt syntax: {\it inlining}.  The definition
%of any stream can be inlined at the point of its instantiation, thereby
%preventing the definition of many small classes that are used only once,
%and, moreover, providing a syntax that reveals the hierarchical
%structure of the streams from the indentation level of the code.  In our
%Java syntax, we make use of anonymous classes for inlining \cite{java}.

Detailed informaiton of the language can be found
in~\cite{streamitcc}. We have written many programs to understand the
scope and limitations of the language. For example, we have written
multiple versions of matrix multiply kernels, sorting algorithms, FIR
filters and FFT filters in StreamIt. We have also implemented many
applications such as a Cyclic Redundancy Checker, an FM Radio with an
Equalizer, a Reed Solomon Decoder for HDTV, a Vocoder~\cite{seneff80},
a Radar Array Front End~\cite{pca}, a GSM Decoder~\cite{gsm}, and the physical
layer of the 3GPP Radio Assess Protocol~\cite{3gpp}.  We plan to
incorporate the knowlege and insight grained from developing these
applications to the next version of StreamIt.
