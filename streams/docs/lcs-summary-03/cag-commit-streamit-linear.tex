% Name your report <group>-<unique-name>.tex, (e.g., sls-jupiter.tex)
% to avoid name collisions.  For \label{} entries within a file,
% please use \<group>-<unique-name>-<your label} (e.g.,
% \label{sls-jupiter-figure}), again to avoid name collisions.  There
% is no need to use unique names with \cite{} since each report has
% its own citation namespace.

% You can also use \formattitle{Title}{Authors} to gain formatting
% control (e.g., footnotes on author names or controlling line breaks
% with \\) and \formatcontents{Title}{Authors}.  Since we did not need
% such control, we simply used \formattitlecontents{Title}{Authors}
% here.  Please do not put any special formatting in
% \formattitlecontents or \formatcontents as this will disturb the
% table of contents.

\formattitlecontents
{A Finite-State Transducer Representation of Phonological Rules}%
{Lee Hetherington}

% My own macros start with <group name><report name> to keep them
% distinct from macros used in other reports.

\newenvironment{slsilhrules}%
  {\begin{center}\begin{tabular}{cccclc}}%
  {\end{tabular}\end{center}}

\newcommand{\slsilhrule}[4]{%
  \{\textit{#1}\}&\textit{#2}&\{\textit{#3}\}&$\Rightarrow$&\textit{#4}&;\\}


% Please use the following \formatsection entries unless they are
% inappropriate: Introduction, Approach, Progress, Future, Research
% Support.
%
% Do not put a blank line after a \formatsection, as this affects the
% formatting.

\formatsection{Introduction}
In our SUMMIT speech recognition system, we have always used
phonological rules to model pronunciation variability.  We use such
rules to transform word phonemic baseforms to graphs of alternate
pronunciations at the phonetic level.  These rules account for many
different phonological phenomena such as place assimilation,
gemination, epenthetic silence insertion, alveolar stop flapping, and
schwa deletion.  These rules primarily depend on input-level phonemic
context, but phonetic realizations may also depend on adjacent
realizations.

Kaplan and Kay \cite{Kaplan94} provide a comprehensive treatment of
phonological rewrite rules where the target, replacement, left
context, and right context are all regular expressions.  Mohri and
Sproat \cite{Mohri96} present an alternative compilation of such rules
that is more efficient, but each rule is still compiled into no less
than five transducers, with the transducers for the individual rules
composed together to implement the entire set of rules.  In this work
we have formulated a simpler phonological rule system that we can
compile into finite-state transducers (FSTs) much more efficiently.

\formatsection{Approach}
Our rules are of the form
\begin{slsilhrules}
  \slsilhrule{$\lambda_1\ldots\lambda_m$}{$\phi$}{$\rho_1\ldots\rho_n$}{$\psi$}
\end{slsilhrules}
Here the left and right contexts $\lambda$ and $\rho$, respectively,
consist of the set of immediately adjacent phonemic contexts that must
be satisfied in order for the rule to apply, transforming target
phoneme $\phi$ to replacement phonetic realization $\psi$.  Here, only
replacement is a regular expression.  Furthermore, if there is a set
of rules with the same target phoneme $\phi$, only the \emph{first}
that matches its context will fire.  Finally, our rules are batch
rules in that all applicable rules are applied simultaneously across
all input positions in one batch.  Rules are also obligatory, meaning
that if a rule matches, the rule will be applied.  Altogether, this
results in rule application that is deterministic.

We also have the ability to place constraints on the realization of
rules $\psi$ so that they can depend on the specific replacements of
adjacent phonemes.  Consider the set of rules:
\begin{slsilhrules}
  \slsilhrule{}{d}{y}{dcl $($d $|$ jh$)$}
  \slsilhrule{}{t}{y}{tcl $($t $|$ ch$)$}
  \slsilhrule{d t}{y}{}{y $|$ $<$\{ch jh\}}
\end{slsilhrules}
Together, these rules can model the palatalization of the stops in
``would you'' and ``hit you'' by mapping a phonemic sequence /d y/ to
a phonetic realization [dcl jh], as well as the non-palatalized
realization [dcl d y].  Here, $<$\{ch jh\} denotes an explicit
reference to a phonetic realization [ch] or [jh] on the left.
Similarly, the notation \{\ldots\}$>$ can be used to specify a surface
phonetic constraint on the right.  See
Figure~\ref{sls-ilh-rules-figure} for a graphical depiction of the
effect of phonological rule application on the phrase ``would you.''

To compile our phonological rules into a transducer $P$, we factor it
into three component transducers such that $P = I \circ R \circ S$.
$I$ enforces input contextual constraints and outputs the particular
rule to apply to each input symbol.  $R$ performs replacement for each
particular rule.  Finally, $S$ enforces surface contextual constrains
such as those specified with the $<$\{\ldots\} and \{\ldots\}$>$
notation.  Please see \cite{Hetherington01} for a complete description
of the rule compilation algorithm.

\formatsection{Progress}
Within our SUMMIT recognition system, we currently use about 200
phonological rules to map from baseform phonemic pronunciations to a
variety of allowable phonetic realizations.  We have 63 phonemic input
symbols, with variations including deletable stop releases, flappable
stops, and silence.  The rules output 71 phonetic output symbols.
Compiling these rules takes about 2.0s on an 866MHz Pentium III and
produces a minimized transducer P containing 294 states and 21,009
transitions.  After compilation, these rules can be applied very
rapidly to a lexicon.  This new phonological rule application system
is about 100 times faster than our previous, non-FST-based approach.

\formatsection{Future}
In the future, we plan to train probabilities on the various
phonological rule realizations.  We will most likely use the EM
algorithm starting from a set of phonetically and orthographically
labelled training utterances.

\formatsection{Research Support}
This research was supported by DARPA under Contract N66001-99-1-8904
monitored through the Naval Command, Control, and Ocean Surveillance
Center.

% Please use \includegraphics to include Encapsulated PostScript figures.

\begin{figure}
\includegraphics[width=\textwidth]{sls-ilh-rules.eps}

\caption{The graph of possible phonetic realizations produced by
	applying our phonological rules to the word sequence
	``would you'' with phonemic pronunciations
	/w uh d y (uw $|$ ax)/.}
\label{sls-ilh-rules-figure}
\end{figure}

% Although the use of BibTeX is highly recommended, you can manually
% format your references as follows:
%
% \begin{thebibliography}{1}
%
% \bibitem{Zue00}
% V.~Zue, S.~Seneff, J.~R. Glass, J.~Polifroni, C.~Pao, T.~J. Hazen, and
% L.~Hetherington, ``\textsc{Jupiter}: A telephone-based conversational
% interface for weather information,'' \textit{IEEE Transactions on
% Speech and Audio Processing}, vol. 8, no.  1, pp. 85--96, Jan. 2000.
% 
% \end{thebibliography}

\bibtex{sls-ilh}
