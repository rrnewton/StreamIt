\section{Fission of a Single Input / Single Output Filter}
\label{sec:single}

In this section we introduce our techniques by defining the
process of fission on a filter with a single input and a single
output. Portions of this process will be used for the more general
techniques discussed in Section~\ref{sec:general}. 

Given a stateless filter $F \in V$ and a fission factor $P$, we will
data-parallelize $F$ by creating $P$ modified copies of the filter
$F_1, ..., F_P$.  Each copy is constructed to execute an equal
percentage firings of $F$ in $S$.  The initialization stage is handled
in a special manner.  If $F$ executes in $I$ (i.e., $M(I,F) > 0$),
then our transformation constructs $F_1$ such that it performs all the
computation and data distribution of $F$ in $I$. $F$ has a single
incoming edge ${U,F}$ and a single outgoing edge ${F, D}$.

\todo{Figure showing the transformation}

We make the following assertions on $F$:

\begin{enumerate}

\item $M(S,F) \mod  P = 0$

\item $C(F) < O(S,F) \times M(S,F)$

\item $M(S,F) \times O(S,F) \ge 2(e(S,F) - o(S,F))$

\end{enumerate}

\todo{Discuss these more!}

\subsection{Initialization Stage Construction}

For the initialization stage the transformation defines the following,
where $1 < x \le P$. First, the initialization
schedule multiplicity and rates of filter $F_x$ are set to zero
because they are not involved in this stage:
$$ M(I,F_x) \equiv o(W_p,F_x) \equiv u(W_p,F_x) \equiv e(W_p,F_x)
\equiv 0 $$
$$ M(I,F_1) \equiv 1 $$

Next we define the properties of $F_1$ in the initialization stage. In
the transformation, $W_P(F_1)$ is constructed to execute all of the
firings of $F_1$ from $I$ in a single firing.  First $W_P(F)$ is
called once, and $W(F)$ is executed $M(I,F) - 1$ times:

\begin{algorithm}
$W_P(F_1)$:
\begin{algorithmic}[1]
\State $W_P(F)$
\For{$M(I,F) - 1$}
\State $W(F)$
\EndFor
\end{algorithmic}
\end{algorithm}

The rates in $I$ of $F_1$ are as follows:
 
\begin{eqnarray*} 
o(W_p,F_1) & \equiv & o(W_p,F) + [(M(I,F) - 1) \times o(W,F)] \\
e(W_p,F_1) & \equiv & \max(e(W_p,F), o(W_p,F_1) + e(W,F)) \\
u(W_p,F_1) & \equiv & u(W_P,F) + [M(I,F - 1) \times u(W_P,F)]
\end{eqnarray*} 

\begin{eqnarray*}
\mt{IW}(I,F_1) & \equiv & (1) \\
\mt{IE}(I,F_1) & \equiv & (\{U,F_1\}) 
\end{eqnarray*}

\begin{eqnarray*} 
\mt{OW}(I,F_1) & \equiv & (1) \\
\mt{OE}(I,F_1) & \equiv & ((\{F_1, D\})) 
\end{eqnarray*} 

\subsection{Steady-State Construction}
Define the following: $1 \le i \le P$.
 
$$ M(S,F_i) \equiv 1 $$

\begin{eqnarray*} 
o(W,F_i) & \equiv & \frac{M(S,F)}{P} \times o(W,F) + e(W,F)\\
e(W,F_i) & \equiv & 0 \\
u(W,F_i) & \equiv & \frac{M(S,F)}{P} \times u(W,F)
\end{eqnarray*} 

\begin{algorithm}
$W(F_i)$:
\begin{algorithmic}[1]
\For{$M(S,F)/P$}
\State $W(F)$
\EndFor
\State {\tt pop(}$e(W,F)${\tt)}
\end{algorithmic}
\end{algorithm}

