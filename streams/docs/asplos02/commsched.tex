\begin{figure}
\centering
\psfig{figure=joiner-buffer-dead.eps,width=4.59in}
\parbox{6in}{\caption{\protect\small Example of deadlock in a
SplitJoin.  As the Joiner is reading items from the stream on the
left, items accumulate in the channels on the right.  On Raw, senders
will block once a channel has four items in it.  Thus, once 10 items
have passed through the Joiner, the system is deadlocked, as the
Joiner is trying to read from the left, but the stream on the right is
blocked.  The Identity filter is blocking on the send of item \#10,
and the Splitter is blocking on the send of item \#20.  If the weights
on the Joiner were (10, 10), the system would be deadlock-free.
\protect\label{fig:joiner-dead}}}

\vspace{48pt}

\psfig{figure=joiner-buffer-live.eps,width=3.97in}
\caption{\protect\small Fixing the deadlock with a buffering Joiner.
The BufferingRoundRobin is an internal StreamIt construct (it is not
part of the language) which reads items from its input channels in the
order in which they arrive, rather than in the order specified by its
weights.  The order of arrival is determined by a simulation of the
stream graph's execution; thus, the system is guaranteed to be
deadlock-free, as the order given by the simulation is feasible for
execution on Raw.  To preserve the semantics of the Joiner, the items
are written to the output channel from the internal buffers in the
order specified by the Joiner's weights.  The ordered items are sent
to the output as soon as they become available.
\protect\label{fig:joiner-live}}
\end{figure}

\section{Communication Scheduler}
\label{sec:communic}

With the nodes of the stream graph assigned to computation nodes of
the target, the next phase of the compiler must map the communication
explicit in the stream graph to the interconnect of the target.  This
is the task of the communication scheduler.  The communication
scheduler maps the infinite FIFO abstraction of the stream channels to
the limited resources of the target.  The communication scheduler must
avoid deadlock and starvation while trying to utilize the parallelism
explicit in the stream graph.

The exact implementation of the communication scheduler is tied to the
communication model of the target.  The simplest mapping would occur
for targets implementing an end-to-end, infinite FIFO abstraction, in
which the scheduler needs only to determine the sender and receiver of
each data item.  This information is easily calculated from the
weights of the Splitters and Joiners.  As the communication model
becomes more constrained, the communication scheduler becomes more
complex, requiring analysis of the stream graph. For targets
implementing a finite, blocking nearest-neighbor communication model,
the exact ordering of tile execution must be specified.

Due to the static nature of StreamIt, the compiler can statically
orchestrate the communication resources.  As described in
Section~\ref{sec:phases}, we create an initialization schedule and a
steady-state schedule that fully describe the execution of the stream
graph.  The schedules can give us an order for execution of the graph
if necessary.  We can generate ordering to minimize buffer length,
maximize parallelism, or minimize latency.

%Thus, we can create a communication scheduler of arbitrary
%detail.  If the architecture must statically orchestrate all aspects
%of communication, the StreamIt language provides this facility.

Deadlock must be carefully avoided in the communication
scheduler. Each architecture requires a different deadlock avoidance
mechanism and we will not go into a detailed explanation of deadlock
here.  In general, deadlock occurs when there is a circular dependence
on resources.  A circular dependence can surface in the stream graph
or in the routing pattern of the layout.  If the architecture does not
provide sufficient buffering, the scheduler must serialize all
potentially deadlocking dependencies.

%Two StreamIt language constructs that can lead to deadlock are
%FeedbackLoops and Joiners. Figure ?? describes the potential for
%deadlock in a FeedbackLoop.  One potential solution is to forbid
%filters of the FeedbackLoop to interleave sends and receives.  This
%has the effect of serializing the FeedbackLoop. In figure ?? we
%descibe the potential for deadlock introduced by Joiner nodes.  A
%solution to this problem is described below.

\subsection{Communication Scheduler for Raw}
\label{sec:rawcommunic}

The communication scheduling phase of the StreamIt compiler maps
StreamIt's channel abstraction to Raw's static network.  As mentioned
in Section \ref{sec:raw}, Raw's static network provides optimized,
nearest neighbor communication.  Tiles communicate using buffered,
blocking sends and receives.  It is the compiler's responsibility to
statically orchestrate the explicit communication of the stream graph
while preventing deadlock.

To statically orchestrate the communication of the stream graph, the
communication scheduler simulates the firing of nodes in the stream
graph, recording the communication as it simulates.  The simulation
does not model the code inside each Filter; instead it assumes that
each Filter fires instantaneously.  This relaxation is possible
because of the flow control of the static network--since sends block
when a channel is full and receives block when a channel is empty, the
compiler needs only to determine the ordering of the sends and
receives rather than arranging for a precise rendezvous between sender
and receiver.

% We simulate the graph for both an initialization schedule and a steady
% state schedule (see Section \ref{sec:phases}).  A ``push schedule'' is
% used for both phases.  We define a push schedule as one that always
% fires the node that is the furthest downstream (among those that have
% enough inputs to fire).  
% 
% A push schedule allows the implementation to
% disregard the FIFO buffers that connect each neighboring node in the
% stream graph.  This is because data items will not be accumulating at
% the source, dest, or intermediate nodes.  Each destination will
% consume the data item as it is produced (modulo the latency of
% routing).

Special care is required in the communication scheduler to avoid
deadlock in SplitJoin constructs.  Figure~\ref{fig:joiner-dead}
illustrates a case where the naive implementation of a SplitJoin would
cause deadlock in Raw's static network.  The fundamental problem is
that some SplitJoins require a buffer of values at the Joiner
node--that is, the Joiner outputs values in a different order than it
receives them.  This can cause deadlock on Raw because the buffers
between channels can hold only four elements; once a channel is full,
the sender will block when it tries to write to the channel.  If this
blocking propagates the whole way from the Joiner to the Splitter,
then the entire SplitJoin is blocked and can make no progress.

To avoid this problem, the communication scheduler implements internal
buffers in the Joiner node instead of exposing the buffers on the Raw
network (see Figure~\ref{fig:joiner-live}).  As the execution of the
stream graph is simulated, the scheduler records the order in which
items arrive at the Joiner, and the Joiner is programmed to fill its
internal buffers accordingly.  At the same time, the Joiner outputs
items according to the ordering given by the weights of the
RoundRobin.  That is, the sending code is interleaved with the
receiving code in the Joiner; no additional items are input if a
buffered item can be written to the output stream.  To facilitate code
generation (Section \ref{sec:codegen}), the maximum buffer size of
each internal buffer is recorded.

Our current implementation of the communication scheduler is overly
cautious in its deadlock avoidance.  All FeedbackLoops are serialized
by the communication scheduler to prevent deadlock.  More precisely,
the loop and body streams of each FeedbackLoop cannot execute in
parallel.  Crossed routes in the layout of the graph are serialized as
well, forcing each path to wait its turn at the contention point.
