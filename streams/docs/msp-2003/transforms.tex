\section{Graph Transformations}
\label{sec:transforms}

In this section we present a set of flexible transformations that can
be used to adjust the hierarchy and communication patterns of a
structured stream graph.  These transformations serve two purposes.
Firstly, they provide a means of deriving a canonical form: a program
representation that is insensitive to changes in the source code and
is easily analyzed by the compiler (for example, the hierarchical
ragged arrays of the partitioner).  Secondly, they serve as the
implementation mechanism by which the compiler arrives at an efficient
executable once it has analyzed the canonical form (for example, the
partitioner's traceback function).

Thus, we consider the transformations in pairs, each of which
represents a step in the opposite direction with respect to the
partitioner's representation.  After describing the transformations,
we describe the sequence in which they are employed when interfacing
to and from the partitioner.

\subsection{Vertical Cut Transformations}

The vertical cut transforms effect the distribution of parallel
streams across a hierarchy of splitjoin constructs (see
Figures~\ref{code:vert} and~\ref{ex:vert}).  The {\tt lowerChildren}
transform can factor any adjacent subset of a splitjoin's children
into a new splitjoin, which becomes a child of the original.  This is
a simple matter of re-arranging the weights in the splitters and
joiners.  It qualifies as a ``vertical cut'' transform because two
adjacent applications can divide a splitjoin with a vertical line.

The {\tt raiseChildren} transform will reverse the effect of any {\tt
lowerChildren} transform.  However, independent applications of this
transform are comparably rare: as detailed in the pseudocode, a child
can only be raised if its splitter type matches that of its parent,
and if the sums of its split and join weights match the corresponding
slots in the parent.

\subsection{Horizontal Cut Transformations}

The horizontal cut transforms are capable of transforming between a
splitjoin and a pipeline of splitjoins (see Figures~\ref{code:horiz}
and~\ref{ex:horiz}).  The {\tt addMatchingSyncPoints} transformation
inputs a rectangular splitjoin--one with pipeline children that have
equal lengths--and factors them into a sequence of splitjoins with
children that have unit length.  For the purpose of program analysis,
any splitjoin can be made rectangular by extending its shorter streams
with {\tt Identity} filters (a pre-defined node in StreamIt.)  Like an
hourglass, this transformation has the effect of squeezing a splitjoin
together at points of synchronization.

The {\tt removeMatchingSyncPoints} transformations directly implements
the inverse of {\tt addMatchingSyncPoints} via a simple re-arrangement
of weights.  Though very simple, this transformation can apply in
practice on interface boundaries where compound data streams are being
interleaved at a static rate.

A more aggressive transformation for synchronization removal is {\tt
removeStructuredSyncPoints} (see Figures~\ref{code:sync}
and~\ref{ex:sync}).  This algorithm removes not only synchronization
points with exactly matching weights, but it also expands any
join/split pair where the set of incoming and outgoing streams can be
partitioned such that no data item is passed between members of
different partitions during a steady-state execution.  The only
disadvantage of this transformation over {\tt
removeMatchingSyncPoints} is that it has the potential to introduce
non-adjacent joiners, which in the StreamIt compiler requires
additional processor resources.  However, like all transformations
described in this paper, the output of {\tt
removeStructuredSyncPoints} is still a structured graph (as the name
would suggest).

We have also implemented horizontal cut transforms for pipelines,
which consist simply of adding or removing a wrapper pipeline around a
sub-segment of the stream.  This is very straightforward (requiring no
change of weights or rates) so we omit the details here.

\subsection{Fusion and Fission Transformations}

Filter fusion describes a transform where several adjacent filters are
combined into one, while filter fission refers to the parallelization
of a filter via conversion to a splitjoin or pipeline.  These
transformations are at the heart of the load balancing partitioner
described in this paper.  A detailed description of fusion, fission,
and other intra-node transformations are described
in~\cite{streamit-asplos}; we restrict our attention to graph-level
transformations in this paper.

\subsection{Transforming to Partitioner's Representation}

To build the hierarchical arrays required by the partitioner, the
following transformations are applied, starting from the leaf nodes
and propagating upward through the stream:
\begin{itemize}

\item {\tt raiseChildren} and {\tt removeStructuredSyncPoints}
wherever possible.

\item {\tt removeMatchingSyncPoints} wherever possible.

\item Convert all non-rectangular splitjoins to rectangular ones by
extending their shorter child pipelines with Identity filters.

\item {\tt addMatchingSyncPoints} on all splitjoins.
\end{itemize}

The last step above will have the effect of replacing every splitjoin
with a pipeline of unit-length splitjoin constructs.  This pipeline is
exactly the representation that is sent to the partitioner; the width
of row {\tt y} is the width of the splitjoin at that location.  

\subsection{Transforming from Partitioner's Representation}

The traceback function is what calculates the sequence of
transformations needed to transform from the partitioner's
representation to the load-balanced set of partitions for program
execution.  In Figure~\ref{code:trace}, the points corresponding to a
vertical and horizontal cut are where traceback recurses after finding
the optimal {\tt xPivot} (for vertical cut) or {\tt yPivot} for
horizontal cut.  In our implementation, these cuts are recorded during
traceback, and then replayed on the stream graph at a later time.  The
only extra step to take is that synchronization must be removed before
any vertical cut is applied.

An example of the conversion to and from the partitioner's
representation appears in Figure~\ref{fig:trans}.

%% \section{Graph Transformations}
%% \label{sec:transforms}

%% We have developed a large suite of graph transformations that can be
%% used to adjust the hierarchy and communication patterns of a
%% structured stream graph.  These transformations--which include filter
%% fusion, filter fission, horizontal cuts, vertical cuts, and aggressive
%% synchronization removal--serve two distinct purposes.  Firstly, they
%% provide a means of deriving a canonical form: a program representation
%% that is insensitive to changes in the source code and is easily
%% analyzed by the compiler (for example, the hierarchical ragged arrays
%% of the partitioner).  Secondly, they serve as the implementation
%% mechanism by which the compiler arrives at an efficient executable
%% once it has analyzed the canonical form (for example, the
%% partitioner's traceback function).  In our final paper, we will
%% describe these transformations in detail, as well as their role in the
%% structured partitioning algorithm.
