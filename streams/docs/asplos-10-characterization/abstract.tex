Stream programs represent an important class of high-performance
computations.  Defined by their regular processing of sequences of
data, stream programs appear most commonly in the context of audio,
video, and digital signal processing, though also in networking,
encryption, and other areas.  Stream programs can be naturally
represented as a graph of independent actors that communicate
explicitly over data channels.

In this work, we provide the most comprehensive characterization of
stream programs that has been published to date.  We focus on the
StreamIt benchmark suite, a set of 67 programs that have been
implemented in the StreamIt programming language.  The suite consists
of 33,800 lines of code, written by over 20 developers during the last
8 years.

We characterize stream programs along several axes, including their
patterns of data flow (sliding windows, stateful actors, feedback
loops, etc.), their scheduling characteristics (input/output rates,
dynamic vs. static rates, actor multiplicity, etc.), and their
utilization of StreamIt language constructs (structured streams,
teleport messaging, etc.).  The lessons learned have implications for
the design of future architectures, languages and compilers for the
streaming domain.

%The suite spans 29 realistic applications, 4 graphics rendering
%pipelines, 19 libraries and kernels, 8 sorting routines, and 7 toy
%examples.
