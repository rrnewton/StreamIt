% This file is a raw Latex listing of the options to the StreamIt compiler.  
% It is put in its own file so that it can be included from both the release
% document and the cookbook.

\newcommand\metavar[1]{$\langle$\emph{#1}$\rangle$}
% don't throw error if previously defined filename command
\providecommand\filename[1]{\textsf{#1}}
\begin{description}
\item [-{}-help]
Displays a summary of common options.

\item [-{}-more-help]                                                                                                   
Displays a summary of advanced options (which are not described below).

\item[-{}-cluster \metavar{n}]
Compile for a cluter or multicore with \metavar{n} nodes.

\item[-{}-library]
Produce a Java file compatible with the StreamIt Java library,
and compile and run it.

\item[-{}-simpleC]
Generate a simple C file that inlines the entire application into a
single function.  This is sometimes more readable than the default
uniprocessor output, but the backend is not fully-featured.

\item[-{}-raw \metavar{n}, -r \metavar{n}]
Compile for an \metavar{n}-by-\metavar{n} Raw processor.

\item[-{}-rstream, -R]
Generate a C-like file to be compiled by the RStream compiler from
Reservoir Labs.

\item[-{}-output \metavar{filename}, -o \metavar{filename}]
Places the resulting binary in \metavar{filename}.

\item [-{}-verbose]
Show intermediate commands as they are executed.
\end{description}

\subsection*{Options available for all backends}
\begin{description}
\item [-O0]
Do not optimize (default).

\item [-O1]
Perform basic optimizations that should improve performance in most
cases. Adds \texttt{--unroll 16 --destroyfieldarray --partition --wbs}.

\item [-O2]
Perform extended optimizations that should improve performance in
most cases, but may also cause the compiler to become unstable.
Adds \texttt{--unroll 256 --destroyfieldarray --partition --wbs --macros}.

\item[-{}-asciifileio]
Specifies that FileReader's and FileWriter's should use ASCII format
rather than binary.

\item [-{}-iterations \metavar{n}, -i\metavar{n}]
Run the program for \metavar{n} steady-state iterations. Defaults to
infinity.  For the uniprocessor, cluster, and simpleC backends, the
number of iterations can also be passed at the command line of the
final executable (\texttt{a.out -i 100}).

\item [-{}-linearreplacement]
Domain-specific optimization: combine adjacent ``linear'' filters in
the program into a single matrix multiplication operation wherever
possible.  Corresponds to the ``linear'' option in the PLDI'03 paper.

\item [-{}-statespace]
In combination with \texttt{--linearreplacement}, performs combination
and optimization of linear statespace filters as described in the
CASES'05 paper.

\item [-{}-unroll \metavar{n}, -u\metavar{n}]
Specify loop unrolling limit. The default value is 0.
\end{description}
          
\subsection*{Options specific to Uniprocessor and Cluster backends}
\begin{description}

\item [-{}-cacheopt]
Performs cache optimizations as described in the LCTES'05 paper.

\item [-{}-l1d \metavar{n}]
Sets the L1 data cache size (in KB) for cache optimizations.  The
default is 8 KB.

\item [-{}-l1i \metavar{n}]
Sets the L1 instruction cache size (in KB) for cache optimizations.
The default is 8 KB.

\item [-{}-l2 \metavar{n}]
Sets the L2 cache size (in KB) for cache optimizations (we assume a
unified L2 cache).  The default is 256 KB.

\item [-{}-linearpartition, -L]
Domain-specific optimization: perform linear replacement and frequency
replacement selectively, based on an estimate of where it is most
beneficial.  Corresponds to the ``autosel'' option in the PLDI'03
paper.  (Relies on FFTW installation.)

\end{description}

\subsection*{Options specific to Raw backend}
\begin{description}
\item [-{}-numbers \metavar{n}, -N\metavar{n}]
Instrument code to gather performance statistics on simulated code
over \metavar{n} steady-state cycles. The results are placed in
\filename{results.out} in the current directory.

\item [-{}-ssoutputs \metavar{n}]
For applications containing a dynamic I/O rate, this option indicates
how many outputs should count as a steady-state when gathering numbers
(with \texttt{--numbers}).

\item [-{}-rawcol \metavar{m}, -c\metavar{m}]
Specify number of columns in Raw processor; --raw specifies number of rows.

\item [-{}-wbs] When laying out communication instructions, use the 
work-based simulator to estimate exactly when items will be produced
and consumed.  This improves the scheduling of routing instructions.

\end{description}
