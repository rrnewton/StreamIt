With the continued miniaturization of the transistor, microprocessor
performance is increasingly dominated by wire delays.  Several
architectures are addressing this problem by replicating processing
units and exposing the communication between units to a software
layer.  There are two basic approaches for compiling to a
communication-exposed target.  A "time multiplexing" approach utilizes
the entire chip for each computation unit, and switches between units
over time.  A "space multiplexing" approach distributes computation
units across the entire chip, running them continuously and in
parallel.

In this paper, we describe a hybrid space-time multiplexing scheme
that provides the flexible load-balancing of time multiplexing while
preserving the locality and latency benefits of space multiplexing.
Our work is done in the context of StreamIt, a high-level stream
programming language, with a backend that targets the Raw
architecture.  Our compiler extracts fine-grained "traces" from a
concurrent stream graph; it then uses space multiplexing for code
within a trace, but time multiplexing to switch between traces.  We
also describe software pipelining of traces and near-optimal code
generation for traces that compute a linear function.  We give
performance results that demonstrate the efficacy of space-time
multiplexing in compiling StreamIt to Raw.
