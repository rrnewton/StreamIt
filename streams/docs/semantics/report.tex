\documentclass[twocolumn, draft]{article}
\usepackage{psfig, wide}
\title{StreaMIT Intermediate Representation \\ Version 0.3-pre}
\author{Stream Worshipers Inc.}


\begin{document}
\maketitle

\section{Basic Ideas}

The stream representation is composed of a number of basic building blocks.
All of these blocks are put together to form a graph, which can be drawn
on a 2D surface with no intersections.  The graph is almost a DAG, except
we allow feedback loops in arrangement of filters.

All blocks can be aggregated into larger superblocks:  Pipes, StreamSelects,
FeedbackLoops, Splits and FilterBlocks.  
All of these superblocks can be named.  Every 
(super)block has a StreamID associated with it.  A message can be sent 
to a StreamID, which will cause all highest-level blocks prepared to
receive messages contained in within the named block
to receive this msg.

A named (super)block can also be reinitialized.  This will cause all the 
in/outflowing buffers to be purged.

\section{Filters}
A new filter type is defined as follows:

\begin{verbatim}
type FooType is Filter
{
// insert data/functions here
};
\end{verbatim}

\subsection{Input/Output}

Each filter is required to have a single input and single output:
\begin{verbatim}
input inputType fooInput;
output outputType fooOutput;
\end{verbatim}

\subsection{Work}

Most of the work done by a filter is described in a Work function:

\begin{verbatim}
Work ()
{
  fooOutput [0] = fooInput [0] + fooInput [1];
  fooInput.release (2);
  fooInput.advance (2);
  fooOutput.push (1);
}
\end{verbatim}

Work accepts no parameters.  It is allowed to send messages.  It is also
allowed to re-initialize other blocks.
Input/output
is accessed as an array.  For input, element 0 is first new element of input,
element 1 is second new element of input, etc.  Element -1 is the first seen
element of input.

\subsection{Init}

When a new filter is being constructed, a constructor (called Init) is called.
This constructor has a chance to initialize the filter with some initial
data (passed as arguments):

\begin{verbatim}
type FooType is Filter
{
  StreamID parent;
  float amplification;
  Init (float a, StreamID p)
  {
    parent = p;
    amplification = a;
  }
};
\end{verbatim}

Init is not allowed to send messages or initialize other blocks.

\subsection{Message}

Each filter is allowed to receive messages:

\begin{verbatim}
Message (Msg fooMsg)
{
// .. handle msg here
}
\end{verbatim}

A message is defined as follows:

\begin{verbatim}
struct Msg
{
  StreamID source;
  int type;
  int length;
  void *data;
};
\end{verbatim}

Message is allowed to send other messages and/or reinitialize blocks.
Message is not allowed to read stream input or output any stream data.
Initialization of a filter causes this filter's message queue to be emptied.
A filter is guaranteed that all of its appropriate predecessors (in the graph)
will have received the msg before it does.

\section{FilterBlock}

FilterBlock is used to aggregate filters and superblocks.  It is a 
single input/single output structure.  It is allowed to have Init and Message
functions, but it is not allowed to have a Work function, nor define its
input or output.

\subsection{Init}

Init function for a FilterBlock is similar to Init function of a Filter.
FilterBlock, however, can also contain Filters/superblocks:

\begin{verbatim}
Init (float amp, float fLow, float fHigh, StreamID p)
{
  LowPass lowPassFilter (aLow);
  HighPass highPassFilter (aHigh);
  Amplifier amplifier (amp);
  parent = p;
}
\end{verbatim}

The above creates three filters connected in series.

\subsection{Message}

The Message function for a FilterBlock looks/works
identically to Filter's Message function.

\section{StreamID}

As seen before, StreamID is used to keep track of other blocks in the 
program.  They can only be assigned values in Init function, only 
values from parameters and/or internal blocks can be used to assign
to a StreamID variable, and a value of a StreamID variable must be
static for every invocation of the Init function (this is somewhat
clumsy - can we fix this somehow?).

StreamID is used to call Init function on a particular block and for
sending messages.  There are several exaples of StreamID in this document.

\section{Pipe}

The above Init function aggregates three filters into one.  While this is
useful, we define another construct to do essentially the same thing
without having to define an all new type.  This new construct will
allow us to aggregate an arbitrary serial connection of fileters/superblocks:

\begin{verbatim}
Init (float amp, float fLow, float fHigh, StreamID p)
{
  Pipe BandPassFilter
  {
    LowPass lowPassFilter (aLow);
    HighPass highPassFilter (aHigh);
  } 
  Amplifier amplifier (amp);
  parent = p;
}
\end{verbatim}

Pipe allows us only to connect single input/single output blocks.

\section{Split}

In order to allow for splitting of single streams into multiples,
we define a Split construct:

\begin{verbatim}
Init ()
{
  Split(RoundRobin, RoundRobin)
  {
    LeftAudioFilter leftAudio;
    RightAudioFilter rightAudio;
  }
}
\end{verbatim}

The first field passed to Split is the type of splitting (RoundRobin,
Duplicate).  Second field passed to Split is type of merge (RoundRobin)
[shouldn't there be more?!!].  The body of Split defines all single input/
single output filters which are part of the Split.

\section{Custom Split/Merge}

The predefined Split construct does not allow for much flexability in
creating interesting designs of stream flow.  Following two base types
are used for defining custom splits and merges:

\begin{verbatim}
type LeftHeavySplit is CustomSplit
{
  input float audio;
  output float leftAudio;
  output float rightAudio;
  Work ()
  {
    leftAudio[0] = audio[0];
    leftAudio[1] = audio[1];
    rightAudio[0] = audio[2];

    audio.release (3);
    audio.advance (3);
    leftAudio.push (2);
    rightAudio.push (1);
  }
};

type LeftHeavyMerge is CustomMerge
{
  input float leftAudio;
  input float rightAudio;
  output float audio;
  Work ()
  {
    audio[0] = leftAudio [0];
    audio[1] = leftAudio [1];
    audio[2] = rightAudio [2];

    leftAudio.release (2);
    leftAudio.advance (2);
    rightAudio.release (1);
    rightAudio.advance (1);
    audio.push (3);
  }
};

type LeftHeavyFilter is FilterBlock
{
  input float inAudio;
  output float outAudio;
  Init ()
  {
    LeftHeavySplit s;
    LeftAudioFilter l;
    RightAudioFilter r;
    LeftHeavyMerge m;
  }
};
\end{verbatim}

Both CustomMerge and CustomSplit are allowed to have Init and Message
functions.  Their functionality and definition syntax is overall very
similar to that of Filter, except Filter is allowed only single input/
single output, where as CustomSplit can have multiple outputs and 
CustomMerge can have multiple inputs.

When using CustomSplit/CustomMerge, we simply declare the Split, followed
by appropriate number of singe input/single output blocks, followed by
a CustomMerge of appropriate number of inputs.

\section{Sources/Sinks}

Sources and Sinks are very similar to Filters, except they either have
no inputs or outputs.  They are defined just like Filters (without
input/output), and used just like Filters.  A FilterBlock can begin/end
its Init function with a Source/Sink.  This will cause it to have
no inputs/outputs.

A block with no input/output can be used in the middle of a FilterBlock (or
other appropriate structure).  The constraint is that after/before block(s) 
with no input/output, a Merge/Join is required.  These can be either
custom or defined as a part of Split construct.

\begin{verbatim}
type Microphone is Source
{
  float vol;
  output float audio;
  Init (float freq, float v)
  {
    vol = v;
    InitHardwareMic (freq);
  }
  Work()
  {
    audio [0] = vol * GetInputFromMic ();
    audio.push (1);
  }
};

type Speaker is Sink
{
  float vol;
  input float audio;
  Init (float v)
  {
    vol = v;
  }
  Work ()
  {
    SendToSpeaker (vol * audio[0]);
  }
};

type Karaoke
{
  StreamID me;
  Init (float micVol, float speakerVol)
  {
    // first RoundRobin is unnecesary..
    Split (RoundRobin, RoundRobin)
    {
      Microphone left (micVol);
      Microphone right (micVol);
    }

    // would give strange sound effect here
    LeftHeavySplit customSplit; 
    Speaker left;
    Speaker right;
    me = this.GetID ();
  }
  Message(Msg volume)
  {
    me.Init (float (thisvolume.type));
  }
};

\end{verbatim}

\section{StreamSelect}

StreamSelect allows us to switch configurations of the stream graph.
The syntax is as follows:

\begin{verbatim}
enum AudioType { Classical, Vocal, Heavy };

Init (AudioType preset, float volume)
{
  Microphone (1);
  Pipe control
  {
    StreamSelect equalizer (preset)
    {
      case Classical:
        FilterBass (0.5);
        FilterTreble (1);
      case Vocal:
        // do nothing here...
        // Vocal can be played 
        // on speaker as is...
      case Heavy:
        FilterBass (1.5);
        FilterTreble (0.75);
    }
    Amplifier (volume);
  }
  Speaker (1);
};
\end{verbatim}

There is a restriction on parameters allowed to control StreamSelect - they
must all be constants computable at compile time.

\section{FeedbackBlock}

Some stream applications require feedback loops for data processing.
Here we define a type that allows us to create feedback loops
in our stream graph.

A FeedbackBlock type is very similar to FilterBlock, but it uses two inputs 
and one output.  The first input
is from the outside stream.  The second input is the feedback input.
The feedback input is defined using FeedbackPath construct.  FeedbackPath
allows us to specify in the Init function what type of FeedbackPath will
be used.  FeedbackPath must be used exactly once per Init function, and
may only be used in FeedbackBlock.

A FeedbackBlock does not have a Work function, but its first non-source
block must be a Merge.

\begin{verbatim}
type NegFeedbackAmplifierMerge is CustomMerge
{
  input float audioIn;
  input float feedback;
  output float audioOut;
  float ampGain;
  float amplification;
  Work ()
  {
    // this basically implements a negative feedback
    // loop with an amp of set gain and a set target
    // amplification.
    float in = audioIn [0];
    float out = feedback [0];
    float negFeed = in + 
      (in + out) / (-amplification + 1);
    float result = - negFeed * ampGain;
    audioOut [0] = result;
	
    audioIn.release (1);
    audioIn.advance (1);
    feedback.release (1);
    feedback.advance (1);
    audioOut.push (1);
  }

  // note: here a is the POSITIVE amplification
  Init (float a, float gain)
  {
    ampGain = gain;
    amplification = -a;
  }
};

type NegFeedbackBuffer is Filter
{
  input float in;
  output float out;
  Work ()
  {
    in [0] = out [0];

    in.release (1);
    in.advance (1);
    out.push (1);
  }

  Init (int delay)
  {
    for (int i = 0; i < delay; i++)
    {
      out [i] = 0;
    }
    out.push (delay);
  }
};

type NegFeedbackAmplifier is NegFeedbackAmp
{
  input float in;
  input float feedback;
  output float out;

  Init(float amp)
  {
    // gain = amp * 10 should be sufficient
    NegFeedbackAmplifierMerge (amp, amp * 10);

    // feedback will be delayed by 2 
    FeedbackPath NegFeedbackBufffer myBuffer (2);
  }
};

\end{verbatim}


%\begin{figure}[t]
%\centerline{\psfig{figure=block.ps}}
%\vspace{-24pt}
%\caption{\protect\small The block construct.}
%\end{figure}

%\section{old stuff}

%\begin{lgrind}
%% Remember to use the lgrind style

\Head{}
\File{code.cc}{2001}{Feb}{20}{09:05}{2114}
\L{\LB{\C{}\1\1_the_base_class_for_blocks}}
\CE{}\L{\LB{\K{c}\K{l}\K{a}\K{s}\K{s}_\V{B}\K{l}\K{o}\K{c}\V{k}_\{}}
\L{\LB{__\C{}\1\1_this_method_is_recognized_by_our_compiler_as_representing_an_instance}}
\CE{}\L{\LB{__\C{}\1\1_of_a_given_object.}}
\CE{}\L{\LB{__\K{s}\K{t}\K{a}\K{t}\K{i}\K{c}_\K{f}\K{i}\K{n}\K{a}\K{l}_\V{B}\K{l}\K{o}\K{c}\V{k}\V{D}\K{e}\K{s}\K{c}_\K{c}\K{r}\K{e}\K{a}\K{t}\K{e}(\K{v}\K{o}\K{i}\K{d}\*_\K{a}\K{r}\K{g}\N{1}_=_\N{0},_\K{v}\K{o}\K{i}\K{d}\*_\K{a}\K{r}\K{g}\N{2}_=_\N{0},_.\,.\,.);}}
\L{\LB{\}}}
\L{\LB{}}
\L{\LB{\C{}\1\1_the_base_class_for_splitters}}
\CE{}\L{\LB{\K{c}\K{l}\K{a}\K{s}\K{s}_\V{S}\K{p}\K{l}\K{i}\K{t}\K{t}\K{e}\K{r}_\{}}
\L{\LB{__\K{s}\K{t}\K{a}\K{t}\K{i}\K{c}_\K{f}\K{i}\K{n}\K{a}\K{l}_\V{S}\K{p}\K{l}\K{i}\K{t}\V{D}\K{e}\K{s}\K{c}_\K{c}\K{r}\K{e}\K{a}\K{t}\K{e}(\K{v}\K{o}\K{i}\K{d}\*_\K{a}\K{r}\K{g}\N{1}_=_\N{0},_\K{v}\K{o}\K{i}\K{d}\*_\K{a}\K{r}\K{g}\N{2}_=_\N{0},_.\,.\,.);}}
\L{\LB{\}}}
\L{\LB{}}
\L{\LB{\C{}\1\1_the_base_class_for_mergers}}
\CE{}\L{\LB{\K{c}\K{l}\K{a}\K{s}\K{s}_\V{M}\K{e}\K{r}\K{g}\K{e}\K{r}_\{}}
\L{\LB{__\K{s}\K{t}\K{a}\K{t}\K{i}\K{c}_\K{f}\K{i}\K{n}\K{a}\K{l}_\V{M}\K{e}\K{r}\K{g}\K{e}\V{D}\K{e}\K{s}\K{c}_\K{c}\K{r}\K{e}\K{a}\K{t}\K{e}(\K{v}\K{o}\K{i}\K{d}\*_\K{a}\K{r}\K{g}\N{1}_=_\N{0},_\K{v}\K{o}\K{i}\K{d}\*_\K{a}\K{r}\K{g}\N{2}_=_\N{0},_.\,.\,.);}}
\L{\LB{\}}}
\L{\LB{}}
\L{\LB{\C{}\1\1_the_base_class_for_feedback_loop_initializers}}
\CE{}\L{\LB{\K{c}\K{l}\K{a}\K{s}\K{s}_\V{L}\K{o}\K{o}\K{p}\V{I}\K{n}\K{i}\K{t}\K{i}\K{a}\K{l}\K{i}\V{z}\K{e}\K{r}_\{}}
\L{\LB{__\V{L}\K{o}\K{o}\K{p}\V{I}\K{n}\K{i}\K{t}\V{D}\K{e}\K{s}\K{c}_\K{c}\K{r}\K{e}\K{a}\K{t}\K{e}(\K{v}\K{o}\K{i}\K{d}\*_\K{a}\K{r}\K{g}\N{1}_=_\N{0},_\K{v}\K{o}\K{i}\K{d}\*_\K{a}\K{r}\K{g}\N{2}_=_\N{0},_.\,.\,.);}}
\L{\LB{__\C{}\1\1_not_sure_about_the_arguments_here,_but_there\4s_some_initializer_function}}
\CE{}\L{\LB{__\K{v}\K{i}\K{r}\K{t}\K{u}\K{a}\K{l}_\K{v}\K{o}\K{i}\K{d}_\K{i}\K{n}\K{i}\K{t}(\K{i}\K{n}\K{t}_\K{d}\K{e}\K{l}\K{a}\V{y},_\K{v}\K{o}\K{i}\K{d}\*\*_\K{d}\K{a}\K{t}\K{a})_=_\N{0};}}
\L{\LB{\}}}
\L{\LB{}}
\L{\LB{\K{c}\K{l}\K{a}\K{s}\K{s}_\V{P}\K{a}\K{r}\V{D}\K{e}\K{s}\K{c};}\Tab{40}{\C{}\1\1_descriptor_for_parallel}}
\CE{}\L{\LB{\K{c}\K{l}\K{a}\K{s}\K{s}_\V{B}\K{l}\K{o}\K{c}\V{k}\V{D}\K{e}\K{s}\K{c}_\K{e}\V{x}\K{t}\K{e}\K{n}\K{d}\K{s}_\V{P}\K{a}\K{r}\V{D}\K{e}\K{s}\K{c};}\Tab{40}{\C{}\1\1_descriptor_for_1-to-1}}
\CE{}\L{\LB{}}
\L{\LB{\C{}\1\1_pipe_must_take_at_least_two_arguments,_and_returns_a_pipeline_of_the}}
\CE{}\L{\LB{\C{}\1\1_enclosed_block_descriptions.}}
\CE{}\L{\LB{\V{B}\K{l}\K{o}\K{c}\V{k}\V{D}\K{e}\K{s}\K{c}_\K{p}\K{i}\K{p}\K{e}(\V{B}\K{l}\K{o}\K{c}\V{k}\V{D}\K{e}\K{s}\K{c}_\K{a}\K{r}\K{g}\N{1}_=_\N{0},_\V{B}\K{l}\K{o}\K{c}\V{k}\V{D}\K{e}\K{s}\K{c}_\K{a}\K{r}\K{g}\N{2}_=_\N{0},_.\,.\,.);}}
\L{\LB{}}
\L{\LB{\C{}\1\1_for_constructing_parallel_stuff_when_for_contents_of_a_parallel}}
\CE{}\L{\LB{\V{P}\K{a}\K{r}\V{D}\K{e}\K{s}\K{c}_\K{p}\K{a}\K{r}(\V{P}\K{a}\K{r}\V{D}\K{e}\K{s}\K{c}_\K{a}\K{r}\K{g}\N{1}_=_\N{0},_\V{P}\K{a}\K{r}\V{D}\K{e}\K{s}\K{c}_\K{a}\K{r}\K{g}\N{2}_=_\N{0},_.\,.\,.);}}
\L{\LB{}}
\L{\LB{\C{}\1\1_the_splitter_construct}}
\CE{}\L{\LB{\V{B}\K{l}\K{o}\K{c}\V{k}\V{D}\K{e}\K{s}\K{c}_\K{s}\K{p}\K{l}\K{i}\K{t}(\V{S}\K{p}\K{l}\K{i}\K{t}\V{D}\K{e}\K{s}\K{c}_\K{s}\K{p}\K{l}\K{i}\K{t}\K{t}\K{e}\K{r},_\V{P}\K{a}\K{r}\V{D}\K{e}\K{s}\K{c}_\K{b}\K{o}\K{d}\V{y},_\V{M}\K{e}\K{r}\K{g}\K{e}\V{D}\K{e}\K{s}\K{c}_\K{m}\K{e}\K{r}\K{g}\K{e}\K{r})}}
\L{\LB{}}
\L{\LB{\C{}\1\1_the_feedback_construct}}
\CE{}\L{\LB{\V{B}\K{l}\K{o}\K{c}\V{k}\V{D}\K{e}\K{s}\K{c}_\K{l}\K{o}\K{o}\K{p}(\V{M}\K{e}\K{r}\K{g}\K{e}\V{D}\K{e}\K{s}\K{c}_\K{m}\K{e}\K{r}\K{g}\K{e}\K{r},_\V{B}\K{l}\K{o}\K{c}\V{k}\V{D}\K{e}\K{s}\K{c}_\K{b}\K{l}\K{o}\K{c}\V{k},_\K{i}\K{n}\K{t}_\K{d}\K{e}\K{l}\K{a}\V{y},_\V{L}\K{o}\K{o}\K{p}\V{I}\K{n}\K{i}\K{t}\V{D}\K{e}\K{s}\K{c}_\K{i}\K{n}\K{i}\K{t})}}
\L{\LB{}}
\L{\LB{\C{}\1\1_the_scoping_for_control_constructs}}
\CE{}\L{\LB{\V{B}\K{l}\K{o}\K{c}\V{k}\V{D}\K{e}\K{s}\K{c}_\K{c}\K{o}\K{n}\K{t}\K{r}\K{o}\K{l}\V{G}\K{r}\K{o}\K{u}\K{p}(\V{B}\K{l}\K{o}\K{c}\V{k}\V{D}\K{e}\K{s}\K{c}_\K{b}\K{l}\K{o}\K{c}\V{k})}}
\L{\LB{}}
\L{\LB{\C{}\1\1_the_construct_for_going_from_a_StreamDesc_to_a_block}}
\CE{}\L{\LB{\V{B}\K{l}\K{o}\K{c}\V{k}_\K{i}\K{n}\K{s}\K{t}\K{a}\K{n}\K{t}\K{i}\K{a}\K{t}\K{e}(\V{B}\K{l}\K{o}\K{c}\V{k}\V{D}\K{e}\K{s}\K{c}_\K{d}\K{e}\K{s}\K{c});}}
\L{\LB{}}
\L{\LB{\C{}\1\1_examples}}
\CE{}\L{\LB{\V{B}\K{l}\K{o}\K{c}\V{k}\V{D}\K{e}\K{s}\K{c}_\K{s}\K{t}\K{r}\K{e}\K{a}\K{m}\N{1}_=_\K{p}\K{i}\K{p}\K{e}(\V{M}\V{y}\V{S}\K{o}\K{u}\K{r}\K{c}\K{e}.\K{c}\K{r}\K{e}\K{a}\K{t}\K{e}(\N{0},\N{1},\S{}\3init\3\SE{}),}}
\L{\LB{}\Tab{24}{_\V{M}\V{y}\V{F}\K{i}\K{l}\K{t}\K{e}\K{r}.\K{c}\K{r}\K{e}\K{a}\K{t}\K{e}(),}}
\L{\LB{}\Tab{24}{_\V{M}\V{y}\V{S}\K{i}\K{n}\V{k}.\K{c}\K{r}\K{e}\K{a}\K{t}\K{e}(\N{1.}\N{0},\S{}\4a\4\SE{}));}}
\L{\LB{}}
\L{\LB{\V{B}\K{l}\K{o}\K{c}\V{k}\V{D}\K{e}\K{s}\K{c}_\K{s}\K{t}\K{r}\K{e}\K{a}\K{m}\N{2}_=_\K{s}\K{p}\K{l}\K{i}\K{t}(\V{D}\V{U}\V{P}\V{L}\V{I}\V{C}\V{A}\V{T}\V{O}\V{R}.\K{c}\K{r}\K{e}\K{a}\K{t}\K{e}(\N{2}),}}
\L{\LB{}\Tab{16}{__________\V{R}\V{O}\V{U}\V{N}\V{D\_}\V{R}\V{O}\V{B}\V{I}\V{N}.\K{c}\K{r}\K{e}\K{a}\K{t}\K{e}(\N{2}),}}
\L{\LB{}\Tab{24}{__\K{p}\K{a}\K{r}(\V{M}\V{y}\V{F}\K{i}\K{l}\K{t}\K{e}\K{r}.\K{c}\K{r}\K{e}\K{a}\K{t}\K{e}(),_\V{M}\V{y}\V{F}\K{i}\K{l}\K{t}\K{e}\K{r}.\K{c}\K{r}\K{e}\K{a}\K{t}\K{e}()));}}
\L{\LB{}}
\L{\LB{\V{B}\K{l}\K{o}\K{c}\V{k}\V{D}\K{e}\K{s}\K{c}_\K{s}\K{t}\K{r}\K{e}\K{a}\K{m}\N{3}_=_\K{l}\K{o}\K{o}\K{p}(\V{M}\V{y}\V{M}\K{e}\K{r}\K{g}\K{e}\K{r}.\K{c}\K{r}\K{e}\K{a}\K{t}\K{e}(),_\K{p}\K{i}\K{p}\K{e}(\K{s}\K{t}\K{r}\K{e}\K{a}\K{m}\N{1},_\K{s}\K{t}\K{r}\K{e}\K{a}\K{m}\N{2}),_\N{5},_}}
\L{\LB{}\Tab{40}{____\V{M}\V{y}\V{I}\K{n}\K{i}\K{t}\K{i}\K{a}\K{l}\K{i}\V{z}\K{e}\K{r}.\K{c}\K{r}\K{e}\K{a}\K{t}\K{e}());}}
\L{\LB{}}
\L{\LB{\C{}\1\1_for_creating_n_copies_of_myfilter_in_parallel}}
\CE{}\L{\LB{\V{P}\K{a}\K{r}\V{D}\K{e}\K{s}\K{c}_\K{b}\K{o}\K{d}\V{y}_=_\K{p}\K{a}\K{r}();}}
\L{\LB{\K{f}\K{o}\K{r}_(\K{i}\K{n}\K{t}_\K{i}=\N{0};_\K{i}\<\K{n};_\K{i}++)_\{}}
\L{\LB{____\K{b}\K{o}\K{d}\V{y}_=_\K{p}\K{a}\K{r}(\K{b}\K{o}\K{d}\V{y},_\V{M}\V{y}\V{F}\K{i}\K{l}\K{t}\K{e}\K{r}.\K{c}\K{r}\K{e}\K{a}\K{t}\K{e}(\K{i}));}}
\L{\LB{\}}}
\L{\LB{\V{B}\K{l}\K{o}\K{c}\V{k}\V{D}\K{e}\K{s}\K{c}_\K{s}\K{t}\K{r}\K{e}\K{a}\K{m}\N{4}_=_\K{s}\K{p}\K{l}\K{i}\K{t}(\V{D}\V{U}\V{P}\V{L}\V{I}\V{C}\V{A}\V{T}\V{O}\V{R}.\K{c}\K{r}\K{e}\K{a}\K{t}\K{e}(\K{n}),_\K{b}\K{o}\K{d}\V{y},_\V{R}\V{O}\V{U}\V{N}\V{D\_}\V{R}\V{O}\V{B}\V{I}\V{N}.\K{c}\K{r}\K{e}\K{a}\K{t}\K{e}(\K{n}));}}
\L{\LB{}}
\L{\LB{\C{}\1\1_run_the_last_example}}
\CE{}\L{\LB{\K{i}\K{n}\K{s}\K{t}\K{a}\K{n}\K{t}\K{i}\K{a}\K{t}\K{e}(\K{s}\K{t}\K{r}\K{e}\K{a}\K{m}\N{4}).\K{r}\K{u}\K{n}();}}

%\end{lgrind}


\end{document}

