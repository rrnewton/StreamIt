\section{Background and Related Work}

\section{Dataflow Graphs}

\begin{itemize}

\item SURE's and Computation Graphs

\item The min-buffers for maximal-rate schedules stuff

\item The ptolemy / SDF / CSDF guys.  Scheduling, buffer minimization,
everything they tried to do.

\item  check out this buffer minimization 
  http://epubs.siam.org/sam-bin/dbq/article/29035

\item uniformization of SARE's:
  http://www.irisa.fr/bibli/publi/pi/2000/1350/1350.html

\end{itemize}

\section{Systems of Affine Recurrence Equations}

SARE/Feautrier, affine dependences, static control flow programs.

Lim/Lam; affine scheduling.

A System of Affine Recurrence Equations (SARE) is a set of equations
${\cal E}_1 \dots {\cal E}_{m_{\cal E}}$ of the following form~\cite{DRV00}:
\begin{equation}
\forall {\vec i} \in {\cal D_{\cal E}}, X({\vec i}) = f_{\cal E}(\dots, Y({\vec h}_{{\cal
E}, Y}({\vec i})), \dots)
\label{eq:sare}
\end{equation}
In this equation, our notation is as follows:
\begin{itemize}

\item $\{X, Y, \dots\}$ is the set of {\it variables} defined by the
SARE.  Variables can be considered as multi-dimensional arrays mapping
a vector of indices ${\vec i}$ to a value in some space ${\cal V}$.

\item ${\cal D_{\cal E}}$ is a polyhedron representing the {\it
domain} of the equation ${\cal E}$.  Each equation for a variable $X$
has a disjoint domain; the domain $D_X$ for variable $X$ is taken as
the convex union of all domains over which $X$ is defined.

\item ${\vec h}_{{\cal E}, Y}$ is a vector-valued affine\footnote{If
the dependence function ${\vec h}$ is a translation (${\vec h}({\vec
i}) = {\vec i} - {\vec k}$ for constant ${\vec k}$), then the SARE is
{\it uniform}, and referred to as a SURE or SRE~\cite{karp67}.}
function, giving the index of variable $Y$ that $X({\vec i})$ depends
on.  A vector-valued function ${\vec h}$ is affine if it can be
written as ${\vec h}({\vec i}) = C{\vec i} + {\vec d}$, where C is a
constant array and ${\vec d}$ is a constant vector that do not vary
with ${\vec i}$.

\item $f_{\cal E}$ is a strict function used to compute the elements
of $X$.

\end{itemize}
