\section{StreamIt Overview}

Most data-flow or signal-processing algorithms can be broken down into
a number of simple blocks with connections between them.  In StreamIt
parlance, the smallest block is a \emph{filter}; it has a single input
and a single output, and its body consists of Java-like code.  Filters
are then connected by placing them into one of three composite blocks:
pipelines, split-joins, and feedback loops.  Each of these structures
also has a single input and a single output, so these blocks can be
recursively composed.

A typical streaming application might be a software FM radio, as shown
in Figure \ref{fig:fmradio}.  The program receives its input from an
antenna, and its output is connected to a speaker.  The main program
is a pipeline with a band-pass filter for the desired frequency, a
demodulator, and an equalizer; the equalizer in turn is made up of a
split-join, where each child adjusts the gain over a particular
frequency range, followed by a filter that adds together the outputs
of each of the bands.

\begin{figure}[htbp]
  \begin{center}
    \includegraphics{cookbook.0}
    \caption{Stream graph for a software FM radio}
    \label{fig:fmradio}
  \end{center}
\end{figure}

Our goal with choosing these constructs was to create a language with
most of the expressiveness of a general data-flow graph structure, but
to keep the block-level abstraction that modern programming languages
offer.  Allowing arbitrary graphs makes scheduling and partitioning
difficult for the compiler.  The hierarchical graph structure allows
the implementation of blocks to be ``hidden'' from users of the block;
for example, an FFT could be implemented as a single filter or as
multiple filters, but so long as there is a stream structure named
``FFT'' somewhere in the program the actual implementation is
irrelevant to other modules that use it.  Since most graphs can be
readily transformed into StreamIt structures, StreamIt is suitable for
working on a wide range of signal-processing applications.

%%% Local Variables:
%%% TeX-master: "cookbook.tex"
%%% End:
