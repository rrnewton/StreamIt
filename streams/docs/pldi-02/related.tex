\section{Related Work}
\label{sec:related}

A large number of programming languages have included a concept of a
stream, with various semantic formalisms; see \cite{survey97} for a
survey.  Those that are perhaps most related to the static-rate
version of StreaMIT are synchronous dataflow languages such as
LUSTRE~\cite{lustre} and ESTEREL~\cite{esterel92} which require a
fixed number of inputs to arrive simultaneously before firing a stream
node.  However, most special-purpose stream languages are functional
instead of imperative, and do not contain features such as messaging
and support for modular program development that are essential for
modern stream applications.  Also, these languages lack the structured
streams of StreamIt, which enable a suite of hierarchical compiler
optimizations and a clean semantics for verifying program
transformations.

At an abstract level, the stream graphs of StreaMIT share a number of
properties with the synchronous dataflow (SDF) domain as considered by
the Ptolemy project~\cite{ptolemyoverview}.  Each node in an SDF graph
produces and consumes a given number of items, and there can be delays
along the arcs between nodes (corresponding loosely to items that are
peeked in StreaMIT). As in StreaMIT, SDF graphs are guaranteed to have
a static schedule, testing for deadlock is decidable, and there have
been many efforts to minimize their memory requirements~\cite{leesdf,
murt2001x1, gov94, goddardmanaging}.  However, nodes such as round
robins that have a cyclic pattern of I/O rates fall outside of SDF and
within the Cyclo-Static domain \cite{lauwereins94geometric} where
there are fewer scheduling results.  To the best of our knowledge, the
phased scheduling algorithm for minimal latency is novel.