\section{Cache Model for Streaming}
\label{sec:cache-model}

From a caching point of view, it is intuitively clear that once a
actor's instruction working set is fetched into the cache, we can
maximize instruction locality by running the actor as many times as
possible.  This of course assumes that the total code size for
all actors in the steady state exceeds the capacity of the
instruction cache.
%% In
%% Figure~\ref{fig:ssi-single} we show a representative breakdown of the
%% code size per actor in a steady state execution of a StreamIt
%% implementation of a Fast Fourier Transform (FFT).
For our benchmarks, the total code size for a steady state
ranges from 2~Kb to over 135~Kb (and commonly exceeds 16~Kb). Thus, while individual actors may have a
small instruction footprint, the total footprint of the actors in a
steady state exceeds a typical instruction cache size.
From these observations, it is evident that we must {\it scale} the
execution of actors in the steady state in order to improve temporal
locality. In other words, rather than running a actor $n$ times per
steady state, we scale it to run $m \times n$ times.
%
%% (e.g., the loop bound for \verb+A_work+ is
%% changed to $\texttt{m} \times 4$ in the example shown earlier). 
We term $m$ the {\it scaling factor}.

The obvious question is: to what extent can we scale the execution of
actors in the steady state? The answer is non-trivial because
scaling, while beneficial to the instruction cache behavior, may
overburden the data cache as the buffers between actors may grow to
prohibitively large sizes that degrade the data cache
behavior. Specifically, if a buffer overflows the cache, then
producer-consumer locality is lost.

In this section we describe a simple and intuitive cache model to
estimate the instruction and data cache miss rates for a steady state
sequence of actor firings. The model serves as a foundation for
reasoning about the cache aware optimizations introduced in this
paper. We develop the model first for the instruction cache, and then
generalize it to account for the data cache.

\subsection{Instruction Cache}

A steady state execution is a sequence of actor firings
$S=(w_1\ldots w_n)$, and a
{\it program execution} corresponds to one or more repetitions of the
steady state. We use the notation $S[i]=w$ to refer to the
work function $w$ that is fired at logical time $i$, and $|S|$ to
denote the length of the sequence. 

Our cache model is simple in that it considers each actor in the
steady state sequence, and determines whether one or more misses are
bound to occur. The miss determination is based on the 
the {\it instruction reuse distance} ($\mt{IRD}$), which is equal to
the number of unique instructions that are referenced between two
executions of the actor under consideration (as they appear in the
schedule). The steady state is a 
compact representation of the whole program execution, and thus, we
simply account for the misses within a steady state, and generalize
the result to the whole program. Within a steady state, an actor is
charged a miss penalty if and only if the number of referenced
instructions since the last execution (of the same actor)
is greater than the instruction cache capacity.

Formally, let $P=\mt{phase}(S,i)$ for $1\le i\le|S|$ represent a
subsequence of $k$ elements of $S$:
\[ 
P[1]=S[i], P[2]=S[i+1], \ldots, P[k]=S[i+k-1],
\]
where $k\in[1,|S|]$ is the smallest integer such that $S[i+k]=S[i]$.
In other words, a phase is a subsequence of $S$ that starts with the
specified actor ($S[i]$) and ends before the next occurence of the same actor
(i.e., there are no interveaning occurences of $S[i]$ in the phase).
Note that because the steady state
execution is cyclic, the construction of the subsequence is allowed to
wrap around the steady state\footnote{In other words, the subsequence
is formed from a new sequence $S'=S|S$ where $|$ represents
concatenation.}. For example, the steady state $S_1=(\texttt{AABB})$
has 
$\mt{phase}(S_1,1)=(\texttt{A})$,
$\mt{phase}(S_1,2)=(\texttt{ABB})$,
$\mt{phase}(S_1,3)=(\texttt{B})$, and
$\mt{phase}(S_1,4)=(\texttt{AAB})$,


Let $I(w)$ equal the code size of work function $w$, and
\[
%\mt{IRD}(S,i)=\sum_{j \in \mt{phase}(S,i)} I(S[j])
\mt{IRD}(S,i)=\sum_{w} I(w)
\]
over all distinct actors $w$ occurring in
$\mt{phase}(S,i)$. We can then determine if a specific work function will
result in an instruction cache miss (on its next firing) by evaluating
the following step function:
\begin{equation}
\label{eq:imiss}
  \mt{IMISS}(S,i) =
    \begin{cases}
      0& \text{if $\mt{IRD}(S,i) \leq C_I$; hit: no cache refill,}\\
      1& \text{otherwise; miss: (some) cache refill.}
    \end{cases}
\end{equation}
In the equation, $C_I$ represents a constant proportional to the
instruction cache size. 

Using 
%the instruction miss ($\mt{IMISS}$) metric in
Equation~\ref{eq:imiss}, we can estimate the instruction miss
rate ($\mt{IMR}$) of a steady state as: 
\begin{equation}
\label{eq:imr}
  \mt{IMR}(S) = \frac{1}{|S|}\sum_{i=1}^{|S|} \mt{IMISS}(S,i).
\end{equation}

The cache model allows us to rank the quality of an
execution ordering: schedules that boost temporal locality
result in miss rates closer to zero, and schedules that do not
exploit temporal locality result in miss rates closer to one.

For example, assuming equal sized actors for the steady
state $S_1=(\texttt{AABB})$ such that
$I(\texttt{A})=I(\texttt{B})=\lceil{C_I/2}\rceil+1$, then the
combined instruction working 
sets exceed the instruction cache. Therefore, 
we expect to suffer a miss at the start of every
steady state because the phase that precedes the execution of
\texttt{A} (at $S_1[1]$) is $\mt{phase}(S_1,2)$ with 
an instruction reuse distance greater than the cache size 
($\mt{IRD}(S_1,2) > C_I$). Similarly, there is a 
miss predicted for the first occurrence of actor \texttt{B} since
$\mt{phase}(S_1,4)=(\texttt{BAA})$ and 
$\mt{IRD}(S_1, 4)>C_I$. Thus, $\mt{IMR}(S_1)=2/4$ whereas for the
following variant $S_2=(\texttt{ABAB})$, $\mt{IMR}(S_2)=1$.
In the case of $S_2$, we know that since the combined
instruction working sets of the actors exceed the cache size, when
actor \texttt{B} is fired following \texttt{A}, it evicts part of
actor \texttt{A}'s instruction working set. Hence when we transition
back to fire actor \texttt{A}, we have to refetch certain
instructions, but in the process, we replace parts of actor
\texttt{B}'s working set. In terms of our model, $\mt{IRD}(S_2,i)>C_I$
for every actor in the sequence, i.e., $1\le i\le|S_2|$.

Note that the
amount of refill is proportional to the number of cache lines that are
replaced when swapping actors, and as such, we may wish to adjust
our cache miss step function ($\mt{IMISS}$). One simple variation is to allow for
some partial replacement without unduly penalizing the overall value
of the metric. Namely, we can allow the constant $C_I$ to be some
fraction greater than the actual cache size. Alternatively, we can use
a more complicated miss function with a more uniform probability
distribution.

\paragraph*{Temporal Locality} In our model, the concept of improving
temporal locality translates to deriving a steady state where, in the
best case, each actor has only one phase that is longer than unit-length.
For example, a permutation of the actors in $S_2$ (where all
phases are of length three) that improves temporal locality
will result in $S_1$, which we have shown has a relatively lower miss rate.


\paragraph*{Execution Scaling}
Another approach to improving temporal locality is to scale the
execution of the actors in the steady state. Scaling increases the
number of consecutive firings of the same actor. In our model, a
scaled steady state has a greater number of unit-length phases (i.e., a
phase of length one and the shortest possible reuse distance).

We represent a scaled execution of the steady state as
$S^m=(w_1^m\dots w_n^m)$: the steady state $S$ is scaled by $m$, which
translates to $m-1$ additional firings of 
each actor. For example, scaling $S_1=(\texttt{AABB})$ by a factor of
two results in
%$S_1^2=(\texttt{\underline{A}A\underline{A}A\underline{B}B\underline{B}B})$
$S_1^2=(\texttt{AAAABBBB})$
and scaling $S_2=(ABAB)$ by the same amount results in 
%$S_2^2=(\texttt{\underline{A}A\underline{B}B\underline{A}A\underline{B}B})$.
$S_2^2=(\texttt{AABBAABB})$;
%; the underlined actors constitute the unscaled steady state.

From Equation~\ref{eq:imiss}, we observe that unit-length phases do
not impact the instruction miss rate  as long as the size of the actor's 
instruction working set  is smaller than the cache
size; we assume this is always the case. Therefore, scaling has the
effect of preserving the pattern of 
miss occurrences while also lengthening the steady state. Mathematically,
we substitute into Equation~\ref{eq:imr}:
\begin{eqnarray}
  \nonumber
  \mt{IMR}(S^m)  &=& \frac{1}{|S^m|}\sum_{i=1}^{|S^m|} \mt{IMISS}(S^m,i) \\
  \label{eq:imrM}
                 &=& \frac{1}{m \times |S|}\sum_{i=1}^{|S|} \mt{IMISS}(S,i).
\end{eqnarray}
The second step is possible because $\mt{IMISS}$ is zero for $m-1$ out
of $m$ executions of each scaled actor.  The result is that the miss
rate is inversely proportional to the scaling factor.
\begin{figure}[t]
\begin{center}
  \psfig{figure=fftc-mult2.eps, width=\columnwidth}
% \nocaptionrule
  \caption{Impact of execution scaling on performance.}
 \label{fig:scaling-data}
\end{center}
\end{figure}

In Figure~\ref{fig:scaling-data} we show a
representative curve relating the scaling factor to overall
performance. The data corresponds to a coarse-grained implementation of 
a Fast Fourier Transform (FFT) running on a Pentium~3 architecture. The
x-axis represents the scaling factors (with increasing values from
left to right). The y-axis represents the execution time and is an
indirect indicator of the miss rate (the two measures are positively
correlated). The execution time improves in accord with our model: 
the running time is shortened as the scaling factor grows larger. There
is however an eventual degradation, and as the sequel will show, it is 
attributed to the data cache performance.


\subsection{Data Cache}

As noted earlier, we can not arbitrarily scale the execution frequency
of a filter without also considering how scaling might impact
the data buffer sizes between filters. In this regard, a filter's
output buffer size is also constrained by the amount of state that a
filter retains with every execution of its work function.

Clearly if a filter has any static data (e.g., state information or
coefficient arrays), then it is prudent to maximize their temporal
access locality. We can define a data-cache miss rate ($\mt{DMR}$) based on
a derivation similar to that for the instruction-cache miss rate:
replace $C_I$ with $C_D$ in Equation~\ref{eq:ims}, and $I(f_i)$ with
$S(f_i)$ when calculating the {\it data reuse distance} ($\mt{DRD}$). 
Here, $C_D$ represents a constant proportional to the data cache size,
and $S(f)$ represents the total size of the static data in the
specified filter.

The presence of static data in a filter implies that we have to limit
the scaling of a filter $f$ such that it does not require more than $C_D -
S(f)$ bytes of buffer space for reading and writing data; otherwise
the data working sets of the filter will overflow the cache and we lose
producer-consumer locality. The buffer requirements are represented by
$Y(f_i)$, and the data-cache model accounts for the dynamic data
requirements by adjusting the data reuse distance as follows:
$\mt{DRD}(f_i) = \sum (S(f_i) + Y(f_i))$ over all distinct filters occurring
in $phase(f_i)$.

Intuitively, the dynamic data accounts for the size of the buffer
required for writing new values, as well as the size of the buffer
necessary for reading values. It also adjusts for any
producer-consumer locality since the output buffer of one filter is
also the input buffer of its neighbor in the stream graph. What this
measure tells us is that we can scale the execution of a filter as
long as its buffer requirements (for reading and writing) do not
exceed the cache, and furthermore, as long as the input-output rates
between producer consumer pairs are not grossly mismatched. Managing the buffer 
requirements is important and motivates a series of cache
optimizations discussed in the next section. In the case of rate
mismatch, the metric tells us that the effective cache size is
reduced, or in other words, the data that is left over after a
producer-consumer firing must be preserved until a future occurrence of
the same pair of filters. This translates to lower data locality and
degrades cache performance.

Mathematically, the dynamic data measure is defined as:
\begin{eqnarray}
  \nonumber
  Y(f_i) &=&\sum_{f_k} min(W(f_k), U(f_k) \times A(f_k)) + \\
  \nonumber
	   &&\sum_{f_k} min(R(f_k), O(f_k) \times A(f_k)) - \\
  \nonumber
         &&\sum_{f_s} min(R(f_s), O(f_s) \times A(f_s))
\end{eqnarray}
with $W(f)$ equal to the output buffer size reserved for writing data, $R(f)$
equal to the input buffer size reserved for reading data, $U(f)$ equal to the
{\tt push} rate (in bytes) of the filter, $O(f)$ equal to the {\tt pop} rate (in
bytes) of the filter, and $A(f)$ equal to the number of occurrences of
filter $f$ in $phase(f_i)$. Also note that the $Y(f_i)$ is defined
over all distinct occurrences of $f_k$ in the phase, and $f_s$
represents the filter that consumes the data produced by $f_k$ (i.e.,
it is $f_k$'s successor in the stream graph, and $f_k$--$f_s$
constitute a producer-consumer pair; $f_s$ is unique unless $f_k$ is a splitter).

The first summand in the equation above quantifies the address space
accessed for writing data. It is  equal to the lesser of the buffer size
reserved for output, and the total number of items produced by
the filter (i.e., the push rate multiplied by the number of times the
work function fired in the phase); this avoids over estimating the
address space when an exceedingly large buffer is reserved but only a
portion of it is used for writing in a phase.

The second term in the equation quantifies the referenced address space 
for reading the input data. This term is also the lesser of two
values: the input buffer size, and the total number of bytes referenced for
reading data.

The third term avoids double counting since the output buffer of one
filter is also the input buffer of its successor in the stream
graph. The term quantifies the amount of data that is consumed by the
producer's successor (which therefore releases a portion of the
address space for use by other filters in the phase).


%% Also complicating matters is the amount of state a actor must retain
%% from one execution of its work function to the next. In the FIR
%% example shown earlier, the state is proportional to the size of
%% the coefficient array (i.e., \texttt{weights}). Thus any scaling of
%% actor firings is further constrainted by the state.


%% We do not account for the cold start
%% effects of an execution trace. This does not affect our model, and can
%% be easily remedied if necessary. To see why this is so, consider the
%% following two execution traces which represent the executions of work
%% functions for two actors \texttt{A} and \texttt{B}:
%% $T_1 = \texttt{A}_1\texttt{A}_2\texttt{B}_3\texttt{B}_4$
%% and 
%% $T_2 = \texttt{A}_1\texttt{B}_2\texttt{A}_3\texttt{B}_4$.
%% In both 
%% traces, the number of cold starts is the same (i.e., two in all).


%% Note that different execution orderings for
%% the same stream graph lead to traces of the same size and hence the
%% comparisons are fair. 

%% Our model predicts that scaling the multipicitly of the actors always
%% leads to lower miss rates. An actor is fired a greater number of times in 
%% the steady state before transitioning to the next actor in the
%% schedule. Specifically, scaling a steady state $S$ by a multiplicity
%% factor $m$ 
%% Hence for example, a program that would generate a trace
%% $T_2$
%% might be as follows:
%% {\small
%% \begin{verbatim}
%% run_steady_state() {
%%   for (i = 0; i < 1; i++) A_work();
%%   for (i = 0; i < 1; i++) B_work();
%% }
%% \end{verbatim}}
%% \noindent but a scaled program that increases temporal locality is as follows:
%% {\small
%% \begin{verbatim}
%% run_steady_state() {
%%   for (i = 0; i < m * 1; i++) A_work();
%%   for (i = 0; i < m * 1; i++) B_work();
%% }
%% \end{verbatim}}
%% \noindent where \texttt{m} (the multiplicity factor) is an integer greater than
%% one.
%% From our model, it is easy to see that scaling reduces the number
%% of misses for a given actor $f$ from $K$ to $\frac{K}{\texttt{m}}- 1$, where $K =
%% \sum \mt{IMS}(f)$ for all occurrences of $f$ in  the execution trace.

%% \subsection{Remarks}

%% Note  that we can  combine the  instruction and  data cache  miss rate
%% equations to yield one measure that estimates the temporal locality of
%% a streaming computation.

%% Also note that while we use  an execution trace to describe our model,
%% it  is possible  to  calculate  the instruction  and  data cache  miss
%% metrics at  compile time.  To do so,  we can leverage  the hierarchical
%% StreamIt   representation  that  dictates   the  ordering   of  actor
%% executions.


%% Intuitively, the dynamic data accounts for the size of the buffer
%% required for writing new values, as well as the size of the buffer
%% necessary for reading values. It also adjusts for any
%% producer-consumer locality since the output buffer of one actor is
%% also the input buffer of its neighbor in the stream graph. What this
%% measure tells us is that we can scale the execution of a actor as
%% long as its buffer requirements (for reading and writing) do not
%% exceed the cache, and furthermore, as long as the input-output rates
%% between producer consumer pairs are not grossly mismatched. Managing the buffer 
%% requirements is important and motivates a series of cache
%% optimizations discussed in the next section. In the case of rate
%% mismatch, the metric tells us that the effective cache size is
%% reduced, or in other words, the data that is left over after a
%% producer-consumer firing must be preserved until a future occurrence of
%% the same pair of actors. This translates to lower data locality and
%% degrades cache performance.

%% Mathematically, the dynamic data measure is defined as:
%% \begin{eqnarray}
%%   \nonumber
%%   Y(f_i) &=&\sum_{f_k} min(W(f_k), U(f_k) \times A(f_k)) + \\
%%   \nonumber
%% 	   &&\sum_{f_k} min(R(f_k), O(f_k) \times A(f_k)) - \\
%%   \nonumber
%%          &&\sum_{f_s} min(R(f_s), O(f_s) \times A(f_s))
%% \end{eqnarray}
%% with $W(f)$ equal to the output buffer size reserved for writing data, $R(f)$
%% equal to the input buffer size reserved for reading data, $U(f)$ equal to the
%% {\tt push} rate (in bytes) of the actor, $O(f)$ equal to the {\tt pop} rate (in
%% bytes) of the actor, and $A(f)$ equal to the number of occurrences of
%% actor $f$ in $phase(f_i)$. Also note that the $Y(f_i)$ is defined
%% over all distinct occurrences of $f_k$ in the phase, and $f_s$
%% represents the actor that consumes the data produced by $f_k$ (i.e.,
%% it is $f_k$'s successor in the stream graph, and $f_k$--$f_s$
%% constitute a producer-consumer pair; $f_s$ is unique unless $f_k$ is a splitter).

%% The first summand in the equation above quantifies the address space
%% accessed for writing data. It is  equal to the lesser of the buffer size
%% reserved for output, and the total number of items produced by
%% the actor (i.e., the push rate multiplied by the number of times the
%% work function fired in the phase); this avoids over estimating the
%% address space when an exceedingly large buffer is reserved but only a
%% portion of it is used for writing in a phase.

%% The second term in the equation quantifies the referenced address space 
%% for reading the input data. This term is also the lesser of two
%% values: the input buffer size, and the total number of bytes referenced for
%% reading data.

%% The third term avoids double counting since the output buffer of one
%% actor is also the input buffer of its successor in the stream
%% graph. The term quantifies the amount of data that is consumed by the
%% producer's successor (which therefore releases a portion of the
%% address space for use by other actors in the phase).
