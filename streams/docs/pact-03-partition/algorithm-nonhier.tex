\section{Partitioning}

This is the beginning of a non-hierarchical rectangular partitioner.

\scriptsize
\begin{verbatim}

can draw horizontal cut when:
1. there are nodes on both sides of the cut
2. NOT right above a joiner or right below a splitter (no diagonal lines crossed)

can draw vertical cut when:
1. there are nodes on both sides of the cut

2. all splitjoin (diagonal) lines crossed lead to 1) a splitter just
   below the top edge of the current rectangle, or 2) a joiner just
   above the bottom edge of the current rectangle

3. the line can be extended to the latest common ancestor (or,
   equivalently, the earliest common descendent) of all nodes in the
   rectangle without crossing any other line.  That is, it represents
   a division of the least common ancestor.


when building graph:
- should space out parallel streams so that joiner on one side doesn't
  clobber a stream on the other.


\end{verbatim}
