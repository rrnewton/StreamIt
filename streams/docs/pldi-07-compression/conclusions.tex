\section{Future Work}

There remain rich areas for future work in computing on compressed
data.  First, as the current transformation has the potential to
increase the size of the file, we plan to explore lightweight
techniques for re-compressing a data stream that is already partially
compressed.  This should be straightforward in the case of Apple
Animation; for example, a run-length encoded unit can be extended
without needing to be rediscovered.

Second, the compressed processing technique can be applied far beyond
the current focus.  In its current form, the technique could be
evaluated on video operations such as thresholding, color depth
reduction, sepia toning, saturation adjustment, and color replacement.
With minor extensions (see Section~\ref{sec:extensions}), the
technique can support video operations such as cropping, padding,
histograms, image flipping, sharpening, and blurring.  The technique
may also have applications in an embedded setting, where it could
offer power savings---for example, in processing the RAW data format
within digital cameras.

Finally, research is underway to apply a similar technique to lossy,
DCT-based compression formats.  The streaming model cf computation
also offers key advantages in this domain, as neighboring actors that
compute linear functions can be algebraically simplified at compile
time~\cite{aalamb}.  For example, a JPEG transcoder typically performs
an iDCT (during decompression), followed by the user's transformation,
followed by a DCT (during compression).  If the user's transformation
is also linear (e.g., brightness adjustment) then all three stages can
be automatically collapsed, thereby eliminating the decompression and
re-compression steps.  Preliminary experiments in this direction
indicate speedups upwards 10x.  By extending the framework to multiple
compression formats, users will be able to write their transformations
once, in a high-level language, and rely on the compiler to map the
computations to each of the compresed domains.

\section{Conclusions}
\label{sec:conclusions}

%% Many of the applications that will drive the next generation of
%% computing systems---digital video editing, computer vision, computer
%% graphics and animation---operate on image and video formats that are
%% universally stored in compressed data formats.  

In order to accelerate operations on compressible data, this paper
presents a general technique for translating programs into the
compressed domain.  Given a natural program that operates on
uncompressed data, our transformation outputs a program that directly
operates on the compressed data format.  We support lossless
compression formats based on LZ77.  In the general case, the
transformed program may need to partially decompress the data to
perform the computation, though this decompression is minimized
throughout the process and significant compression ratios are
preserved without resorting to an explicit re-compression step.

Though our transformations would likely prove intractable in a
language such as C, they are quite straightforward in the synchronous
dataflow model.  Synchronous dataflow is supported by high-level
languages such as StreamIt and is a natural fit for media processing
applications.  Because each actor in a synchronous dataflow graph has
a regular pattern of communication, it can easily be wrapped in our
compressed execution driver.  A similar transformation may be possible
in a functional language, if the compiler can detect that a function
is repeatedly applied across a window of data.

Our techniques demonstrate excellent speedups in our experimental
evaluation.  Across a suite of 12 videos in Apple Animation format,
computing directly on compressed data offers a speedup roughly
proportional to the compression ratio.  For pixel transformations
(brightness, contrast, inverse) speedups range from 3.1x to 235x, with
a median of 19x; for video compositing operations (overlays and
mattes) speedups range from 1.0x to 35x, with a median of 7.4x.  While
previous researchers have used special-purpose compressed processing
techniques to obtain speedups on lossy, DCT-based codecs, we are
unaware of a comparable demonstration for lossless video compression.
As digital films and animated features have embraced lossless formats
for the editing process, the speedups obtained may have significant
practical value.
