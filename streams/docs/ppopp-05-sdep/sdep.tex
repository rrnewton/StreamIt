\section{Stream Dependence Function}

In this section, we describe our model of computation for streams and
define the stream dependence function, $\sdep$.

Our model of computation is Synchronous Dataflow~\cite{LM87-i}.  A
Synchronous Dataflow graph is a directed graph where nodes represent
filters and edges represent FIFO communication channels.  Each filter
has an atomic execution step where it consumes some items from its
input channels and produces some items on its output channels.  The
number of items produced or consumed on each channel is fixed and
known at compile time.

An exection $\phi$ of a dataflow graph is an ordered sequence of
filter firings.  Let $\phi[i]$ denote the $i$th filter appearing in
execution $\phi$, and let $|\phi \wedge F|$ denote the number of times
that filter $F$ appears in $\phi$.

An execution is legal if the dataflow requirements are respected; that
is, for all $i$, the sequential firing of filters $\phi[0]$ through
$\phi[i-1]$ leaves enough items on the communication channels for
$\phi[i]$ to fire atomically.  Let $\Phi$ denote the set of legal
executions.  Note that while $\Phi$ is infinite, each $\phi \in \Phi$
is finite.

The stream dependence function, $\sdep$, describes the dependences
between filter firings in the graph.  Informally, $\sdepf{A}{B}(n)$
represents the minimum number of times that filter $A$ must execute to
make it possible for filter $B$ to execute $n$ times.  This dependence
is meaningful only if there is a directed path in the stream graph
from $A$ to $B$; otherwise, $\sdep$ will have a value of zero.
Because the I/O rates of each filter in the stream graph are known at
compile time, $\sdep$ is also a static relation.  The following is a
formal definition of $\sdep$ using the notations from above:
\begin{definition}(SDEP)
\begin{center}
$\sdepf{A}{B}(n)~~ = ~~\mbox{MIN}~~|\phi \wedge A|$ \\
~~~~~~~~~~~~~\raisebox{5pt}[0pt]{$~_{\phi \in \Phi, |\phi \wedge B| = n}$}
\end{center}
\end{definition}
This equation reads: over all legal executions in which $B$ fires $n$
times, $\sdepf{A}{B}(n)$ is the minimum number of times that $A$ fires.

An example of $\sdep$ appears in
Figure~\ref{fig:sdep}. \todo{Example}.

\section{Calculating SDEP}

$\sdep$ function for a given graph $G$:

{\scriptsize
\begin{verbatim}
int SDEP_{A<-B} (n) {
  if no path from A to B in G then
    return 0
  else
    cycles = floor(steady(B) / n)
    return cycles * steady(A) + sdepCore(A,B)[n mod steady(B)]
  endif
}

int[] sdepCore(A,B) {
  result = { 0 }
  count = 0
  do {
    child <- most downstream filter of G that can fire
    simulate(child)
    if child = A then
      count++;
    else if child = B then
      result = result o count
    endif
  } loop until (all filters f \in G have fired steady(f) times)
  return result
}
\end{verbatim}}

The most ``downstream'' filter is the one reachable via the longest
acyclic path from a source node (a node that consumes zero items).
