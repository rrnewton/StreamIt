\section{Space-Time Scheduling}
\label{sec:scheduling}
\subsection{Steady-State Schedule}

\begin{figure}
\centering
\psfig{figure=space-time-ex-1.ps, width=3in}
\caption{Something should go here.
\protect\label{fig:1d}}
\end{figure}

what we get at this stage

3d bin packing translation

What the pieces look like 

saman's figures...

We arrive at a solution to the 3d bin packing by using simulated
annealing \cite{simanneal}, a type of iterative improvement.  A
detailed explanation of simulated annealing is beyond the scope of
this paper.  Simulated annealing is a form of stochastic
hill-climbing. Unlike most other methods for cost function
minimization, simulated annealing is suitable for problems where there
are many local minima.  Simulated annealing achieves its success by
allowing the system to go uphill with some probability as it searches
for the global minima.  As the simulation proceeds, the probability of
climbing uphill decreases.


what it does!

\subsection{Prologue Schedule}
how contructed, what it guarantees.

\subsection{Peek Initialization Schedule}
what is it for?

how constructed



