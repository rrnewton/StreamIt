\section{Future Research}
\label{section:future}

Realtime processing is an important part of embedded systems and video
codecs. StreamIt specifies timing constraints through message delivery
semantics but these are logical scheduling constraints divorced from
any absolute notion of time. No constraints can be specified 
regarding the maximum latency from stream graph input to 
output. One direction for future work would be to
examine these realtime performance issues.

MPEG-2 is interesting from a language 
and compiler research perspective, but hardware advances have made MPEG-2 
coding relatively easy. 
However, the new H.264 standard presents serious challenges to 
modern architectures and is designed to push performance well into the 
future. 
H.264 is the video coding scheme specified in Part 10 of the MPEG-4 
video specification~\cite{MPEG4}. H.264 reflects many advances in video and image compression 
research. The major specification changes and their consequences for 
a stream-based implementation are detailed.

\paragraph{Video Organization}
\begin{itemize}
\item Macroblocks can occur in any order within a picture. 
This will require additional messages associated with each 
macroblock indicating its position within an image.
\item Macroblocks may be composed of blocks of many different 
sizes (8x8, 16x8, 16x16, 8x16), and block sizes may change 
from macroblock to macroblock. This will make programmable 
splitjoins and dynamic filter reconfiguration especially 
important as language features.
\item Blocks are always composed of 4x4 subblocks, which 
have their own motion vectors.
\item I and P pictures are no longer automatically considered 
reference pictures; instead, pictures are explicitly declared 
as reference pictures and may be used as reference pictures 
for an indeterminate length of time. The picture reordering 
filter will be substantially more complicated. 
\end{itemize}

\paragraph{Spatial Coding}
\begin{itemize}
\item The DCT is replaced by a wavelet transform. The 
wavelet transform incorporates the quantization scale 
factors. The transform is simple to implement and 
simplifies the spatial decoding process from a 
programmer's standpoint, although little can be reused 
from MPEG-2.
\item A block-aliasing mechanism removes edge-artifacts between 
macroblock boundaries. The spatial coding pipeline can be
naturally extended to include an additional filter or 
subgraph that handles this transformation.
\end{itemize}

\paragraph{Motion Prediction}
\begin{itemize}
\item Motion vectors may now refer to data contained in other 
macroblocks in the same frame. Instead of referring to the 
previous 2 decoded reference frames, P and B pictures may 
now refer to any of the past 32 decoded reference frames. 
The optimization to share read-only message sent data
between filters will be necessary.
\item Because blocks have variable size and are composed 
of subblocks with their own motion vectors, 
the motion prediction filters are substantially complicated
in both encoders and decoders. An encoder will only 
benefit from the huge variety of prediction options
if it can expose a high degree of coarse grained
parallelism in the motion estimation subgraph.
\item Motion vectors include a scaling factor (for zooms) 
and an amplitude factor (for image fades). 
Motion vectors are also accurate 
to a quarter pixel resolution instead of half pixel 
resolution. This particular aspect of motion prediction
can be accommodated by small changes to the existing
prediction formation process. 
\end{itemize}

\paragraph{Variable Length Coding}
\begin{itemize}
\item The bit-stream syntax uses many new variable bit 
rate encoding algorithms which have higher compression 
rates than Huffman coding. The bitstream also makes a 
distinction about the importance of data and provides 
options for variable rate error correction. Bitstream 
parsing is therefore dramatically more complicated and 
good inter-language interfaces for this non-streaming 
computation are critical.
\end{itemize}

StreamIt should be well suited for MPEG-4 decoding and 
encoding. Encoding, in particular, should benefit
from the extreme kinds of parallelization that StreamIt
can expose. MPEG-4 provides a huge variety of motion
prediction options for encoding any particular video 
and exploring the space of potential compression options
will require a scalable parallel encoder implementation. 