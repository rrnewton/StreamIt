\section{Related Work}
Printz's ``signal flow graphs'' includes nodes that performed a
sliding window~\cite{printz91thesis}.  Printz presents an
implementation strategy translating the sharing requirement of data
parallelization of peeking filters into communication.  However, his
approach does not alter the steady-state of the graph to reduce
sharing to neighboring fission products, and it does not further alter
the steady-state to reduce sharing to under a given threshold.
Instead, Printz's technique parallelizes the communication required by
sharing via a pipelined sequence of transfers between neighboring
cores on the iWarp machine.  Finally, Printz did not implement his
strategy; instead his evaluation relied on an model of a parallel
architecture.

In the Warp project, the AL language~\cite{tseng89thesis,tseng90} had
a window operation for use with arrays.  The AL compiler targeted the
Warp machine and translated loops into systolic computation.  The AL
compiler did alter the blocking of distributed loop iterations to
reduce the sharing requirement for the sharing of the sliding window
among iterations of the loop.  However, this was in the context of an
imperative language with parallel arrays and considered a different
application class: matrix computation. 

The Brook language requires a programmer to explicitly represent the
communication of sliding windows by specifying that a filter reads
overlapping portions of the input. Liao et al. map Brook to multicore
processors by leveraging the affine partitioning
model~\cite{liao06brook}. However, they do not alter the steady-state
to reduce the number of items shared between the data parallel
products of sliding window filters.

Other languages have included the notion of a sliding window.  The ECOS
graphs language allows actors to specify how many items are read but
not consumed~\cite{huang_ecos_1992}; the Signal language allows access
to the window of values that a variable assumed in the
past~\cite{le_guernic_signal--data_1986}; and the SA-C language
contains a two-dimensional windowing
operation~\cite{draper_compiling_2001}.  However, to the best of our
knowledge, translation systems for these languages do not utilize
sliding windows to improve parallelism and reduce inter-core
communication.
