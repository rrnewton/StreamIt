\section{Generalized Fission of Sliding Window Filters}
\label{sec:fission}

This section describes our technique for the fission of sliding window
filters. Our technique for filter fission employs an optimization
termed {\it synchronization removal}.  This transformation is applied
to nodes calculating the identity function in the general graph such
that they can be removed, but their communication patterns remain
embedded in the general graph. A description of synchronization
removal is beyond the scope of this paper, for a complete exposition,
please see~\cite{mgordon-phd}.

Before we describe the fission transformation, we need to list the
preconditions that must hold before fission of $f$ by $P$ in the general
graph can be performed:

\begin{enumerate}
\item The items remaining of the input buffer of the original
filter after initialization must be less than the number of items
dequeued by each fission product:
\begin{equation}
C(f) < (M(S,f) / P) \cdot o(W, f)
\label{eq:fiss-precond1}
\end{equation}

\item In this version of the algorithm we create
  fission products with equal work, an equal division of the original
  steady-state multiplicity of $f$.  This restriction can be relaxed
  via simple modifications to the algorithm.  The following must hold:
\begin{equation}
M(S,f) \mod P = 0 
\label{eq:mod-fiss}
\end{equation}
\end{enumerate}

\noindent As Section~\ref{sec:data-par} describes, both of these
preconditions can be enforced on any filter by increasing the
steady-state of the graph. The details of fission on a filter of the
general stream graph are given in Figure~\ref{fig:general-fission}:
(a) gives the definition of the original filter $f$, (b) lists the 11
steps of the algorithm, and (c) gives details for steps 1-9 of the
transformation.

\begin{figure*}
\centering
\includegraphics[width=\textwidth]{figures/general-fission.pdf}
\caption[Fission of a node in the general stream graph.]{Fission of a
  node $f$ by $P$ in the general stream
  graph.\label{fig:general-fission}}
\end{figure*}

\begin{figure*}
\centering
\includegraphics[width=\textwidth]{figures/split-pattern.pdf}
\caption[The output distribution required for general
fission.]{
The steady-state output distribution installed for identity node
$ID_I$ by general fission.  Red edges denote items that are shared
across fission products via duplication. If $C(f) > dup$, $C(f) - dup$ items at the end of the
steady-state input are distributed to $f_1$. \label{fig:split-pattern}}
\end{figure*}

The general fission transformation creates two identity nodes ($ID_I$
and $ID_O$) that are encoded to implement the data distribution for
the fission products.  Figure~\ref{fig:split-pattern} illustrates
the pattern of communication between $ID_I$ and the products of $f$.
This pattern is common to the transformation for all filters we seek
to fiss that meet the preconditions of the transformation.

The result of the fission transformation has the following properties:
\begin{itemize}
\item No item is read by more than 2 fission products.
\item A fission product does not need to remember items across
  steady-state executions of itself. The peeking of the original
  filter $f$ is now encoded in the sharing across fission products
  achieved via the duplication pattern.
\item Only the first fission product $f_1$ is required to receive the $C(f)$
  initialization  items because $C(F) < (M(S,f) / p) \cdot o(W, f)$,
  and it will consume the $C(f)$ items on its first invocation.
\item The presence of the $C(f)$ items in the input buffer after
  initialization must be accounted for by shifting the read pattern
  for the fission products.  The first fission product $f_1$ is offset by
  $C(f)$ items in that it reads its first $C(f)$ items from the
  previous execution stage.  In the steady-state, $f_1$ executing at
  steady-state iteration $i$ shares items with $f_P$ executing at
  steady-state iteration $i-1$.
\item The computation and communication performed by $f$ during the
  initialization stage is transferred completely to the first fission
  product, $f_1$.  Since, by construction, only $f_1$ requires the
  items remaining after the initialization stage.  The other fission
  products are idle during this stage.
\end{itemize}

The final steps of the general fission transformation applies
\textsc{SynchRemove} to remove the identity filters, and stitch the
communication directly between the fission products and $f$'s
producer(s) and consumer(s).
 
%  The process includes the
% following steps (Figure~\ref{fig:general-fission} illustrates steps
% 1-9): 
% \begin{enumerate}
% \item Create $P$ copies of $f$ and set their rates
% and work functions according to Figure~\ref{fig:general-fission}.
% \item Create two identity nodes, $ID_I$ and $ID_O$, that will encode
%   the distribution for the fission.
% \item Move the initialization stage computation of $f$ to $f_1$
%   according to Figure~\ref{fig:general-fission}. 
% \item Move input distribution of $f$ to $ID_I$
% replacing occurrences of $f$ with $ID_I$ in edges.
% \item Move output distribution of $f$ to $ID_O$, replacing
% occurrences of $f$ with $ID_O$ in edges.
% \item Create the fission duplication pattern in the
% output distribution of $ID_I$.
% \item Create a round robin joining pattern for the output identity
%   filter $ID_O$ to receive from each fission product.
% \item For each node $p$ that is a producer of $f$, replace the
%  occurrences of $f$ with $O_I$ in the edges of the dupsets of $p$'s
%  output distribution.
% \item For each node $c$ that is a consumer of $f$, replace the
%  occurrences of $f$ with $O_O$ in incoming edges $c$'s input
%  distribution.
% \item \textsc{SynchRemove}($ID_I$)
% \item \textsc{SynchRemove}($ID_O$)
% \end{enumerate}


% To understand the transformation, we first need to understand the item
% distribution and sharing that is required by fission on a filter $f$
% that adheres to the preconditions above.
% Figure~\ref{fig:fission-sharing2} gives another example of the input
% items required by fission products.  In this example, both:

% \[ C(f) = \mt{dup}_f < (M(S,f) / p) \cdot o(W, f) \]
% \[ M(S, f) \mod P = 0\]

% \noindent so $f$ adheres to the preconditions stipulated above for
% general fission.  In the example, the items read for $f$ plus its
% fission products for $P=4$ are shown for the initialization plus two
% steady-states.  After the initialization stage, $C(f) = 2$ items are
% enqueued to the input buffer by the producer(s) to $f$, and $M(S,f)
% \cdot o(W, f) = 16$ items are enqueued by the producer(s) for each
% steady-state.  Examining the sharing requirement for fission products of
% the second steady-state, we can see a pattern emerge with the following
% features: