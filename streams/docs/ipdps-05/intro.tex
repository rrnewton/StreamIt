\Section{Introduction}

Image compression of still and motion pictures plays an important role
in Internet and multimedia applications, digital appliances such as
HDTV, and handheld devices. The compression process allows images and
video to be encoded with a much smaller amount of data, and often with
a negligible loss in quality. The reduction in data decreases storage
requirements (important for embedded devices) and provides higher
effective transmission rates (important for Internet enabled devices).

Current programming practices often require developers to implement
compression algorithms in low level languages, and arduously
tune their code for performance. This methodology is not cost
effective, as architecture-specific code is not portable, leading to
multiple implementations of the same codec. The process is made more
challenging by the continuous evolutions of the standards, as new
innovations are realized in a rapidly growing digital multimedia
market.

A typical compression algorithm involves three types of operations:
data representation, lossy compression, and lossless
compression. These operations are semi-autonomous, exhibit data and
pipeline parallelism, and easily fit into a sequence of distinct
processing stages. As such, image and video compression is a good
match for the streaming model of computation where data is transformed
by a series of filters, usually organized in well structured
topologies.  Stream programming models afford certain advantages in
terms of programmability, robustness, and achieving high performance.

This paper describes an implementation of the widely used MPEG-2
compression standard in StreamIt~\cite{streamitcc}, a high-level
architecture-independent language for streaming computations. Our goal
is to deliver a unified development environment that captures all
aspects of stream application development without sacrificing either
performance or programmability. This paper details the salient
processing steps of the MPEG-2 decoder in StreamIt, and compares some
of the important implementation details to a reference C
implementation of the decoder.

The StreamIt programming model allows the programmer to build an
application by connecting components together into a stream graph,
where the nodes represent filters that transform the data communicated
along the edges. In StreamIt, the programmer is relieved of the burden
of explicit buffer management and complex index expressions for
multi-dimensional data.  StreamIt also exposes the inherent
parallelism and communication topology of the application, thereby
empowering the compiler to perform many stream-aware
optimizations~\cite{agrawal05cases,gordon02asplos,lamb03pldi,sermulins05lctes}
that elude other languages. The end result is a clean, malleable,
and efficient portable code.
