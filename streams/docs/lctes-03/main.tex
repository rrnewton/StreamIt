% Michal Karczmarek


%\documentclass[times,11pt,twoside,oneandahalfspace]{mitthesis}
%\documentclass[times,11pt,twoside,draft,oneandahalfspace]{mitthesis}
%\documentclass[10pt,oneandahalfspace,twoside,openany]{mitthesis} %\pagestyle{drafthead}
%\documentclass[10pt,oneandahalfspace,twoside,openany,draft]{mitthesis} \pagestyle{drafthead}
%\documentclass[10pt,singlespace,twoside,openany,largemargin]{mitthesis} \pagestyle{drafthead}
%\pagestyle{drafthead}

\documentclass[runningheads,roman]{article}

%\usepackage{lgrind}
\usepackage{verbatim}
\usepackage{epsfig, graphicx}
\usepackage{enumerate}
\usepackage{fullpage}
%\usepackage{doublespace}
%\setstretch{1.05}
%\usepackage{algorithm}
%\usepackage{algorithmic}

%\pagestyle{plain}
\newtheorem{theorem}{Theorem}
\newtheorem{proof}{Proof}
\newtheorem{lemma}{Lemma}
\newtheorem{corollary}{Corollary}
\newtheorem{definition}{Definition}


\pagestyle{headings}

%\mainmatter

\title{Phased Scheduling of Stream Programs}
\author{Michal Karczmarek, William Thies and Saman
Amarasinghe \\
Laboratory for Computer Science \\
        Massachusetts Institute of Technology \\
        Cambridge, MA  02139 \\
\texttt{\{karczma, thies, saman\}@lcs.mit.edu} \vspace{-24pt}}
\date{}

\begin{document}
\maketitle

\newcommand{\Signed}{\emph{signed}}
\newcommand{\Unsigned}{\emph{unsigned}}
\newcommand{\Byte}{\emph{byte}}
\newcommand{\Char}{\emph{char}}
\newcommand{\Short}{\emph{short}}
\newcommand{\Int}{\emph{int}}
\newcommand{\Long}{\emph{long}}
\newcommand{\Float}{\emph{float}}
\newcommand{\Double}{\emph{double}}
\newcommand{\LongDouble}{\emph{long double}}
\newcommand{\Void}{\emph{void}}
\newcommand{\Struct}{\emph{struct}}
\newcommand{\Union}{\emph{union}}
\newcommand{\Reference}{\emph{reference}}
\newcommand{\New}{\emph{new}}
\newcommand{\Class}{\emph{class}}
\newcommand{\Malloc}{\emph{malloc}}
\newcommand{\Vastart}{\emph{va\_start}}
\newcommand{\Vaarg}{\emph{va\_arg}}
\newcommand{\Vaend}{\emph{va\_end}}
\newcommand{\Goto}{\emph{goto}}
\newcommand{\For}{\emph{for}}
\newcommand{\Printf}{\emph{printf}}
\newcommand{\NULL}{\textbf{NULL}}
\newcommand{\ANSIC}{\textbf{ANSI C}}
\newcommand{\SUIF}{\textbf{SUIF}}
\newcommand{\IEEE}{\textbf{IEEE}}
\newcommand{\Lod}{\textbf{lod}}
\newcommand{\Str}{\textbf{str}}
\newcommand{\Cal}{\textbf{cal}}

\newcommand{\Null}{\emph{null}}
\newcommand{\Tableswitch}{\emph{tableswitch}}
\newcommand{\Lookupswitch}{\emph{lookupswitch}}

\newcommand{\StreamIt}{\tt{StreamIt}}
\newcommand{\filter}{\tt{Filter}}
\newcommand{\filters}{{\filter}s}
\newcommand{\pipeline}{\tt{Pipeline}}
\newcommand{\pipelines}{{\pipeline}s}
\newcommand{\splitjoin}{\tt{SplitJoin}}
\newcommand{\splitjoins}{{\splitjoin}s}
\newcommand{\feedbackloop}{\tt{FeedbackLoop}}
\newcommand{\feedbackloops}{{\feedbackloop}s}
\newcommand{\splitter}{\tt{splitter}}
\newcommand{\splitters}{{\splitter}s}
\newcommand{\joiner}{\tt{joiner}}
\newcommand{\joiners}{{\joiner}s}
\newcommand{\duplicate}{\tt{Duplicate}}
\newcommand{\roundrobin}{\tt{RoundRobin}}
\newcommand{\work}{\tt{work}}
\newcommand{\infwavefront}{\tt{infromation wavefront}}
\newcommand{\Input}{\tt{input}}
\newcommand{\Output}{\tt{output}}
\newcommand{\Channel}{\tt{channel}}
\newcommand{\Channels}{\tt{channel}s}
\newcommand{\stream}{\tt{stream}}
\newcommand{\streams}{\tt{stream}s}
\newcommand{\operator}{\tt{operator}}
\newcommand{\SDF}{\tt{SDF}}
\newcommand{\C}{\tt{C}}

\newcommand{\subsubsubsection}[1]{\vspace{10pt}\noindent{\bf{#1}:}}

\newcommand{\myitem}{\vspace{-6pt} \item}


\vspace{0.1in}

\begin{abstract}
Applications structured around some notion of a ``stream'' are
becoming increasingly important and widespread. \cite{Rix98}
provides evidence that streaming media applications are already
consuming most of the cycles on consumer machines, and their use
is continuing to grow. The streaming computation model is
pervasive and ranges from small, embedded systems (ex. cell
phones) to large, computationally powerful machines (ex. cell base
stations). In this paper, we describe a novel technique for
scheduling execution of synchronous data flow streaming
applications exhibiting hierarchical properties. A vital aspect in
compiling such programs is finding an efficient schedule. The
technique presented here focuses on producing schedules that are
optimized for the amount of space required for buffering and
storing the schedule. A wide variety of real-life applications and
a few synthetic applications are surveyed. Applications benefit
from an average 14.5\% decrease in buffer requirements, with a
peak of 93\% savings in buffer size. No application requires more
space than the most popular technique used today (Single
Appearance Scheduling).
\end{abstract}

%% -*-latex-*-

\title{Linear State-Space Analysis and Optimization of StreamIt Programs}

\author{Sitij Agrawal}
\department{Department of Electrical Engineering and Computer Science}
% If the thesis is for two degrees simultaneously, list them both
% separated by \and like this:
% \degree{Doctor of Philosophy \and Master of Science}
\degree{Master of Engineering in Computer Science and Engineering}
\degreemonth{August}
\degreeyear{2004}
\thesisdate{August 26, 2004}

%% By default, the thesis will be copyrighted to MIT.  If you need to copyright
%% the thesis to yourself, just specify the `vi' documentclass option.  If for
%% some reason you want to exactly specify the copyright notice text, you can
%% use the \copyrightnoticetext command.
%\copyrightnoticetext{\copyright IBM, 1990.  Do not open till Xmas.}

% If there is more than one supervisor, use the \supervisor command
% once for each.
\supervisor{Saman Amarasinghe}{Associate Professor}

% This is the department committee chairman, not the thesis committee
% chairman.  You should replace this with your Department's Committee
% Chairman.
\chairman{Arthur C. Smith}{Chairman, Department Committee on Graduate Students}

% Make the titlepage based on the above information.  If you need
% something special and can't use the standard form, you can specify
% the exact text of the titlepage yourself.  Put it in a titlepage
% environment and leave blank lines where you want vertical space.
% The spaces will be adjusted to fill the entire page.  The dotted
% lines for the signatures are made with the \signature command.
\maketitle

% The abstractpage environment sets up everything on the page except
% the text itself.  The title and other header material are put at the
% top of the page, and the supervisors are listed at the bottom.  A
% new page is begun both before and after.  Of course, an abstract may
% be more than one page itself.  If you need more control over the
% format of the page, you can use the abstract environment, which puts
% the word "Abstract" at the beginning and single spaces its text.

%% You can either \input (*not* \include) your abstract file, or you can put
%% the text of the abstract directly between the \begin{abstractpage} and
%% \end{abstractpage} commands.

% First copy: start a new page, and save the page number.
\cleardoublepage
% Uncomment the next line if you do NOT want a page number on your
% abstract and acknowledgments pages.
% \pagestyle{empty}
\setcounter{savepage}{\thepage}
\begin{abstractpage}
Applications that are structured around some notion of a "stream"
are becoming increasingly important and widespread.  There is
evidence that streaming media applications are already consuming
most of the cycles on consumer machines \cite{Rix98}, and their
use is continuing to grow.  {\StreamIt} is a language and compiler
specifically designed for modern stream programming.  Despite the
prevalence of these applications, there is surprisingly little
language and compiler for practical, large-scale stream
programming.  {\StreamIt} is a language and compiler specifically
designed for modern stream programming.  The {\StreamIt} langauge
holds two goals: first, to provide high-level stream abstractions
that improve programmer productivity and program robustness within
the streaming domain; second, to serve as a common machine
language for grid-based processors.  At the same time, {\StreamIt}
compiler aims to perform stream-specific optimizations to achieve
the performance of an expert programmer.  This thesis develops
several techniques for scheduling execution of {\filters} in
{\StreamIt}.  The work focuses on correctness as well as
minimizing buffering requirements and stored schedule size.

\end{abstractpage}

% Additional copy: start a new page, and reset the page number.  This way,
% the second copy of the abstract is not counted as separate pages.
% Uncomment the next 6 lines if you need two copies of the abstract
% page.
% \setcounter{page}{\thesavepage}
% \begin{abstractpage}
% Applications that are structured around some notion of a "stream"
are becoming increasingly important and widespread.  There is
evidence that streaming media applications are already consuming
most of the cycles on consumer machines \cite{Rix98}, and their
use is continuing to grow.  {\StreamIt} is a language and compiler
specifically designed for modern stream programming.  Despite the
prevalence of these applications, there is surprisingly little
language and compiler for practical, large-scale stream
programming.  {\StreamIt} is a language and compiler specifically
designed for modern stream programming.  The {\StreamIt} langauge
holds two goals: first, to provide high-level stream abstractions
that improve programmer productivity and program robustness within
the streaming domain; second, to serve as a common machine
language for grid-based processors.  At the same time, {\StreamIt}
compiler aims to perform stream-specific optimizations to achieve
the performance of an expert programmer.  This thesis develops
several techniques for scheduling execution of {\filters} in
{\StreamIt}.  The work focuses on correctness as well as
minimizing buffering requirements and stored schedule size.

% \end{abstractpage}

\cleardoublepage

\section*{Acknowledgments}

    First, I would like to thank my family for all their support throughout my college years.
I would like to thank the members of the StreamIt group - in particular Jasper Lin and David Maze - for patiently
answering all my questions and for helping me understand the StreamIt language and compiler. 
I would like to thank Andrew Lamb, whose work on linear analysis of StreamIt programs provided the foundation for 
my own work on state-space analysis. His thesis and well-constructed code were invaluable to me. Rodric Rabbah, another
member of our group, gave me excellent comments about the writing in this thesis.
I would like to thank my advisor, Saman Amarasinghe, for giving me the opportunity to work on the StreamIt project and for funding
my research. Finally, I would like to thank Bill Thies for guiding me through every step of my project. That
I was able to complete this thesis is a testament to his mentoring ability. I could not have done it without him.


%%%%%%%%%%%%%%%%%%%%%%%%%%%%%%%%%%%%%%%%%%%%%%%%%%%%%%%%%%%%%%%%%%%%%%
% -*-latex-*-

%\input{contents}
\section{Introduction}

The domain of stream programs is important because it stands at the
intersection of trends in applications and architectures.  Stream
programming naturally represents applications such as audio, video,
digital signal processing, and data analysis; applications that are
increasing prevalent as computing moves towards data-centric
applications and to the mobile and embedded space.  Also, by virtue of
their structure -- a graph of independent computational nodes (termed
{\it filters}) with explicit and regular communication -- stream
programs are a natural fit for exploiting coarse-grained parallelism
suitable for multicore architectures.  The interest in streaming
applications has spawned a number of streaming languages that target
the streaming domain, including StreamIt~\cite{streamitcc},
Brook~\cite{brook04}, Cg~\cite{cg03},
SPUR~\cite{spur05samos}, Spidle~\cite{spidle03}, Lime~\cite{lime10},
and SPL~\cite{spl09}.

In a stream program, filters define an atomic execution step that
repeats for many iterations; each execution step discards a number of
data items from filter's input edge.  Often, a filter does not discard
all the data items that it read for the current execution step,
requiring these inspected (but not discarded) items for a future
iteration (or iterations) of the filter.  This type of filter is
described as performing a sliding window computation on its
input. Sliding window computations are prevalent in stream programs.
Examples of sliding window computations include FIR filters; moving
averages and differences; error correcting codes; motion estimation;
and network packet inspection.  A recent study of a large streaming
benchmark suite written in the StreamIt programming language finds
that 17 of the 30 real-world benchmarks include at least one filter
that performs a sliding window computation~\cite{streamit-suite}.


Figure~\ref{fig:fir-nopeeking} shows how to perform a sliding
window FIR filter via state carried between iterations of a filter.
This implementation is difficult for the compiler to analyze and
reason about.  Some programming languages (e.g., Brook, Lime,
StreamIt, and IBM SPL) go so far as to include idioms that directly
represent sliding window computation, allowing the programmer to
specify, for each filter, the size of the window and the number of
items discarded after an execution of the filter.
Figure~\ref{fig:fir-peeking} shows how language extensions of the
StreamIt programming language elegantly expose sliding windows for
compiler analysis and optimization.

A goal of stream programming is to directly expose to the software
layer the necessary information to enable automatic management of
coarse-grained parallelism.  Stream programs expose multiple forms of
parallelism: pipeline parallelism that exists between producers and
consumers; task parallelism that exists between pairs of filters on
parallel branches of the stream graph; and data parallelism that
exists when a filter is stateless and can thus be replicated.  Data
parallelism is the most attractive, as it provides load-balanced and
limitless parallelism (as long as input data is available).  A filter
that is stateful, and cannot be data-parallelized, becomes a limit to
parallelization scalability, as the work of that filter cannot be
divided; the most load-intensive stateful filter becomes a
bottleneck.

This paper presents a compiler framework for data-parallelizing
filters that perform sliding window computations when the properties
of the sliding window can be calculated statically.  If sliding window
filters required state, this state would represent a new
parallelization bottleneck.  Sliding windows are the bottleneck in 11
of the 17 real-world benchmarks in the StreamIt Benchmark Suite that
contain sliding windows~\cite{streamit-suite}.  For example, examining
the Channelvocoder benchmark, this state would limit scalability to 18
cores, whereas our techniques scale to at least 64 cores.

Data-parallelizing a filter is performed via a transformation termed
{\it fission} (verb form {\it fiss})~\cite{streamit-asplos}.  Fission
is the process of data-parallelizing a stateless filter by duplicating
the filter a certain number of ways, assigning duplicates to distinct
cores, and correctly distributed input data to and collecting output
data from the duplicates.  The duplicated filters are referred to as
{\it products}.  When a sliding window is present, fission is
accomplished by duplicating certain input items since they are
required by multiple products.  This duplication translates into
inter-core communication, a limiting factor for scalability when
targeting multicore architectures.

Previous approaches duplicate each input data item to all products,
with products ignoring (decimating) items that are not
needed~\cite{streamit-asplos}.  We will show that this strategy limits
scalability for multicores by requiring too much inter-core
communication.  In contrast, our strategy precisely routes each input
item to the minimal set of product filters that requires the item.
Unlike previous work, our techniques are defined on
multiple input and multiple output filters, removing the need to
introduce synchronization filters that serialize data before and
after the product filters.  

Our techniques operate on {\it static-rate} stream graphs, meaning
that the number of items produced, the number of items consumed, and
the number of items inspected by each filter can be determined
statically.  Because of this property, a steady-state schedule of
filter firings can be calculated that does not grow buffers and can be
executed indefinitely~\cite{lee87}.  Our techniques are conscious of
the spatial locality between producers and consumers.  Our framework
includes techniques that can determine when spatial locality can be
increased by altering the steady-state schedule.  When applicable, our
techniques can reduce the overall sharing (and thus inter-core
communication) requirement to below a threshold percent of the total
input communication for each sliding window filter that is
data-parallelized. 

\begin{figure}[t]
\centering
\subfigure[]{\includegraphics[width=3.3in]{figures/fir-nopeeking.pdf}\label{fig:fir-nopeeking}}
\subfigure[]{\includegraphics[width=3.3in]{figures/fir-peeking.pdf}\label{fig:fir-peeking}}
\caption[Two implementations of an FIR filter.]{\label{fig:fir-code}
  Two StreamIt implementations of an FIR filter:
   (a) the non-peeking version implemented via a
  stateful circular buffer; and (b) the peeking version. Only steady-state implementation is
  given.}
\end{figure}

The framework presented is defined on a model of computation that is
agnostic of source language.  To evaluate our techniques we have
implemented them in the context of the StreamIt compiler
infrastructure~\cite{gordon-asplos06}.  Our transformations are guided
by the parallelization management techniques presented
in~\cite{gordon-asplos06}.  We employ 3 real-world benchmarks from the
StreamIt Benchmark Suite~\cite{streamit-suite} that include sliding
window computation.  We demonstrate the effectiveness of our
techniques by comparing them to previously published techniques on 2
multicore architectures: a 16-core SMP shared-memory multicore and the
64-core distributed-memory Tilera TILE64.  We show that
our techniques are required to achieve scalable parallelization on
both architectures, achieving a 6.7x mean speedup on the 16-core SMP
and a 1.8x mean speedup on the 64-core distributed memory multicore
over a previously published technique.

\subsection{Contributions}
This paper makes the following contributions:
\begin{itemize}
  % \myitem{Motivation for Exposing Sliding Windows in Stream
  %   Languages}: Without exposing sliding windows in the language, it
  % requires heroic effort by the compiler to analyze the access patterns
  % of such a filter. Without success, the compiler will not be able to
  % data-parallelize these filters.  This will prevent robust 
  % parallelization scalability for streaming applications.

  \myitem{Generalized Fission of Sliding Window Filters}: We present a
  transformation that fisses sliding window filters with multiple
  input and multiple outputs.  The technique also supports filters
  that with multiple schedules of execution.  General fission defines
  a precise pattern of communication of input data to the products
  that can be reasoned upon by our other techniques.

  \myitem{Sharing Reduction}: We are the first to present a technique
  that decides when it is possible to decrease the amount of sharing
  between fission products by altering the steady-state of a stream
  graph, thus decreasing inter-core communication.  The technique
  reasons about all the sliding window filters of the stream graph,
  and when possible, reduces the sharing requirement to below a given
  threshold percent of the total input of the filters. 

  \myitem{Data Parallelization of Stream Graph}: We present a
  framework for data-parallelizing all of the filters of a stream
  graph employing the fission transformation on individual filters and
  applying sharing reduction when possible.  This framework optimizes
  for spatial locality and enables the compiler to automatically and
  effectively manage parallelization across varying multicore
  architectures.

  \myitem{Enable Robust Parallelization Scaling for Multicores}: For
  streaming applications with sliding window computation, previously
  published data-parallelization transformations do not scale for our
  target multicores. Our techniques enable robust parallelization
  scalability by reducing inter-core communication.  We achieve a 17x
  mean parallelization speedup for a 16-core SMP and a 62.3x mean
  parallelization speedup for the 64-core TILE64 across our benchmarks.

\end{itemize}

% \begin{figure}[t]
% \centering
% \begin{subfloat}
% \begin{minipage}[b]{0.45\textwidth}
% \eightpoint
% \begin{verbatim}
% float->float filter FIR(int N) {
%   int srcBuffer[N];
%   int srcEnd = 0; 
%   ...
%   work push 1 pop 1 {
%     srcBuffer[srcEnd] = pop();
%     float sum = 0;
%     for (int i=0; i<N; i++) {
%       sum += weights[i] * srcBuffer[(srcEnd + i + 1) % N];
%     }
%     push(sum);
%     srcEnd = (srcEnd + 1) % N;
%   }
% }
% \end{verbatim}
% \vspace{-8pt}
% \end{minipage}%
% \caption{ \label{fig:fir-nopeeking}}
% \end{subfloat}%
% \qquad
% \begin{subfloat}
% \begin{minipage}[b]{0.45\textwidth}
% \eightpoint
% \begin{verbatim}
% float->float filter FIR(int N) {
%   ...
%   work push 1 pop 1 peek N {
%     float sum = 0;
%     for (int i=0; i<N; i++) {
%       sum += weights[i] * peek(i);
%     }
%     push(sum);
%     pop();
%   }
% }
% \end{verbatim}
% \vspace{-18pt}
% \end{minipage}
% \caption{ \label{fig:fir-streamit}}
% \end{subfloat}
% \caption[Two implementations of an FIR filter.]{\label{fig:fir-code}
%   Two StreamIt implementations of an FIR filter:
%    \subref{fig:fir-nopeeking} the non-peeking version implemented via a
%   stateful circular buffer; and \subref{fig:fir-streamit} the peeking version. Only steady-state implementation is
%   given.}
% \end{figure}

%\section{{\StreamIt} Language}
\label{chpt:streamit}

This chapter introduces relevant constructs of the {\StreamIt}
language.  Syntax is not explored here, as it is not relevant to
{\StreamIt} scheduling.

Section \ref{sec:streamit:struct} introduces the structured
streaming concept, while Section \ref{sec:streamit:messages}
introduces the low bandwidth messaging semantics of {\StreamIt}.

\subsection{Structure}
\label{sec:streamit:struct}

Perhaps the most distinguishing feature of {\StreamIt} language is
that it introduces structure to the concept of stream computation.
{\StreamIt} concept of structure is conceptually similar to
structured constructs in functional languages such as \C.

In {\StreamIt} programs are composed out of streaming components
called streams.  Each stream is a single-input, single-output
component, possibly made up of a hierarchical composition of other
streams. Streams can only be arranged in a limited number of ways,
using {\pipelines}, {\splitjoins}, and {\feedbackloops}.  Data
passed between {\filters} is read from and written to {{\Channels}}.
Figure \ref{fig:structure} contains examples of various
{\StreamIt} streams.  The restrictions on arrangement of streams
enforces the structure imposed by {\StreamIt}.

\begin{figure}\begin{center}
\begin{minipage}{2in}
\centering
\psfig{figure=filter.eps,width=0.7in} \\
{\protect\small (a) A {\filter}}
\end{minipage}
~
\begin{minipage}{2in}
\centering
\psfig{figure=pipeline.eps,width=0.7in} \\
{\protect\small (b) A {\pipeline} with $n$ children.}
\end{minipage}
~
\begin{minipage}{2in}
\centering
\psfig{figure=splitjoin.eps,width=2in} \\
{\protect\small (c) A {\splitjoin} with $n$ children}
\end{minipage}
~
\begin{minipage}{2in}
\centering
\psfig{figure=feedback.eps,width=1.5in} \\
{\protect\small (d) A {\feedbackloop}}
\end{minipage}
\end{center}
\caption{All {\StreamIt} streams}
\label{fig:structure}
\end{figure}

\subsubsection{\filters}

The basic unit of computation in {\StreamIt} is the {\filter}. The
central aspect of a filter is the {\work} function, which
describes the filter's atomic execution step. Within the {\work}
function, the filter can communicate with its neighbors using the
{\Input} and {\Output} channels, which are typed FIFO queues
declared during initialization of a {\filter}.  Figure
\ref{fig:structure}(a) depicts a {\filter}.

{\filters} also have the restriction of requiring a static amount
of data to be consumed and produced for each execution of a
{\work} function.  The amount of data produced by a {\filter} $F$
upon execution of its {\work} function is called a push amount,
denoted $push$. The amount of data consumed from {\Input}
{{\Channel}} by a {\filter} $F$ upon execution of its {\work}
function is called a pop amount, denoted $pop$.  {\filters} may
require that additional data be available in the {\Input}
{{\Channel}} for the {\filter} to examine.  This data can be read by
the {\filter}'s {\work} function, but it will not be consumed, and
will remain in the {{\Channel}} for the next execution of the
{\work} function.  The amount of data necessary on the {\Input}
{{\Channel}} to execute {\filter}'s {\work} function is called peek
amount, denoted $peek$.  Note, that for all {\filters} $peek
>= pop$.  Extra peek amount is the amount of data required on by
the {\filter} that will be read but will not be consumed, namely
$peek - pop$.  The \emph{peek}, \emph{pop} and \emph{push} values
in Figure \ref{fig:structure}(a) correspond to the $peek$, $pop$
and $push$ amounts of the {\filter}'s {\work} function.

A {\filter} can be a source, if it does not consume any data, but
it produced data.  Namely, a {\filter} is a source if it has $peek
= pop = 0$. Likewise, a {\filter} can be a sink, if it consumes
data, but does not produce any, or $push = 0$.

\subsubsection{\pipelines}

{\pipelines} are used to connect {\StreamIt} structures in a chain
fashion: each child stream's output is the next child stream's
input. {\pipelines} have no {\work} function, as they do not perform
any computation themselves. {\pipelines} are simply containers of
other {\StreamIt} structures.  Figure \ref{fig:structure}(b) depicts
a {\pipeline}.

\subsubsection{\splitjoins}

{\splitjoins} are used to specify independent parallel structures
that diverge from a common {\splitter} and merge into a common
{\joiner}.  There are two types of {\splitters}:
\begin{enumerate}[(a)]
\item %(a)
{\duplicate}, which replicates each data item and sends a copy to
each parallel stream, and

\item %(b)
{\roundrobin} $(w_0,\dots, w_{n-1})$, which sends the first $w_0$
items to the first stream, the next $w_1$ items to the second
stream, and so on.  If all $w_i$ are equal to $0$, all child
streams of the {\splitjoin} must be sources.
\end{enumerate}

{\roundrobin} is also the only type of a {\joiner} supported in
{\StreamIt}; its function is analogous to a {\roundrobin} {\splitter}.

Figure \ref{fig:structure}(c) depicts a
{\splitjoin}.

\subsubsection{\feedbackloops}
\label{sec:explain-fl}

{\feedbackloops} are used to create cycles in the stream graph. A
{\feedbackloop} contains a {\joiner}, a body stream, a {\splitter}, and
a loop stream.  Figure \ref{fig:structure}(d) depicts a
{\feedbackloop}.

A {\feedbackloop} has an additional feature required to allow a
{\feedbackloop} to begin computation: since there is no data on
the feedback path at first, the stream instead inputs data from a
special function defined by the {\feedbackloop}.  The amount of
data pushed onto the feedback path is called delay amount, denoted
$delay_{fl}$, for a {\feedbackloop} $fl$.

\subsection{Messages}
\label{sec:streamit:messages}

In addition to passing data between {\filters} using structured
streams, {\StreamIt} provides a method for low-bandwidth data
passing, similar to a combination of sending messages and function
calls. Messages are sent from within the body of a {\filter}'s
{\work} function, perhaps to change a parameter in another
{\filter}. The sender can continue to execute while the message is
en route. When the message arrives at its destination, a special
message receiver method is called within the destination
{\filter}. Since message delivery is asynchronous, there can be no
return value; only void methods can be message targets. This
allows the send to continue execution while the message is en
route - the sender does not have to wait for the receiver to
receive the message and send a return value back. If the receiver
wants to send a return value to the sender, it can send a message
back to the sender.

Although message delivery in {\StreamIt} is asynchronous in
principle, {\StreamIt} does include semantics to restrict the
latency of delivery of a message. Since {\StreamIt} does not
provide any shared resources to {\filters} (including global
memory, global clock, etc), the timing mechanism uses a concept of
flow of information.

One motivating example for messaging in {\StreamIt} can be found in
cell phone processing application. Modern cellular phone protocols
involve a technique called frequency hopping - the cell phone base
station selects a new frequency or channel for the phone to
communicate with the base station and informs the phone of this
change.  The phone must switch to the new channel within a certain
amount of time, or it risks losing connection with the base
station.

If the phone decoder application is written in {\StreamIt}, the
{\filter} controlling the antenna and the {\filter} which will process
control signals are likely far apart, and may not have a simple
way of communicating data directly with each other. In {\StreamIt},
the {\filter} which decodes control signals can simply send a
message to the {\filter} controlling the antenna. The message can be
sent with a specific latency corresponding to the timing required
by the base station. When the antenna controller receives the
message it can change the appropriate settings in the hardware to
switch to the appropriate new frequency, without having to wait
for the appropriate time. The timing of delivery is taken care of
by {\StreamIt}.

\subsubsection{Information Wavefronts}

When a data item enters a stream, it carries with it some new
information. As execution progresses, this information cascades
through the stream, affecting the state of {\filters} and the values
of new data items which are produced. We refer to an information
wavefront as the set of {\filter} executions that first sees the
effects of a given input item. Thus, although each {\filter}'s {\work}
function is invoked asynchronously without any notion of global
time, two invocations of a work function occur at the same
information-relative time if they operate on the same information
wavefront.

\subsubsection{Message Sending}

Messages can be sent upstream or downstream between any two
{\filters}.  Sending messages across branches of a {\splitjoin} is not
legal.  Timing of message delivery uses the concept of information
wavefront.  The sender specifies that the message is supposed to
be delivered with a certain delay of information wavefront. The
delays are specified as ranges, $[l_0, l_1], l_0 \le l_1$. $l_0$
and $l_1$ specify the information wavefront in executions of the
{\work} function of the sender {\filter}.

If the message is being sent downstream, the sender specifies that
the receiver will receive the data just before it sees the
information wavefront produced by the sender between $l_0$ and
$l_1$ executions of its {\work} function from when it sends the
message. If the message is being sent upstream, the sender
specifies that the receiver must receive the message just before
it produces an information wavefront the sender will see between
$l_0$ and $l_1$ executions of its {\work} function from when it
sends the message.

Message sending is meant to be a low-bandwidth method of
communication between {\filters}. Message sending is not a fast
operation and is intended not to interfere with the high bandwidth
{\StreamIt} communication and processing.  However, depending on how
tight the latency constraints are (both the magnitude of the
latency as well as the range), declaring that messages can be sent
may slow program execution down considerably.

Figure \ref{fig:message-example} presents an example of a
{\pipeline} in which the last {\filter} sends a message to the first
{\filter}.  {\filter}$_3$ sends a message to {\filter}$_0$.  The message
is sent with latency $[3,8]$. This means that after at least 3 and
at most 8 executions of sender's {\work} function, it will see data
produced by the receiver just after receiving the message.

\begin{figure}\begin{center}
\begin{minipage}{4in}
\centering
\psfig{figure=message-example.eps,width=1.5in} \\
\end{minipage}
\end{center}
\caption{Example of a {\pipeline} with a message being sent}
\label{fig:message-example}
\end{figure}

\subsection{The {\StreamIt} Language}
\label{sec:streamit}

{\StreamIt} is an architecture-independent programming language for
high-performance streaming applications.  This section contains a very
brief overview of the semantics of {\StreamIt}.  We do not conncern
ourselves with the syntax of the language, as it is not relevant to
scheduling stream graphs. A more detailed description of the design
and rationale for {\StreamIt} can be found in~\cite{thies02streamit}
or on our website~\cite{streamitweb}.

\subsubsection{Language Constructs}

The basic unit of computation in {\StreamIt} is the {\filter}. A
{\filter} is a single-input, single-output block with a user-defined
procedure for translating input items to output items.  Every
{\filter} contains a {\work} function, which is comprised of one or
more atomic phases that the filter cycles through during its
steady-state execution. A filter can optionally declare a {\tt
prework} function that executes instead of {\tt work} on the first
invocation of the filter, if special startup behavior is desired.
Filters communicate with their neighbors via FIFO queues, called
{\Channels}, using the intuitive operations of {\tt push(value)}, {\tt
pop()}, and {\tt peek(index)}, where {\tt peek} returns the value at
position {\tt index} without dequeuing the item.  The number of items
that are pushed, popped, and peeked\footnote{{\small We define $peek$
as the total number of items read, including the items popped.  Thus,
we always have that $peek \ge pop$.}} on each invocation are declared
with each phase of the {\work} function.

%% {\StreamIt}'s representation of a filter is an improvement over
%% general-purpose languages.  In a procedural language, the analog of a
%% filter is a block of statements in a complicated loop nest.  There is
%% no clear abstraction barrier between one filter and another, and
%% high-volume stream processing is muddled with global variables and
%% control flow. The loop nest must be re-arranged if the input or output
%% ratios of a filter changes, and scheduling optimizations further
%% inhibit the readability of the code.

%% In an object-oriented language, one could implement a stream
%% abstraction as a library.  This avoids some of the problems associated
%% with a procedural loop nest, but the programming model is complicated
%% by efficiency concerns--to optimize cache performance, the entire
%% application processes blocks of data that complicate and obscure the
%% underlying algorithm.

%% In contrast to these alternatives, {\StreamIt} places the filter in its
%% own independent unit, making explicit the parallelism and inter-filter
%% communication while hiding the grungy details of scheduling and
%% optimization from the programmer.

{\StreamIt} provides three primitives for composing {\filters} into
hierarchical streams (see Figure~\ref{fig:steady-state}).  The
{\pipeline} construct cascades a set of filters in sequence, with the
output of one connected to the input of the next.  The {\splitjoin}
construct is used to specify independent parallel streams that diverge
from a common {\splitter} and merge into a common {\joiner}--for
example, in the Equalizer of Figure~\ref{fig:radio-ascoded}.  Within a
{\splitjoin}, a {\splitter} collects $o_s$ data from its {\Input}
{\Channel} and pushes $w_{s,i}$ data to its $i$th {\Output}
{\Channel}.  Likewise, a {\joiner} collects $w_{j,i}$ items from its
$i$th input {\Input}{\Channel} and pushes $u_j$ data to its
{\Output}{\Channel}.  {\StreamIt} currently supports two types of
{\splitters} ({\duplicate} and {\roundrobin}) and one type of
{\joiner} (\roundrobin).

The last control construct provides a means for creating cycles in the
stream graph: the {\feedbackloop}. A {\feedbackloop} contains a
{\joiner}, a body operator, a {\splitter}, and a loop operator.  A
{\feedbackloop} has an additional feature required to allow it to
begin computation: since at first there is no data on any {\Channels},
the program inserts data from a special function defined by the
{\feedbackloop} onto the {\Channel} connecting the loop child and the
{\joiner}. The amount of data pushed onto the feedback path is called
delay amount, denoted $delay_{fl}$, for a {\feedbackloop} $fl$.

\subsubsection{Design Rationale}

{\StreamIt} differs from other stream languages in the single-input,
single-output hierarchical structure that it imposes on streams.  This
structure aims to help the programmer by defining clean, composable
modules that admit a linear textual representation.  In addition, it
helps the compiler by restricting certain analyses to a local level
rather than dealing with global properties of a graph.

In the context of scheduling, hierarchy is also useful because it
allows for the separate compilation of program componenets.  This
enables creation of standardized libraries and their distribution in
binary form, rather than source code.  This ability may become
important as streaming languages become more widely used for larger
applications.

\begin{figure}[t]
\begin{center}
\hspace{0.1in} \psfig{figure=radio-ascoded.eps, width=2.8in}
\vspace{-36pt} \caption{\protect\small Block diagram of an FM
Radio. \protect\label{fig:radio-ascoded}}
%% \begin{minipage}{0.46in}
%% \centering
%% \psfig{figure=pipeline.eps,width=0.46in} \\
%% \end{minipage}
%% ~
%% \begin{minipage}{1.0in}
%% \centering
%% \psfig{figure=splitjoin.eps,width=0.57in} \\
%% \end{minipage}
%% ~
%% \begin{minipage}{1.02in}
%% \centering
%% \psfig{figure=feedback.eps,width=1.02in} \\
%% \end{minipage}
%% \\ ~ \\ {\bf \protect\small (a) \pipeline ~~~~ (b) \splitjoin ~~~~ (c) \feedbackloop}
%% \caption{\protect\small Stream structures supported by {\StreamIt}.
%% \protect\label{fig:structures}} \vspace{-12pt}
\end{center}
\end{figure}

%% \subsection{Messages}

%% {\StreamIt} provides a dynamic messaging system for passing irregular,
%% low-volume control information between filters and other stream
%% operators.  Messages
%% are sent from within the body of a filter's {\tt work} function,
%% perhaps to change a parameter in another filter.  The central aspect
%% of the messaging system is a sophisticated timing mechanism that
%% allows filters to specify when a message will be received relative to
%% the flow of data items between the sender and the receiver.  With the
%% messaging system, {\StreamIt} is equipped to support full application
%% development--not just high-bandwidth data channels, but also events,
%% control, and re-initialization.

\section{General {\StreamIt} Scheduling Concepts}
\label{chpt:sched-basic}

This chapter introduces the general concepts used for scheduling
{\StreamIt} programs.  Concepts presented here are are common with
other languages \cite{ptolemyoverview} \cite{esterel92}
\cite{lustre}.

Section \ref{sec:exec-model} presents the {\StreamIt} execution
model. Section \ref{sec:steady-state} introduces the concept of a
steady state and shows how to calculate it. Section
\ref{sec:init-peeking} explains the need for initialization of
{\StreamIt} program. Section \ref{sec:general:schedules} introduces
simple notation for expressing schedules while Section
\ref{sec:sched-vs-buffer} presents the tradeoff between schedule
and buffer storage requirements.

\subsection{{\StreamIt} execution model}
\label{sec:exec-model}

A {\StreamIt} program is represented by a directed graph, $G = (N,
E)$.  A node in $G$ is either a {\filter}, a {\splitter} or a
{\joiner}. Edges in $G$ represent data {\Channels}.  Each node in
$G$ takes data from its {\Input} {\Channel}(s), processes this data,
and puts the result on the {\Output} {\Channel}(s).  Each data
{\Channel} is simply a FIFO queue.

Each {\filter} node $n_f$ has exactly one incoming edge and one
outgoing edge.  The incoming edge is referred to as an {\Input}
{{\Channel}}, while the outgoing edge is called an {\Output}
{{\Channel}}. A {\splitter} node $n_s$ has exactly one incoming edge
({\Input} {\Channel}), but has multiple outgoing edges ({\Output}
{\Channels}). A {\joiner} node has multiple incoming edges ({\Input}
{\Channels}) but only one outgoing edge ({\Output} {\Channel}).

Each node of graph $G$ can be executed.  An execution of a node
causes some data to be collected from the node's {\Input}
{\Channel}(s), the data to be processed and the result to be put on
the {\Output} {\Channel}(s).  An execution of a node transfers the
smallest amount of data across the node - it is an atomic
operation.  {\StreamIt} uses a static data flow model, meaning
that every execution of a node $n$ will require the same amount of
data to be present on node's {\Input} {\Channel}(s) for consumption or
inspection, same amount to be consumed from the {\Input} {\Channel}(s)
and same amount of data to be pushed onto its {\Output} {\Channel}(s).

Each {\filter} node $n_f$ is associated with a 3-tuple $(e_f, o_f,
u_f)$. These three values represent the rate of data flow for the
{\filter} for each execution.  The first value represents the
amount of data necessary to be present in its {\Input} {\Channel} in
order to execute the {\filter}.  This is also called the peek
amount of the {\filter}.  The second value represents the amount
of data which will be consumed by the {\filter} from its {\Input}
{\Channel}. This is called the pop amount of the {\filter}.  Note,
that $e_f \ge o_f$. The final value represents the amount of data
that will be put on the {\Output} {\Channel} of the {\filter}. This is
called the push amount of a {\filter}.  The amount of data present
in the {\Input} {{\Channel}} of a {\filter} node $n_f$ is denoted
$in_f$, while data present in the {\Output} {{\Channel}} is denoted
$out_f$.

Each {\splitter} node $n_s$ is associated with a tuple $(o_s,
w_s)$. The first value represents the amount amount of data that
will be consumed by $n_s$ from its {\Input} {\Channel}. Thus, in order
to execute $n_s$, there must be at least $o_s$ data in its {\Input}
{\Channel}. $w_s$ is a vector of integers, each representing the
amount of data that will be pushed onto a corresponding {\Output}
{\Channel} of $n_s$.  The amount of data present in the {\Input}
{{\Channel}} of a {\splitter} node $n_s$ is denoted $in_s$, while
data present in the $i$th {\Output} {{\Channel}} is denoted
$out_{s,j}$.

Each {\joiner} node $n_j$ is associated with a tuple $(w_j, u_j)$.
The first value is a vector of integers, each representing the
amount of data that will be consumed by $n_j$ from its
corresponding {\Input} {\Channels}.  In order to execute $n_j$, each
of its {\Input} {\Channels} must have at least as much data in it as
the corresponding value in $w_j$ indicates.  $u_j$ represents the
amount of data that will be pushed by $n_j$ onto its {\Output}
{\Channel}. The amount of data present in the $i$th {\Input}
{{\Channel}} of a {\joiner} node $n_j$ is denoted $in_{j,i}$, while
data present in the {\Output} {{\Channel}} is denoted $in_s$.

A schedule for a {\StreamIt} program is a list of executions of
nodes of graph $G$.  The list describes the order in which these
nodes are to be executed.  In order for a schedule to be legal, it
must satisfy two conditions.  The first one is that for every
execution of a node, a sufficient amount of data must be present
on its {\Input} {\Channel}(s), as described above.  The second is that
the execution of the schedule must require a finite amount of
memory.

\subsection{Steady State}
\label{sec:steady-state}

A {\StreamIt} schedule is an ordered list of firings of nodes in the
{\StreamIt} graph.  Every firing of a node consumes some data from
{\Input} {{\Channel}}(s) and pushes data onto the {\Output} {{\Channel}}(s).

One of the most important concepts in scheduling streaming
applications is the steady state schedule.  A steady state
schedule is a schedule that the program can repeatedly execute
forever.  It has a property that the amount of data buffered up
between any two nodes does not change from before to after the
execution of the steady state schedule.  This property is
important, because it allows the compiler to statically schedule
the program at compile time, and simply repeat the schedule
forever at runtime.  A schedule without this property cannot be
repeated continuously.  This is because the delta in amount of
data buffered up on between nodes will continue accumulating,
requiring an infinite amount of buffering space.

A steady state of a program is a collection of number of times
that every node in the program needs to execute in a steady state
schedule.  It does not impose an order of execution of the nodes
in the program.

Not every {\StreamIt} program has a steady state schedule.  As will
be explained in Section \ref{sec:calc-min-steady}, it is possible
for a program to have unbalanced production and consumption of
data in {\splitjoins} and {\feedbackloops}.  The amount of data
buffered continually increases, and cannot be reduced, thus making
it impossible to create a steady state schedule for them.  It is
also possible that a {\feedbackloop} does not have enough data
buffered up internally in order to complete execution of a full
steady state, and thus deadlocks. Programs without a valid steady
state schedule are not considered valid {\StreamIt} programs. In
other words, all valid {\StreamIt} programs have a steady state
schedule.

\subsubsection{Minimal Steady State}

The size of a steady state is defined as the sum of all executions
of all the nodes in the program per iteration of the steady state.

\begin{definition}
A steady state of stream $s$ is represented by vector $m$ of
non-negative integers. Each of the elements in $m$ represents the
number of times a corresponding node in $s$ must be executed in
the steady state.
\end{definition}

Note that $m$ does not impose an order of execution of nodes. Size
of a steady state is the total number of executions of all the
nodes in the steady state, and is represented by $\sum_i m_i$.

Next we will summerize the properties of schedules prsented in
\cite{lee87static}.

\begin{theorem}[Minimal Steady State Uniqueness]
A {\StreamIt} program that has a valid steady state, has a unique
minimal steady state.
\end{theorem}

\begin{proof}[Minimal Steady State Uniqueness]
Assume that there are two different minimal steady states with
same size.  Let $m$ and $q$ denote vectors representing the two
steady states. Let $\sum_i m_i$ denote size of schedule $m$ and
$\sum_i q_i$ denote size of schedule $q$. Note that since both $m$
and $q$ are minimal steady states, $\sum_i m_i = \sum_i q_i$.
Since the schedules are different, there must be some $j$ for
which $m_j \ne q_j$. Assume without loss of generality that $m_j <
q_j$. Since a steady state does not change the amount of data
buffered between nodes, the node producing data for node $i$ must
also execute less times than corresponding node in $q$. Similarly,
the node consuming data produced by node $j$ also must execute
less times than the corresponding node in schedule $q$. Since a
{\StreamIt} program describes a connected graph, it follows that
$\forall i, m_i < q_i$.  Thus $\sum_i m_i \ne \sum_i q_i$, which
is a contradiction. Thus there cannot be two different minimal
steady state.
\end{proof}

\begin{corollary}[Minimal Steady State Uniqueness]
\label{corollary:minimal-state}
The additional property we have from the above proof is that if
$m$ represents a minimal steady and $q$ any other steady state,
then $\forall i, m_i < q_i$.
\end{corollary}

\begin{lemma}[Composition of Steady Schedules]
\label{lemma:composition}
If $m$ and $q$ are two steady states for a {\StreamIt} program, then
$m + q$ is also a steady state.
\end{lemma}

The above lemma is true because neither $m$ nor $q$ change the
amount of data buffered in the {{\Channels}}.  Thus a composition of
the steady states does not change the amount of data buffered in
the {{\Channels}}, which makes the composition also a steady schedule.

\begin{corollary}[Composition of Steady Schedules]
\label{corollary:composition}
If $m$ and $q$ are two steady states, and $\forall i, m_i > q_i$,
then $w = m - q$ is also a steady state.
\end{corollary}

If $q$ is a steady state and $m = w + q$ is a steady state, then
$w$ must not change the amount of data buffered in {{\Channels}}. Thus
$w$ must be a steady state.

\begin{theorem}[Multiplicity of Minimal Steady States]
If a {\StreamIt} program has a valid steady state, then all its
steady states are strict multiples of its minimal steady state.
\label{thm:multiplicity}
\end{theorem}

\begin{proof}[Multiplicity of Minimal Steady State]
Assume that there exists a steady state that is not a multiple of
the minimal steady state.  Let $m$ denote the minimal steady
state. Let $q$ denote the other steady state.  Note that $w = q -
m$ is still a steady state, as long as all elements of $w$ remain
non-negative (by Corollary \ref{corollary:composition}).  Repeat
subtracting $m$ from $q$ until no more subtractions can be
performed without generating at least one negative element in
vector $w$.  Since $q$ is not a multiple of $m$, $w \ne 0$. But
since we cannot subtract $m$ from $w$ any further, $\exists i, m_i
> w_i$.  Since $m$ is a minimal steady state and $w$ is a steady
state, this is impossible due to Corollary
\ref{corollary:minimal-state}. Thus there are no steady states
that are not multiples of the minimal steady schedule.
\end{proof}

\subsubsection{Calculating Minimal Steady State}
\label{sec:calc-min-steady}

This section presents equations used for calculating minimal
steady states.  Minimal steady states are calculated recursively
in a hierarchical manner. That is, a minimal steady state is
calculated for all children streams of {\pipeline}, {\splitjoin} and
{\feedbackloop}, and then the schedule is computed for the actual
parent stream using these minimal states as atomic executions.
This yields a minimal steady state because all child streams must
execute their steady states (to avoid buffering changes), and all
steady states are multiples of the minimal steady states (per
Theorem \ref{thm:multiplicity}).  Executing a full steady state of
a stream is referred to as "executing a stream".

\subsubsubsection{Notation of Steady States}

In this section, the notation for $peek$, $pop$ and $push$ will be
extended to mean entire streams in their minimal steady state
execution.  That is, a {\pipeline} $p$ will consume $o_p$ data,
produce $u_p$ data and peek $e_p$ data on every execution of its
steady state.  Again, in the hierarchical view of {\StreamIt}
programs, a child stream of a {\pipeline} will execute its steady
state atomically.

A steady state of a stream $s$ is represented by a set $S_s$ of
elements, $S_s = \{m, N, c, v\}$. The set includes a vector $m$,
which describes how many times each {\StreamIt} node of the stream
will be executed in the steady state, a corresponding ordered set
$N$ which stores all the nodes of the stream, a vector $c$, which
holds values $[e_s, o_s, u_s]$ for stream $s$, and a vector $v$
which holds number of steady state executions of all direct
children of $s$. $m$ and $v$ are not the same vector, because $m$
refers to nodes in the subgraph, while $v$ refers only to the
direct children, which may be {\filters}, {\pipelines},
{\splitters} and {\feedbackloops}.

For a stream $s$, set $S$ is denoted as $S_s$ and the elements of
$S_s$ are denoted as $S_{s,m}$, $S_{s,N}$, $S_{s,c} and S_{s,v}$.

Note, that a steady state does not say anything about the ordering
of the execution of nodes, only how many times each node needs to
be executed to preserve amount of data buffered by the stream.

\begin{figure}
\begin{center}

\begin{minipage}{1.5in}
\centering \psfig{figure=pipeline-steady-state.eps,width=0.6in} \\
{\protect\small (a) A sample {\pipeline}}
\end{minipage}
~
\begin{minipage}{1.5in}
\centering \psfig{figure=splitjoin-steady-state.eps,width=1.2in} \\
{\protect\small (b) A sample {\splitjoin}}
\end{minipage}
~
\begin{minipage}{2in}
\centering \psfig{figure=feedback-steady-state.eps,width=1.0in} \\
{\protect\small (c) A sample {\feedbackloop}.  The $L$ {\filter} has
been flipped upside-down for clarity.\\$peek_L = pop_L = 5, push_L
= 6$}
\end{minipage}

\caption{Sample {\StreamIt} streams} \label{fig:steady-state}

\end{center}
\end{figure}

\subsubsubsection{\filter}

Since {\filters} do not have any internal buffering, their minimal
steady state is to execute the {\filter}'s {\work} function once. This
is the smallest amount of execution a {\filter} can have.

Thus, for a {\filter} $f$,

\begin{displaymath}
S_f = \left\{[1], \{f\}, { \left[
\begin{array}{c}e_f\\o_f\\u_f
\end{array}
\right]}, [] \right\}
\end{displaymath}

Notice that $S_{f,v}$ is empty, because a {\filter} does not have
any children.

\subsubsubsection{\pipeline}

Let the {\pipeline} $p$ have $n$ children and let $p_i$ denote
the $i$th child of the {\pipeline} (counting from {\Input} to
{\Output}, starting with 0, the children may be streams, not
necessarily {\filters}). We must find $S_p$.

We start with calculating all $S_{p_i}, i \in \{0, \dots, n-1\}$.
This task is achieved recursively.

Next we find a fractional vector $v''$ such that executing each
$p_i$ $v_i''$ times will not change the amount of data buffered in
the {\pipeline} and the first child is executed exactly once.  Since
the children streams are executed fractional amount of times, we
calculate the amount of data they produce and consume during this
execution by multiplying $S_{p_i,c_o}$ and $S_{p_i,c_u}$ by
$v_i''$. Thus $v''$ must have the following property

\begin{displaymath}
v_0'' = 1, \forall i \in \{0,\dots,n-1\}, v_i'' * u_{p_i} =
v_{i+1}'' * o_{p_{i-1}}
\end{displaymath}

We compute $v''$ as follows.  The first child executes once, thus
$v_0'' = 1$.  The second child must execute $v_1'' = {u_{p_0}
\over {o_{p_1}}}$ times to ensure that all data pushed on the the
first {{\Channel}} is consumed by the second child.  The third child
must execute $v_2'' = v_1'' {u_{p_1} \over o_{p_2}} = {u_{p_0}
\over o_{p_1}} {u_{p_1} \over o_{p_2}}$ times to ensure that it
consumes all the data produced by the second child. Thus,

\begin{displaymath}
v_i'' = {\prod_{j = 0}^{i-1} u_{p_j} \over \prod_{j=1}^i o_{p_j}}
\end{displaymath}

Next we will find an integral vector $v'$ such that executing each
$p_i$ $v_i'$ times will not change the amount of data buffered in
the {\pipeline}.  $v'$ will be a valid steady state of the
{\pipeline}.

In order to calculate $v'$ we multiply $v''$ by $\prod_{j=1}^{n-1}
o_{p_j}$.  Thus

\begin{displaymath}
v'_i = \left({\prod_{j = 0}^{i-1} u_{p_j} \over \prod_{j=1}^i
o_{p_j}} \right) \left(\prod_{j=1}^{n-1} o_{p_j} \right) = \left(
\prod_{j=0}^{i-1} u_{p_j} \right) \left( \prod_{j=i+1}^{n-1}
o_{p_j} \right)
\end{displaymath}

Now we find an integral vector $v$, such that, for some positive
integer $g$, $v' = g * v$, and $\sum_i v_i$ is minimal.  In other
words, we find the greatest integer $g$, such that $v' = g * v$,
with $v$ consisting of integers.  $v$ represents the minimal
steady state for pipeline $p$.

This is achieved by finding the $\gcd$ of all elements in $v'$,
and dividing $v'$ by $g$.  Thus

\begin{displaymath}
v = {v' \over \gcd(v')}
\end{displaymath}

$v$ represents the number of times each child of $p$ will need to
execute its steady state in order to execute the minimal steady
state of $p$, thus $S_{p,v} = v$.  $v$ holds a steady state
because amount of data buffered in $p$ does not change, and it is
a minimal steady state, because $\sum_i v_i$ is minimal.

We construct set $S_p$ as follows:\footnote{Here we use symbol
$\circ$ to denote concatenation of vectors and sets.  Thus $[1\ 2\
3] \circ [4\ 5\ 6] = [1\ 2\ 3\ 4\ 5\ 6]$ and $\{A\ B\ C\} \circ
\{D\ E\ F\} = \{A\ B\ C\ D\ E\ F\}$.}

\begin{displaymath}
S_p = \left\{ \begin{array}{c}
v_0 * S_{p_0,m} \circ \dots \circ
v_{n-1}
* S_{p_{n-1}, m}, S_{p_0, N} \circ \dots \circ S_{P_{n-1}, N}, \\
\left[
\begin{array}{c}
e_{p_0} + (v_0 - 1) * o_{p_0} \\
v_0 * o_{p_0} \\
v_{n-1} * u_{p_{n-1}}
\end{array}\right], v \end{array} \right\}
\end{displaymath}

An example is presented in Figure \ref{fig:steady-state} (a). For
this {\pipeline}, we have the following steady states for all
children of the {\pipeline}:

\begin{displaymath}
\begin{array}{lrlr}
S_A = & \left\{[1], \{A\}, { \left[
\begin{array}{c} 1 \\ 1 \\ 3
\end{array}
\right]}, [] \right\}, &

S_B = & \left\{[1], \{B\}, { \left[
\begin{array}{c} 3 \\ 2 \\ 3
\end{array}
\right]}, [] \right\} \\ \\

S_C = & \left\{[1], \{D\}, { \left[
\begin{array}{c} 2 \\ 2 \\ 1
\end{array}
\right]}, [] \right\}, &

S_D = & \left\{[1], \{D\}, { \left[
\begin{array}{c} 5 \\ 3 \\ 1
\end{array}
\right]}, [] \right\} \\

\end{array}
\end{displaymath}

Using the steady states above, we get the following vector $v'$:

\begin{displaymath}
v' = \left[
\begin{array}{c}
(2 * 2 * 3)\\
(3) (2 * 3) \\
(3 * 3) (3) \\
(3 * 3 * 1)
\end{array}
\right] = \left[
\begin{array}{c}
12\\ 18\\ 27\\ 9
\end{array}
\right]
\end{displaymath}

We now calculate $g = \gcd(v') = \gcd(12,18,27,9) = 3$.  We thus
have

\begin{displaymath}
v = {v' \over 3} = {1 \over 3} \left[
\begin{array}{c}
12\\ 18\\ 27\\ 9
\end{array}
\right] = \left[
\begin{array}{c}
4\\ 6\\ 9\\ 3
\end{array}
\right]
\end{displaymath}

Finally, we construct $S_p$:

\begin{displaymath}
S_p = \left\{
\begin{array}{c}
4 S_{A,m} \circ 6 S_{B,m} \circ 9 S_{C,m} \circ 3
S_{D,m}, S_{A,N} \circ S_{B,N} \circ S_{C,N} \circ S_{D,N} \\
\left[
\begin{array}{c}
1 + (4-1) * 1 \\
4 * 1 \\
3 * 1
\end{array}\right],
\left[ \begin{array}{c} 4\\ 6\\ 9\\ 3 \end{array}
\right]
\end{array}
\right\}
\end{displaymath}

\subsubsubsection{\splitjoin}

Let the {\splitjoin} have $n$ children and let $sj_i$ denote the
$i$th child of the {\splitjoin} (counting from left to right,
starting with 0).  Let $sj_s$ and $sj_j$ denote the {\splitter} and
the {\joiner} of the {\splitjoin}, respectively. Let $w_{s,i}$ denote
the number of items sent by the {\splitter} to $i$th child on
{\splitter}'s every execution. Let $w_{j,i}$ denote the number of
items consumed by the {\joiner} from the $i$th child on {\joiner}'s
every execution.  We are computing $S_{sj}$.

We start by calculating all $S_{sj_i}, i \in \{0, \dots, n-1\}$.

Next we compute a fraction vector $v''$ and a fraction $a_j''$
such that executing the {\splitter} exactly once, each child $sj_i$
$v_i''$ times and the {\joiner} $a_j''$ times does not change the
amount of data buffered in the {\splitjoin}. Again, since $v''$ and
$a_j''$ are fractions, we multiply the steady-state pop and push
amounts by appropriate fractions to obtain the amount of data
pushed and popped.  For convenience we define $a_s''$ to be the
number of executions of the {\splitter} and set it to 1.

\begin{comment}
\begin{displaymath}
v'', a_j'', a_s'' \ne 0, \forall i \in \{0,\dots,n-1\}, a_s'' *
w_{s, i} = v_i'' * o_{sj_i}, v_i'' * u_{sj_i} = a_j'' * w_{j, i}
\end{displaymath}
\end{comment}

We thus have that each child $sj_i$ must execute $v_i'' = {w_{s,i}
\over o_{sj_i}}$ times. To compute the number of executions of the
{\joiner}, $a_j''$, we select an arbitrary $k$th child ($0 \le k <
n$) and have that the {\joiner} executes $a_j'' = {{w_{s,k} \over
o_{s_k}}{u_{sj_k} \over w_{j,k}}}$ times.

Next we compute integer vector $v'$ and integers $a_s$ and $a_j$
such that executing the {\splitter} $a_s$ times, each child $sj_i$
$v_i'$ times and the {\joiner} $a_j$ times still does not change the
amount of data buffered in the {\splitjoin}. We do this by
multiplying $a_s''$, $v''$ and $a_j''$ by $w_{j,k}
\left(\prod_{r=0}^{n-1}o_{sj_r}\right)$. Thus we get

\begin{displaymath}
\begin{array}{rl}
a_s' = & w_{j,k} \left(\prod_{r=0}^{n-1}o_{sj_r}\right) \\
v_i' = & w_{j,k} \left(\prod_{r=0}^{n-1}o_{sj_r}\right) * {w_{s,i}
\over o_{sj_i}} = w_{s,i} * w_{j_k} \left( \prod_{r=0}^{i-1}
o_{s_r} \right) \left( \prod_{r=i+1}^{n-1} o_{s_r} \right)
\\
a_j' = & w_{j,k} \left(\prod_{r=0}^{n-1}o_{sj_r}\right) *
{{w_{s,k} \over o_{s_k}}{u_{sj_k} \over w_{j,k}}} = w_{s,k} *
u_{sj_k} * \left( \prod_{r=0}^{k-1} o_{s_r} \right)
\left( \prod_{r=k+1}^{n-1} o_{s_r} \right) \\
\end{array}
\end{displaymath}

Now we use $v'$, $a_s'$ and $a_j'$ to compute minimal steady state
of the {\splitjoin}.  Since $v'$, $a_s'$ and $a_j'$ represent a
steady state, they represent a strict multiple of the minimal
steady state.  Thus we find the multiplier by computing $g$, the
$\gcd$ of all elements in $v'$ and integers $a_s'$ and $a_j'$, and
dividing $v'$, $a_s'$ and $a_j'$ by $g$.  We have that

\begin{displaymath}
\begin{array}{rl}
g = & \gcd(v', a_s', a_j') \\
v = & v' \over g \\
a_s = &  a_s' \over g \\
a_j = & a_j' \over g
\end{array}
\end{displaymath}

Finally, we use $v$, $a_s$ and $a_j$ to construct $S_{sj}$:

\begin{displaymath}
S_{sj} = \left\{
\begin{array}{c}
v_0 * S_{sj_0,m} \circ \dots \circ v_{n-1} * S_{sj_{n-1}, m} \circ
[a_s\ a_j] , \\
S_{sj_0, N} \circ \dots \circ S_{sj_{n-1}, N} \circ \{sj_s,
sj_j\},
\\ \left[
\begin{array}{c}
n_s * o_{s} \\
n_s * o_{s} \\
n_j * u_{j} \\
\end{array}\right], \\
v \circ [a_s] \circ [a_j]
\end{array}\right\}
\end{displaymath}

Figure \ref{fig:steady-state} (b) depicts a sample {\splitjoin}. The
following are the steady states of the {\splitjoin}'s children: $$
\begin{array}{lrlr} S_A = & \left\{[1], \{A\}, { \left[
\begin{array}{c} 2 \\ 2 \\ 1
\end{array}
\right]}, [] \right\}, & S_B = & \left\{[1], \{B\}, { \left[
\begin{array}{c} 3 \\ 2 \\ 6
\end{array}
\right]}, [] \right\}
\end{array}
$$ For this {\splitjoin}, we select $k = 0$ (we use the left-most child
to compute $a_j'$).  We get the following $v'$, $a_s'$ and $a_j'$

\begin{displaymath}
\begin{array}{rl}
v' = & \left[
\begin{array}{c}
2 * 2 (2)\\
1 * 2 (2)
\end{array}
\right] = \left[
\begin{array}{c}
8 \\ 4
\end{array}
\right] \\
a_s' = & 1 * 2 (2 * 2) = 8 \\
a_j' = & 2 * 1 (2 * 2) = 8
\end{array}
\end{displaymath}

Thus $\gcd(u', a_s', a_j') = \gcd(8,4,8,8) = 4$.  Now we obtain

\begin{displaymath}
\begin{array}{rl}
v = & {v \over 4} = {1 \over 4} \left[
\begin{array}{c}
8 \\ 4
\end{array}
\right] =  \left[
\begin{array}{c}
2 \\ 1
\end{array}
\right]\\
a_s = & {a_s' \over 4} = {8 \over 4} = 2 \\
a_j' = & {a_j' \over 4} = {8 \over 4} = 2
\end{array}
\end{displaymath}

Finally, we construct $S_{sj}$:

\begin{displaymath}
S_{sj} = \left\{
\begin{array}{c}
2 * S_{sj_0, m} \circ 1 * S_{sj_1, m} \circ [2\ 2], \\
S_{sj_0, N} \circ S_{sj_1, N} \circ \{sj_s, sj_j\}, \\
\left[
\begin{array}{c}
2 * 3 \\ 2 * 3 \\ 2 * 4
\end{array}
\right], \left[
\begin{array}{c}
2 \\ 1 \\ 2 \\ 2
\end{array}\right]
\end{array} \right\}
\end{displaymath}

\begin{figure}\begin{center}
\begin{minipage}{2in}
\centering \psfig{figure=splitjoin-illegal.eps,width=2in}
\end{minipage}
\end{center}
\caption{An illegal {\splitjoin}} \label{fig:splitjoin-illegal}
\end{figure}

It is important to note, that it is not always possible to compute
a unique $v''$ for all possible {\splitjoins}. The reason is that
unbalanced production/consumption ratios between different
children of a {\splitjoin} can cause data to buffer up infinitely.

\begin{definition}[Valid {\splitjoin}] A {\splitjoin} is valid
{\emph iff} $\forall k, 0 \le k < n-1, a_{j,k}'' = a''_{j,k+1}$,
using notation of $a_{j,k}''$ to indicate that $k$th child of the
{\splitjoin} was used to compute the value of $a_j''$.
\end{definition}

An example of an illegal {\splitjoin} is depicted in Figure
\ref{fig:splitjoin-illegal}.  The rates of throughput of data for
the left child mean that for every execution of the {\splitter}, the
{\joiner} needs to be executed exactly once to drain all data
entering the {\splitjoin}.  The rates of throughput of data for the
right child mean that for every execution of the {\splitter}, the
{\joiner} needs to be executed exactly twice to drain all data
entering the {\splitjoin}. That means that consumption of data by
the {\joiner} will be relatively slower on the right side, causing
data to buffer up. This means that the given {\splitjoin} does not
have a steady state.

If a {\splitjoin} is such that it does not have a steady state, it
is considered an illegal {\splitjoin}.  It cannot be executed
repeatedly without infinite buffering, so a practical target for
{\StreamIt} cannot execute it.  The calculations presented here
assume that the {\splitjoin} is legal.  In order to check if a given
{\splitjoin} is legal, we test if selecting a different child for
calculation of $a_j''$ yields a different $a_j''$. If it does,
then the two paths tested have different production/consumption
rates, and the {\splitjoin} does not have a steady state.

\subsubsubsection{\feedbackloop}

Let {\feedbackloop} $fl$ have children $B$ (the body child) and $L$
(the feedback loop child). Let the {\joiner} and the {\splitter} of
the {\feedbackloop} be denoted $fl_j$ and $fl_s$. Let $w_{j,I}$ and
$w_{j,L}$ denote the number of data items consumed by the {\joiner}
from the {\Input} {{\Channel}} to the {\feedbackloop} and from $fl_L$,
respectively.  Let $w_{s,O}$ and $w_{s,F}$ denote the number of
data items pushed by the {\splitter} onto the {\feedbackloop}'s {\Input}
{{\Channel}} and to $fl_L$ respectively.  We are computing $S_{fl}$.

First we calculate $S_{B}$ and $S_{L}$.

Now we compute a fractional vector $v'' = [a_B''\ a_L''\ a_s''\
a_j'']$ such that executing the body child $a_B''$ times, the
{\splitter} $a_s''$ times, the loop child $a_F''$ times and the
{\joiner} $a_j''$ times will not change the amount of data buffered
up in the {\feedbackloop}.  Thus

\begin{displaymath}
\begin{array}{rcl}
a_B' * u_B & = & a_s' * o_s \\
a_L' * u_B & = & a_j' * w_{j, L} \\
a_s' * w_{s, F} & = & a_L' * o_B \\
a_j' * u_j & = & a_B' * o_B \\
\end{array}
\end{displaymath}

We begin with setting $a_j'' = 1$. $B$ needs to be executed $a_B''
= u_j \over o_B$ times, the {\splitter} needs to be executed $a_s''
= {u_j \over o_B}{u_B \over o_s}$ times and $L$ needs to be
executed $a_L'' = {u_j \over o_B}{u_B \over o_s}{w_{s,L} \over
o_L}$ times. Furthermore, in order to assure that the
{\feedbackloop} has a valid steady state, we continue going around
the loop, the {\joiner} must require ${u_j \over o_B}{u_B \over
o_s}{w_{s,L} \over o_L}{u_L \over w_{j,L}} = 1$.  If this
condition is not satisfied, the {\feedbackloop} does not have a
steady state. This is a necessary, but not a sufficient condition
for a {\feedbackloop} to be valid.

Next we compute an integer vector $v' = [a_B'\ a_L'\ a_s'\ a_j']$
such that executing B $a_B'$ times, {\splitter} $a_s'$ times, L
$a_L'$ times and {\joiner} $a_j'$ times will not change the amount
of data buffered in the {\splitjoin}. We do this by multiplying
$v''$ by $o_B * o_s * o_L$.

\begin{displaymath}
\begin{array}{rl}
a_B' = & u_j * o_s * o_L \\
a_L' = & u_j * u_B * w_{s,L} \\
a_j = & o_B * o_s * o_L \\
a_s = & u_j * u_B * o_L
\end{array}
\end{displaymath}

We now use $v'$ to compute $v = [a_B\ a_L\ a_s\ a_j]$, a minimal
steady state for the {\feedbackloop}.  We do this by finding an
integer $g$, the $\gcd$ of all elements in $v'$ and computing $v =
{v' \over g}$.

Finally, we construct $S_{fj}$ as follows:

\begin{displaymath}
S_{fj} = \left\{
\begin{array}{c}
a_B * S_{B,m} \circ a_L * S_{L,m} \circ [a_s \ a_j], \\
S_{B,N} \circ S_{L,N} \circ \{fl_s, fl_j\}, \\
\left[\begin{array}{c}
a_j * w_{j,I} \\
a_j * w_{j,I} \\
a_s * w_{s,O}
\end{array} \right], v
\end{array} \right\}
\end{displaymath}

Figure \ref{fig:steady-state}(c) depicts a sample {\feedbackloop}.
The following are the steady states of the {\splitjoin}'s children:
$$
\begin{array}{lrlr} S_B = & \left\{[1], \{B\}, { \left[
\begin{array}{c} 2 \\ 2 \\ 1
\end{array}
\right]}, [] \right\}, & S_L = & \left\{[1], \{L\}, { \left[
\begin{array}{c} 5 \\ 5 \\ 6
\end{array}
\right]}, [] \right\}
\end{array}
$$ We compute $v'$ for this {\feedbackloop}:

\begin{displaymath}
v' = \left[
\begin{array}{c}
5 * 3 * 5 \\
5 * 1 * 3 \\
5 * 1 * 5 \\
2 * 3 * 5
\end{array}\right] = \left[
\begin{array}{c}
75 \\
15 \\
25 \\
30
\end{array}\right]
\end{displaymath}

Thus $g = \gcd(75,15,25,30) = 5$ and

\begin{displaymath}
v = {1 \over 5} \left[
\begin{array}{c}
15 \\
3 \\
5 \\
6
\end{array}\right]
\end{displaymath}

Finally, we construct $S_{fl}$

\begin{displaymath}
S_{fl} = \left\{
\begin{array}{c}
15 * S_{B, m} \circ 3 * S_{L, m} \circ [5\ 6], \\
S_{B, N} \circ S_{L, N} \circ \{fl_s, fl_j\}, \\
\left[
\begin{array}{c}
6 * 2 \\ 6 * 2 \\ 5 * 3
\end{array}
\right], \left[
\begin{array}{c}
15 \\ 3 \\ 5 \\ 6
\end{array}\right]
\end{array} \right\}
\end{displaymath}

\subsection{Initialization for Peeking}
\label{sec:init-peeking}

Consider a {\filter} $f$, with peek amount of 2 and a pop amount of
1.  When a {\StreamIt} program is first run, there is no data
present on any of the {{\Channels}}.  This means that for the first
execution, filter $f$ requires that two data items be pushed onto
its {\Input} {{\Channel}}.  After the first execution of $f$, it will
have consumed one data item, and left at least one data item on
its {\Input} {{\Channel}}.  Thus in order to execute $f$ for the second
time, at most one extra data item needs to be pushed onto $f$'s
{\Input} {{\Channel}}.  The same situation persists for all subsequent
executions of $f$ - at most one additional data item is required
on $f$'s {\Input} {{\Channel}} in order to execute $f$.

This example illustrates that first execution of a {\filter} may
require special treatment.  Namely, the source for {\filter}'s data
may need to push more data onto {\filter}'s {\Input} {{\Channel}} for
{\filter}'s first execution.  Due to this condition, a {\StreamIt}
program may need to be initialized before it can enter steady
state execution.

There are other constraints (latency constraints) which may
require more complex initialization.  These will be discussed in
Chapter \ref{chpt:constrained}.

After an execution, a {\filter} $f$ must leave at least $e_f - o_f$
data on its {\Input} {{\Channel}}.  Thus, if the only constraints on
initialization are peek-related, it is a sufficient condition for
entering steady state schedule that $\forall f \in {\filters}, in_f
\ge e_f - o_f$.

Specific strategies for generating initialization schedules for
peeking will be presented in Chapter \ref{chpt:hierarchical} and
Chapter \ref{chpt:phased}.

\subsection{Schedules}
\label{sec:general:schedules}

 Once a program has been
initialized, it is ready to execute its steady state. In order to
do this, a steady state schedule needs to be computed. The steady
states computed above do not indicate the ordering of execution of
the nodes, only how many times the nodes need to be executed.

A schedule is an ordering of nodes in a {\StreamIt} streams. In
order to execute the schedule, we iterate through all of its nodes
in order of appearance and execute them one by one.  For example
in order to execute schedule $\{ABBCCBBBCC\}$ we would execute
node A once, then node B, node B again, C two times, B three times
and C twice again, in that order.

In order to shorten the above schedule we can run-length encode
it.  The schedule becomes $\{A \{2B\}\{2C\}\{3B\}\{3C\}\}$.

\subsection{Schedule Size vs. Buffer Size}
\label{sec:sched-vs-buffer}

\begin{figure}
\begin{center}

\psfig{figure=pipeline-buffers.eps,width=0.6in} \caption[4 {\filter}
{\pipeline}]{Sample 4 {\filter} {\pipeline}.  This {\pipeline} is
the same as one in Figure \ref{fig:steady-state} (a), except that
its children do not peek extra data} \label{fig:pipeline-buffers}
\end{center}
\end{figure}

When creating a schedule, two very important properties of it are
schedule size and amount of buffering required.  Schedule size
depends on encoding the schedule in an efficient way, while amount
of space required depends only on order of execution of nodes. The
two are related, however, because order of execution of {\filters}
affects how efficiently the schedule can be encoded.

For example, execution of {\filters} in {\pipeline} depicted in Figure
\ref{fig:pipeline-buffers} can be ordered in two simple ways, one
resulting in a large schedule but minimal amount of buffering, the
other resulting in a small schedule but a large amount of
buffering.

The steady schedule of the {\pipeline} in Figure
\ref{fig:pipeline-buffers} executes {\filter} $A$ 4 times, {\filter}
$B$ 6 times, {\filter} $C$ 9 times and {\filter} $D$ 3 times. Writing
out a schedule that requires minimal buffering results in schedule
$\{AB\{2C\}BCDAB\{2C\}ABCDB\{2C\}ABCD\}$.  This schedule requires
a buffer for 4 data items between {\filters} $A$ and $B$, 4 items
between $B$ and $C$ and 3 items between $C$ and $D$, resulting in
total buffers size 11, assuming data items in all buffers require
the same amount of space. The schedule itself has 18 entries.

To compare, writing the schedule in the most compact method we get
$$\{4A\}\{6B\}\{9C\}\{3D\}$$  This schedule requires a buffer for
12 data items between {\filters} $A$ and $B$, 18 items between $B$
and $C$, and 9 data items between $C$ and $D$, resulting in total
buffers size 39.  The schedule has 4 entries.

We can compare the storage efficiency of these two schedules by
assuming that one data item in a buffer requires $x$ amount of
memory and each entry in a schedule requires $y$ amount of memory.
Thus the two schedules will require the same amount of storage to
store themselves and execute if $11 x + 18 y = 39 x + 4 y$.

\begin{displaymath}
\begin{array}{rcl}
11 x + 18 y & = & 39 x + 4 y \\
14 y & = & 28 x \\
y & = & 2x
\end{array}
\end{displaymath}

Thus the smaller schedule is more efficient if every data item
requires less than twice the amount of storage than every entry in
the schedule.

One of the difficulties in scheduling {\StreamIt} programs lies in
finding a good set of trade-offs between schedule size and
buffering requirements.

\begin{comment}

\subsection{Minimum Buffer Size between {\filters}}

As illustrated above, the amount of buffering in a {\pipeline} can
be affected greatly by the order of executions of {\filters} in the
{\pipeline}.  The following equation calculates the minimal buffer
size required in order for two {\filters} to be able to push data
between each other indefinitely in the most buffer-efficient way.
Buffers this size cannot always be achieved, because some
components require that data be buffered up for execution (ex.
{\feedbackloops} require data to exist internally in order to
execution to advance) or because extra latency constrains require
additional buffering.

\begin{equation}
buffer_{A \to B} = \left\lceil {{peek_B} \over {\gcd(push_A,
pop_B)}} - 1 \right\rceil \gcd (push_A, pop_B) + push_A
\end{equation}

\emph{I can explain this equation, but I cannot prove it.  what
should I do with this?  it's not necessary for the thesis, but it
is a neat result we never published (PLDI submission), nor have I
seen it in any other papers (nobody does peeking, so it can't be
anywhere else)}

\end{comment}

\section{Pseudo Single Appearance Hierarchical Scheduling}
\label{chpt:hierarchical}

In this section we present Pseudo Single Appearance Hierarchical
Scheduling, a technique which is quite effective for scheduling
{\StreamIt} programs, but which cannot schedule all programs, and
which may require the buffers to be very large.

\begin{comment}
Section \ref{sec:hierarchical:motivation} provides some motivation
for hierarchical scheduling.
\end{comment}
Section \ref{sec:hierarchical-notation} presents the notation used
for hierarchical schedules. Section \ref{sec:sas} provides an
algorithm for computing pseudo single appearance hierarchical
schedules.

\begin{comment}
\subsection{Motivation}
\label{sec:hierarchical:motivation}

As has been explained in Section \ref{sec:sched-vs-buffer}, the
ordering of execution of nodes in a {\StreamIt} program can have a
significant effect on the amount of resources necessary to execute
the schedule.  The two important factors to consider when creating
the schedule is amount of buffering necessary to execute the
schedule, and the amount of space necessary to store the schedule.
The amount of buffering necessary is controlled by the ordering of
execution of nodes of the {\StreamIt} graph.  The amount of storage
necessary to store the schedule is controlled by the encoding of
the schedule.  As a general rule, ordering which minimizes the
buffering space requirements is fairly irregular and difficult to
encode efficiently.

\begin{figure}
\centering \psfig{figure=hierarchical-sample.eps,width=1in}
\caption[Stream for hierarchical scheduling]{A sample stream used
for hierarchical scheduling.} \label{fig:sample-sj}
\end{figure}

One technique used for encoding schedules is to form loop-nests of
sub-schedules and repeat them multiple times, until a steady-state
schedule is reached.  For example, the stream in Figure
\ref{fig:sample-sj} has a following steady state:

\begin{displaymath}
S_{s} = \left\{ \left[
\begin{array}{c}
9 \\ 6 \\ 18 \\ 18 \\ 4 \\ 4
\end{array} \right], \left[
\begin{array}{c}
A \\ C \\ D \\ split \\ join \\ B
\end{array}
\right], \left[
\begin{array}{c}
54 \\ 54 \\ 40
\end{array}
\right], \left[
\begin{array}{c}
9 \\ 2 \\ 4
\end{array}
\right]\right\}
\end{displaymath}

\noindent Thus one steady state schedule for this stream can be
$$\{9\{A\{2 split\}\{2D\}\}\}\{2\{\{3C\}\{2 split\}\{2B\}\}\}$$
Here, $\{A\{2 split\}\{2D\}\}$ and $\{\{3C\}\{2 split\}\{2B\}\}$
are the inner nests, executed 9 and 2 times respectively.

If, the overall schedule has every {\StreamIt} node appear only
once (as in the example above), the technique is called Single
Appearance Scheduling \cite{bhattacharyya94looped}. One of
difficulties in using Single Appearance Scheduling is finding a
good way to form loop-nests for the sub-schedules, because the
buffering requirements can grow quite large.  An example of this
has been presented in Section \ref{sec:sched-vs-buffer}.

{\StreamIt} provides the scheduler with a pre-existing
hierarchical structure. While it is possible to use techniques
developed for Single Appearance Scheduling to create valid
schedules for {\StreamIt} programs, Single Appearance Scheduling
does not satisfy all requirements of an effective {\StreamIt}
scheduler. This is because some {{\feedbackloops}} cannot be
scheduled using Single Appearance Scheduling techniques. This
difficulty arises because the amount of data provided to the
{{\feedbackloop}} by the $delay_{fl}$ variable is not sufficient to
perform a complete steady-state execution of the loop, thus
preventing the schedule for the {{\feedbackloop}} to be encoded with
only a single appearance of every node in the schedule.

The solution to this problem is to have the same node appear
multiple times in the schedule.  While this solves the problem of
inability to schedule some {{\feedbackloops}}, it introduces another
problem: which nodes should appear several times, and how many
times should they be executed on each appearance.  The solution
proposed here goes half-way to solve the problem. A more effective
solution will be proposed in Chapter \ref{chpt:phased}.

In hierarchical scheduling we use the pre-existing structure
(hierarchy) to determine the nodes that belong in every loop-nest.
Basically, every stream receives its own loop-nest, and treats
steady-state execution of its children as atomic (even if those
children are streams whose executions can be broken down into more
fine-grained steps). In the example above, the {\pipeline} has a
{\splitjoin} child.  The {\splitjoin} is responsible for
scheduling its children (nodes $C$, $B$, $split$ and $join$).  The
{\pipeline} will use the {\splitjoin}'s schedule to create its
own steady state schedule. Here the {\splitjoin}'s schedule can be
$T_{sj} = \{\{9 split\}\{3C\}\{9D\}\{2 join\}\}$, thus making the
{\pipeline}'s schedule \footnote{Notation for this schedule is
explained in next section (Section
\ref{sec:hierarchical-notation}).} $$T_{pipe} = \{\{9A\}\{2
T_{sj}\}\{4B\}\} = \{\{9A\}\{2\{\{9 split\}\{3C\}\{9D\}\{2
join\}\}\}\{4B\}\}$$


The problem of inability to schedule some {{\feedbackloops}} is
alleviated by allowing {{\feedbackloop}} to interleave the execution
of its children (the body, the loop, and the {\splitter} and
{\joiner}).  This results in {{\feedbackloop}} containing multiple
appearances of its children. All other streams use their
children's schedules in their schedules only once. This technique
is called Pseudo Single Appearance Scheduling, since it results in
schedules that are very similar to proper single appearance
schedules. While it does not allow scheduling of all
{{\feedbackloops}} (a {{\feedbackloop}} may have a child which
requires more data for steady state execution then made available
by the $delay_{fl}$ variable) it has been found to be very
effective, and only one application has been found which cannot be
scheduled using this technique.
\end{comment}

\subsection{Notation}
\label{sec:hierarchical-notation}

\begin{comment}
The notation in the above example, is very similar to that
presented in Section \ref{sec:sched-vs-buffer}.  A number in front
of a node represents that the node is meant to be executed a
certain number of times.  The one big difference is that
$\{2T_{sj}\}$ means that the schedule for the {\pipeline} is meant
to be executed twice.  Since $T_{sj} = \{\{9
split\}\{3C\}\{9D\}\{2 join\}\}$, $\{2T_{sj}\}$ is same as
$\{2\{\{9 split\}\{3C\}\{9D\}\{2 join\}\}\}$.

This means that to execute $T_{pipe}$, node $A$ is executed 9
times, schedule $T_{sj}$ is executed twice and node $B$ is
executed twice, in that order. To execute $T_{sj}$, the {\splitter}
is executed 9 times, node $C$ is executed 3 times, node $D$ 9
times and the {\joiner} twice.

Thus, writing the schedule of $T_{pipe}$ into a flat schedule (one
with no loop-nests) results in schedule $\{9A\}\{9
split\}\{3C\}\{9D\}\{2 join\}\{9 split\}\{3C\}\{9D\}\{2
join\}\{4B\}$.

In other words, $T_{sj}$ is a loop-nest, which can be executed
multiple times. When storing a schedule, $T_{sj}$ is stored only
once, and every use of $T_{sj}$ becomes the reference to the
actual schedule.
\end{comment}

A hierarchical schedule is a schedule which uses the hierarchy of
the program to create a schedule. Each hierarchical component in
the program has a corresponding schedule. This schedule only
consists of executions of steady state schedules of the
component's direct children. For example, if a {\splitjoin} has a
{\pipeline} child, the {\pipeline} will have a steady schedule $T_p$,
while the {\splitjoin} will have a steady schedule $T_{sj}$, which
will include $T_p$.

A hierarchical steady schedule for a stream $s$ will be denoted by
$T_s$, while an initialization schedule for a stream $s$ will be
denoted $I_s$. A {\splitter} of a {\splitjoin} $sj$ or a
{{\feedbackloop}} $fl$ will be denoted as $split_{sj}$ or
$split_{fl}$, while the {\joiner} will be denoted as $join_sj$ or
$join_{fl}$.

This section will continue using the notation for $e$, $o$ and $u$
extended to streams.
\begin{comment}
That is, for a stream $s$,
$e_s$ will represent the amount of data needed by $s$ on its
{\Input} {{\Channel}} in order to execute its minimal steady state
schedule; $o_s$ represents the amount of data consumed by from its
{\Input} {{\Channel}} $s$ during execution of its steady state
schedule; and $u_s$ represents the amount of data pushed by $s$
onto its {\Output} {{\Channel}}.
\end{comment}
The notation for $e$, $o$ and $u$ will also be extended to
initialization schedules.  Namely, $e^i_s$ represents the amount
of data required by stream $s$ on its {\Input} {{\Channel}} in
order to execute the initialization schedule for $s$; $o^i_s$
represents the amount of data consumed by $s$ from its {\Input}
{{\Channel}} during its initialization schedule; and $u^i_s$
denotes the amount of data pushed by $s$ onto its {\Output}
{{\Channel}} during execution of its initialization schedule.  The
initialization schedules are created in such a way, that after all
streams have executed their initialization schedules, the program
is ready to enter its steady state execution.

\begin{comment}
Note, that it is possible that a stream $s$ has $u^i_s \ne
0$. An example of this will be presented in Section
\ref{sec:sas-fl}.
\end{comment}

A hierarchical schedule for a stream $s$ is denoted as $$H_s =
\left\{T_s, I_s, c_s = \left[\begin{array}{c}
e_s\\o_s\\u_s\end{array}\right], c^i_s =
\left[\begin{array}{c}e^i_s\\o^i_s\\u^i_s\end{array}\right]\right\}$$

\subsection{Pseudo Single-Appearance Hierarchical Scheduling}
\label{sec:sas}

\begin{comment}
\begin{figure}
\begin{minipage}{1.5in}
\centering \psfig{figure=pipeline-steady-state.eps,width=0.6in} \\
{\protect\small (a) A sample {\pipeline}}
\end{minipage}
~
\begin{minipage}{1.5in}
\centering \psfig{figure=splitjoin-steady-state.eps,width=1.2in} \\
{\protect\small (b) A sample {\splitjoin}}
\end{minipage}
~
\begin{minipage}{2.5in}
\centering \psfig{figure=feedback-hierarchical.eps,width=1.0in} \\
{\protect\small (c) A sample {{\feedbackloop}}.\\ $delay_{fl} = 15$ \\
The $L$ {\filter} has been flipped upside-down for clarity. \\$e_L
= 9, o_L = 5, u_L = 6$ }
\end{minipage}
\caption{Sample {\StreamIt} streams used for Pseudo
Single-Appearance Hierarchical Scheduling}
\label{fig:hierarchical-schedule}
\end{figure}
\end{comment}

A single appearance schedule is a schedule in which every node
appears exactly once. The advantage of single appearance schedules
is that they are very small. The disadvantage of single appearance
schedules is that they can require large amount of buffering
between {\filters}. Not all valid {\StreamIt} programs can be
scheduled using single appearance schedules. This is because a
{\feedbackloop} may provide not enough data in its feedback path.
Pseudo single appearance scheduling allows a large number of
applications to be scheduled, while keeping the size of the
schedule small. There still exist applications which cannot be
scheduled.

Due to space constraints, algorithms presented below are not
detailed. Refer to \cite{karczma-thesis} for details.

\begin{comment}
This section will develop hierarchical scheduling techniques to
create initialization and steady state schedules. A simple
implementation of the hierarchical scheduling creates a
single-appearance schedule.  While single-appearance scheduling is
quite effective in scheduling {\StreamIt} programs, it is also easy
to construct programs that have {{\feedbackloops}} that are impossible
to schedule.  To alleviate the problem, the single-appearance
scheduling was slightly modified to allow {{\feedbackloops}} to
schedule programs using hierarchical push scheduling.  This does
not solve the problem altogether (some {{\feedbackloops}} are still
impossible to schedule using this technique), but this technique
is able to schedule many programs which cannot be scheduled with a
simple single-appearance scheduler.

Sample streams for techniques described here are taken from Figure
\ref{fig:hierarchical-schedule}.  The streams in Figure
\ref{fig:hierarchical-schedule} are identical to those in Figure
\ref{fig:steady-state} with exception of the {{\feedbackloop}}.
\end{comment}

\subsubsection{\filter}

An execution of a {\filter} is an atomic operation.  Thus a steady
state schedule for a {\filter} $f$ is simply $T_f = (f)$.

A {\filter} has no internal buffering.  Thus there is no need to
initialize a {\filter} for its steady state.  {\filters} may,
however, peek data.  That means that in order to enter a steady
state, sufficient amount of data must be pushed onto {\filter}'s
{\Input} {{\Channel}}.  Thus, for a {\filter} $f$, $e^i_f = e_f -
o_f$. Thus we have $$H_f = \left\{\{f\}, \{\}, \left[
\begin{array}{c}
e_f\\o_f\\u_f
\end{array}\right], \left[
\begin{array}{c}
e_f-o_f\\0\\0\end{array}\right] \right\}
$$

\subsubsection{{\pipelines} and {\splitjoins}}

{\large \bf put some math here}

Scheduling {\pipelines} and {\splitjoin} consists of computing a
single appearance schedule for the {\pipeline} or {\splitjoin}.
This means that the resulting schedule contains each child of the
{\pipeline} or {\splitjoin} exactly once.

\subsubsubsection{Initialization Schedule} In order for a
{\pipeline} or {\splitjoin} to be initialized, all their children
must have executed their own initialization schedules.

The initialization schedule for a {\pipeline} is calculated as
follows. For every child stream of the {\pipeline}, the amount of
data necessary to initialize all the streams below it is
calculated. For $k$th stream, that amount is denoted $init_k$. If
the {\pipeline} has $n$ children, then for the bottom-most child,
$p_{n-1}$, that amount is $init_{n-1} = e^i_{p_{n-1}}$. The data
to the $k$th child is provided by the $k-1$ child, during its
initialization and subsequent execution of its steady state
schedule. Thus the $k-1$ child must execute its steady state
schedule $l_{k-1} = \left\lceil init_k - u^i_{p_{k-1}} \over
u_{p_{k-1}} \right\rceil$ times.  The amount of data required for
initialization of the {\pipeline} by the $k-1$ child is
$init_{k-1} = e^i_{p_{k-1}} + l_{k-1} * o_{p_{k-1}}$.

Now, the initialization schedule is simply constructed by
iterating over all children of the {\pipeline}, from top to
bottom, and concatenating all initialization and appropriate
steady state schedules.

Constructing an initialization schedule for a {\splitjoin} is
similar to the {\pipeline}. Every stream child will execute only
their initialization schedule. The {\splitter} will execute as
many times as necessary to provide enough data for all the
children to initialize. The {\joiner} will not execute. The
schedule is created by concatenating executions of the
{\splitter}, followed by initialization schedules of all the
children.

\begin{comment}
In order to create an initialization schedule of a
{\pipeline}, all of {\pipeline}'s children's initialization
schedules must be executed. Every child must execute its
initialization schedule before it can execute its steady-state
schedule.  Some children may require some data in order to execute
their initialization schedules. The upstream children provide this
data to them by first executing their own initialization schedule,
and then their steady-state schedule. Thus, in the final form, the
execution of a {\pipeline}'s initialization schedule first
executes the initialization schedule of the top-most child, then
executes the steady-state schedule this child several times, then
the initialization schedule of the second-from-the-top child,
followed by executing this child's steady-state schedule several,
etc, until the bottom-most child is reached.  Since the
bottom-most child does not need to provide any data {\pipeline}'s
downstream children (there aren't any), the bottom-most child only
executes its initialization schedule.

The initialization schedule is calculated as follows. At every
stream of the {\pipeline}, the amount of data necessary to
initialize all the streams below is calculated. For $k$th stream,
that amount is denoted $init_k$. If the {\pipeline} has $n$
children, then for the bottom-most child, $p_{n-1}$, that amount
is $init_{n-1} = e^i_{p_{n-1}}$. The data to the $k$th child is
provided by the $k-1$ child, during its initialization and
subsequent execution of its steady state schedule.  The
initialization provides $u^i_{p_{k-1}}$ data items. Thus the $k-1$
child must execute its steady state schedule $l_{k-1} =
\left\lceil init_k - u^i_{p_{k-1}} \over u_{p_{k-1}} \right\rceil$
times.  The amount of data required for initialization of the
{\pipeline} by the $k-1$ child is $init_{k-1} = e^i_{p_{k-1}} +
l_{k-1} * o_{p_{k-1}}$.

This calculation is performed for all children of the {\pipeline},
starting at the last (bottom-most) child, and moving up.  For the
sample {\pipeline} in Figure \ref{fig:hierarchical-schedule}(a), the
values computed are:

\begin{displaymath}
\begin{array}{lr}
\begin{array}{rl}
l_3 & = 0 \raisebox{-0.2in}{ } \\
l_2 & = \left\lceil init_3 - u^i_C \over u_C \right\rceil = \left\lceil 2 - 0 \over 1 \right\rceil = 2 \raisebox{-0.2in}{ } \\
l_1 & = \left\lceil init_2 - u^i_B \over u_B \right\rceil =
\left\lceil 4 - 0 \over 3 \right\rceil = 1 \raisebox{-0.2in}{ } \\
l_0 & = \left\lceil init_1 - u^i_A \over u_A \right\rceil =
\left\lceil 4 - 0 \over 3 \right\rceil = 2 \raisebox{-0.2in}{ } \\
\end{array} &
\begin{array}{rl}
init_3 & = e^i_D + l_3 * o_D = 2 + 0 * 1 = 2 \raisebox{-0.2in}{ }\\
init_2 & = e^i_C + l_2 * o_C = 0 + 2 * 2 = 4 \raisebox{-0.2in}{ }\\
init_1 & = e^i_B + l_1 * o_B = 1 + 1 * 3 = 4 \raisebox{-0.2in}{ }\\
init_0 & = e^i_A + l_0 * o_A = 0 + 2 * 1 = 2 \raisebox{-0.2in}{ }\\
\end{array}
\end{array}
\end{displaymath}

Now, the initialization schedule is simply constructed by
iterating over all children of the {\pipeline}, from top to
bottom, and concatenating all initialization and appropriate
steady state schedules.  Thus $I_p = \{I_A\{2T_A\} I_{B}T_B
I_C\{2T_C\} I_D\}$.

Finally, we need to compute the amount of data peeked, popped and
pushed by the {\pipeline} during its initialization.

The amount of data popped is simply the amount of data popped by
the top-most child when executing the {\pipeline}'s initialization
schedule, that is the amount of data popped by the first child
during its own initialization plus the amount of data popped
during its steady-state execution times number of steady state
executions. That is $o^i_p = o^i_{p_0} + l_0 * o_{p_0}$.

Similarly, the amount of data pushed by the {\pipeline} is simply
the amount of data pushed by the bottom-most child during its
initialization. Remember that the bottom-most child never executes
its steady-state schedule.  Thus $u^i_p = u^i_{p_{n-1}}$.

Computing the amount of data peeked by the {\pipeline} during
initialization may be a little more complicated, because unlike
popping and pushing, peeking is not accumulative. Luckily, we can
rely on our knowledge of structure of the {\StreamIt} graph to
calculate the amount of data peeked by a {\pipeline}. We know that a
{\pipeline} is a single-input structure. We also know that this
single input will lead directly into a {\StreamIt} node.  There are
only three possibilities for what this node will be.

\begin{itemize}
\item If {\pipeline}'s first node is a {\filter} $f$ (the first child
of the {\pipeline} is a {\filter} or a {\pipeline} with a {\filter} as its
first node) then the extra amount of data peeked by the {\pipeline}
on initialization will be $e^i_f - o^i_f$. If the first child is a
{\filter}, then $p_0$ is $f$ and the extra amount peeked is also
$e^i_{p_0} - o^i_{p_0}$.  If the first child is a {\pipeline} with a
{\filter} first node, we can show by induction that this {\pipeline}'s
extra peek amount will also be $e^i_{p_0} - o^i_{p_0}$.

\item If {\pipeline}'s first node is a {\splitter} (the first child of
the {\pipeline} is a {\splitjoin} or a {\pipeline} with a {\splitter} as
its first node) then the extra amount of data peeked by the
{\pipeline} on initialization will be 0, because {\splitters} never
peek. Furthermore, for the same reason, the amount of extra data
peeked by the first child on its initialization will also be zero,
or $e^i_{p_0} - o^i_{p_0}= 0$.

\item If {\pipeline}'s first node is a {\joiner} (the first child of
the {\pipeline} is a {{\feedbackloop}} or a {\pipeline} with a {\joiner} as
its first node) then the amount of extra data peeked by the
{\pipeline} on initialization will be 0, for the same reasons as
above. And again $e^i_{p_0} - o^i_{p_0}= 0$.
\end{itemize}

Thus on initialization, the {\pipeline} will have an extra peek
amount of $e^i_{p_0} - o^i_{p_0}$, and the total amount of data
peeked by the {\pipeline} for initialization is $e^i_p = (e^i_{p_0}
- o^i_{p_0}) + l_0
* o_{p_0}$.
\end{comment}

\subsubsubsection{Steady State Schedule}
\begin{comment}
The steady state state schedule is calculated as a
single-appearance schedule.
\end{comment}
Calculation of a single-appearance schedule for a {\pipeline} starts
with computing $S_p$, the steady state for the {\pipeline}. The
steady state schedule simply executes every child $p_i$ of the
{\pipeline} $S_{p,v,i}$ times. The topmost child is executed first,
then the second child, and so on.

Similarly, calculation of a single-appearance schedule for a
{\splitjoin} starts with computing $S_{sj}$. The children are
executed the appropriate number of times, starting with the
{\splitter}, then all the stream children, and finally the {\joiner}.

The consumption and production of data for the steady state
schedule is already calculated by the steady state, and is
$S_{p,c}$ or $S_{sj,c}$.

For our examples in Figure \ref{fig:steady-state}(a) and (b) we
have the following steady state schedules:

\begin{displaymath}
\begin{array}{lr}
H_p = \left\{\begin{array}{l}\left\{\begin{array}{c}\{4T_A\}\\
\{6T_B\} \\ \{9T_C\}\\ \{3T_D\}\end{array}\right\}, \left\{\begin{array}{c}I_A\\
I_B\\ I_C\\ I_D\end{array}\right\},\end{array} \left[
\begin{array}{c}
4\\4\\3
\end{array}\right], \left[
\begin{array}{c}
0\\0\\0
\end{array}\right] \right\}, &
H_{sj} = \left\{\left\{\begin{array}{c}\{2\ split\}\\ \{2T_A\} T_B
\\ \{2\ join\}\end{array}\right\}, \left\{\begin{array}{c}split\\I_A\\I_B\end{array}\right\},
\left[
\begin{array}{c}
6\\6\\8
\end{array}\right], \left[
\begin{array}{c}
3\\3\\0
\end{array}\right] \right\}
\end{array}
\end{displaymath}

\begin{comment}
\subsubsection{\splitjoins}

Creating a schedule for a {\splitjoin} is essentially identical to
scheduling a {\pipeline}.  The initialization schedule only needs to
compute how many times the {\splitter} needs to be executed, and
construct the actual schedule.  The steady state schedule is
constructed by concatenating steady state schedule of {\splitjoin}'s
children, the {\splitter} and {\joiner}.

For our example in Figure \ref{fig:hierarchical-schedule}(b), the
steady state is

\begin{displaymath}
S_{sj} = \left\{ \left[
\begin{array}{c} 2 \\ 1 \\ 2 \\ 2 \end{array}\right], \left\{
\begin{array}{c} A \\ B \\ {\splitter} \\ {\joiner} \end{array} \right\},
\left[ \begin{array}{c} 6 \\ 6 \\ 8
\end{array} \right], \left[
\begin{array}{c}
2 \\ 1 \\ 2 \\ 2
\end{array}\right]\right\}
\end{displaymath}

\subsubsubsection{Initialization} In order to initialize a
{\splitjoin}, all its children must execute their initialization
schedules.  The only requirement for executing those schedules is
that they have been provided with sufficient data on their
{\Input} {{\Channels}}.  Since the {\splitter} provides data for
all the children of a {\splitjoin}, it is the only element of a
{\splitjoin} that must execute its steady state schedule.

For $k$th child of a {\splitjoin}, the {\splitter} must provide
$e^i_{sj_k}$ data items.  One execution of the {\splitter} causes
it to push $w_{s,k}$ data items toward the $k$th child.  Thus the
{\splitter} must execute at least $l_k = \left\lceil e^i_{sj_k}
\over w_{s,k} \right\rceil$ times.  In order to find out how many
times the {\splitter} needs to execute to initialize all children,
$l_s$, we simply find the maximum $l_k$. Thus $l_s = {\max \atop
k}(l_k)$.

In the sample {\splitjoin} from Figure
\ref{fig:hierarchical-schedule}(b), we get following $l_k$s:

\begin{displaymath}
\begin{array}{rl}
l_0 = & \left\lceil e^i_A \over w_{s,0} \right\rceil = \left\lceil
0 \over 2 \right\rceil = 0 \raisebox{-0.2in}{ }\\
l_1 = & \left\lceil e^i_B \over w_{s,1} \right\rceil = \left\lceil
1 \over 1 \right\rceil = 1 \raisebox{-0.2in}{ }\\
\end{array}
\end{displaymath}

The maximum $l_k$s is 1, thus $l_s = 1$, the {\splitter} must be
executed once for initialization.

The initialization schedule is constructed by concatenating an
appropriate number of executions of the {\splitter} and
initialization schedules of all the children.  Thus in our
example, $I_{sj} = \{split\ I_A\ I_B\}$.

The consumption of an initialization schedule of a  {\splitjoin} is
computed as follows:  $e^i_{sj} = u^i_{sj} = l_s * o_{sj_s}$ and
$u^i_{sj} = 0$. The peeking and popping amounts are simply the
amount of data popped by the {\splitter} for every one of its
executions times the number of times it is executed.  The {\joiner}
is never executed, thus the push amount is 0.

Thus for our example, $e^i_{sj} = u^i_{sj} = 1 * 3 = 3$ and
$u^i_{s} = 0$.

\subsubsubsection{Steady State} Similarly to the algorithm for
{\pipeline}, the steady state is constructed by using $S_{sj,v}$
to concatenate the executions of the {\splitter}, all children of
the {\splitjoin} and the {\joiner} together.

For our example, the steady state schedule is simply $$T_{sj} =
\{\{2\ split\}\{2T_A\}T_B\{2\ join\}\}$$

The consumption vector, $c_{sj}$ is the same as $S_{sj,c}$.

Thus the hierarchical schedule for the {\splitjoin} in Figure
\ref{fig:hierarchical-schedule}(b) is

\begin{displaymath}
H_{sj} = \left\{\{\{2\ split\}\{2T_A\}T_B\{2\ join\}\},\{split\
I_A\ I_B\}, \left[
\begin{array}{c}
6\\6\\8
\end{array}\right], \left[
\begin{array}{c}
3\\3\\0
\end{array}\right] \right\}
\end{displaymath}
\end{comment}

\subsubsection{{\feedbackloops}}
\label{sec:sas-fl}

\subsubsubsection{Initialization Schedule} Initialization of a
{\feedbackloop} is similar to initializing a {\pipeline}. First
the children are iterated over to find out how many times they
need to execute to initialize the {\feedbackloop}, and then they
are iterated in reverse and their executions concatenated
appropriately. The order of original traversal is loop child,
{\splitter}, body child and {\joiner}. The loop child only
executes its initialization schedule, while other children may
need to execute their steady state schedules.

It is possible that the {\joiner} will require more than
$delay_{fl}$ data from its second input (the feedback path). If
this is the case, then this algorithm cannot schedule such a
{\feedbackloop}. This does not mean that the loop is not
schedulable, only that it cannot be scheduled using pseudo single
appearance scheduling.

\begin{comment}
Scheduling of {{\feedbackloops}} is a task that can be made
difficult, if the amount of data provided for the {{\feedbackloop}}
by the $delay_{fl}$ value is low. Before a {\StreamIt} program
begins executing, the {{\feedbackloop}} needs to be provided with
some data in one of the internal {{\Channels}}. Without this data,
the {\splitter} and the {\joiner} of the {{\feedbackloop}} will not
be able to execute, because they will never have sufficient data
on their input {{\Channels}}. This is a consequence of the
{{\feedbackloop}} having a cyclical structure.

\begin{figure}
\centering
\psfig{figure=feedback-non-scheduleable.eps,width=1.2in}
\caption[Example of non-shedulable {{\feedbackloop}}]{Sample
{{\feedbackloop}}. If this {{\feedbackloop}} has a $delay_{fl}$ value
set to 7, it does not have a steady state schedule which will
allow it to execute forever. If the $dealy_{fl}$ value is
increased by 1 to 8, the {{\feedbackloop}} has a steady state
schedule of $\{join\{2B\}\{5 split\}L\ join\{2B\}\{5 split\}L\ $
$join\{2B\}\{5 split\}L\{2\ join\}\{4B\}\{10 split\}\{2L\}\ \}$.}
\label{fig:feedback-non-schedulable}
\end{figure}

The difficulty in scheduling {{\feedbackloops}} is that if the amount
of data made available to the {{\feedbackloop}} by the $delay_{fl}$
value (as explained in Section \ref{sec:explain-fl}) is small,
there will be very limited number of ways to execute the
{{\feedbackloop}}.  In fact, it is possible that the amount of data
available to the {{\feedbackloop}} is so small, it cannot reach and
complete an execution of a steady state schedule. An example of
such {{\feedbackloop}} is presented in Figure
\ref{fig:feedback-non-schedulable}.

Here we will use {{\feedbackloop}} from Figure
\ref{fig:hierarchical-schedule}(c).  The steady state schedule for
this {{\feedbackloop}} is

\begin{displaymath} S_{fl} = \left\{
\begin{array}{c} \left[
\begin{array}{c}
15 \\ 3 \\ 5 \\ 6 \end{array}\right], \left\{
\begin{array}{c} B \\ L \\ {\splitter} \\ {\joiner} \end{array}\right\}, \left[
\begin{array}{c}
12 \\ 12 \\ 15
\end{array}
\right], \left[
\begin{array}{c}
15 \\ 3 \\ 5 \\ 6
\end{array}\right]
\end{array} \right\}
\end{displaymath}

\subsubsubsection{Initialization Schedule} Initialization for the
{{\feedbackloop}} is calculated in a similar way to initialization
of a {\pipeline}.  The number of steady state executions of the
children of the {{\feedbackloop}} is denoted $l_B$ and $l_L$ for
the body child and the loop child, respectively. The number of
executions of the {\splitter} is denoted $l_s$ and the {\joiner}
is denoted $l_j$.

Since the initial data is inserted into the buffer between the
loop child and the {\joiner} (as explained in \ref{sec:explain-fl}),
it follows that the loop child should initialize last - it will be
the last one receive data to initialize. Since the computation of
the initialization schedule is similar to the way it was done for
{\pipeline}, we will start with the child which is executed last,
namely the loop child. Similarly as with {\pipeline}, the which is
initialized last does not execute its steady state schedule for
initialization, thus we set $l_L = 0$. The {\splitter} must provide
the loop child with just enough data to initialize, the body child
must provide the {\splitter} with just enough data for the {\splitter}
to pass enough data to the loop child, etc. Thus,

\begin{displaymath}
\begin{array}{rl}
l_L & = 0 \\
l_{s} & = \left\lceil {o^i_{fl_L} \over w_{s,1}} \right\rceil \raisebox{-0.2in}{ } \\
l_{B} & = \left\lceil o_{s} * l_{s} - u^i_{fl_B} \over u_{fl_B} \right\rceil \raisebox{-0.2in}{ } \\
l_{j} & = \left\lceil o^i_{fl_B} + l_B * o_{fl_B} \over u_{j}
\right\rceil \raisebox{-0.2in}{ }
\end{array}
\end{displaymath}

This initialization schedule will only be valid if there is enough
data provided between the loop child and the {\joiner}, or
$delay_{fl} \ge l_j * w_{j,1}$.  If this condition does not hold,
the {{\feedbackloop}} cannot be scheduled using pseudo
single-appearance algorithm.

Referring to the example Figure
\ref{fig:hierarchical-schedule}(c), we obtain the following values
for $n$s:

\begin{displaymath}
\begin{array}{rl}
l_{s} & = \left\lceil {4 \over 3} \right\rceil = 2 \raisebox{-0.2in}{ } \\
l_{B} & = \left\lceil 2 * 3 - 0 \over 1 \right\rceil = 6 \raisebox{-0.2in}{ } \\
l_{j} & = \left\lceil 0 + 6 * 2 \over 5 \right\rceil = 3 \raisebox{-0.2in}{ } \\
\end{array}
\end{displaymath}

Furthermore, since $delay_{fl} = 15$, we have $15 \ge 3 * 3$, thus
a valid initialization schedule can be constructed.

The initialization schedule is constructed by concatenating
executions of the {\joiner}, body child, {\splitter} and the loop
child.  The body child will execute both its initialization
schedule as well as its steady state schedule, while the loop
child will only execute its initialization schedule.

Thus for our example we get $I_{fl} = \{\{3\ join\} I_B \{6T_B\}
\{2\ split\} I_L\}$.

We now compute the consumption of data for the initialization
schedule of the {{\feedbackloop}}: $e^i_{fl} = o^i_{fl} = n_j *
w_{j,0}$ and $u^i_{fl} = n_s * w_{s,0}$. Similarly as in
computation for the {\splitjoin}, these values are simply the
production and consumption of the {\splitter} and {\joiner} from their
appropriate {\Input} and {\Output} channels multiplied by the number
of times the {\splitter} and {\joiner} are executed during
initialization schedule.

In our example, $e^i_{fl} = o^i_{fl} = 3 * 2 = 6$ and $u^i_{fl} =
3 * 3 = 9$.  Note that the {{\feedbackloop}} pushes data out during
its initialization.

Finally, we compute the amount of data present in {{\Channels}} after
initialization. These amounts are important because they will be
used to compute the steady state schedule of the {{\feedbackloop}}.
These amounts were not necessary for computation of steady state
schedules of {\pipeline} and {\splitjoin}. These amounts are
calculated by simply subtracting the amount of data popped from a
{{\Channel}} from amount of data pushed into a {{\Channel}}. Here we
adopted the notation for
{\Input} and {\Output} {{\Channel}} from Section \ref{sec:exec-model}.

\begin{displaymath}
\begin{array}{rl}
in^i_B = & l_j * u_j - l_B * o_{fl_B} \\
out^i_B = & u^i_{fl_B} + l_B * u_{fl_B} - l_s * o_s\\
in^i_L = & l_s * w_{s, 1} - l_L * o_{fl_L} \\
out^i_L = & delay_{fl} + u^i_{fl_L} + l_L * u_{fl_L} - l_j * w_{j,1} \\
\end{array}
\end{displaymath}
\end{comment}

\subsubsubsection{Steady State Schedule} Computing the steady
state schedule for a {{\feedbackloop}} is more complicated than
for the other streams.  The reason for this is that
{{\feedbackloops}} may require a non single-appearance schedule
due to small $delay_{fl}$ value, while other {\StreamIt} construct
can always be scheduled using single-appearance schedules.

\begin{comment}
The algorithm used for creating of a steady state schedule
will work in several phases.  The amount of data present in
{{\Channels}} between the children of the {{\feedbackloop}}, the
{\joiner} and the {\splitter} is kept track of to determine which
element is allowed to execute.
\end{comment}

The algorithm for creating a steady state schedule of a
{{\feedbackloop}} iterates over the elements of the {{\feedbackloop}}
in order of ({\joiner}, body child, {\splitter}, loop child). Each
element is executed as many times as possible, considering the
amount of data required and available to execute the element. Each
execution of an element is appended to the steady state schedule.

This iteration is repeated until either all elements have executed
their steady state number of times, or until a complete iteration
has been performed with no element being able to execute. The
first case indicates a successful completion of the algorithm. The
second case indicates a failure - the algorithm is unable to
schedule the {{\feedbackloop}}.

Table \ref{tab:sas-fl} illustrates the execution of this algorithm
for {{\feedbackloop}} from Figure \ref{fig:steady-state}. In the
table, the first row and the last row have the same amount of data
buffered in {{\Channels}}, thus indicating that a full steady
state schedule has indeed been computed.
\begin{comment}
Furthermore, the last entry considering execution of $B$ has
sufficient data to execute $B$ 5 times, but only executes it 4
times to ensure that a steady state schedule is constructed.
\end{comment}

\begin{table} \centering \scriptsize
\begin{tabular}{|l|c|c|c|c|c|c|c|c|c|}
\hline \multicolumn{4}{|c|}{data items in buffer} & \multicolumn{4}{c|}{executions left} & \parbox{0.5in}{element considered} & executions \\
\cline{1-8} $in_B$ & $out_B$ & $in_L$ & $out_L$ & $split$ & B & $join$ & L & & \\
\hline  1   &   6   &   6   &   0   &   5   &   9   &   4   &   3   &   $split$ &   2   \\
\hline  1   &   0   &   12  &   0   &   3   &   9   &   4   &   3   &   $L$ &   1   \\
\hline  1   &   0   &   7   &   6   &   3   &   9   &   4   &   2   &   $join$  &   2   \\
\hline  11  &   0   &   7   &   0   &   3   &   9   &   2   &   2   &   $B$ &   5   \\
\hline  1   &   5   &   7   &   0   &   3   &   4   &   2   &   2   &   $split$ &   1   \\
\hline  1   &   2   &   10  &   0   &   2   &   4   &   2   &   2   &   $L$ &   1   \\
\hline  1   &   2   &   5   &   6   &   2   &   4   &   2   &   1   &   $join$  &   2   \\
\hline  11  &   2   &   5   &   0   &   2   &   4   &   2   &   1   &   $B$ &   4   \\
\hline  3   &   6   &   5   &   0   &   2   &   0   &   2   &   1   &   $split$ &   2   \\
\hline  3   &   0   &   11  &   0   &   0   &   0   &   0   &   1   &   $L$ &   1   \\
\hline  3   &   0   &   6   &   6   &   0   &   0   &   0   &   0   &       &       \\
\hline
\end{tabular}
\caption[Execution of Steady-State algorithm on sample
{{\feedbackloop}}]{Trace of execution of steady-state algorithm on
sample {{\feedbackloop}} from Figure \ref{fig:steady-state}(c). The
executions left amount is the number of executions left for a
particular child to complete a steady state execution of the
{{\feedbackloop}}. Onece this value reaches 0, the element is not
executed anymore, even if it has data to execute.}
\label{tab:sas-fl}
\end{table}

\begin{comment}
The schedule resulting from the above computation is $$T_{fl}
= \{\{2\ join\} \{6T_B\} \{2\ split\} T_L \{2\ join\} \{5T_B\}\
split\ T_L \{2\ join\} \{4T_B\} \{2\ split\} T_L\}$$ This schedule
is obtained by going through Table \ref{tab:sas-fl} from top to
bottom and concatenating the appropriate number of executions of
every child of the {{\feedbackloop}}, as listed in the "executions"
column.

The steady state consumption $c_{fl}$ is again simply $S_{fl, c}$.
\end{comment}

Thus the hierarchical schedule for the {\feedbackloop} is:

\begin{displaymath}
H_{fl} = \left\{
\begin{array}{l}
\begin{array}{c}\{\{2\ join\} \{6T_B\} \{2\ split\} T_L \{2\ join\} \{5T_B\},\\
split\ T_L \{2\ join\} \{4T_B\} \{2\ split\} T_L\},\end{array}\\
\{\{3\ join\} I_B \{6T_B\} \{2\ split\} I_L\}, \left[
\begin{array}{c}
12\\12\\15
\end{array}\right], \left[
\begin{array}{c}
6\\6\\9
\end{array}\right] \end{array}\right\}
\end{displaymath}

\begin{comment}
Scheduling {{\feedbackloops}} requires some extra care, as explained
above.  Once again, steady schedule multiplicities are computed,
but this time, the amount of data buffered between the {\joiner},
$body$, {\splitter} and $loop$ is required in order to perform the
algorithm.

The first step in the algorithm is to execute the {\joiner} as many
times as possible, depending on how much data is available between
the $loop$ and the {\joiner}, up to the number permitted in
executing a steady schedule. Data is transferred between buffers
at known rates, and buffering is adjusted appropriately. Next the
$body$ is executed as much as possible, followed by the {\splitter},
followed by the $loop$.  If, after executing this sequence, the
{\joiner}, $body$, {\splitter} or $loop$ need to be executed more
times in order to complete a steady schedule, this execution is
repeated until the steady schedule is completed.

It is possible, that the algorithm above deadlocks - there is not
enough data for any of the children to advance.  This does not
necessarily mean that the {{\feedbackloop}} has no legal schedule.
This is because pseudo single-appearance scheduling is a coarse
scheduling technique.  Furthermore, this problem is not caused by
hierarchical scheduling.  Figure
\ref{fig:feedback-non-schedulable} contains an example of a
{{\feedbackloop}} that cannot be scheduled using any single appearance
technique.
\end{comment}

\section{Phased Scheduling}
\label{chpt:phased}

We now propose Phased Scheduling, a technique which allows to
schedule all valid {\StreamIt} programs, and which allows for
better control of trade-off between schedule size and buffer size.

Section \ref{sec:phased:intro} provides an introduction to and
explanation of Phased Scheduling. Section \ref{sec:min-latency}
presents a Minimal Latency Schedule implementation using Phased
Scheduling.

\subsection{Phased Scheduling}
\label{sec:phased:intro}

The pseudo single-appearance hierarchical scheduling technique
presented in Section \ref{chpt:hierarchical}, while quite
effective in scheduling simple applications, cannot schedule a
small number of tight {{\feedbackloops}}. Furthermore, the
technique is quite inflexible when it comes to attempting to
create a different tradeoff between schedule size and buffer size.
Phased scheduling solves both of these problems.

Phased schedule is a hierarchical schedule, just like the pseudo
single appearance schedule. Every stream uses only the schedule of
its immediate child streams to create its own schedule. Phased
schedules, however, consist of several steps, called phases. All
of the phases must be executed in order to guarantee correctness.
Once all of the phases have been executed, the schedule has been
executed. A parent stream can interleave execution of its
children's phases in its own schedule, to provide any level of
granularity desired.

\begin{comment}
The schedules created using single appearance hierarchical
scheduling tend to be quite small at the expense of larger
buffering requirements.  A quite simple situation when such
tradeoff is not desired, could be if the schedule is being stored
in a large cheap ROM device, while the RAM used for buffering data
is more expensive.  It is also quite possible that latency
constraints cannot be satisfied by a single appearance
hierarchical schedule. Clearly, a more flexible technique is
required for scheduling.

A key observation in hierarchical scheduling is that each
component only needs to worry about the data that enters or leaves
its children.  The amount of buffering done internally in a child
is not noticeable or important to the parent component. This
observation changes slightly if latency constraints are placed on
the computation. Namely, the important information to keep track
of is amount of data that leaves or enters children as well as
amount of data that crosses latency constraint boundaries.

This observation leads to a conclusion that scheduling execution
of the {\StreamIt} programs using hierarchical scheduling can be
simpler than scheduling the entire program all at once (scheduling
the program all at once requires tracking all buffers and latency
constraints at once).  Phased scheduling is a concept that expends
on hierarchical scheduling, but does not require that a stream has
a single or pseudo single appearance schedule.  Each stream is
allowed to have multiple sub-schedules, also called phases. Each
phase consists of phases of the children of the stream that will
be executed to execute the phase. The phases must be executed in
correct order. When all of the initialization phases of a stream
have been executed, the stream has executed its initialization
schedule and is ready to enter steady state execution. When all of
the steady state phases of a stream have been executed in order,
the entire steady state schedule for the stream has been executed.
\end{comment}

The granularity of splitting the steady state schedule into phases
is left up to the specific scheduler.  Different streams can use
different granularities of execution.  In principle, the parent
should not need to know the scheduling granularity of its
children. The only exception to this rule are {{\feedbackloops}},
which can have children which are not scheduled tightly enough to
allow the {{\feedbackloop}} to execute. An example of that may be a
pseudo single-appearance hierarchical scheduling algorithm
described in Section \ref{sec:sas} implemented using phase
scheduling.

\begin{comment}
One important observation to make is that it makes little sense to
have phases which do not consume or produce any data, and which do
not have data cross any latency boundaries.  This is because such
phases can easily be merged with preceding or following phases
without any effect on ability to schedule a particular program.
This observation allows to easily bound the size of the resulting
schedules to be the sum of executions of first child, last child
and children with latency boundaries. For example, the {\pipeline}
in Figure \ref{fig:hierarchical-schedule} executes its first
child, {\filter} A, 4 times in steady state execution, and its last
child, {\filter} D, 9 times. Thus a phasing schedule of this
pipeline should at most have $4+9=13$ phases.
\end{comment}

\subsection{Minimal Latency Phased Scheduling}
\label{sec:min-latency}

One of the problems with pseudo single-appearance scheduling is
that it cannot schedule all legal {\StreamIt} programs.  A program
with a {{\feedbackloop}} can have requirements for tight execution
that cannot be satisfied using a pseudo single-appearance
schedule, leading to deadlock. Phasing scheduling can alleviate
this problem by allowing the program to be scheduled in a more
fine-grained manner. Minimal latency scheduling is an example of a
specific scheduling strategy that solves the problem of deadlock.
Minimal latency schedule is a schedule that requires a minimal
amount of input data in order to output data. In other words, a
minimal latency schedule only buffers as much data as is
absolutely necessary. This means that if a {\feedbackloop} can be
scheduled, a minimal latency schedule will be able to schedule it.

\begin{comment}
A minimal latency schedule is not necessarily single
appearance. In fact, very few applications can have their minimal
latency schedules expressed as a single appearance schedule.  One
of the consequences of this is that minimal latency schedules
require more space for storage of the schedule. Use of phasing
scheduling facilitates creation of acceptably small minimal
latency schedules.  In spirit of hierarchical scheduling, every
component is scheduled separately, in hierarchical order.

One important consequence of phased scheduling, one that is
highlighted when calculating a minimal latency schedule, is that
every phase is allowed to consume a different amount of data and
produce a different amount of data.

\subsubsection{Peeking}

Phased scheduling has interesting consequence for peeking
calculations.  The reason for this is that not all phases must
consume data, thus not all phases will peek.  The amount of
peeking done by a stream is important for creating an
initialization schedule.  It is thus important to remember that
the amount of peeking done by a stream is not necessarily the
amount of peeking done by that stream in its first phase, because
on first phase, the stream may not consume or peek any data.
\end{comment}

\subsubsection{Notation}

\begin{comment}
We extend the notation for peeking, popping and pushing to
include phases. $u^m_s$ will denote amount of data pushed by the
$m$th phase of stream $s$, $o^m_s$ will denote amount of data
popped by the $m$th phase of stream $s$ and $e^m_s$ will denote
amount of data peeked by $m$th phase of stream $s$.
\end{comment}

A phasing schedule of a stream $s$ is a set $P_s$ of elements,
$P_s = \{T_s, I_s, c_s, c^i_s\}$.  The first element, $T_s$
denotes the phases used for the steady state schedule of $s$.
$I_s$ denotes the phases used for the initialization schedule of
$s$. $c_s$ and $c^i_s$ are defined identically to their
definitions in hierarchical schedules: $c_s$ is the consumption
rate of the stream during its steady state execution and $c^i_s$
is the consumption rate of the initialization schedule.

$T_s$ and $I_s$ are defined by identical structures.  Both are
defined as sets of phases. The only real difference between $T_s$
and $I_s$ is that $T_s$ will be executed indefinitely, while $I_s$
will be executed only once. A phase $A$ is defined as $A = \{E,
c\}$.  $E$ is an ordered list of phases and nodes that are to be
executed in order to execute the phase.  $c$ is the consumption of
the phase, with respect to its stream.

\begin{comment}
As an example, here is a minimal latency schedule for the sample
stream in Figure \ref{fig:sample-sj}.  First, the schedule for the
internal {\splitjoin}:

\begin{displaymath} \small
P_{sj} = \left\{
\begin{array}{c}
T_{sj} = \left\{
\begin{array}{c}
A_{sj,0} = \left\{\{\{6\ split\}\{2C\}\{5D\}\ join\},
\left[\begin{array}{c}12\\12\\12\end{array}\right]\right\}, \\
A_{sj,1} = \left\{\{\{3\ split\}C\{4D\}\ join\},
\left[\begin{array}{c}6\\6\\12\end{array}\right]\right\}
\end{array}\right\}, \\
I_{sj} = \left\{ \right\}, \\
c_{sj} = \left[ \begin{array}{c} 18 \\ 18 \\ 24 \end{array}
\right], c^i_{sj} = \left[ \begin{array}{c} 0 \\ 0 \\ 0
\end{array} \right]
\end{array} \right\}
\end{displaymath}

\noindent And the following is a schedule for the {\pipeline}:

\begin{displaymath} \small
P_p = \left\{
\begin{array}{c}
T_p = \left\{
\begin{array}{c}
A_{p,0} = \left\{ \{\{3A\}A_{sj,0}\ B\}, \left[\begin{array}{c} 18 \\ 18 \\ 10 \end{array}\right]\right\}, \\
A_{p,1} = \left\{ \{\{2A\}A_{sj,1}\ B\}, \left[\begin{array}{c} 12 \\ 12 \\ 10 \end{array}\right]\right\}, \\
A_{p,2} = \left\{ \{\{3A\}A_{sj,0}\ B\}, \left[\begin{array}{c} 18 \\ 18 \\ 10 \end{array}\right]\right\}, \\
A_{p,3} = \left\{ \{A\ A_{sj,1}\ B\}, \left[\begin{array}{c} 6 \\ 6 \\ 10 \end{array}\right]\right\} \\
\end{array}\right\}, \\
I_p = \left\{ \right\}, \\
c_p = \left[ \begin{array}{c} 54 \\ 54 \\ 40 \end{array} \right],
c^i_p = \left[ \begin{array}{c} 0 \\ 0 \\ 0 \end{array} \right]
\end{array}
\right\}
\end{displaymath}
\end{comment}

\subsubsection{\filter}

Since {\filters} have no internal buffering and only one {\work}
function, their schedules are simple.  They contain a single
phase, which in turn contains a single execution of the filter's
{\work} function.  Although in principle, a {\filter} does not need to
be executed to be initialized, it may require some data to be
buffered for its execution.  This means that if $e_f > o_f$, we
insert an artificial initialization phase to phasing schedules of
{\filters}:

\begin{displaymath} \small
P_p = \left\{
\begin{array}{c}
T_p = \left\{
\begin{array}{c}
A_{f,0} = \left\{ \{f\}, \left[\begin{array}{c} e_f \\ o_f \\ u_f \end{array}\right]\right\} \\
\end{array}\right\}, \\
I_p = \left\{ A^i_{f,0} = \left\{ \{ \}, \left[\begin{array}{c}e_f - o_f \\ 0 \\ 0 \end{array}\right]\right\} \right\}, \\
c_p = \left[ \begin{array}{c} e_f \\ o_f \\ u_f \end{array}
\right], c^i_p = \left[ \begin{array}{c} e_f - o_f \\ 0 \\ 0
\end{array} \right]
\end{array}
\right\}
\end{displaymath}

\subsubsection{{\pipeline}, {\splitjoin} and {\feedbackloop}}

Technique used for calculating minimal latency phasing schedule
for a {\pipeline}, {\splitjoin} and {{\feedbackloop}} is similar to
the technique used to create a pseudo single-appearance
hierarchical schedule for a {{\feedbackloop}}. Every phase is
computed separately. Every phase knows how much data has been left
in internal buffers by the previous phase.  The goal is to create
a phase that consumes the minimum amount of data from the {\Input}
{{\Channel}} in order to push at least one data item out to the
{\Output} {{\Channel}}. Once the minimum amount of data has been
consumed by the stream, the maximum amount of data possible is
pushed out of the stream without consuming any more data. This is
meant to prevent unnecessary buffering of data internally within
streams, and reduce the number of phases necessary to compute a
complete schedule.

One important technique used for creating phased schedules is
borrowing of data from {\Channels}.  When a child is being
executed, it is allowed to borrow some data from the {\Channel},
and expect that the upstream child will provide the right amount
of data in the {\Channel} for real execution. In the tables, this
means that amount of data can fall below 0.
\begin{comment}
This is obviously illegal during real execution for any
{\Channel}. Some {\Channels}, however, have even stricter
restrictions. If the node reading from a {\Channel} peeks more
than it pops, the amount of data in the {\Channel} during real
execution cannot fall below the $peek-pop$ amount.
\end{comment}
Since some {\filters} require a positive number of data items in the
{\Channel} , we also need to keep track of amount of data needed
from a {\Channel}, as can be seen in Table\ref{tbl:min-lat-sj}.

The initialization schedule starts with no internally buffered
data (with exception of {{\feedbackloops}}) and executes as many
phases as is necessary to ensure that all children have executed
all of their initialization phases. Once that has been achieved,
the steady state schedule is created. The only difference between
computation of an initialization and steady state schedules is
that the steady state schedule stops executing children early, if
they have already executed all the phases allocated to them for
the steady state, while the initialization schedule continues
executing until all initialization phases of all children have
been executed.

The only significant difference between the algorithms used for
minimal latency scheduling of different stream types ({\pipeline},
{\splitjoin} and {{\feedbackloop}}) is the order with which children
of the stream are considered for execution.

\begin{comment}
For an $i$th child of a stream $s$ (stream $s_n$), the number of
phases that must be executed for its steady state schedule to be
complete is $S_{s,v,i} * |P_{s_i,T}|$.
\end{comment}

For a {\pipeline}, the order with which children are considered for
execution is as follows.  First all the children are considered
for execution moving from bottom to top.  The last child executes
just enough phases to produce some data.  The child directly above
it executes just enough phases to provide sufficient data for the
child below to execute its child.  This process is repeated until
the top-most child is reached.  At this point the direction of
traversal is reversed. This time, the top-most child is skipped,
and the second top-most child is considered.  It only executes as
many phases as it can, while only using data already buffered
between it and the child above it. Then, the child below it is
executed in the same way. This is repeated until the bottom-most
child is reached. The number of phases executed by each child is
added up, and the phases are inserted in order (all phases of
every child together, in order, iterating from top-most child down
to bottom-most child).  This constitutes one complete phase of the
{\pipeline}.

For a {\splitjoin}, the process is similar, but the children are
first executed bottom up starting with {\joiner}, then the stream
children and finally the {\splitter}. After the children are
executed from top to bottom (excluding the {\splitter}), consuming
only data already available to them in {\Channels}.

\begin{comment}
Using the sample {\pipeline} from Figure
\ref{fig:hierarchical-schedule}(a), the following are phasing
schedules for {\filters} A, B, C and D:

\begin{displaymath} \small
P_A = \left\{
\begin{array}{c}
T_A = \left\{
\begin{array}{c}
A_{A,0} = \left\{ \{A\}, \left[\begin{array}{c} 1 \\ 1 \\ 3 \end{array}\right]\right\} \\
\end{array}\right\}, \\
I_A = \left\{ \right\},  c_A = \left[ \begin{array}{c} 1 \\ 1 \\
3 \end{array} \right], c^i_B = \left[ \begin{array}{c} 0 \\ 0 \\ 0
\end{array} \right]
\end{array}
\right\}
\end{displaymath}

\begin{displaymath} \small
P_B = \left\{
\begin{array}{c}
T_B = \left\{
\begin{array}{c}
A_{B,0} = \left\{ \{B\}, \left[\begin{array}{c} 3 \\ 2 \\ 3 \end{array}\right]\right\} \\
\end{array}\right\}, \\
I_B = \left\{ A^i_{B,0} = \left\{ \{ \}, \left[\begin{array}{c}1 \\ 0 \\ 0 \end{array}\right]\right\} \right\}, \\
c_B = \left[ \begin{array}{c} 3 \\ 2 \\ 3 \end{array} \right],
c^i_B = \left[ \begin{array}{c} 1 \\ 0 \\ 0
\end{array} \right]
\end{array}
\right\}
\end{displaymath}

\begin{displaymath} \small
P_C = \left\{
\begin{array}{c}
T_C = \left\{
\begin{array}{c}
A_{C,0} = \left\{ \{C\}, \left[\begin{array}{c} 2 \\ 2 \\ 1 \end{array}\right]\right\} \\
\end{array}\right\}, \\
I_C = \left\{ \right\}, c_C = \left[ \begin{array}{c} 2 \\ 2 \\ 1
\end{array} \right], c^i_C = \left[ \begin{array}{c} 0 \\ 0 \\ 0
\end{array} \right]
\end{array}
\right\}
\end{displaymath}

\begin{displaymath} \small
P_D = \left\{
\begin{array}{c}
T_D = \left\{
\begin{array}{c}
A_{D,0} = \left\{ \{D\}, \left[\begin{array}{c} 5 \\ 3 \\ 1 \end{array}\right]\right\} \\
\end{array}\right\}, \\
I_D = \left\{ A^i_{D,0} = \left\{ \{ \}, \left[\begin{array}{c}2 \\ 0 \\ 0 \end{array}\right]\right\} \right\}, \\
c_D = \left[ \begin{array}{c} 5 \\ 3 \\ 1 \end{array} \right],
c^i_D = \left[ \begin{array}{c} 2 \\ 0 \\ 0
\end{array} \right]
\end{array}
\right\}
\end{displaymath}

\noindent Table \ref{tbl:min-lat-pipe} shows a trace of execution
of the algorithm on the {\pipeline} from Figure
\ref{fig:hierarchical-schedule}(a).

\begin{table} \centering
\small
\begin{tabular}{|c|c|c|c|c|c|c|c|c|c|c|}
\hline
\multicolumn{3}{|c|}{data in {{\Channel}}} & \multicolumn{4}{c|}{\parbox{1in}{\centering phase executions left}} & \parbox{0.5in}{\centering child considered} & \parbox{0.8in}{\centering phases executed} & \parbox{0.8in}{\centering {\pipeline} consumption} \\
\cline{1-7} $in_B$ & $in_C$ & $in_D$ & A & B & C & D & & & \\

\hline 0 (0) & 0 (0) & 0 (-2) & 0 & 1 & 0 & 0 & C & $\{2 A_{C,0}\}$ & $[0\ 0\ 0]$ \\
\hline 0 (0) & -4 (-4) & 2 (0) & 0 & 1 & 0 & 0 & B & $A^i_{B,0}, \{2A_{B,0}\}$ & $[0\ 0\ 0]$ \\
\hline -4 (-5) & 2 (0) & 2 (0) & 0 & 0 & 0 & 0 & A & $\{2A_{A, 0}\}$ & $[2\ 2\ 0]$ \\
\hline 2 (0) & 2 (0) & 2 (0) & 0 & 0 & 0 & 0 & B & - & $[0\ 0\ 0]$ \\
\hline 2 (0) & 2 (0) & 2 (0) & 0 & 0 & 0 & 0 & C & $A_{C,0}$ & $[0\ 0\ 0]$ \\
\hline 2 (0) & 0 (0) & 3 (0) & 0 & 0 & 0 & 0 & D & - & $[0\ 0\ 0]$ \\
\hline 2 (0) &  0 (0) &  3 (0) & \multicolumn{7}{|c|}{init phase 0 done, init done} \\
\hline 2 (0) & 0 (0) & 3 (0) & 4 & 6 & 9 & 3 & D & $A_{D,0}$ & $[0\ 0\ 1]$ \\
\hline 2 (0) & 0 (0) & 0 (-2) & 4 & 6 & 9 & 2 & C & $\{2 A_{C,0}\}$ & $[0\ 0\ 0]$ \\
\hline 2 (0) & -4 (-4) & 2 (0) & 4 & 6 & 7 & 2 & B & $\{2 A_{B,0}\}$ & $[0\ 0\ 0]$ \\
\hline -2 (-3) & 2 (0) & 2 (0) & 4 & 4 & 7 & 2 & A & $A_{A,0}$ & $[1\ 1\ 0]$ \\
\hline 1 (0) & 2 (0) & 2 (0) & 3 & 4 & 7 & 2 & B & - & $[0\ 0\ 0]$ \\
\hline 1 (0) & 2 (0) & 2 (0) & 3 & 4 & 7 & 2 & C & $A_{C,0}$ & $[0\ 0\ 0]$ \\
\hline 1 (0) & 0 (0) & 3 (0) & 3 & 4 & 6 & 2 & D & - & $[0\ 0\ 0]$ \\
\hline 1 (0) &  0 (0) &  3 (0) &  \multicolumn{7}{|c|}{phase 0 done} \\
\hline 1 (0) & 0 (0) & 3 (0) & 3 & 4 & 6 & 2 & D & $A_{D,0}$ & $[0\ 0\ 1]$ \\
\hline 1 (0) & 0 (0) & 0 (-2) & 3 & 4 & 6 & 1 & C & $\{2 A_{C,0}\}$ & $[0\ 0\ 0]$ \\
\hline 1 (0) & -4 (-4) & 2 (0) & 3 & 4 & 4 & 1 & B & $\{2 A_{B,0}\}$ & $[0\ 0\ 0]$ \\
\hline -3 (-4) & 2 (0) & 2 (0) & 3 & 2 & 4 & 1 & A & $\{2 A_{A,0}\}$ & $[2\ 2\ 0]$ \\
\hline 3 (0) & 2 (0) & 2 (0) & 1 & 2 & 4 & 1 & B & $A_{B,0}$ & $[0\ 0\ 0]$ \\
\hline 1 (0) & 5 (0) & 2 (0) & 1 & 1 & 4 & 1 & C & $\{2 A_{C,0}\}$ & $[0\ 0\ 0]$ \\
\hline 1 (0) & 1 (0) & 4 (0) & 1 & 1 & 2 & 1 & D & - & $[0\ 0\ 0]$ \\
\hline 1 (0) &  1 (0) &  4 (0) &  \multicolumn{7}{|c|}{phase 1 done} \\
\hline 1 (0) & 1 (0) & 4 (0) & 1 & 1 & 2 & 1 & D & $A_{D,0}$ & $[0\ 0\ 1]$ \\
\hline 1 (0) & 1 (0) & 1 (-1) & 1 & 1 & 2 & 0 & C & $A_{C,0}$ & $[0\ 0\ 0]$ \\
\hline 1 (0) & -1 (-1) & 2 (0) & 1 & 1 & 1 & 0 & B & $A_{B,0}$ & $[0\ 0\ 0]$ \\
\hline -1 (-2) & 2 (0) & 2 (0) & 1 & 1 & 1 & 0 & A & $A_{A,0}$ & $[1\ 1\ 0]$ \\
\hline 2 (0) & 2 (0) & 2 (0) & 0 & 0 & 1 & 0 & B & - & $[0\ 0\ 0]$ \\
\hline 2 (0) & 2 (0) & 2 (0) & 0 & 0 & 1 & 0 & C & $A_{C,0}$ & $[0\ 0\ 0]$ \\
\hline 2 (0) & 0 (0) & 3 (0) & 0 & 0 & 0 & 0 & D & - & $[0\ 0\ 0]$ \\
\hline 2 (0) &  0 (0) &  3 (0) &  \multicolumn{7}{|c|}{phase 2 done, steady state schedule done} \\
\hline
\end{tabular}
\caption[Trace of execution of Minimal Latency Scheduling on a
{\pipeline}]{Trace of execution of Minimal Latency Scheduling
Algorithm on {\pipeline} from Figure
\ref{fig:hierarchical-schedule}(a). In the "data in {{\Channel}}"
columns the first value represents the actual number of data in
the {{\Channel}}, which can be negative if more data has been popped
from the {{\Channel}} than has been pushed into it.  This is due to
borrowing of data from {\Channels}. The second value represents the
minimal number of data items that the downstream {\filter} has
inspected beyond the 0th data. This value can be higher than the
negative amount of data in the {{\Channel}} because a {\filter} may
peek at data without consuming it.  In general, for a {\filter}
$f$, the amount of data needed on its input {{\Channel}} is $\max(0,
-(in_f - (e_f - o_f)))$. The needed amount is 0 until the
downstream {\filter} is executed for the first time.}
\label{tbl:min-lat-pipe}
\end{table}

\noindent The following is the resulting phasing schedule:

\begin{displaymath} \small
P_p = \left\{
\begin{array}{c}
T_p = \left\{
\begin{array}{c}
A_{p,0} = \left\{ \{A\{2B\}\{3C\}D\}, \left[\begin{array}{c} 1 \\ 1 \\ 1 \end{array}\right]\right\}, \\
A_{p,1} = \left\{ \{\{2A\}\{3B\}\{4C\}D\}, \left[\begin{array}{c} 2 \\ 2 \\ 1 \end{array}\right]\right\}, \\
A_{p,2} = \left\{ \{A\ B\{2C\}D\}, \left[\begin{array}{c} 1 \\ 1 \\ 1 \end{array}\right]\right\} \\
\end{array}\right\}, \\
I_p = \left\{ A^i_{p,0} = \left\{
\{\{2A\}A^i_{B,0}\{2B\}\{3C\}A^i_{D,0}\}, \left[\begin{array}{c} 2  \\2 \\ 0 \\
\end{array}\right]\right\}
\right\}, \\
c_p = \left[ \begin{array}{c} 4 \\ 4 \\ 3 \end{array} \right],
c^i_p = \left[ \begin{array}{c} 2 \\ 2 \\ 0 \end{array} \right]
\end{array}
\right\}
\end{displaymath}

\subsubsection{\splitjoin}

As explained above, the only difference between the algorithm for
a {\pipeline} and a {\splitjoin} is the order in which the children
streams are considered for execution.  In a {\pipeline}, the
children are considered from the bottom-most child to the top
child, and then from second top-most child down to the bottom most
child again.  A {\splitjoin} has only three levels of direct
children in it: the top is a {\splitter}, the middle is formed by
all the child streams of the {\splitjoin} and the bottom is the
{\joiner}.  To schedule a {\splitjoin}, the children are also
considered in the bottom to top and top to bottom order, but the
child streams are also considered from left to right (this choice
is arbitrary - the order does not affect the number of child phase
executions per phase of the {\splitjoin}).

Using the sample {\splitjoin} from Figure
\ref{fig:hierarchical-schedule}(b), the following are phasing
schedules for {\filters} A and B:

\begin{displaymath} \small
P_A = \left\{
\begin{array}{c}
T_A = \left\{
\begin{array}{c}
A_{A,0} = \left\{ \{A\}, \left[\begin{array}{c} 2 \\ 2 \\ 1 \end{array}\right]\right\} \\
\end{array}\right\}, \\
I_A = \left\{ \right\}, c_A = \left[ \begin{array}{c} 2 \\ 2 \\ 1
\end{array} \right], c^i_A = \left[ \begin{array}{c} 0 \\ 0 \\ 0
\end{array} \right]
\end{array}
\right\}
\end{displaymath}

\begin{displaymath} \small
P_B = \left\{
\begin{array}{c}
T_B = \left\{
\begin{array}{c}
A_{B,0} = \left\{ \{B\}, \left[\begin{array}{c} 3 \\ 2 \\ 6 \end{array}\right]\right\} \\
\end{array}\right\}, \\
I_B = \left\{ A^i_{B,0} = \left\{ \{ \}, \left[\begin{array}{c} 1 \\ 0 \\ 0 \end{array}\right]\right\} \right\}, \\
c_B = \left[ \begin{array}{c} 3 \\ 2 \\ 6 \end{array} \right],
c^i_B = \left[ \begin{array}{c} 1 \\ 0 \\ 0
\end{array} \right]
\end{array}
\right\}
\end{displaymath}

\noindent Table \ref{tbl:min-lat-sj} shows execution of the
algorithm on the {\splitjoin} from Figure
\ref{fig:hierarchical-schedule}(b).

\begin{table} \centering \small
\begin{tabular}{|c|c|c|c|c|c|c|c|c|c|c|c|}
\hline
\multicolumn{4}{|c|}{data in {{\Channel}}} & \multicolumn{4}{c|}{\parbox{1in}{\centering phase executions left}} & \parbox{0.5in}{\centering child considered} & \parbox{0.6in}{\centering phases executed} & \parbox{0.6in}{\centering {\pipeline} consumption} \\
\cline{1-8} split & A & B & join & $in_A$ & $out_A$ & $in_B$ & $out_B$ & & & \\
\hline
\hline 0 (0) & 0 (0) & 0 (0) & 0 (0) & 0 & 0 & 1 & 0 & join & - & $[0\ 0\ 0]$ \\
\hline 0 (0) & 0 (0) & 0 (0) & 0 (0) & 0 & 0 & 1 & 0 & A & - & $[0\ 0\ 0]$ \\
\hline 0 (0) & 0 (0) & 0 (0) & 0 (0) & 0 & 0 & 1 & 0 & B & $A^i_{B,0}$ & $[0\ 0\ 0]$ \\
\hline 0 (0) & 0 (0) & 0 (-1) & 0 (0) & 0 & 0 & 0 & 0 & split & split & $[3\ 3\ 0]$ \\
\hline 2 (0) & 0 (0) & 1 (0) & 0 (0) & 0 & 0 & 0 & 0 & A & $A^i_{A,0}$ & $[0\ 0\ 0]$ \\
\hline 0 (0) & 1 (0) & 1 (0) & 0 (0) & 0 & 0 & 0 & 0 & B & - & $[0\ 0\ 0]$ \\
\hline 0 (0) & 1 (0) & 1 (0) & 0 (0) & 0 & 0 & 0 & 0 & join & - & $[0\ 0\ 0]$ \\
\hline 0 (0) &  1 (0) &  1 (0) &  0 (0) & \multicolumn{7}{|c|}{init phase 0 done, init done} \\
\hline 0 (0) & 1 (0) & 1 (0) & 0 (0) & 2 & 2 & 1 & 2 & join & join & $[0\ 0\ 4]$ \\
\hline 0 (0) & 0 (0) & 1 (0) & -3 (-3) & 2 & 2 & 1 & 2 & A & - & $[0\ 0\ 0]$ \\
\hline 0 (0) & 0 (0) & 1 (0) & -3 (-3) & 2 & 2 & 1 & 1 & B & $A_{B,0}$ & $[0\ 0\ 0]$ \\
\hline 0 (0) & 0 (0) & -1 (-2) & 3 (0) & 2 & 2 & 0 & 2 & split & $\{2split\}$ & $[6\ 6\ 0]$ \\
\hline 4 (0) & 0 (0) & 1 (0) & 3 (0) & 0 & 2 & 0 & 2 & A & $\{2A_{A,0}\}$ & $[0\ 0\ 0]$ \\
\hline 0 (0) & 2 (0) & 1 (0) & 3 (0) & 0 & 0 & 0 & 0 & B & - & $[0\ 0\ 0]$ \\
\hline 0 (0) & 2 (0) & 1 (0) & 3 (0) & 0 & 0 & 0 & 1 & join & join  & $[0\ 0\ 4]$ \\
\hline 0 (0) &  1 (0) &  1 (0) &  0 (0) & \multicolumn{7}{|c|}{phase 0 done, steady state schedule done} \\
\hline
\end{tabular}
\caption[Execution of Minimal Latency Scheduling Algorithm on a
{\splitjoin}]{Execution of Minimal Latency Scheduling Algorithm on
{\splitjoin} from Figure \ref{fig:hierarchical-schedule}(b).}
\label{tbl:min-lat-sj}
\end{table}

The trace of the execution shows that even though it is strictly
necessary to traverse the children of the stream second time from
bottom to top, doing so can pay off in reducing the number of
phases necessary to construct a phasing schedule.  Namely, in its
first steady state execution, the {\splitter} needs to push enough
data to execute the {\joiner} again, thus eliminating a need for an
additional phase.

Once all the phases are computed, the phasing schedule is
constructed. For every phase, the number of child phases executed
is added up, and the actual schedule is constructed by
concatenating all the phases of all the children, starting with
the {\splitter}, all stream children (as listed from left to right)
and finally the {\joiner}. The following is the resulting phasing
schedule:

\begin{displaymath} \small
P_{sj} = \left\{
\begin{array}{c}
T_{sj} = \left\{
\begin{array}{c}
A_{sj,0} = \left\{ \{\{2 split\}\{2A\}B\{2 join\}\}, \left[\begin{array}{c} 6 \\ 6 \\ 8 \end{array}\right]\right\} \\
\end{array}\right\}, \\
I_{sj} = \left\{ A^i_{sj,0} = \left\{
\{split\ A^i_{A,0}\ A^i_{B,0}\}, \left[\begin{array}{c} 3 \\ 3 \\ 0 \\
\end{array}\right]\right\}
\right\}, \\
c_{sj} = \left[ \begin{array}{c} 6 \\ 6 \\ 8 \end{array} \right],
c^i_{sj} = \left[ \begin{array}{c} 3 \\ 3 \\ 0 \end{array}
\right],
\end{array}
\right\}
\end{displaymath}

\subsubsection{{\feedbackloop}}
\end{comment}

Scheduling of {{\feedbackloops}} is again similar to the above
algorithms.  The children's phases are executed in order of
({\splitter}, body child, {\joiner}, body child, {\splitter}, loop
child).  The {\splitter} tries to execute exactly one time on its
first iteration.  The body child and the {\joiner} execute just
enough times to provide data for the {\splitter} to perform its
first execution.  Then the body child, {\splitter} and the loop
child are executed as many times as possible with the data
available to them on their {\Input} {{\Channels}}.

The one big difference between {{\feedbackloop}} and the other streams
({\pipeline} and {\splitjoin}) is that in scheduling a {{\feedbackloop}},
the {\joiner} is {\emph not} allowed to borrow elements from $out_L$
{{\Channel}}.  That is in the trace table, the $out_L$ entry is never
allowed to become negative.  The reason for this is that
{{\feedbackloops}} are cyclical structures, and allowing the {\joiner}
to borrow elements from $out_L$ would cause a full cycle of
borrowing, leading to deadlock.

\begin{comment}
This one condition does not prevent from scheduling any legal
{{\feedbackloops}}.  The reason for this is that before the
{{\feedbackloop}} is initialized, there is data pushed onto the
$out_L$ {{\Channel}}.  At the end of scheduling of any phase, all
available data is pushed through the {{\feedbackloop}} into the
$out_L$ {{\Channel}}.  Thus any available free data is already always
stored in the $out_L$ {{\Channel}}, and there is no additional data to
borrow from in a {{\feedbackloop}}.

If the algorithm is unable to schedule an execution of the {\joiner}
in a phase without borrowing data from $out_L$ {{\Channel}}, then the
{{\feedbackloop}} cannot be scheduled.

\begin{lemma}[{{\feedbackloop}} Scheduling]
If all children of a {{\feedbackloop}} are scheduled using minimal
latency scheduling algorithm, then if the {{\feedbackloop}} cannot
be scheduled using the minimal latency scheduling algorithm then
there is no valid schedule for this {{\feedbackloop}}.
\end{lemma}

We believe this lemma to be true because minimal latency
scheduling always consumes the minimal amount of data to produce
some data, and produces the maximal amount of data possible given
the amount of data it consumes.  Thus no data is being buffered up
in {{\Channels}} and if the {{\feedbackloop}} cannot be scheduled, then
the $delay_{fl}$ value is too low and does not provide enough data
to complete a steady state execution. A formal proof is left for
future work.

We will again use the sample {{\feedbackloop}} from Figure
\ref{fig:hierarchical-schedule}(c).  The following are the phasing
schedules for {\filters} B and L:

\begin{displaymath} \small
P_B = \left\{
\begin{array}{c}
T_B = \left\{
\begin{array}{c}
A_{B,0} = \left\{ \{B\}, \left[\begin{array}{c} 2 \\ 2 \\ 1 \end{array}\right]\right\} \\
\end{array}\right\}, \\
I_B = \left\{ \right\}, \\
c_B = \left[ \begin{array}{c} 2 \\ 2 \\ 1 \end{array} \right],
c^i_B = \left[ \begin{array}{c} 0 \\ 0 \\ 0
\end{array} \right]
\end{array}
\right\}
\end{displaymath}

\begin{displaymath} \small
P_L = \left\{
\begin{array}{c}
T_L = \left\{
\begin{array}{c}
A_{L,0} = \left\{ \{L\}, \left[\begin{array}{c} 9 \\ 5 \\ 6 \end{array}\right]\right\} \\
\end{array}\right\}, \\
I_L = \left\{ A^i_{L,0} = \left\{ \{ \}, \left[\begin{array}{c} 4 \\ 0 \\ 0 \end{array}\right]\right\} \right\}, \\
c_L = \left[ \begin{array}{c} 9 \\ 5 \\ 6 \end{array} \right],
c^i_L = \left[ \begin{array}{c} 4 \\ 0 \\ 0
\end{array} \right]
\end{array}
\right\}
\end{displaymath}

\noindent Table \ref{tbl:min-lat-fl} shows execution of the
algorithm on the {{\feedbackloop}} from Figure
\ref{fig:hierarchical-schedule}(c).

\begin{table} \centering
\scriptsize
\begin{tabular}{|c|c|c|c|c|c|c|c|c|c|c|c|}
\hline
\multicolumn{4}{|c|}{data in {{\Channel}}} & \multicolumn{4}{c|}{\parbox{1in}{\centering phase executions left}} & \parbox{0.5in}{\centering child considered} & \parbox{0.6in}{\centering phases executed} & \parbox{0.6in}{\centering {\pipeline} consumption} \\
\cline{1-8} $in_B$ & $out_B$ & $in_F$ & $out_F$ & join & B & split & F & & & \\
\hline 0 (0) & 0 (0) & 0 (0) & 15 (0) & 0 & 0 & 0 & 1 & split & split & $[0\ 0\ 3]$ \\
\hline 0 (0) & -3 (3) & 3 (0) & 15 (0) & 0 & 0 & 0 & 1 & B & $\{3A_{B,0}\}$ & $[0\ 0\ 0]$ \\
\hline -6 (6) & 0 (0) & 3 (0) & 15 (0) & 0 & 0 & 0 & 1 & join & $\{2\ join\}$ & $[4\ 4\ 0]$ \\
\hline 4 (0) & 0 (0) & 3 (0) & 9 (0) & 0 & 0 & 0 & 1 & B & $\{2A_{B,0}\}$ & $[0\ 0\ 0]$ \\
\hline 0 (0) & 2 (0) & 3 (0) & 9 (0) & 0 & 0 & 0 & 1 & split & - & $[0\ 0\ 0]$ \\
\hline 0 (0) & 2 (0) & 3 (0) & 9 (0) & 0 & 0 & 0 & 1 & F & - & $[0\ 0\ 0]$ \\

\hline 0 (0) &  2 (0) &  3 (0) &   9 (0) & \multicolumn{7}{|c|}{init phase 0 done} \\

\hline 0 (0) & 2 (0) & 3 (0) & 9 (0) & 0 & 0 & 0 & 1 & split & split & $[0\ 0\ 3]$ \\
\hline 0 (0) & -1 (1) & 6 (0) & 9 (0) & 0 & 0 & 0 & 1 & B & $\{A_{B,0}\}$ & $[0\ 0\ 0]$ \\
\hline -2 (2) & 0 (0) & 6 (0) & 9 (0) & 0 & 0 & 0 & 1 & join & join & $[2\ 2\ 0]$ \\
\hline 3 (0) & 0 (0) & 6 (0) & 6 (0) & 0 & 0 & 0 & 1 & B & $\{A_{B,0}\}$ & $[0\ 0\ 0]$ \\
\hline 1 (0) & 1 (0) & 6 (0) & 6 (0) & 0 & 0 & 0 & 1 & split & - & $[0\ 0\ 0]$ \\
\hline 1 (0) & 1 (0) & 6 (0) & 6 (0) & 0 & 0 & 0 & 1 & F & $\{A^i_{F,0}\}$ & $[0\ 0\ 0]$ \\

\hline 1 (0) &  1 (0) &  6 (0) &   6 (0) & \multicolumn{7}{|c|}{init phase 1 done, init done} \\

\hline 1 (0) & 1 (0) & 6 (0) & 6 (0) & 6 & 15 & 5 & 3 & split & split & $[0\ 0\ 3]$ \\
\hline 1 (0) & -2 (2) & 9 (0) & 6 (0) & 6 & 15 & 4 & 3 & B & $\{3A_{B,0}\}$ & $[0\ 0\ 0]$ \\
\hline -5 (5) & 1 (0) & 9 (0) & 6 (0) & 6 & 12 & 4 & 3 & join & join & $[2\ 2\ 0]$ \\
\hline 0 (0) & 1 (0) & 9 (0) & 3 (0) & 5 & 12 & 4 & 3 & B & - & $[0\ 0\ 0]$ \\
\hline 0 (0) & 1 (0) & 9 (0) & 3 (0) & 5 & 12 & 4 & 3 & split & - & $[0\ 0\ 0]$ \\
\hline 0 (0) & 1 (0) & 4 (0) & 9 (0) & 5 & 12 & 4 & 3 & F & $\{A_{F,0}\}$ & $[0\ 0\ 0]$ \\

\hline 0 (0) &  1 (0) &  4 (0) &   9 (0) & \multicolumn{7}{|c|}{phase 0 done} \\

\hline 0 (0) & 1 (0) & 4 (0) & 9 (0) & 5 & 12 & 4 & 2 & split & split & $[0\ 0\ 3]$ \\
\hline 0 (0) & -2 (2) & 7 (0) & 9 (0) & 5 & 12 & 3 & 2 & B & $\{2A_{B,0}\}$ & $[0\ 0\ 0]$ \\
\hline -4 (4) & 0 (0) & 7 (0) & 9 (0) & 5 & 10 & 3 & 2 & join & join & $[2\ 2\ 0]$ \\
\hline 1 (0) & 0 (0) & 7 (0) & 6 (0) & 4 & 10 & 3 & 2 & B & - & $[0\ 0\ 0]$ \\
\hline 1 (0) & 0 (0) & 7 (0) & 6 (0) & 4 & 10 & 3 & 2 & split & - & $[0\ 0\ 0]$ \\
\hline 1 (0) & 0 (0) & 7 (0) & 6 (0) & 4 & 10 & 3 & 2 & F & - & $[0\ 0\ 0]$ \\

\hline 1 (0) & 0 (0) & 7 (0) & 6 (0) & \multicolumn{7}{|c|}{phase 1 done} \\

\hline 1 (0) & 0 (0) & 7 (0) & 6 (0) & 4 & 10 & 3 & 2 & split & split & $[0\ 0\ 3]$ \\
\hline 1 (0) & -3 (3) & 10 (0) & 6 (0) & 4 & 10 & 2 & 2 & B & $\{3A_{B,0}\}$ & $[0\ 0\ 0]$ \\
\hline -5 (5) & 0 (0) & 10 (0) & 6 (0) & 4 & 7 & 2 & 2 & join & join & $[2\ 2\ 0]$ \\
\hline 0 (0) & 0 (0) & 10 (0) & 3 (0) & 3 & 7 & 2 & 2 & B & - & $[0\ 0\ 0]$ \\
\hline 0 (0) & 0 (0) & 10 (0) & 3 (0) & 3 & 7 & 2 & 2 & split & - & $[0\ 0\ 0]$ \\
\hline 0 (0) & 0 (0) & 10 (0) & 3 (0) & 3 & 7 & 2 & 2 & F & $\{A_{F,0}\}$ & $[0\ 0\ 0]$ \\

\hline 0 (0) & 0 (0) & 5 (0) & 9 (0) & \multicolumn{7}{|c|}{phase 2 done} \\

\hline 0 (0) & 0 (0) & 5 (0) & 9 (0) & 3 & 7 & 2 & 1 & split & split & $[0\ 0\ 3]$ \\
\hline 0 (0) & -3 (3) & 8 (0) & 9 (0) & 3 & 7 & 1 & 1 & B & $\{3A_{B,0}\}$ & $[0\ 0\ 0]$ \\
\hline -6 (6) & 0 (0) & 8 (0) & 9 (0) & 3 & 4 & 1 & 1 & join & $\{2\ join\}$ & $[4\ 4\ 0]$ \\
\hline 4 (0) & 0 (0) & 8 (0) & 3 (0) & 1 & 4 & 1 & 1 & B & $\{2A_{B,0}\}$ & $[0\ 0\ 0]$ \\
\hline 0 (0) & 2 (0) & 8 (0) & 3 (0) & 1 & 2 & 1 & 1 & split & - & $[0\ 0\ 0]$ \\
\hline 0 (0) & 2 (0) & 8 (0) & 3 (0) & 1 & 2 & 1 & 1 & F & - & $[0\ 0\ 0]$ \\

\hline 0 (0) & 2 (0) & 8 (0) & 3 (0) & \multicolumn{7}{|c|}{phase 3 done} \\

\hline 0 (0) & 2 (0) & 8 (0) & 3 (0) & 1 & 2 & 1 & 1 & split & split & $[0\ 0\ 3]$ \\
\hline 0 (0) & -1 (1) & 11 (0) & 3 (0) & 1 & 2 & 0 & 1 & B & $\{A_{B,0}\}$ & $[0\ 0\ 0]$ \\
\hline -2 (2) & 0 (0) & 11 (0) & 3 (0) & 1 & 1 & 0 & 1 & join & join & $[2\ 2\ 0]$ \\
\hline 3 (0) & 0 (0) & 11 (0) & 0 (0) & 0 & 1 & 0 & 1 & B & $\{A_{B,0}\}$ & $[0\ 0\ 0]$ \\
\hline 1 (0) & 1 (0) & 11 (0) & 0 (0) & 0 & 0 & 0 & 1 & split & - & $[0\ 0\ 0]$ \\
\hline 1 (0) & 1 (0) & 11 (0) & 0 (0) & 0 & 0 & 0 & 1 & F & $\{A_{F,0}\}$ & $[0\ 0\ 0]$ \\

\hline 1 (0) & 1 (0) & 6 (0) & 6 (0) & \multicolumn{7}{|c|}{phase 4 done, steady state schedule done} \\
\hline
\end{tabular}
\caption[Execution of Minimal Latency Scheduling Algorithm on a
{{\feedbackloop}}]{Execution of Minimal Latency Scheduling Algorithm
on {{\feedbackloop}} from Figure
\ref{fig:hierarchical-schedule}(c).} \label{tbl:min-lat-fl}
\end{table}

Once the number of executions of children's phases is known for
every phase of the {{\feedbackloop}}'s schedule, the phasing schedule
can be constructed.  For every phase, the children of the
{{\feedbackloop}} are iterated over in order of ({\joiner}, body child,
{\splitter}, loop child) and for every child the appropriate number
of phases is inserted into the schedule.  Below is the schedule
for {{\feedbackloop}} in Figure \ref{fig:hierarchical-schedule}(c):

\begin{displaymath} \small
P_{fl} = \left\{
\begin{array}{c}
T_{fl} = \left\{
\begin{array}{c}
A_{fl,0} = \left\{ \{join\ \{3B\}\ split\ F\}, \left[\begin{array}{c} 2 \\ 2 \\ 3 \end{array}\right]\right\}, \\
A_{fl,1} = \left\{ \{join\ \{2B\}\ split\}, \left[\begin{array}{c} 2 \\ 2 \\ 3 \end{array}\right]\right\}, \\
A_{fl,2} = \left\{ \{join\ \{3B\}\ split\ F\}, \left[\begin{array}{c} 2 \\ 2 \\ 3 \end{array}\right]\right\}, \\
A_{fl,3} = \left\{ \{\{2\ join\}\{5B\}\ split\}, \left[\begin{array}{c} 4 \\ 4 \\ 3 \end{array}\right]\right\}, \\
A_{fl,4} = \left\{ \{join\ \{2B\}\ split\ F\}, \left[\begin{array}{c} 2 \\ 2 \\ 3 \end{array}\right]\right\} \\
\end{array}\right\}, \\
T_{fl} = \left\{
\begin{array}{c}
A^i_{fl,0} = \left\{ \{\{2\ join\}\{5B\}\ split\}, \left[\begin{array}{c} 4 \\ 4 \\ 3 \end{array}\right]\right\}, \\
A^i_{fl,1} = \left\{ \{join\ \{2B\}\ split\}, \left[\begin{array}{c} 2 \\ 2 \\ 3 \end{array}\right]\right\} \\
\end{array}\right\}, \\
c_{fl} = \left[ \begin{array}{c} 12 \\ 12 \\ 15 \end{array}
\right], c^i_{fl} = \left[ \begin{array}{c} 6 \\ 6 \\ 6
\end{array} \right],
\end{array}
\right\}
\end{displaymath}
\end{comment}

\begin{table}[t] \centering  \scriptsize
\begin{tabular}{|c|c|c|c|c|c|c|c|c|c|c|c|}
\hline
\multicolumn{4}{|c|}{data in {{\Channel}}} & \multicolumn{4}{c|}{\parbox{1in}{\centering phase executions left}} & \parbox{0.5in}{\centering child considered} & \parbox{0.6in}{\centering phases executed} & \parbox{0.6in}{\centering {\pipeline} consumption} \\
\cline{1-8} split & A & B & join & $in_A$ & $out_A$ & $in_B$ & $out_B$ & & & \\
\hline 0 (0) & 0 (0) & 0 (0) & 0 (0) & 0 & 0 & 1 & 0 & join & - & $[0\ 0\ 0]$ \\
\hline 0 (0) & 0 (0) & 0 (0) & 0 (0) & 0 & 0 & 1 & 0 & A & - & $[0\ 0\ 0]$ \\
\hline 0 (0) & 0 (0) & 0 (0) & 0 (0) & 0 & 0 & 1 & 0 & B & $A^i_{B,0}$ & $[0\ 0\ 0]$ \\
\hline 0 (0) & 0 (0) & 0 (-1) & 0 (0) & 0 & 0 & 0 & 0 & split & split & $[3\ 3\ 0]$ \\
\hline 2 (0) & 0 (0) & 1 (0) & 0 (0) & 0 & 0 & 0 & 0 & A & $A^i_{A,0}$ & $[0\ 0\ 0]$ \\
\hline 0 (0) & 1 (0) & 1 (0) & 0 (0) & 0 & 0 & 0 & 0 & B & - & $[0\ 0\ 0]$ \\
\hline 0 (0) & 1 (0) & 1 (0) & 0 (0) & 0 & 0 & 0 & 0 & join & - & $[0\ 0\ 0]$ \\
\hline 0 (0) &  1 (0) &  1 (0) &  0 (0) & \multicolumn{7}{|c|}{init phase 0 done, init done} \\
\hline 0 (0) & 1 (0) & 1 (0) & 0 (0) & 2 & 2 & 1 & 2 & join & join & $[0\ 0\ 4]$ \\
\hline 0 (0) & 0 (0) & 1 (0) & -3 (-3) & 2 & 2 & 1 & 2 & A & - & $[0\ 0\ 0]$ \\
\hline 0 (0) & 0 (0) & 1 (0) & -3 (-3) & 2 & 2 & 1 & 1 & B & $A_{B,0}$ & $[0\ 0\ 0]$ \\
\hline 0 (0) & 0 (0) & -1 (-2) & 3 (0) & 2 & 2 & 0 & 2 & split & $\{2split\}$ & $[6\ 6\ 0]$ \\
\hline 4 (0) & 0 (0) & 1 (0) & 3 (0) & 0 & 2 & 0 & 2 & A & $\{2A_{A,0}\}$ & $[0\ 0\ 0]$ \\
\hline 0 (0) & 2 (0) & 1 (0) & 3 (0) & 0 & 0 & 0 & 0 & B & - & $[0\ 0\ 0]$ \\
\hline 0 (0) & 2 (0) & 1 (0) & 3 (0) & 0 & 0 & 0 & 1 & join & join  & $[0\ 0\ 4]$ \\
\hline 0 (0) &  1 (0) &  1 (0) &  0 (0) & \multicolumn{7}{|c|}{phase 0 done, steady state schedule done} \\
\hline
\end{tabular}
\caption[Execution of Minimal Latency Scheduling Algorithm on a
{\splitjoin}]{Execution of Minimal Latency Scheduling Algorithm on
{\splitjoin} from Figure \ref{fig:steady-state}(b). In the "data
in {{\Channel}}" columns, the first value represents the actual
number of data in the {{\Channel}}, which can be negative due to
data borrowing. The second value is the minimal number of
additional data items needed in the {\Channel}.}
\label{tbl:min-lat-sj}
\end{table}

Table \ref{tbl:min-lat-sj} contains a trace of execution of our
algorithm on the sample {\splitjoin} from Figure
\ref{fig:steady-state}(b). Below is the phasing schedule for the
{\splitjoin}. Note that this example does produce a single
appearance schedule.

\begin{displaymath} \scriptsize
P_{sj} = \left\{
\begin{array}{c}
T_{sj} = \left\{
\begin{array}{c}
A_{sj,0} = \left\{ \{\{2 split\}\{2A\}B\{2 join\}\}, \left[\begin{array}{c} 6 \\ 6 \\ 8 \end{array}\right]\right\} \\
\end{array}\right\}, \\
I_{sj} = \left\{ A^i_{sj,0} = \left\{
\{split\ A^i_{A,0}\ A^i_{B,0}\}, \left[\begin{array}{c} 3 \\ 3 \\ 0 \\
\end{array}\right]\right\}
\right\}, \\
c_{sj} = \left[ \begin{array}{c} 6 \\ 6 \\ 8 \end{array} \right],
c^i_{sj} = \left[ \begin{array}{c} 3 \\ 3 \\ 0 \end{array}
\right],
\end{array}
\right\}
\end{displaymath}

%\subsection{Latency Constrained Scheduler}

\begin{figure*}[t]
\begin{center}
\psfig{figure=constrained-example.eps,height=3.5in}
\caption{{\small   Example   of    construction   of   a   constrained
schedule. The $\sdepf{R}{S}$ function for filters $R$ and $S$ is given
in  Table \ref{tab:sdepconst}. The  blob between  filters $R$  and $S$
illustrates other possible stream elements. $R$ sends a message to $S$
with  latency $[1,2]$.  Executions  of  the blob  are  omitted, it  is
assumed  that at the  point $S$  executes, the  blob has  drained data
provided by $R$.}}
\end{center}
\vspace{-12pt}
\label{fig:sdepconst}
\end{figure*}


\begin{table*}[t]
{\small
\begin{tabular}{|c|c|c|c|c|c|c|c|c|} \hline
{\bf $sdepf{R}{S}$} & $9n+2$ & $9n+3$  & $9n+5$  & $9n+5$ & $9n+6$ &
$9n+8$ & $9n+9$ & $9n+9$ \\ \hline
{\bf Execs of S} & $8n+1$ & $8n+2$ & $8n+3$ & $8n+4$ & $8n+5$ & $8n+6$ &
$8n+7$ & $8n+8$ \\ \hline
\end{tabular}}
\caption{\small $sdepf{R}{S}$ function for example in Figure \ref{fig:sdepconst}. This particular $\sdep$ function was obtained by setting $push_R=2$, $pop_S=3$ and making the blob between $R$ and $S$ into a filter that pops 3 and pushes 4 every iteration of its work function. No initialization due to peeking is necessary in this example.}
\label{tab:sdepconst}
\end{table*}

In this section we decribe latency constrained scheduling, an 

The example  in Figure~\ref{fig:sdepconst} illustrates  the scheduling
of a single constraint: a  message sent upstream with latency $[1,2]$.
The resulting schedule consists  of two parts, an {\it initialization}
schedule and a  {\it steady state} schedule.  The  former is necessary
to ensure that during the actor's steady state execution, it may check
for messages  at proper boundaries.  For  the sake of  clarify, we use
$lastReceived$  to represent  the firing  of  $S$ that  sent the  last
message which has been  received by $R$; $n_S$ and  $n_R$ respectively indicate
the number of firings of $S$ and $R$.

In order  to compute  the initialization schedule,  $R$ is  allowed to
fire  $SDEP(minLatency)-1$ times,  while $S$  fires as  many  times as
possible.   This assures  that when  the steady  schedule  begins, all
firings  of  filters  will  contribute  to  new  message  sending  and
receiving,   thus  allowing   the  steady   state  schedule   to  fire
perpetually. In our example, the initialization schedule does not send
or receive any messages.

Once the  initialization schedule is  known, the steady state  schedule is
computed by  allowing the receiver $R$ to  fire continuously, provided
it  does not violate  the constraint  that a  message sent  during the
$lastReceived+1$ firing of $S$ must  reach $R$ according to the specified
latency;  this  is indicated  by  {\it  "Oldest  msg to  receive"}  in
Figure~\ref{fig:sdepcons}.  Now the  sender  $S$ is  executed as  many
times as possible, given the data provided via $R$. At this point, $R$
can receive the messages sent  by $S$ in  its executions
$[lastReceived+1  ... \min(n_S,$"{\it Newest msg to receive}"$)]$.


\section{Results}
\label{chpt:results}

This section presents results of creating schedules using
techniques described in Sections \ref{chpt:hierarchical} and
\ref{chpt:phased}.

Section \ref{sec:results:apps} presents the applications used for
testing.  Section \ref{sec:results:results} presents the results and
analysis.

\subsection{Applications}
\label{sec:results:apps}

Our benchmark suite contains 17 applications. Out of those
applications, 15 represent useful practical computation taken from
real-life applications, while two were chosen to highlight the
effectiveness of phased scheduling.

SJPeek1024 and SJPeek31 are synthetic benchmarks, designed to
highlight strengths of phased schedules. SJPeek1024 requires an
initialization schedule which benefits from finer granularity of
minimum latency schedule. SJPeek31 contains a push/pop mismatch
which causes a combinatorial blow-up using SAS.

Nine test applications (BitonicSort, FFT, FilterBank, FIR, Radio,
GSM, 3GPP, Radar and Vocoder) are applications used in
\cite{Gordo02}. BitonicSort performs a 32 element bitonic sort.
FFT performs a 64-element FFT, FilterBank is an 8 channel filter
bank.  FIR is a 64-tap FIR application. Radio is an FM radio
decoder with an equalizer.  3GPP is a 3GPP Radio Access Protocol
application. Radar is a radar array front-end application. Vocoder
is a 28 channel Vocoder.

Two test applications (QMF and CD-DAT) are applications used in
another publication on scheduling streaming applications
(\cite{murthy99buffer}). QMF is a filter bank application. CD-DAT
is a sample rate conversion application. The code inside of the
{\filters} has not been implemented. QMF application is a
qmf12\_3d.  It was slightly modified to account for {\StreamIt}
{\splitters} and {\joiners} not allowing any computation. The
high-pass and low-pass filtering in the {\splitters} has been
moved to just after data has been separated into two channels. The
re-combining of data in the {\joiners} has been moved to a
{\filter} just after the {\joiners}. The low and high pass filters
have also been given a peek amount of 16 so they can perform their
function in the way intended in {\StreamIt}.
 CD-DAT is exactly the same application as
that described in \cite{murthy99buffer}.

The remaining 4 applications were chosen from our sample
applications used for testing the StreamIt compiler. HDTV performs
a HDTV signal decoding/encoding. CFAR implements PCA Constant
False Alarm Rate detection. Block Matrix Mult performs a blocked
matrix multiplication application - it multiplies a 12x12 matrix
by a 9x12 matrix in blocks of 3x3 submatrices. Trellis performs
trellis encoding/decoding.

\begin{comment}

\subsection{Methodology}
\label{sec:results:methodology}

The following data has been collected: number of nodes, number of
node executions per steady state, schedule size and buffer size
for pseudo single appearance and minimal latency schedules.

\subsubsection{Schedule Compression}

\end{comment}

\begin{comment}
\subsubsection{Sinks}

Any application in {\StreamIt} must receive its data from outside,
and its data must be sent to outside. {\filters} that receive and
send data to outside are called sinks and sources. In particular,
sinks have the property of having $u_f = 0$ while sources have
$e_f = o_f = 0$. Sinks are problematic for minimal latency
scheduling, because they do not produce any data. Thus any
schedule of a sink operator is a minimal latency schedule. This
leads to the minimal latency schedule of the outer-most
{\pipeline} becoming a single appearance schedule, thus destroying
some of the benefit of using phased scheduling. The technique
used for scheduling {\pipeline} sinks has been discussed in
\cite{karczma-thesis}.

This problem has been alleviated by detecting sinks at the end of
a {\pipeline} and scheduling them in a unique way. Namely, a simple
attempt is made to minimize the amount of storage necessary to
store the phases of the {\pipeline}.

Let the amount of storage necessary to store one data item in
{\Input} {\Channel} to the sink be $x$, the amount of storage necessary
to store a phase be $y$, the sink consume $a$ data per steady
state execution of its parent {\pipeline} and $b$ be the number of
phases of the parent pipeline, then we have that amount of storage
necessary to store the phases and the buffer is
$$ {ax \over b} + by $$ We want to minimize this amount, with $b$
being the variable. We take a derrivative of the above expression,
set it to zero and solve:

\begin{displaymath}
\begin{array}{rcl}
-{ax \over b^2} + y & = & 0 \\
yb^2 & = & ax \\
b & = & \sqrt{ax \over y}
\end{array}
\end{displaymath}

For simplicity, we set $x = y = 1$, thus obtaining that $b =
\sqrt{a}$.

Now, for every phase of the parent {\pipeline} of the sink, the sink
is executed $\sqrt{a}$ times on the first step of scheduling a
phase of the {\pipeline}.
\end{comment}

\subsection{Results}
\label{sec:results:results}

\begin{figure}[t]
\psfig{figure=buffer-graph.eps,width=3.35in}
\caption{Buffer sizes.}
\end{figure}

\begin{figure}[t]
\psfig{figure=code-graph.eps,width=3.35in}
\caption{Code sizes.}
\end{figure}

\begin{figure}[t]
\psfig{figure=total-size-graph.eps,width=3.35in}
\caption{Sum of code size and buffer size.}
\end{figure}

\begin{table*} \centering \small
\begin{tabular}{|c|c|c|c|c|c|c|}
\hline benchmark & \parbox{0.5in}{\centering number of nodes} & \parbox{0.5in}{\centering number of node executions} & \multicolumn{2}{c|}{pseudo single appearance} & \multicolumn{2}{c|}{minimal latency} \\
\cline{4-7} & & & \parbox{0.5in}{\centering schedule size} & \parbox{0.5in}{\centering buffer size} & \parbox{0.5in}{\centering schedule size} & \parbox{0.5in}{\centering buffer size} \\
\hline SJPeek31 & 6 & 12063 & 8 & 19964 & 24 & 874 \\
\hline HDTV & 170 & 390038 & 230 & 550692 & 1190 & 28300 \\
\hline CD-DAT & 6 & 612 & 6 & 1021 & 64 & 72 \\
\hline CFAR & 4 & 193 & 7 & 193 & 9 & 129 \\
\hline SJPeek1024 & 6 & 3081 & 8 & 7168 & 13 & 4864 \\
\hline Block Matrix Mult & 43 & 1956 & 48 & 4212 & 56 & 3132 \\
\hline Vocoder & 117 & 415 & 156 & 1285 & 205 & 1094 \\
\hline Radar & 68 & 161 & 68 & 332 & 68 & 332 \\
\hline BitonicSort & 370 & 468 & 370 & 2112 & 370 & 2112 \\
\hline 3GPP & 94 & 356 & 104 & 986 & 108 & 970 \\
\hline Trellis & 14 & 301 & 14 & 538 & 17 & 499 \\
\hline FIRfine & 132 & 152 & 132 & 1560 & 132 & 1560 \\
\hline FilterBank & 53 & 312 & 95 & 2063 & 116 & 1991 \\
\hline QMF & 65 & 184 & 85 & 1225 & 85 & 1225 \\
\hline Radio & 30 & 43 & 35 & 1351 & 35 & 1351 \\
\hline FFT & 26 & 448 & 26 & 3584 & 26 & 3584 \\
\hline GSM & 47 & 3356 & - & - & 64 & 3900 \\
\hline
\end{tabular}
\caption{Results of running pseudo single appearance and minimal
latency scheduling algorithms on various applications.}
\label{tbl:results}
\end{table*}

\begin{comment}
\begin{figure}
\centering \psfig{figure=kz-1.eps,width=6in} \caption[Buffer
storage space savings of Phased Minimal Latency schedule vs.
Hierarchical schedule.]{Buffer storage space savings of Phased
Minimal Latency schedule vs. Hierarchical schedule. All data in
all {\Channels} is assume to consume same amount of space.}
\end{figure}

\begin{figure}
\centering \psfig{figure=kz-2.eps,width=6in} \caption[Storage
usage comparison]{Storage usage for compressed Minimal Latency
Phased schedule vs. Hierarchical schedule. Left bars are for
Hierarchical schedules. Numbers are normalized to total storage
required by Hierarchical schedule. Each entry in every schedule
and data items in all {\Channels} are assumed to consume same
amount of space.}
\end{figure}
\end{comment}

Table \ref{tbl:results} presents buffer and schedule sizes
necessary to execute various applications using the algorithms
developed in this thesis.

The GSM application cannot be scheduled using pseudo
single-appearance algorithm, because it has a loop which is too
tight for execution under the SAS.

Several applications show a very large improvement in buffer size
necessary for execution.  Namely, CD-DAT decreases from 1021 to
72, a 93\% improvement. \cite{murthy99buffer} reports a buffer
size of 226 after applying buffer merging techniques. Our
improvement is due to reducing the combinatorial growth of the
buffers using phased scheduling.

Our synthetic benchmarks decrease from 7168 to 4864 and from 19964
to 12063, a 32\% and 40\% improvements. The first improvement is
due to creating fine grained phases which allow the initialization
schedule to transfer smaller amount of data and allow the children
of a {\splitjoin} to drain their data before the {\splitter}
provides them with more. This improvement is only created in
presence of peeking. The second improvement is due to reducing
combinatorial growth and due to finer grained schedules to deal
with peeking.

Other applications show no or little improvement in buffer
requirements. As expected, no application requires more buffer
space.

\mysection{Related Work}
\label{sec:related}

This paper builds directly on the work done to analyze and optimize
linear components in StreamIt graphs \cite{Lamb}. We extend the
theoretical framework for linear analysis to state space analysis in
order to apply our optimizations to a wider class of applications.
Specifically, state space analysis applies to filters with persistent
state, and feedback loops can be combined into a single state space
representation; neither of these cases are handled by linear analysis.
The extension from linear analysis to state space analysis required a
fundamental change to the underlying representation, as well as a
complete reformulation of the rules for combination and expansion.
Moreover, this paper introduces novel optimizations of state removal
and minimal parameterization, both of which operate on the state space
representation.

Several other groups are researching methods for automated DSP
optimizations. SPIRAL \cite{Spiral} is a system developed to generate
libraries of DSP transforms. These libraries are designed for specific
architectures, and can be re-optimized when hardware is upgraded or
replaced. Other such libraries that have been designed include a
package for linear algebra manipulations by the ATLAS project
\cite{Atlas} and portable high-performance FFTs (Fast Fourier
Transforms) \cite{fftw}.

Aside from StreamIt, other programming languages have been designed
for streaming data. Synchronous languages which target embedded
applications include LUSTRE \cite{Lustre}, Esterel \cite{Esterel}, and
Signal \cite{Signal}. Other stream-based languages are Occam
\cite{Occam}, SISAL \cite{sisal}, and StreamC \cite{streamc}.  Some of
these languages are designed to exploit vector and parallel
processing. However, none of these languages have compilers that run
state space or linear analysis.

\section{Conclusion}

This paper shows that the inter-node dependences of a Cyclo-Static
Dataflow Graph can be cleanly represented as a System of Affine
Recurrence Equations.  In combination with Feautrier's array dataflow
analysis~\cite{Feautrier01} for deducing intra-node dependences, this
establishes the SARE as a unified analysis and optimization framework
for high-level DSP programming models.

We believe that the precise affine dependence framework provided by
the SARE representation will enable a powerful suite of node
optimizations in dataflow graphs.  The SARE is a robust and
well-established framework within the systolic and scientific
communities, with methods for graph parameterization, automatic
parallelization, and storage optimization.  We propose optimizations
such as decimation propagation and node fission that are first
applications of these techniques to the signal processing domain.

%\section{Initialization for Peeking}

This section will develop a simple algorithm for constructing an
initialization schedule.  The algorithm developed here will use
hierarchy in a similar way section \ref{sec:calc-min-steady} used
hierarchy to compute steady schedules.  Schedules computed here
are not minimal, but remain reasonably small.

The amount of data required for initialization by stream $s$ will
be denoted by $init^{pop}_s$. The amount of data produced by
initialization schedule will be denoted by $init^{push}_{s}$. This
hierarchical technique will simplify calculations, but may be
unable to compute an initialization schedule for {\feedbackloops}
that would otherwise be possible to schedule.  An algorithm for
computing a valid initialization schedule for any stream that can
be initialized will be presented in Section \ref{sec:min-latency}.

\subsubsection{Notation for Initialization Schedules}

An initialization schedule for a stream $s$ is a set $I_s$ with
elements $I_s = \{c, u\}$.  The set includes a vector $c$ which
holds values $[e_s^{init}, o_s^{init}, u_s^{init}]$ for the
initialization schedule of the stream and a vector $u$ which
stores how many times each of the direct children of $s$ will
execute their steady state.  The elements are denoted $I_{s,S}$,
$I_{s,c}$ and $I_{s,u}$.

\subsubsection{\filter}

Since {\filters} do not buffer any data, they do not require
initialization schedules.  They may, however, require some data
for their initialization.  For a {\filter} $f$, this amount of data
is $e_f - o_f$, as explained above.  {\filters} will not produce any
data during initialization.

An initialization schedule of a {\filter} is thus $I_f = \{[e_f-o_f,
0, 0], \{\}\}$.

\subsubsection{\pipeline}

\begin{algorithm}
\caption{Single Appearance Initialization Counts for a {\pipeline}}
{\bf }($p$). Given a {\pipeline} $p$, calculate how many times each
child needs to execute its steady state to initialize the
{\pipeline} for peeking.
\begin{algorithmic}
\STATE compute $S_{p_{n-1}}$; $u_{n-1} = 0$ \FOR{$i=n-2$ downto 0}
\STATE compute $S_{p_i}$ \STATE $u_i = {I_{p_{i+1}, c_0} + u_{i+1}
* S_{p_{i+1}, c_1} - I_{p_i, c_2} \over I_{p_i, c_2}}$ \ENDFOR
\STATE $I_p = \{[I_{p_0, c_0} + (u_0 - 1) * S_{p_0, c_0}, u_0 *
S_{p_0, c_0}, I_{p_{n-1}, c_2}] ,u\}$
\end{algorithmic}
\end{algorithm}

Let $s_i$ denote $i$th child stream of the {\pipeline}.  Also, let
$m_i$ denote the number of times the $i$th child will execute its
steady state during execution of {\pipeline}'s initialization
schedule.

The $i$th stream needs to provide at least $init^{pop}_{s_{i+1}} +
m_{i+1} * pop_{s_{i+1}}$ data for {\pipeline}'s next child.  For the
last child $m_{n-1} = 0$, thus it only needs
$init^{pop}_{s_{n-1}}$ data to initialize.  Knowing the amount of
data required by the last child, we can compute how much data all
other children require, and thus compute how many times all other
children need to execute their steady state schedules.

In order to provide enough data for $s_{i+1}$, $s_{i}$ is going to
execute its initialization schedule, producing
$init^{push}_{s_{i}}$ data items, and then it is going to execute
its steady state schedule $m_i$ times to provide any additional
required data.  The $i$th child will need to execute its steady
state schedule $m_i = \left\lceil { init^{pop}_{s_{i+1}} + m_{i+1}
* pop_{s_{i+1}} - init^{push}_{s_i} \over{push_{s_i}}}
\right\rceil$ times.

Finally, since the first child of the {\pipeline} is directly
receiving the data that is meant to enter the {\pipeline}, it
follows that $init^{pop}_{p} = init^{pop}_{s_0} + m_0 *
pop_{s_0}$. Similarly, $init^{push}_{pipeline} =
init^{push}_{s_{i-1}}$.

Once all $m_i$s are known, the initialization schedule is
constructed according to the following algorithm:

\begin{singlespace}
\begin{verbatim}
initialization schedule (p) = empty for i = 0 .. n-1
    initialization schedule (p) += initialization schedule (s_i)
    for j = 1 .. m_i
        initialization schedule (p) += steady schedule (s_i)
    end for
end for
\end{verbatim}
\end{singlespace}

Notice, that since the children of the {\pipeline} use their steady
schedules in order to push extra data into the buffers, they may
be pushing more data than required, thus causing the pipeline to
consume more data for its initialization schedule than absolutely
required.  This over-estimation will propagate up with each
{\StreamIt} stream that contains this {\pipeline}.

We again use the example from Figure \ref{fig:steady-state} (a).
Since all children of the {\pipeline} are {\filters}, they do not have
any initialization schedules, and their $init^{push}_{s_i} = 0$.
By inspection we obtain

\begin{displaymath}
\begin{array}{rl}
init^{pop}_A = & 1-1=0 \\
init^{pop}_B = & 3-2 = 1 \\
init^{pop}_C = & 2 - 2 = 0 \\
init^{pop}_D = & 5-3=2 \\
\\
m_3 = & 0 \\
m_2 = & \left \lceil 2 + 3 * 0 - 0 \over 1 \right \rceil = 2\\
m_1 = & \left \lceil 0 + 2 * 2 - 0 \over 3 \right \rceil = 2\\
m_0 = & \left \lceil 1 + 2 * 2 - 0 \over 3 \right \rceil = 2\\
\end{array}
\end{displaymath}

Thus the initialization schedule is $(AABBCC)$.  Also,
$init^{pop}_p = 0 + 2 * 1 = 2$ and $init^{push}_p = 0$.

\subsubsection{\splitjoin}

\begin{algorithm}
\caption{Single Appearance Initialization Counts for a {\pipeline}}
{\bf }($p$). Given a {\pipeline} $p$, calculate how many times each
child needs to execute its steady state to initialize the
{\pipeline} for peeking.

\begin{algorithmic}

\STATE compute $S_{p_{n-1}}$; $u_{n-1} = 0$

\FOR{$i=n-2$ downto 0}

\STATE compute $S_{p_i}$

\STATE $u_i = {I_{p_{i+1}, c_0} + u_{i+1} * S_{p_{i+1}, c_1} -
I_{p_i, c_2} \over I_{p_i, c_2}}$

\ENDFOR

\STATE $I_p = \{[I_{p_0, c_0} + (u_0 - 1) * S_{p_0, c_0}, u_0 *
S_{p_0, c_0}, I_{p_{n-1}, c_2}] ,u\}$

\end{algorithmic}
\end{algorithm}

Initializing a {\splitjoin} is done by executing the {\splitter}
enough times to provide enough data for all the children to
initialize.  The {\joiner} is never run, so there may be some data
buffered between the children streams and the {\joiner}.

Let $s_i$ denote the $i$th child stream of the {\splitjoin},
$w_{s,i}$ denote the amount of data pushed by the {\splitter} of the
{\splitjoin} towards the $i$th child during an execution of the
{\splitter}, and $w_{j,i}$ denote the amount of data popped by the
{\joiner} from the $i$th child during its execution.

The {\splitter} must execute enough times to provide every child of
the {\splitjoin} with enough data to initialize.  For $i$th child,
the required number of executions is $\left\lceil init^{pop}_{s_i}
\over w_{s,i} \right\rceil$.  The {\splitter} needs to execute the
maximum amount of times required by any the children, that is
$m_{split} = \max \left\lceil init^{pop}_{s_i} \over w_{s,i}
\right\rceil, \forall i \in \{0,\dots,n-1\}$.  Thus the amount of
data required for initialization of a {\splitjoin} is
$init^{pop}_{sj} = pop_{split} * m_{split}$. Since the joiner will
never get executed, $init^{push}_{sj} = 0$.

Once $m_{\splitter}$ is has been calculated, the initialization
schedule is constructed according to the following algorithm:

\begin{singlespace}
\begin{verbatim}
initialization schedule (sj) = empty

for i = 1 .. m_{splitter}
  initialization schedule (sj) += {\splitter} execution
end for

for i = 0 .. n-1
    initialization schedule (sj) += initialization schedule (s_i)
end for
\end{verbatim}
\end{singlespace}

The following computes an initialization schedule for the example
{\splitjoin}  from Figure \ref{fig:steady-state} (b). Both children
of the {\splitjoin} are {\filters}, thus they do not have
initialization schedules and their $init^{push}_{s_i} = 0$. By
inspection, we obtain

\begin{displaymath}
\begin{array}{rl}
init^{pop}_A = & 2-2=0 \\
init^{pop}_B = & 3-2 = 1 \\
\\
m_{split} = & \max({0 \over 2}, {1 \over 1}) = \max (0, 1) = 1
\end{array}
\end{displaymath}

Thus the initialization schedule for this {\splitjoin} is simply
$({\splitter})$.  We also get $init^{pop}_{sj} = 1
* 3 = 3$ and $init^{push}_{sj} = 0$.

\subsubsection{\feedbackloop}

The final {\StreamIt} component left to initialize is the
{\feedbackloop}. Let $s_b$ be the body stream of the {\feedbackloop},
and $s_l$ be the feedback path stream of the {\feedbackloop} (thus
$push_{s_b}$ is the amount of data pushed by $s_b$ per steady
state execution, etc).  Let $m_{s_b}$ be the number of times $s_b$
needs to be executed in order to properly initialize the
{\feedbackloop} and $m_{s_l}$ be the number of times $s_l$ needs to
be executed to initialize the {\feedbackloop}.

Initialization for the {\feedbackloop} is calculated in a similar
way to initialization of a {\pipeline}.  Since the initial data is
inserted into the buffer between the loop stream and the {\joiner},
it follows that calculation of initialization requirements should
start from the loop stream as a "last" element - it will be
execute last in the initialization schedule. That means that the
loop stream will execute its initialization schedule, but it will
not execute its steady schedule, namely $m_{s_l} = 0$.  Thus we
have

\begin{displaymath}
\begin{array}{rl}
m_{split} & = \left\lceil {init^{pop}_{s_l} \over w_{s,1}} \right\rceil \\
m_{s_b} & = \left\lceil pop_{split} * m_{split} -
init^{push}_{s_b} \over push_{s_b} \right\rceil \\
m_{join} & = \left\lceil init^{pop}_{s_b} + pop_{s_b} * m_{s_b}
\over { push_{s_b}} \right\rceil
\end{array}
\end{displaymath}

We also calculate the overall consumption and production of data
during initialization of the {\splitjoin}

\begin{displaymath}
\begin{array}{rl}
init^{pop}_{fl} & = w_{j,0} * m_{join} \\
init^{push}_{fl} & = w_{s, 0} * m_{split}
\end{array}
\end{displaymath}

Now the schedule can be constructed by using the following
algorithm:

\begin{singlespace}
\begin{verbatim}
initialization schedule (fl) = empty for i = 1 .. m_{join}
  initialization schedule (fl) += {\joiner} execution
end for initialization schedule (fl) += initialization schedule
(s_b) for i = 0 .. m_{s_b}
    initialization schedule (fl) += steady state schedule (s_b)
end for for i = 1 .. m_{split}
  initialization schedule (fl) += {\splitter} execution
end for initialization schedule (fl) += initialization schedule
(s_l)
\end{verbatim}
\end{singlespace}

It is important to note, that this initialization schedule is not
legal if there isn't enough data in the buffer between the {\joiner}
and the loop stream, that is if $delay_{fl} < w_{j,1} *
m_{joiner}$.  This condition being true, does not mean, however,
that no initialization schedule exists for the particular
{\feedbackloop}.  The reason for this is that executing entire
steady state schedules of the body stream may consume more data
than is actually necessary to provide enough data for the loop
stream to receive $init^{pop}_{s_l}$ data.  Also the
$init^{pop}_{s_b}$ and $init^{pop}_{s_l}$ values may be larger
than necessary, because they also may use their children's steady
schedules in initialization.

An algorithm for initializing any legal stream structure is
presented in Section \ref{sec:min-latency}

The following computes an initialization schedule for the example
{\feedbackloop} from Figure \ref{fig:steady-state} (c). Both
children of the {\feedbackloop} are {\filters}, thus they do not have
initialization schedules and their $init^{push}_{s} = 0$. By
inspection, we obtain

\begin{displaymath}
\begin{array}{rl}
init^{pop}_B = & 3-2 = 1 \\
init^{pop}_L = & 7-5 = 2 \\
\\
m_{split} = & \left\lceil 2 \over 3 \right\rceil = 1 \\
m_{B} = & \left\lceil 3 * 1 - 0 \over 1 \right\rceil = 3 \\
m_{join} = & \left\lceil 1 + 2 * 3 \over 5 \right\rceil = 2 \\
\end{array}
\end{displaymath}

Thus the initialization schedule for this {\feedbackloop} is
$({\joiner}\ {\joiner}\ BBB\ {\splitter})$.  We also get
$init^{pop}_{fl} = 2 * 2 = 4$ and $init^{push}_{fl} = 3 * 1 = 3$.

\section{Steady Schedules}

\subsection{Filter}

The scheduling of a {\filter} is very simple.  Since a {\filter} has
no sub-components (it is an atomic unit), a steady schedule for a
{\filter} is simply an execution of the {\filter}.  Thus, for a
{\filter} $f$, $P_f = \{f, \{f\}, \{[e_f,o_f,u_f]\}\}$

\subsection{Pipeline}

Scheduling a {\pipeline} $p$ first requires calculating the steady
state $S_p$ and phasing schedules for all the children of $p$,
$P_{p_i}$. Once the steady state has been calculated,
multiplicities of execution of each child are known, and the
children are simply scheduled to execute an appropriate number of
times in a row, starting from first child.

\begin{algorithm}
\label{alg:sa-pipeline} \caption{Single Appearance Schedule for a
{\pipeline}} {\bf SASPipeline}($p$).  Given a {\pipeline} $p$,
calculate a phasing Single Appearance Schedule for $p$.
\begin{algorithmic}
\STATE compute $S_p$; $phase = \{\}$ \FOR{$i=0$ to $n_p$}
\FOR{$j=0$ to $S_{p,u,i}$} \FOR{$k=0$ to $|P_{p_i, P}|$} \STATE
$phase = phase + P_{p_i, P, k}$ \ENDFOR \ENDFOR \ENDFOR \STATE
$P_p = \{p, \{phase\}, \{S_{p,c}\}\}$
\end{algorithmic}
\end{algorithm}

This technique works here, because {\pipelines} do not have any
cycles between their children (though their children may have
cycles, ie. {\feedbackloops}), and because a correct initialization
schedule is assumed to have been executed.  Notice, that it is not
necessary to know the amount of data buffered between children of
the {\pipeline} in order to use this algorithm, because once the
{\pipeline} has been initialized, all the needed data will be
provided by the steady state schedule.

\subsection{SplitJoin}

Scheduling a {\splitjoin} is essentially identical to scheduling a
{\pipeline}.  Once steady schedule multiplicities are computed, the
{\splitter} is executed the appropriate number of times, followed by
all the immediate children, and finally the {\joiner}.

\begin{algorithm}
\label{alg:sa-pipeline} \caption{Single Appearance Schedule for a
{\splitjoin}} {\bf SASSplitJoin}($sj$).  Given a {\splitjoin} $sj$,
calculate a phasing Single Appearance Schedule for $sj$.
\begin{algorithmic}
\STATE compute $S_{sj}$; $phase = \{\}$ \FOR{$j=0$ to
$S_{sj,u,n_{sj}}$} \STATE $phase = phase + {\splitter}$ \ENDFOR
\FOR{$i=0$ to $n_p$} \FOR{$j=0$ to $S_{p,u,i}$} \FOR{$k=0$ to
$|P_{p_i, P}|$} \STATE $phase = phase + P_{p_i, P, k}$ \ENDFOR
\ENDFOR \ENDFOR \FOR{$j=0$ to $S_{sj,u,n_{sj} + 1}$} \STATE $phase
= phase + {\joiner}$ \ENDFOR \STATE $P_p = \{p, \{phase\},
\{S_{p,c}\}\}$
\end{algorithmic}
\end{algorithm}

Similarly to {\pipelines}, this technique works because {\splitjoins}
have no cycles, and because the {\splitjoin} is assumed to have been
initialized properly.

\section{Min Latency}

\subsection{Pipeline}

In order to create a steady state schedule for a {\pipeline}, an
initialization schedule must already exist (or at least its
results must have been computed).  This was not the case with
single appearance schedule, because just the assumption that the
{\pipeline} was properly initialized allowed for execution of the
all the child streams from top to bottom.  With minimal latency
scheduling, the amount of data buffered up in the {\pipeline} (or
any other stream, for that matter) makes a big difference in which
child streams need to produce (and thus possibly consume) data,
and which do not.

One consequence of this interaction between the initialization and
steady schedules is that the initialization schedule may affect
the size of the steady schedule.  The difference comes from the
fact, that a steady schedule doesn't necessarily process all the
data buffered up and ready for processing.  This means that it is
possible that the steady schedule will have an additional phase at
the end, that will simply push some data around internally to the
{\pipeline}.  The phase will not produce any data (otherwise, the
phase would be necessary anyway), but it may consume some data (to
complete the amounts required to execute a steady schedule) and/or
push the data in internal buffers lower, in order to return to the
buffering state from the beginning of the steady schedule.

To simplify the calculation of how many times each child needs to
get executed during initialization, initialization is computed
exactly the same way as steady schedule.  Namely, the bottom most
stream is executed enough (minimal number of) times to produce
some data, the stream above is executed just enough times to
provide enough data to the stream below, and so on, until either a
child stream does not need to fire, or the top most child is
reached. If the amount of additional data needed by $n+1$ child is
$m$, and the $n$th child is about to execute $p$th phase of
$phases_n$, then the number of executions of the $n$th child is
computed using following algorithm:

\begin{singlespace}
\begin{verbatim}
while (m > 0)
  fired_n = fired_n + 1
  m = m - push^p_{p_n}
  p = (p + 1) \% phases_n
\end{verbatim}
\end{singlespace}

The result of running this algorithm is that while the
initialization schedule grows a little larger than necessary, the
steady schedule remains as small as possible.  The reason for this
is that all the streams will automatically execute the right
number of times to provide enough data for the bottom most child
to fire enough times.

\subsubsection{\splitjoin}

{\splitjoin} schedule is computed in essentially the same way as
{\pipeline}, except that number of firings of child streams depends
on the {\joiner} and the number of firings of the {\splitter} depends
on the child streams.  Algorithm is omitted here for brevity.

\subsubsection{\feedbackloop}

{\feedbackloop} schedule is also scheduled in a way very similar to
the {\pipeline} schedule.  Since the data is output from the
{\splitter}, the calculation begins by setting number of executions
of the {\splitter} to 1.  Number of firings of the $body$ is
computed, then the {\joiner} and finally the $loop$.

Computing of the schedule for the {\feedbackloop} is guaranteed to
succeed, if the {\feedbackloop} has a valid schedule.  This is
because each component in the {\feedbackloop} is scheduled in a way
that requires minimal amount of data input in order to produce
some output.  Thus, if a deadlock is detected in the
{\feedbackloop}, there genuinly is not enough data in the
{\feedbackloop} to execute the {\splitter}, thus output data.

\section{Notation and equations}

Steady State $T_s = \{s, N, m, c, u\}$.
\begin{itemize}
\myitem $s$ - the stream itself

\myitem $L$ - children of the stream

\myitem $m$ - multiples of execution of the children

\myitem $c$ - peek/push/pop for $s$
\end{itemize}

\noindent Phasing Schedule $P^p_s = \{S, I\}$.
\begin{itemize}
\myitem $S$ - steady state phasing schedule phases

\myitem $I$ - phasing init schedule phases
\end{itemize}

\noindent a set of phases (steady state or init) $S = \{H, c\}$
\begin{itemize}
\myitem $H$ - the actual phases \myitem $c$ - peek/pop/push for
the set
\end{itemize}

\noindent Phase $H_s = \{s, A, c, u\}$.
\begin{itemize}
\myitem $s$ - the stream itself

\myitem $A$ - sub-phases to be executed for this phase

\myitem $c$ - peek/push/pop for the phase

\myitem $u$ - {\bf don't need - so don't use!} multiples for
direct children, {\splitter} and {\joiner}.  {\splitter} and {\joiner} are
the $n$th and the $n+1$st element.
\end{itemize}

\noindent Single Appearance $S_s = \{H, I\}$
\begin{itemize}
\myitem $H$ - steady state phase (single one!) \myitem $I$ -
initialization phase (single one!)
\end{itemize}

\subsection{Steady State}

The steady state is easy to determine.  It's already described,
and I don't want to go into it here.

$c$ vector describes the TRUE peek.

\subsection{Scheduling}

All scheduling is actually done within the framework of phased
scheduling.  Initialization has multiple phases (doh!), so I'll
have to be careful writing this up.


\subsection{Initialization Schedule}

Initialization schedules are done in a very simple way - the child
that is meant to output data is executed at least once.  Because
of feedbackloops, there may be more than one phase of
initialization (for min-latency).

\subsection{Single Appearance}

\subsubsection{Filter}

for {\filter} $f$:

\begin{displaymath}
\begin{array}{rl}
H = & \{f, [f], [1], [e_f, o_f, u_f], [1]\} \\
I = & \{f, [], [], [e_f - o_f], [1]\} \\
S_f = & \{H, I\}
\end{array}
\end{displaymath}

\subsubsection{Pipeline}

for {\pipeline} $p$, children $p_i$, $0 \le i < n_p$, $n_p$ children

\begin{algorithm}
\label{alg:sas-pipeline} \caption{Create a Single Appearance
Schedule for a {\pipeline}} {\bf SASPipeline}($T_p$). Given a steady
state for a {\pipeline}, create a single appearance schedule for it.
\begin{algorithmic}
\STATE no clue how to do this
\end{algorithmic}
\end{algorithm}

\subsection{Min-Latency}

\subsubsection{Peek/Pop/Push from phases}

\begin{algorithm}
\label{alg:ph-consumption} \caption{Computing Consumption of a Set
of Phases} {\bf ConsPhases}($H$).  Given a set of phases $H$,
compute the amount of data peeked/popped/pushed by this set of
phases.
\begin{algorithmic}
\STATE $e = 0, o = 0, u = 0$
\FOR{$i=0$ to $|H|-1$}
\STATE $e = \max(e, o + H_{i, c, e})$
\STATE $o = o + H_{i, c, o}$
\STATE $u = u + H_{i, c, u}$
\ENDFOR
\STATE return $[e, o, u]$
\end{algorithmic}
\end{algorithm}

\begin{algorithm}
\label{alg:init-peek} \caption{Create a Phasing schedule out of
set of phases for Steady State and Initialization Schedules} {\bf
MakePhases}($H^S, H^I$). Given sets of phases to execute for
steady state and initialization schedules, construct a proper
phasing schedule.
\begin{algorithmic}
\STATE $c_S = {\bf ConsPhases} (H^S)$
\STATE $c_I = {\bf ConsPhases} (H^I)$
\STATE $c_{I,e} = \max(c_{I,e}, c_{I,o} + c_{S,e} - c_{S,o})$
\STATE $S = \{H^S, c_S\}$
\STATE $I = \{H^I, c_I\}$
\STATE return $\{S, I\}$
\end{algorithmic}
\end{algorithm}

\begin{algorithm}
\label{alg:get-phase} \caption{Return an appropriate phase of a
schedule.  Phase 0 is the first initialization stage of the
schedule.  Once all initialization stages have been exhausted, all
steady state phases are returned, with wrap-around.} {\bf
GetPhase}(n, $P$)
\begin{algorithmic}
\IF{$n < |P_{I, A}|$}
\STATE return $P_{I, A, n}$
\ELSE
\STATE return $P_{S, A, (n - |P_{I,A}|) {\bf\ mod\ } |P_{S, A}|}$
\ENDIF
\end{algorithmic}
\end{algorithm}

%\subsubsection{\filter}

\begin{algorithm}
\label{alg:min-lat-filter} \caption{Return a min-latency phasing
schedule for a {\filter}} {\bf MLFilter}().
\begin{algorithmic}
\STATE $H = \{f, \{f\}, [e_f, o_f,u_f]\}$
\STATE return ${\bf MakePhases} (\{H\}, \emptyset)$
\end{algorithmic}
\end{algorithm}

%\subsubsection{\pipeline}

\begin{algorithm}
\label{alg:min-lat-init-pipeline} \caption{Return a set of phases
that execute a Minimum-Latency initialization schedule for a
{\pipeline}, the amount of data buffered as a result of
initialization, and how many phases each child has executed.} {\bf
MLInitStagesPipeline} $()$
\begin{algorithmic}
\STATE Let $b$ vector represent amount of data stored between children of
$p$
\STATE Let $d$ vector represent phase/stage being currently executed; 0
represents first initialization stage.


\STATE $b = 0$, $d = 0$

\STATE {\bf Compute how many times each child will get
executed at minimum}
\STATE Let $f$ represent amount of data needed by the $i$th child
to be initialized.  $f$ has $n_p+1$ elements, last child being
fictional and not requiring any initialization.
\STATE $f = 0$

\FOR{$i=n_p-1$ downto $0$}
\WHILE{$d_i < |P_{p_i, I, A}|$ {\bf or} $b_{i+1} < f_{i+1}$}
\STATE $H' = {\bf GetPhase}(P_{p_i}, d_i)$, $d_i = d_i + 1$
\STATE $f_i = \max(f_i, b_i + H'_{c, e})$, $b_i = b_i - H'_{c,
o}$, $b_{i+1} = b_{i+1} + H'_{c, u}$
\ENDWHILE
\ENDFOR

\STATE {\bf Compute initialization stages:}
\STATE Let $u$ represent minimum number of executions of each
child
\STATE $u = d$, $d = 0$, $b = 0$
\STATE $stages = \emptyset$
\WHILE{$\neg \forall i, u_{i} \le 0$}
\STATE{\bf Compute a stage}
\STATE Let $H$ be the stage I'm about to construct
\STATE Let $m$ be the number of phase executions that each child
will perform in this stage
\STATE $H = \{p, \emptyset, [0,0,0]\}$, $b_0 = 0$, $b_{n_p} = 0$, $m = 0$
\STATE $need = 1$
\FOR{$i = n_p-1$ downto $0$}
\STATE $nextNeed = 0$
\WHILE{$need > 0$}
\STATE $H' = {\bf GetPhase}(P_{p_i}, d_i + m_i)$, $m_i = m_i + 1$, $u_i = u_i - 1$
\STATE $need = need - H'_{c, u}$, $nextNeed = \max(nextNeed, H'_{c, e} - b_i)$
\STATE $b_i = b_i - H'_{c, o}$, $b_{i+1} = b_{i+1} + H'_{c, u}$
\ENDWHILE
\STATE $need = nextNeed$
\ENDFOR
\STATE $H_{c,e} = need$
\FOR{$i=0$ to $n_p-1$}
\WHILE{\bf true}
\STATE $H' = {\bf GetPhase}(P_{p_i}, d_i + m_i)$
\IF{$H'_{c, e} \le b_i$}
\STATE $m_i = m_i + 1$, $u_i = u_1 - 1$
\STATE $b_i = b_i - H'_{c, o}$, $b_{i+1} = b_{i+1} + H'_{c, u}$
\ELSE
\STATE {\bf break}
\ENDIF
\ENDWHILE
\ENDFOR
\STATE $H_{c,o} = -b_0$, $H_{c,u} = b_{n_p}$
\STATE{\bf Create the stage:}
\FOR{$i=0$ to $n_p-1$}
\WHILE{$m_i \ne 0$}
\STATE $H_A = H_A \circ {\bf GetPhase}(P_{p_i}, d_i)$
\STATE $d_i = d_i + 1$, $m_i = m_i - 1$
\ENDWHILE
\ENDFOR
\STATE $phases = phases \circ H$
\ENDWHILE
\STATE return $\{phases, b, d\}$
\end{algorithmic}
\end{algorithm}

\begin{algorithm}
\label{alg:min-lat-steady-pipeline} \caption{Return a set of
phases that execute a Minimum-Latency steady state schedule for a
{\pipeline}} {\bf MLSteadyStagesPipeline} $(T_p, b^{init}_p,
d^{init}_p)$. $T_p$ is the steady state for the {\pipeline}.
$b^{init}_p$ is the amount of data stored between children of the
{\pipeline} after initialization, $d^{init}_p$ is the number of
phases and stages executed by the initialization schedule.
\begin{algorithmic}
\STATE Let $b$ vector represent amount of data stored between children of
$p$
\STATE Let $d$ vector represent phase/stage being currently executed; 0
represents first initialization stage.
\STATE Let $u$ be number of phase executions for each child for a full steady state execution of the {\pipeline}.

\STATE $b = b^{init}_p$, $d = d^{init}_p$, $\forall i, u_i = T_{p, m, i} * |P_{p, S, A}|$
\STATE {\bf Compute steady state:}
\STATE $phases = \emptyset$
\WHILE{$u_{n_p -1} \ne 0$}
\STATE{\bf Compute a phase}
\STATE Let $H$ be the phase I'm about to construct
\STATE Let $m$ be the number of phase executions that each child
will perform in this phase
\STATE $H = \{p, \emptyset, [0,0,0]\}$, $b_0 = 0$, $b_{n_p} = 0$, $m = 0$
\STATE $need = 1$
\FOR{$i = n_p-1$ downto $0$}
\STATE $nextNeed = 0$
\WHILE{$need > 0$ {\bf and} $u_i \ne 0$}
\STATE $H' = {\bf GetPhase}(P_{p_i}, d_i + m_i)$, $m_i = m_i + 1$, $u_i = u_i - 1$
\STATE $need = need - H'_{c, u}$, $nextNeed = \max(nextNeed, H'_{c, e} - b_i)$
\STATE $b_i = b_i - H'_{c, o}$, $b_{i+1} = b_{i+1} + H'_{c, u}$
\ENDWHILE
\STATE $need = nextNeed$
\ENDFOR
\STATE $H_{c,e} = need$
\FOR{$i=0$ to $n_p-1$}
\WHILE{$u_i \ne 0$}
\STATE $H' = {\bf GetPhase}(P_{p_i}, d_i + m_i)$
\IF{$H'_{c, e} \le b_i$}
\STATE $m_i = m_i + 1$, $u_i = u_1 - 1$
\STATE $b_i = b_i - H'_{c, o}$, $b_{i+1} = b_{i+1} + H'_{c, u}$
\ELSE
\STATE {\bf break}
\ENDIF
\ENDWHILE
\ENDFOR
\STATE $H_{c,o} = -b_0$, $H_{c,u} = b_{n_p}$
\STATE{\bf Create a phase:}
\FOR{$i=0$ to $n_p-1$}
\WHILE{$m_i \ne 0$}
\STATE $H_A = H_A \circ {\bf GetPhase}(P_{p_i}, d_i)$
\STATE $d_i = d_i + 1$, $m_i = m_i - 1$
\ENDWHILE
\ENDFOR
\STATE $phases = phases \circ H$
\ENDWHILE
\STATE return $phases$
\end{algorithmic}
\end{algorithm}

\begin{algorithm}
\label{alg:min-lat-pipeline} \caption{Return a min-latency phasing
schedule for a {\pipeline}} {\bf MLPipeline}().
\begin{algorithmic}
\STATE $T = {\bf SSPipeline} ()$
\STATE $\{I, b^{init}, d^{init}\} = {\bf MLInitStagesPipeline}()$
\STATE $H = {\bf MLSteadyPhasesPipeline}(T, b^{init}, d^{init})$
\STATE return ${\bf MakePhases} (H, I)$
\end{algorithmic}
\end{algorithm}

\section{Old Latency}

\subsubsection{\filters}

The transfer function for amount of information carried by an item
of data before and after a {\filter} is quite simple.  The amount of
information consumed and produced by the {\filter} needs to remain
constant.  Using notation for a {\pipeline} $p$, the transfer
function for a {\filter} is $info_{p_n} = info_{p_{n-1}} * {o_{p_n}
\over u_{p_n}}$.

\subsubsection{{\splitters} and {\joiners}}

The transfer function for the amount of information carried by an
item of data through a {\splitter} or a {\joiner} is also very simple,
however it is not very intuitive, compared to the {\filter} transfer
function. An intuitive function would preserve the amount of
information carried across a {\splitter} or a {\joiner}, and split or
merge the information appropriately across all branches.  Such an
approach, however, can result in a situation where a {\joiner} is
joining data items with different amount of information carried in
different branches.  This type of a situation is difficult to
handle, because for simplicity all data items in a single buffer
should carry exactly the same amount of information.

The approach that solves the problem above is to not preserve the
amount of information across all branches when data is being split
or joined. Instead, every branch (including the input to a
{\splitter} or output of a {\joiner}) consumes or produces the same
amount of information on every iteration.

Using the {\splitjoin} notation, amount information per data item
transferred to $n$th branch of a {\splitter} is $info_{split_n} =
info_{input} * {o_{split} \over u_{split,n}}$, and the amount of
information per data item transferred from $n$th branch of a
{\joiner} is $info_{output} = info_{join_n} * {o_{join,n} \over
u_{join}}$.  The same equations will work for {\feedbackloops}, with
the property that going through a loop will preserve the amount of
information carried by a data item.

\subsubsection{checking for messages}

 by a {\filter} $f_0$ to {\filter} $f_1$. that consumes
(and thus produces) $info_{f_0}$ amount of information, and
delivered to {\filter} $f_1$ that consumes $info_{f_1}$ amount of
information on every execution of its {\work} function, then the
destination {\filter} must check for delivery of new at least
messages every $\left \lfloor {(l_1-l_0)
* info_{f_0}} \over info_{f_1} \right \rfloor$ firings of its
{\work} function. If this equation yields 0, {\filter} $f_1$ must
check for new messages before every firing of its {\work} function.

%  -- this is minor, and I don't know how to explain it well
%Note, that this frequency of checking for messages to deliver may
%not be sufficient to satisfy the latency requirements.  If the
%schedule enforces very tight information buffering (very close to
%absolute minimums or maximums, as explained below), it is possible
%that the destination {\filter} needs to check for message delivery
%more often.  This effect is schedule specific, and needs to be
%computed on a case-by-case basis.
%

\subsubsection{information and latency}

Next, the amount of information buffered between the source and
the destination of a message needs to be known.  As explained
earlier, there are three types of messages.  The easiest type of
message to handle in terms of buffering is the downstream positive
delay message.  Those messages impose no information buffering
requirements on the schedule.  This is because the information
wavefront that the message needs to be delivered with cannot
possibly have passed the destination filter - it cannot even have
passed the source filter.

The next type of message to consider is the downstream negative
delay message.  A downstream negative delay message will be
delivered to the destination \emph{before} the current source
{\filter} information wavefront reaches the destination {\filter}.
 This type of a message generates a minimal buffering requirement
on the schedule.  There is no maximal requirement, because if the
amount of information buffered up   between the source and
destination is large, the message will not be delivered on the
first iteration of checking for messages to be delivered, but on
some subsequent iteration.  At the minimum, the message must be
delivered just before the minimum latency wavefront reaches the
destination {\filter}.  Just after the destination {\filter} is
executed, the amount of data buffered up is reduced by
$info_{dest}$.  This is the situation when the minimal amount of
information is stored between the source and destination.  Thus,
the amount of information stored between the two {\filters} should
be at least $info_{src} * (-l_0) + info_{dest} {e_{dest} -
o_{dest} \over o_{dest}} - \min (info_{src}, info_{dest})$.

The case of upstream positive delay message is very similar to
downstream negative delay message.  The destination {\filter} must
receive the message before it produces the minimal allowed
information wavefront.  Thus there must be less than $info_{src}
* (l_1 + {e_{src} - o_{src} \over o_{src}})$ information
between the destination and source {\filters}.

The last step required for a complete framework relating latency
and information is to compute the amount of information between
two {\filters}.  This task is actually quite simple.  Given two
{\filters}, $f_0$ and $f_1$, the algorithm selects the upstream
{\filter}, and follows any non-cyclic downstream path towards the
other {\filter}.  Amount of information stored along the path
followed is summed up, and represents the information stored
between the two {\filters}.  Note, that this algorithm will follow
the body path of {\feedbackloops}, and select any branch of a
{\splitjoin}.  The reason this algorithm works is because {\filters}
do not destroy or create information when their {\work} functions
execute, and because amount of data stored along any branch of a
{\splitjoin} is the same.


\bibliographystyle{plain}
\bibliography{references}

\appendix
\section{Details of Calculating Minimal Steady States}
\label{apx:eqs}

This appendix presents equations used for calculating minimal
steady states.  Minimal steady states are calculated recursively
in a hierarchical manner. That is, a minimal steady state is
calculated for all children streams of {\pipeline}, {\splitjoin}
and {{\feedbackloop}}, and then the schedule is computed for the
actual parent stream using these minimal states as atomic
executions. This yields a minimal steady state because all child
streams must execute their steady states (to avoid buffering
changes), and all steady states are multiples of the minimal
steady states (per Theorem \ref{thm:multiplicity}).  Executing a
full steady state of a stream is referred to as ``executing a
stream.''

\begin{figure}
\begin{center}

\begin{minipage}{1.5in}
\centering \psfig{figure=pipeline-steady-state.eps,width=0.6in} \\
{\protect\small (a) A sample {\pipeline}}
\end{minipage}
~
\begin{minipage}{1.5in}
\centering \psfig{figure=splitjoin-steady-state.eps,width=1.2in} \\
{\protect\small (b) A sample {\splitjoin}}
\end{minipage}
~
\begin{minipage}{2in}
\centering \psfig{figure=feedback-steady-state.eps,width=1.0in} \\
{\protect\small (c) A sample {{\feedbackloop}}.  The $L$ {\filter}
has been flipped upside-down for clarity.\\$peek_L = pop_L = 5,
push_L = 6$}
\end{minipage}

\caption{Sample {\StreamIt} streams. These are identical to
streams in Figure \ref{fig:steady-state}, except for the
{\pipeline}.} \label{fig:app:steady-state}

\end{center}
\end{figure}

\subsection{\filter}

Since {\filters} do not have any internal buffering, their minimal
steady state is to execute the {\filter}'s {\work} function once.
This is the smallest amount of execution a {\filter} can have.

Thus, for a {\filter} $f$,

\begin{displaymath}
S_f = \left\{[1], \{f\}, { \left[
\begin{array}{c}e_f\\o_f\\u_f
\end{array}
\right]}, [] \right\}
\end{displaymath}

Notice that $S_{f,v}$ is empty, because a {\filter} does not have
any children.

\subsection{\pipeline}

Let the {\pipeline} $p$ have $n$ children and let $p_i$ denote the
$i$th child of the {\pipeline} (counting from {\Input} to
{\Output}, starting with 0, the children may be streams, not
necessarily {\filters}). We must find $S_p$.

We start with calculating all $S_{p_i}, i \in \{0, \dots, n-1\}$.
This task is achieved recursively.

Next we find a fractional vector $v''$ such that executing each
$p_i$ $v_i''$ times will not change the amount of data buffered in
the {\pipeline} and the first child is executed exactly once.
Since the children streams are executed fractional amount of
times, we calculate the amount of data they produce and consume
during this execution by multiplying $S_{p_i,c_o}$ and
$S_{p_i,c_u}$ by $v_i''$. Thus $v''$ must have the following
property

\begin{displaymath}
v_0'' = 1, \forall i \in \{0,\dots,n-2\}, v_i'' * u_{p_i} =
v_{i+1}'' * o_{p_{i+1}}
\end{displaymath}

We compute $v''$ as follows.  The first child executes once, thus
$v_0'' = 1$.  The second child must execute $v_1'' = {u_{p_0}
\over {o_{p_1}}}$ times to ensure that all data pushed on the the
first {{\Channel}} is consumed by the second child.  The third
child must execute $v_2'' = v_1'' {u_{p_1} \over o_{p_2}} =
{u_{p_0} \over o_{p_1}} {u_{p_1} \over o_{p_2}}$ times to ensure
that it consumes all the data produced by the second child. Thus,

\begin{displaymath}
\forall i \in \{1,\dots,n-1\}\ v_i'' = {\prod_{j = 0}^{i-1}
u_{p_j} \over \prod_{j=1}^i o_{p_j}}
\end{displaymath}

Next we will find an integral vector $v'$ such that executing each
$p_i$ $v_i'$ times will not change the amount of data buffered in
the {\pipeline}.  $v'$ will be a valid steady state of the
{\pipeline}.

In order to calculate $v'$ we multiply $v''$ by $\prod_{j=1}^{n-1}
o_{p_j}$.  Thus

\begin{displaymath}
v'_i = \left({\prod_{j = 0}^{i-1} u_{p_j} \over \prod_{j=1}^i
o_{p_j}} \right) \left(\prod_{j=1}^{n-1} o_{p_j} \right) = \left(
\prod_{j=0}^{i-1} u_{p_j} \right) \left( \prod_{j=i+1}^{n-1}
o_{p_j} \right)
\end{displaymath}

Now we find an integral vector $v$, such that, for some positive
integer $g$, $v' = g * v$, and $\sum_i v_i$ is minimal.  In other
words, we find the greatest integer $g$, such that $v' = g * v$,
with $v$ consisting of integers.  $v$ represents the minimal
steady state for pipeline $p$.

This is achieved by finding the $\gcd$ of all elements in $v'$,
and dividing $v'$ by $g$.  Thus

\begin{displaymath}
v = {v' \over \gcd(v'_0,\dots,v'_{n-1})}
\end{displaymath}

$v$ represents the number of times each child of $p$ will need to
execute its steady state in order to execute the minimal steady
state of $p$, thus $S_{p,v} = v$.  $v$ holds a steady state
because amount of data buffered in $p$ does not change, and it is
a minimal steady state, because $\sum_i v_i$ is minimal.

We construct set $S_p$ as follows:\footnote{Here we use symbol
$\circ$ to denote concatenation of vectors and sets.  Thus $[1\ 2\
3] \circ [4\ 5\ 6] = [1\ 2\ 3\ 4\ 5\ 6]$ and $\{A\ B\ C\} \circ
\{D\ E\ F\} = \{A\ B\ C\ D\ E\ F\}$.}

\begin{displaymath}
S_p = \left\{ \begin{array}{c} v_0 * S_{p_0,m} \circ \dots \circ
v_{n-1}
* S_{p_{n-1}, m}, S_{p_0, N} \circ \dots \circ S_{p_{n-1}, N}, \\
\left[
\begin{array}{c}
e_{p_0} + (v_0 - 1) * o_{p_0} \\
v_0 * o_{p_0} \\
v_{n-1} * u_{p_{n-1}}
\end{array}\right], v \end{array} \right\}
\end{displaymath}

An example is presented in Figure \ref{fig:app:steady-state} (a).
For this {\pipeline}, we have the following steady states for all
children of the {\pipeline}:

\begin{displaymath}
\begin{array}{lrlr}
S_A = & \left\{[1], \{A\}, { \left[
\begin{array}{c} 1 \\ 1 \\ 3
\end{array}
\right]}, [] \right\}, &

S_B = & \left\{[1], \{B\}, { \left[
\begin{array}{c} 3 \\ 2 \\ 3
\end{array}
\right]}, [] \right\} \\ \\

S_C = & \left\{[1], \{D\}, { \left[
\begin{array}{c} 2 \\ 2 \\ 1
\end{array}
\right]}, [] \right\}, &

S_D = & \left\{[1], \{D\}, { \left[
\begin{array}{c} 5 \\ 3 \\ 1
\end{array}
\right]}, [] \right\} \\

\end{array}
\end{displaymath}

Using the steady states above, we get the following vector $v'$:

\begin{displaymath}
v' = \left[
\begin{array}{c}
(2 * 2 * 3)\\
(3) (2 * 3) \\
(3 * 3) (3) \\
(3 * 3 * 1)
\end{array}
\right] = \left[
\begin{array}{c}
12\\ 18\\ 27\\ 9
\end{array}
\right]
\end{displaymath}

We now calculate $g = \gcd(v') = \gcd(12,18,27,9) = 3$.  We thus
have

\begin{displaymath}
v = {v' \over 3} = {1 \over 3} \left[
\begin{array}{c}
12\\ 18\\ 27\\ 9
\end{array}
\right] = \left[
\begin{array}{c}
4\\ 6\\ 9\\ 3
\end{array}
\right]
\end{displaymath}

Finally, we construct $S_p$:

\begin{displaymath}
S_p = \left\{
\begin{array}{c}
4 S_{A,m} \circ 6 S_{B,m} \circ 9 S_{C,m} \circ 3
S_{D,m}, S_{A,N} \circ S_{B,N} \circ S_{C,N} \circ S_{D,N} \\
\left[
\begin{array}{c}
1 + (4-1) * 1 \\
4 * 1 \\
3 * 1
\end{array}\right],
\left[ \begin{array}{c} 4\\ 6\\ 9\\ 3 \end{array} \right]
\end{array}
\right\}
\end{displaymath}

\subsection{\splitjoin}

Let the {\splitjoin} have $n$ children and let $sj_i$ denote the
$i$th child of the {\splitjoin} (counting from left to right,
starting with 0).  Let $sj_s$ and $sj_j$ denote the {\splitter}
and the {\joiner} of the {\splitjoin}, respectively. Let $w_{s,i}$
denote the number of items sent by the {\splitter} to $i$th child
on {\splitter}'s every execution. Let $w_{j,i}$ denote the number
of items consumed by the {\joiner} from the $i$th child on
{\joiner}'s every execution.  We are computing $S_{sj}$.

We start by calculating all $S_{sj_i}, i \in \{0, \dots, n-1\}$.

Next we compute a fraction vector $v''$ and a fraction $a_j''$
such that executing the {\splitter} exactly once, each child
$sj_i$ $v_i''$ times and the {\joiner} $a_j''$ times does not
change the amount of data buffered on any {\Channel} in the
{\splitjoin}. Again, since $v''$ and $a_j''$ are fractions, we
multiply the steady-state pop and push amounts by appropriate
fractions to obtain the amount of data pushed and popped.  For
convenience we define $a_s''$ to be the number of executions of
the {\splitter} and set it to 1.

\begin{comment}
\begin{displaymath}
v'', a_j'', a_s'' \ne 0, \forall i \in \{0,\dots,n-1\}, a_s'' *
w_{s, i} = v_i'' * o_{sj_i}, v_i'' * u_{sj_i} = a_j'' * w_{j, i}
\end{displaymath}
\end{comment}

We thus have that each child $sj_i$ must execute $v_i'' = {w_{s,i}
\over o_{sj_i}}$ times. To compute the number of executions of the
{\joiner}, $a_j''$, we select an arbitrary $k$th child ($0 \le k <
n$) and have that the {\joiner} executes $a_j'' = {{w_{s,k} \over
o_{s_k}}{u_{sj_k} \over w_{j,k}}}$ times.

Next we compute integer vector $v'$ and integers $a_s$ and $a_j$
such that executing the {\splitter} $a_s$ times, each child $sj_i$
$v_i'$ times and the {\joiner} $a_j$ times still does not change
the amount of data buffered on any {\Channel} in the {\splitjoin}.
We do this by multiplying $a_s''$, $v''$ and $a_j''$ by $w_{j,k}
\left(\prod_{r=0}^{n-1}o_{sj_r}\right)$. Thus we get

\begin{displaymath}
\begin{array}{rl}
a_s' = & w_{j,k} \left(\prod_{r=0}^{n-1}o_{sj_r}\right) \\
v_i' = & w_{j,k} \left(\prod_{r=0}^{n-1}o_{sj_r}\right) * {w_{s,i}
\over o_{sj_i}} = w_{s,i} * w_{j_k} \left( \prod_{r=0}^{i-1}
o_{s_r} \right) \left( \prod_{r=i+1}^{n-1} o_{s_r} \right)
\\
a_j' = & w_{j,k} \left(\prod_{r=0}^{n-1}o_{sj_r}\right) *
{{w_{s,k} \over o_{s_k}}{u_{sj_k} \over w_{j,k}}} = w_{s,k} *
u_{sj_k} * \left( \prod_{r=0}^{k-1} o_{s_r} \right)
\left( \prod_{r=k+1}^{n-1} o_{s_r} \right) \\
\end{array}
\end{displaymath}

Now we use $v'$, $a_s'$ and $a_j'$ to compute minimal steady state
of the {\splitjoin}.  Since $v'$, $a_s'$ and $a_j'$ represent a
steady state, they represent a strict multiple of the minimal
steady state.  Thus we find the multiplier by computing $g$, the
$\gcd$ of all elements in $v'$ and integers $a_s'$ and $a_j'$, and
dividing $v'$, $a_s'$ and $a_j'$ by $g$.  We have that

\begin{displaymath}
\begin{array}{rl}
g = & \gcd(v', a_s', a_j') \\
v = & v' \over g \\
a_s = &  a_s' \over g \\
a_j = & a_j' \over g
\end{array}
\end{displaymath}

Finally, we use $v$, $a_s$ and $a_j$ to construct $S_{sj}$:

\begin{displaymath}
S_{sj} = \left\{
\begin{array}{c}
v_0 * S_{sj_0,m} \circ \dots \circ v_{n-1} * S_{sj_{n-1}, m} \circ
\left[\begin{array}{c}a_s\\a_j\end{array}\right] , \\
S_{sj_0, N} \circ \dots \circ S_{sj_{n-1}, N} \circ \{sj_s,
sj_j\},
\\ \left[
\begin{array}{c}
n_s * o_{s} \\
n_s * o_{s} \\
n_j * u_{j} \\
\end{array}\right], \\
v \circ [a_s] \circ [a_j]
\end{array}\right\}
\end{displaymath}

Figure \ref{fig:steady-stat}e (b) depicts a sample {\splitjoin}.
The following are the steady states of the {\splitjoin}'s
children: $$
\begin{array}{lrlr} S_A = & \left\{[1], \{A\}, { \left[
\begin{array}{c} 2 \\ 2 \\ 1
\end{array}
\right]}, [] \right\}, & S_B = & \left\{[1], \{B\}, { \left[
\begin{array}{c} 3 \\ 2 \\ 6
\end{array}
\right]}, [] \right\}
\end{array}
$$ For this {\splitjoin}, we select $k = 0$ (we use the left-most child
to compute $a_j'$).  We get the following $v'$, $a_s'$ and $a_j'$

\begin{displaymath}
\begin{array}{rl}
v' = & \left[
\begin{array}{c}
2 * 2 (2)\\
1 * 2 (2)
\end{array}
\right] = \left[
\begin{array}{c}
8 \\ 4
\end{array}
\right] \\
a_s' = & 1 * 2 (2 * 2) = 8 \\
a_j' = & 2 * 1 (2 * 2) = 8
\end{array}
\end{displaymath}

Thus $\gcd(u', a_s', a_j') = \gcd(8,4,8,8) = 4$.  Now we obtain

\begin{displaymath}
\begin{array}{rl}
v = & {v \over 4} = {1 \over 4} \left[
\begin{array}{c}
8 \\ 4
\end{array}
\right] =  \left[
\begin{array}{c}
2 \\ 1
\end{array}
\right]\\
a_s = & {a_s' \over 4} = {8 \over 4} = 2 \\
a_j' = & {a_j' \over 4} = {8 \over 4} = 2
\end{array}
\end{displaymath}

Finally, we construct $S_{sj}$:

\begin{displaymath}
S_{sj} = \left\{
\begin{array}{c}
2 * S_{sj_0, m} \circ 1 * S_{sj_1, m} \circ \left[\begin{array}{c}2\\2\end{array}\right], \\
S_{sj_0, N} \circ S_{sj_1, N} \circ \{sj_s, sj_j\}, \\
\left[
\begin{array}{c}
2 * 3 \\ 2 * 3 \\ 2 * 4
\end{array}
\right], \left[
\begin{array}{c}
2 \\ 1 \\ 2 \\ 2
\end{array}\right]
\end{array} \right\}
\end{displaymath}

\begin{figure}\begin{center}
\begin{minipage}{1in}
\centering \psfig{figure=splitjoin-illegal.eps,width=2in}
\end{minipage}
\end{center}
\caption{An illegal {\splitjoin}} \label{fig:splitjoin-illegal}
\end{figure}

It is important to note, that it is not always possible to compute
a unique $v''$ for all possible {\splitjoins}. The reason is that
unbalanced production/consumption ratios between different
children of a {\splitjoin} can cause data to buffer up infinitely.

\begin{definition}[Valid {\splitjoin}] A {\splitjoin} is valid
{\emph iff} $\forall k, 0 \le k < n-1, o_{sj_k}, u_{sj_k},
o_{sj_{k+1}}, u_{sj_{k+1}} \ne 0$ we have $a_{j,k}'' =
a''_{j,k+1}$, using notation of $a_{j,k}''$ to indicate that $k$th
child of the {\splitjoin} was used to compute the value of
$a_j''$.
\end{definition}

An example of an illegal {\splitjoin} is depicted in Figure
\ref{fig:splitjoin-illegal}.  The rates of throughput of data for
the left child mean that for every execution of the {\splitter},
the {\joiner} needs to be executed exactly once to drain all data
entering the {\splitjoin}.  The rates of throughput of data for
the right child mean that for every execution of the {\splitter},
the {\joiner} needs to be executed exactly twice to drain all data
entering the {\splitjoin}. That means that consumption of data by
the {\joiner} will be relatively slower on the right side, causing
data to buffer up. This means that the given {\splitjoin} does not
have a steady state.

If a {\splitjoin} is such that it does not have a steady state, it
is considered an illegal {\splitjoin}.  It cannot be executed
repeatedly without infinite buffering, so a practical target for
{\StreamIt} cannot execute it.  The calculations presented here
assume that the {\splitjoin} is legal.  In order to check if a
given {\splitjoin} is legal, we test if selecting a different
child for calculation of $a_j''$ yields a different $a_j''$. If it
does, then the two paths tested have different
production/consumption rates, and the {\splitjoin} does not have a
steady state.

\subsection{{\feedbackloop}}

Let {{\feedbackloop}} $fl$ have children $B$ (the body child) and
$L$ (the feedback loop child). Let the {\joiner} and the
{\splitter} of the {{\feedbackloop}} be denoted $fl_j$ and $fl_s$.
Let $w_{j,I}$ and $w_{j,L}$ denote the number of data items
consumed by the {\joiner} from the {\Input} {{\Channel}} to the
{{\feedbackloop}} and from $fl_L$, respectively.  Let $w_{s,O}$ and
$w_{s,F}$ denote the number of data items pushed by the
{\splitter} onto the {{\feedbackloop}}'s {\Input} {{\Channel}} and
to $fl_L$ respectively.  We are computing $S_{fl}$.

First we calculate $S_{B}$ and $S_{L}$.

Now we compute a fractional vector $v'' = [a_B''\ a_L''\ a_s''\
a_j'']$ such that executing the body child $a_B''$ times, the
{\splitter} $a_s''$ times, the loop child $a_L''$ times and the
{\joiner} $a_j''$ times will not change the amount of data
buffered up in any {\Channel} in the {{\feedbackloop}}.  Thus

\begin{displaymath}
\begin{array}{rcl}
a_B'' * u_B & = & a_s'' * o_s \\
a_L'' * u_B & = & a_j'' * w_{j, L} \\
a_s'' * w_{s, F} & = & a_L'' * o_B \\
a_j'' * u_j & = & a_B'' * o_B \\
\end{array}
\end{displaymath}

We begin with setting $a_j'' = 1$. $B$ needs to be executed $a_B''
= {u_j \over o_B}$ times, the {\splitter} needs to be executed
$a_s'' = {u_j \over o_B}{u_B \over o_s}$ times and $L$ needs to be
executed $a_L'' = {u_j \over o_B}{u_B \over o_s}{w_{s,L} \over
o_L}$ times. Furthermore, in order to assure that the
{{\feedbackloop}} has a valid steady state, we continue going
around the loop, the {\joiner} must require ${u_j \over o_B}{u_B
\over o_s}{w_{s,L} \over o_L}{u_L \over w_{j,L}} = 1$.  If this
condition is not satisfied, the {{\feedbackloop}} does not have a
steady state. This is a necessary, but not a sufficient condition
for a {{\feedbackloop}} to be valid.

Next we compute an integer vector $v' = [a_B'\ a_L'\ a_s'\ a_j']$
such that executing B $a_B'$ times, {\splitter} $a_s'$ times, L
$a_L'$ times and {\joiner} $a_j'$ times will not change the amount
of data buffered in the {\splitjoin}. We do this by multiplying
$v''$ by $o_B * o_s * o_L$.

\begin{displaymath}
\begin{array}{rl}
a_B' = & u_j * o_s * o_L \\
a_L' = & u_j * u_B * w_{s,L} \\
a_j = & o_B * o_s * o_L \\
a_s = & u_j * u_B * o_L
\end{array}
\end{displaymath}

We now use $v'$ to compute $v = [a_B\ a_L\ a_s\ a_j]$, a minimal
steady state for the {{\feedbackloop}}.  We do this by finding an
integer $g$, the $\gcd$ of all elements in $v'$ and computing $v =
{v' \over g}$.

Finally, we construct $S_{fj}$ as follows:

\begin{displaymath}
S_{fj} = \left\{
\begin{array}{c}
a_B * S_{B,m} \circ a_L * S_{L,m} \circ [a_s \ a_j], \\
S_{B,N} \circ S_{L,N} \circ \{fl_s, fl_j\}, \\
\left[\begin{array}{c}
a_j * w_{j,I} \\
a_j * w_{j,I} \\
a_s * w_{s,O}
\end{array} \right], v
\end{array} \right\}
\end{displaymath}

Figure \ref{fig:app:steady-state}(c) depicts a sample
{{\feedbackloop}}. The following are the steady states of the
{\splitjoin}'s children:
$$
\begin{array}{lrlr} S_B = & \left\{[1], \{B\}, { \left[
\begin{array}{c} 2 \\ 2 \\ 1
\end{array}
\right]}, [] \right\}, & S_L = & \left\{[1], \{L\}, { \left[
\begin{array}{c} 5 \\ 5 \\ 6
\end{array}
\right]}, [] \right\}
\end{array}
$$ We compute $v'$ for this {{\feedbackloop}}:

\begin{displaymath}
v' = \left[
\begin{array}{c}
5 * 3 * 5 \\
5 * 1 * 3 \\
5 * 1 * 5 \\
2 * 3 * 5
\end{array}\right] = \left[
\begin{array}{c}
75 \\
15 \\
25 \\
30
\end{array}\right]
\end{displaymath}

Thus $g = \gcd(75,15,25,30) = 5$ and

\begin{displaymath}
v = {v' \over 5} = \left[
\begin{array}{c}
15 \\
3 \\
5 \\
6
\end{array}\right]
\end{displaymath}

Finally, we construct $S_{fl}$

\begin{displaymath}
S_{fl} = \left\{
\begin{array}{c}
15 * S_{B, m} \circ 3 * S_{L, m} \circ [5\ 6], \\
S_{B, N} \circ S_{L, N} \circ \{fl_s, fl_j\}, \\
\left[
\begin{array}{c}
6 * 2 \\ 6 * 2 \\ 5 * 3
\end{array}
\right], \left[
\begin{array}{c}
15 \\ 3 \\ 5 \\ 6
\end{array}\right]
\end{array} \right\}
\end{displaymath}


%\section{Diagrams of Test Applications}
\label{apx:apps}

This appendix presents the applications used for testing and
collecting results in this thesis.

There are two formats of Figures in this appendix. CD-DAT, QMF and
the two SJ\_PEEK benchmarks have nodes denoted by ovals. The name
of {\splitters} and {\joiners} indicates their type ({\duplicate} or
{\roundrobin}) and possible splitting or joining amounts (if a
{\roundrobin} {\splitter} or {\joiner} has no numbers, they're all
unity). The name of {\filters} has format $(pop, peek) name (push)$.
{\pipelines} and {\splitjoins} are represented by rectangles, and
their names are given in their top left corner.

The format for the other figures is similar, but the peek, pop and
push amounts for {\filters} is given explicitly.

\begin{figure}
\centering \psfig{figure=bitonic.eps,width=6in} \caption{Diagram
of Bitonic Sort Application}
\end{figure}

\begin{figure}
\centering \psfig{figure=cddat.eps,width=1.5in} \caption{Diagram
of CD-DAT Application}
\end{figure}

\begin{figure}
\centering \psfig{figure=fft.eps,width=3in} \caption{Diagram of
FFT Application}
\end{figure}

\begin{figure}
\centering \psfig{figure=filterbank.eps,width=6in}
\caption{Diagram of Filter Bank Application}
\end{figure}

\begin{figure}
\centering \psfig{figure=fir.eps,width=0.5in} \caption{Diagram of
FIR Application}
\end{figure}

\begin{figure}
\centering \psfig{figure=fm.eps,width=5in} \caption{Diagram of
Radio Application}
\end{figure}

\begin{figure}
\centering \psfig{figure=gsm.eps,width=6in} \caption{Diagram of
GSM Application}
\end{figure}

\begin{figure}
\centering \psfig{figure=3gpp.eps,width=2.5in} \caption{Diagram of
3GPP Application}
\end{figure}

\begin{figure}
\centering \psfig{figure=qmf.eps,width=6in} \caption{Diagram of
QMF Application}
\end{figure}

\begin{figure}
\centering \psfig{figure=radar.eps,width=6in} \caption{Diagram of
Radar Application}
\end{figure}

\begin{figure}
\centering \psfig{figure=sj1024.eps,width=3in} \caption{Diagram of
SJ\_PEEK\_1024 Application}
\end{figure}

\begin{figure}
\centering \psfig{figure=sj31.eps,width=3in} \caption{Diagram of
SJ\_PEEK\_31 Application}
\end{figure}

\begin{figure}
\centering \psfig{figure=vocoder.eps,width=5in} \caption{Diagram
of Vocoder Application}
\end{figure}


\end{document}
