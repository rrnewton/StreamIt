\section{Conclusions and Future Work}
\label{sec:conc}

This paper presents StreamIt, a novel language for high-performance
streaming applications.  Stream programs are emerging as a very
important class of applications with distinct properties from other
recognized application classes.  This paper develops fundamental
programming constructs for the streaming domain.

The primary goal of StreamIt is to raise the abstraction level in
stream programming without sacrificing performance.  We have argued
that StreamIt's mechanisms for filter definition, filter composition,
messaging, and re-initialization will improve programmer productivity
and program robustness within the streaming domain.

Also, we believe that StreamIt is a viable common machine language for
grid-based architectures (e.g., \cite{smartmemories,raw,trips}), just
as C is a common machine language for von-Neumann machines.  StreamIt
abstracts away the target's granularity, memory layout, and network
interconnect, while capturing the notion of independent processors
that communicate in regular patterns.  Communication in StreamIt
corresponds naturally to grid-based machines, as neighbors are
connected by high-bandwidth channels, but non-neighbors can exchange
low-bandwidth messages without relying on a global clock or a
particular network topology.  We are developing fission and fusion
algorithms that can automatically adjust the granularity of a stream
graph to match that of a given target.

We have a number of extensions planned for the next version of the
StreamIt language.  The current version is designed primarily for
uniform one-dimensional data processing, but constructs for
hierarchical frames of data could be useful for image processing.
Moreover, a future version will support dynamically varying I/O rates
of the filters in the stream.  We expect that such support will
require new language constructs--for instance, a type-dispatch
splitter that routes items to the components of a SplitJoin based on
their type, and a fall-through joiner that pulls items from any stream
in a SplitJion as soon as they are produced.

An orthogonal area of future research is to develop a clean high-level
syntax for StreamIt.  The Java-based syntax has many advantages,
including programmer familiarity, availability of compiler frameworks
and a robust language specification.  However, the resulting StreamIt
syntax is cumbersome.

Our immediate focus is on developing a high-performance optimizing
compiler for StreamIt 1.0.  As described in \cite{streamittech622},
the structure of StreamIt can be exploited by the compiler to perform
a wide range of stream-specific optimizations.  Our goal is to match
the performance of hand-coded applications, such that the abstraction
benefits of StreamIt come with no performance penalty.



