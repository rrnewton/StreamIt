\documentclass[10pt]{article}

\usepackage{epsfig}
\usepackage{amsmath}
\usepackage{fullpage}

\begin{document}

  \begin{titlepage}
    \begin{center}
      {\LARGE
	~ \\ ~ \\ ~ \\ ~ \\ ~ \\ ~ \\ ~ \\ ~ \\
	{\bf PCA Morphware Virtual Machine Plan \\ ~ \\}
      }
      {\Large
	June 16, 2003 \\ ~ \\
      }
      {\large
	Georgia Institute of Technology \\
	Massachusetts Institute of Technology \\ Raytheon \\ Reservoir Labs \\ Stanford University \\ University of Southern California \\ ~ \\ ~ \\ 
	Based on earlier revisions by: \\ ~ \vspace{-8pt} \\
	University of Texas at Austin \\ IBM Austin Research Laboratory
      }
    \end{center}
  \end{titlepage}

  \newcommand{\mt}[1]{\mbox{\it #1}}
  \newcommand{\todo}[1]{\framebox{\bf #1}}
  \newcommand{\sss}[1]{\medskip \noindent {\bf #1} \smallskip}
  \newcommand{\ssss}[1]{\medskip \noindent {\bf #1:}}

  \tableofcontents

%%   \clearpage
%%   \section*{Preface}
%%   Copy from UT/IBM proposal.
%%   \section{Introduction}
%%   Copy from UT/IBM proposal.
%%   \section{Overview}
%%   Copy from UT/IBM proposal.

%%   \section{Virtual Machine Metadata Description and Interfaces}
%%   Still in progress.

%%  \clearpage
%%  \newcommand{\taba}[1]{\parbox{2.4in}{ ~ \vspace{-0pt} \\ #1 \vspace{-6pt} \\ }}
\newcommand{\tabb}[1]{\parbox{3.6in}{ ~ \vspace{-0pt} \\ #1 \vspace{-6pt} \\ }}
\newcommand{\mytable}[2]{
  {\small
    \begin{tabular}{|l|l|} \hline
      Attribute Name & #1 \\
      \hline \hline
      #2
    \end{tabular}
  }
}
\newcommand{\enttable}[1]{\mytable{Type - Units / Interpretation}{#1}}
\newcommand{\sumtable}[1]{\mytable{Summary}{#1}}
\newcommand{\justtable}[1]{\mytable{Summary and Justification}{#1}}
\newcommand{\entry}[3]{\taba{#1} & \tabb{#2 \\ #3} \\ \hline}
\newcommand{\summary}[2]{\taba{#1} & \tabb{#2} \\ \hline}
\newcommand{\just}[3]{\taba{#1} & \tabb{#2 \\ {\it #3}} \\ \hline}

\section{Virtual Machine Metadata Description and Interfaces}

An architecture is represented as a directed graph, where each node
represents a processor or memory unit and each edge represents a
uni-directional communication channel.  In the following, we describe
the essential properties of a processor node, a memory node, and a
communication channel.  An outline of the changes from the previous
proposal appears in Figure~\ref{fig:archdiff}.

\begin{figure}[t]
\begin{center}
\framebox[6.5in]{
\begin{minipage}{6in}

\begin{itemize}

\item {\bf Architecture description is a graph rather than a tree.}

\item {\bf Network properties are expressed pairwise,} as edges
between nodes, instead of as an interface at the nodes themselves.

\item {\bf No distinction is made between threaded processors and
streaming processors,} except for a new MASTER property to specify
control processors.

\end{itemize}
\caption{Outline of changes to the metadata description.
  \protect\label{fig:archdiff}}
\end{minipage}}
\end{center}
\end{figure}

\subsection{Processor Nodes}

A processor has a single program counter, as well as its own
instruction memory and address space.  Its work capacity can be
modeled as a superscalar.

\sss{Processor Node Properties (New)}

We introduce the following properties that were not in the previous
proposal: \\

\enttable{

  \entry{VM\_PROP\_PROC\_MASTER} {VM\_NODE\_TYPE\_PROC} {The master
    node that can control this processor as a slave in a batch
    dispatch mode.  If no such node exists, then this processor is its
    own master.  (In the terminology of the previous proposal,
    ``streaming'' processors should have a ``threaded'' master, while
    all ``threaded'' processors should be their own master.)}

  \entry{VM\_PROP\_PROC\_SIMD} {int32 - SIMD/Vector speedup factor.}
    {The amount by which the processor will speed up (over that
    specified by the ILP property) if the computation is data parallel.}

  \entry{VM\_PROP\_PROC\_REGS} {int32 - Number of registers} {The
    number of registers on the processor (possibly to include
    scratchpad or cache space?)}

  \entry {VM\_PROP\_PROC\_DMEMBANK} {VM\_NODE\_TYPE\_MEM} {The default
    memory node in which to hold data for this processor.  Though data
    could possibly be held in several memory nodes, this gives the
    compiler a clue as to the lowest-cost memory node that was
    intended for this processor.}

  \entry{VM\_PROP\_PROC\_IMEMBANK} {VM\_NODE\_TYPE\_MEM} {The default
  memory node in which to hold instructions for this processor.}

  \entry{VM\_PROP\_PROC\_ISIZE} {int32 - Number of words} {The number
    of words required to store a RISC instruction for this processor.}
}

\newpage
\sss{Processor Node Properties (Unmodified)}

We retain the following properties exactly as they appear in the
previous proposal: \\

\sumtable{

  \summary{VM\_PROP\_PROC\_ILP} {ILP capability (e.g., number of ALUs
    on the processor).}

  \summary{VM\_PROP\_PROC\_FREQ} {Processor frequency.}

  \summary{VM\_PROP\_OP\_LATENCY} {Instruction latency for a ``typical''
    instruction in cycles.}

  \summary{VM\_PROP\_PROC\_CONTEXT\_SWITCH} {Expected number of cycles for a
    context switch.}
}

\sss{Processor Node Properties (Removed)}

We remove the following properties that appeared in the previous
proposal: \\

\justtable{

  \just{VM\_PROP\_PROC\_TYPE} {Configurations/instruction sets this
    processor resource supports.} {Removed in favor of the MASTER
    property above.}
  
  \just{VM\_PROP\_PROC\_SPEED} {Speed of the processor (e.g., expected
    MIPS or SPEC for a ``typical'' workload).} {Removed for
    simplification.}
  
  \just{VM\_PROP\_PROC\_DMA\_CHANNELS} {DMA channels on this
    processor.} {Removed because network is encapsulated in
    connections.}

  \just{VM\_PROP\_PROC\_MEMBANKS} {Dedicated memory structures for the
    processor.} {Removed in favor of DMEMBANK and IMEMBANK properties.}

  \just {VM\_PROP\_PROC\_PROC\_NETIFS} {Dedicated network interfaces
    for this processor.}  {Removed because network is encapsulated in
    connections.}

}

\sss{Processor Node Configuration Settings}

As before (for properties that still exist), plus new ``Get''
configurations for the new properties.

\sss{Processor Node Statistics}

As before.

\subsection{Memory Nodes}

As before, except without the SRF node type.  SRF's can be modeled as
``regular'' data memory.

\subsection{Edges}

Each edge of the graph represents a ``connection'' for uni-directional
communication between two nodes.  A connection represents a set of
physical or virtual channels that share a set of properties.

Note that all types of nodes can be connected (i.e., there are
memory-memory, processor-memory, memory-processor, and
processor-processor connections.)  Also note that there can be
multiple connections between two nodes (or between a node and itself)
if there are communication channels with different characteristics.

\sss{Connection Properties}

Each edge in the architectural graph has the following properties: \\

\enttable{

  \entry{VM\_PROP\_CHAN\_TYPE} {VM\_CHAN\_TYPE} {The type of channels
  in this connection.  Must be one of the following:
  VM\_CHAN\_TYPE\_VIRTUAL, meaning that bandwidth is split between
  channels that are active at a given time, or
  VM\_CHAN\_TYPE\_PHYSICAL, meaning that each channel has the same
  bandwidth, independent of how many are currently active.}
  
  \entry{VM\_PROP\_CHAN\_NUM} {int32 - Number of channels} {Maximum \#
  of active channels.  We assume all of these channels are uniform (if
  not, then create another connection between the two nodes.)  }

  \entry{VM\_PROP\_CHAN\_BANDWIDTH} {float - Bandwidth bytes/sec}
  {Bandwidth for one active channel; scaling to multiple channels
  depends on type, as described above.}

  \entry{VM\_PROP\_CHAN\_SETUP\_COST} {float - nanoseconds} {The time
  required to setup a new connection between two nodes.}

  \entry{VM\_PROP\_CHAN\_BLOCKING\_FACTOR} {int32 - Number of words}
  {The minimum number of words that must be sent at once to fully
  utilize the bandwidth.}

  \entry{VM\_PROP\_CHAN\_OUTPUT\_WINDOW} {int32 - Number of words}
  {The maximum number of words that can be in flight before the sender
  blocks.}

  \entry{VM\_PROP\_CHAN\_INPUT\_WINDOW} {int32 - Number of words}
  {The maximum number of words that the receiver can see for
  out-of-order access (i.e., the maximum peek amount at receiver.)}

  \entry{VM\_PROP\_CHAN\_CONTROL} {set of VM\_NODE\_TYPE\_PROC} {A
  list of processors that can control the transfer of data across this
  connection.  This property is needed for the case of memory-memory
  connections where such a processor is not clear from the graph.  For
  processor-memory connections, this property contains only the
  processor connected to this edge; for processor-processor
  connections, this property contains only the source of the edge.}

}


%%   \section{Threaded Virtual Machine API}
%%   Copy from each of the following sections of UT/IBM proposal:
%%   \subsection{Processor}
%%   \subsection{Memory}
%%   \subsection{Network}

  \clearpage
  \clearpage
\section{Processor}

An overview of the changes to the virtual machine API appears in
Figure~\ref{fig:vmdiff}.

\begin{figure}[t]
\begin{center}
\framebox[6.5in]{
\begin{minipage}{6in}

\begin{center}
\underline{Stream Kernel API}
\end{center}

\begin{itemize}

\item {\bf The VM is object-oriented.}  Kernels, streams, and stream
graphs are all represented as objects.  Among other benefits, this
makes explicit which data are local to kernels.

\item {\bf Input and output streams are strictly typed within
kernels.}  This preserves type information and avoids casts that would
have resulted from generic STREAM\_DESCRIPTOR accesses in the previous
proposal.

\end{itemize}

\begin{center}
\underline{Stream Processor API}
\end{center}

\begin{itemize}

\item {\bf Stream objects are annotated with the node in which they
are stored.}  A stream object resides in a given memory or processor
node for its entire lifetime, and is declared with that node.

\item {\bf Static stream operations are represented as an explicit
stream graph,} in which kernels are the nodes and streams are the
edges.  The graph representation exposes parallelism and communication
patterns, and gives scheduling freedom to the low-level compiler.

\item {\bf Memory management and processor-processor 
communication are integrated into the stream graph,} as a set of
pre-defined kernels.  This replaces the previous proposal for
processor memory operations and stream network protocols.

\end{itemize}

\caption{Outline of changes to the virtual machine API.
  \protect\label{fig:vmdiff}}
\end{minipage}}
\end{center}
\end{figure}

\subsection{Threaded Processor API}

No changes.

\subsection{Stream Kernel API}

To simplify the presentation, we describe the stream kernel API before
the stream processor API.

\subsubsection{Overview}

Each kernel is now represented as a C++ object, with the following
components:

\begin{enumerate}

\item A constructor, which receives the input and output streams for
the kernel, as well as any other initialization settings.

\item A {\it work} function that represents the steady-state
execution step.

\item (Optional) A {\it prework} function that is called instead of
{\it work} on the first execution.

\item (Optional) Data members, which represent local kernel data that
are preserved between invocations of {\it work}.

\end{enumerate}

For example, a kernel for a simple amplifier could be as follows:
{\small
\begin{verbatim}
    class AmplifierKernel : Kernel_1_1 <float, float> {
      private int N;

      AmplifierKernel(istream<float> in, ostream<float> out, int _N) : Kernel_1_1(in, out) {
        N = _N
      }

      void work(istream<float> in, ostream<float> out) {
        out.push(in.pop() * N);
      }

      void work_info(istream<float> in, ostream<float> out) {
        KernelInfo.setPop(in, 1);
        KernelInfo.setPush(out, 1);
        KernelInfo.isSIMD();
      }
    }  
\end{verbatim}}
As evident in the example, input and output streams are represented by
objects that support {\tt push} and {\tt pop} operations.  In the
following sections, we describe the API for streams
(Section~\ref{sec:kerstreams}) and the API for kernels
(Section~\ref{sec:kernels}).

\subsubsection{Stream Objects}
\label{sec:kerstreams}

As in the previous proposal, a {\it stream} is a C++ data type that
represents a sequence of items of a given type.  Streams are
instantiated as part of the stream processor API, but their member
functions can only be invoked from the stream kernel API\footnote{Most
of the member functions are in {\tt istream} and {\tt ostream}, which
can only be invoked from the stream kernel API.  However, the member
functions declared in {\tt stream} must be invoked from the stream
processor API instead.}.

In this proposal, each kernel views a stream as either an {\tt
istream} or {\tt ostream}, depending on whether it is using the stream
for input or output.  An {\tt istream} provides {\tt pop}, {\tt peek},
and {\tt eos} functions, while an {\tt ostream} supports {\tt push}
and {\tt push\_eos}, as described below: {\small
\begin{verbatim}
    class base_stream {}

    template <class T>
    class istream : base_stream {
      // peek at item at position <offset> without dequeuing it
      T peek(int offset);

      // pop next item off of the stream
      T pop();

      // whether or not input is empty; that is, whether the next item
      // in the stream is an end-of-stream marker
      boolean eos();
    }

    template <class T>
    class ostream : base_stream {
      // push value onto channel
      void push(T val);

      // push an end-of-stream marker onto the stream, indicating that
      // no more data will be written.  Only needed for "source" kernels
      // that have no input streams but need to indicate completion.
      void push_eos();
    }  
\end{verbatim}}

The processor API uses the {\tt stream} class, which inherits from
both {\tt istream} and {\tt ostream}. It is described in
Section~\ref{sec:procstreams}.

\subsubsection{Kernel Objects}
\label{sec:kernels}

Each kernel is described as a subclass of a basic kernel class, such
as the following:
{\small
\begin{verbatim}
    template <class I1, class I2, class O1>
    class Kernel_2_1 {
      // construct a kernel with the given input and output streams
      Kernel(istream<I1> in1, istream<I2> in2, ostream<O1> out1);

      // steady-state execution step
      virtual void work(istream<I1> in1, istream<I2> in2, ostream<O1> out1);

      // (optional) execution step for first invocation
      void prework(istream<I1> in1, istream<I2> in2, ostream<O1> out1);

      // annotations for work function
      virtual void work_info(istream<I1> in1, istream<I2> in2, ostream<O1> out1);

      // annotations for prework function
      void prework_info(istream<I1> in1, istream<I2> in2, ostream<O1> out1);
    }  
\end{verbatim}}

The {\tt Kernel\_2\_1} base class is named as such because it contains
two input streams and one output stream.  In order to propagate type
information into the work function, the kernel is templated on the
types of the input and output streams.  Though it might be infeasible
to manually define all such base classes, the low-level compiler can
identify each appearance of Kernel\_N\_M and interpret it accordingly.

Each kernel provides a constructor in which the input and output
streams are specified.  These streams are then made available as
parameters to {\tt work} and {\tt prework}.  This is supported
automatically by the low-level compiler; no call to {\tt work} or {\tt
prework} appears in the output of the high-level compiler.

\sss{Annotations}

There are several pieces of information that are available to the
high-level compiler which should be transferred to the low-level
compiler in the form of annotations.  Each annotation takes the form
of a function call to the {\tt KernelInfo} class.  Calls from the {\tt
work\_info} function apply to {\tt work}, while calls from {\tt
prework\_info} apply to {\tt prework}.
{\small
\begin{verbatim}
    class KernelInfo {
      // these functions declare the push, pop, and maximum peek rate
      // of a work or prework function with respect to a given stream.
      static void setPush(base_stream str, int push, boolean static=true);
      static void setPop(base_stream str, int pop, boolean static=true);
      static void setPeek(base_stream str, int maxPeek, boolean static=true);

      // these functions indicate a dynamic rate that is unknown at compile
      // time.  However, they provide an optional estimate of the average
      // runtime rate to help the low-level compiler.
      static void setDynamicPush(base_stream str, float push_estimate=UNKNOWN);
      static void setDynamicPop(base_stream str, float pop_estimate=UNKNOWN);
      static void setDynamicPeek(base_stream str, float peek_estimate=UNKNOWN);

      // indicates that a work or prework function is data-parallel 
      // and fit for SIMD execution
      static void isSIMD();
    }  
\end{verbatim}}

All arguments to annotations must be compile-time constants.  For
example, in a kernel from a MergeSort, the input rates are marked
dynamic (with an estimate of 0.5 items per stream on average) because
they depend on values from the input streams:
{\small
\begin{verbatim}
    class MergeKernel : Kernel_2_1 <int, int> {
      MergeKernel(istream<int> in1, 
                  istream<int> in2, 
                  ostream<int> out) : Kernel_2_1(in1, in2, out) {}

      void work(istream<float> in1, istream<int> in2, ostream<int> out) {
        if (in1.peek(0) < in2.peek(0)) {
          out.push(in1.pop());
        } else {
          out.push(in2.pop());
        }
      }

      void work_info(istream<float> in1, istream<int> in2, ostream<int> out) {
        KernelInfo.setDynamicPush(in1, 0.5);
        KernelInfo.setDynamicPop(in2, 0.5);
        KernelInfo.setPush(out, 1);
      }
    }  
\end{verbatim}}

All streams in {\tt work} and {\tt prework} functions must be
annotated with their input and output rates.  If a peek rate is
omitted, it is assumed to be equal to the pop rate.

\sss{Kernel Restrictions}

Only a subset of C++ is supported from within a kernel; restrictions
are listed in Figure~\ref{fig:restrict}.  Note also that this proposal
does not provide special support for directly accessing streams (e.g.,
in the SRF).  Random access to a stream can be achieved by using the
{\tt peek} operation without any {\tt pop}'s.

\begin{figure}[t]
\framebox[6.5in]{
\begin{minipage}{6in}

\begin{enumerate}

\item No pointers.

\item No dynamic memory allocation.

\item No accesses to global memory.

\item No GOTO statements (all control flow is structured).

\item No recursive functions (all function calls have inline
semantics).

\item No references to templates or objects besides the classes
described in this document.  Further, no creation or casting of
objects within kernels.

\item The kernel constructor must receive all its arguments by value
(not by reference.)  Also, the kernel constructor cannot invoke any
member functions of {\tt istream} or {\tt ostream}.

\item Supported opcodes are only the logical, arithmetic, and boolean
operations found in C (no special-purpose DSP operations).

\item Supported types include 32-bit {\tt float}, 32-bit {\tt int},
16-bit {\tt short}, 8-bit {\tt byte}, {\tt boolean}, arrays with a
fixed (int literal) length, and {\tt struct}'s containing members of
any other type.

\end{enumerate}

\caption{Restrictions on C++ code within kernels.\protect\label{fig:restrict}}
\end{minipage}}
\end{figure}

\subsection{Stream Processor API}

\subsubsection{Overview}
\label{sec:streamover}

As in the previous proposal, a control thread in a restricted subset
of C++ is used to manage stream memory and to supervise the execution
of stream kernels.  However, in this proposal, all static stream
operations are represented by explicit stream graphs, in which kernel
objects are the nodes and stream objects are the edges.  The network
model is also integrated into the graph representation: each stream
object is annotated with the memory bank in which it resides, and
pre-defined network kernels are used to communicate between streams in
different memories.  Dynamic operations and dynamic control flow are
fully supported in the code that constructs stream graphs and
transitions between them.

An example of the stream processor API is as follows:
{\small
\begin{verbatim}
    // declare streams with their size and the location in which they are held
    stream<float> s_raw(1000, MEM1), s1(100, SRF1), s2(100, PROC2), s3(1000, SRF1);

    // set up a graph to do some audio filtering and compression
    Graph compress(new FileReader(s_raw, "input.dat"),
                   new Copy(s_raw, s1),
                   new FIRFilter(s1, s2),
                   new RunLengthEncode(s2, s3));

    // run the kernels on PROC2
    compress.start(PROC2);
    compress.wait();

    // if the output is still too large, run additional compression
    stream<float> s4(1000, SRF1);
    if (s3.length() > SIZE_THRESHOLD) {
      Graph compressMore(new ZipCompress(s3, s4));
      compressMore.start(PROC2);
      compressMore.wait();
    } else {
      Graph copy(new Copy(s3, s4));
      copy.start(PROC2);
      copy.wait();
    }

    // store the result to "output.dat"
    Graph store(new FileWriter(s4, "output.dat"));
    store.start(PROC2);
    store.wait();
\end{verbatim}}

The above code fragment illustrates several aspects of the stream
processor API.  In the following, Section~\ref{sec:procstreams}
describes the processor's view of the {\tt stream} type;
Section~\ref{sec:streamgraph} explains the construction of graphs from
kernels and streams; and Section~\ref{sec:predef} describes a
pre-defined set of kernels for dealing with memory and network
operations (such as the {\tt Copy} kernel above).

\subsubsection{Stream Objects}
\label{sec:procstreams}

The processor API uses the {\tt stream} class, which inherits from
both {\tt istream} and {\tt ostream}:
{\small
\begin{verbatim}
    template <class T> 
    class stream : istream <T>, ostream <T> {
      // make a stream that is buffered in units of <size> and allocated
      // at <address> of memory node <location>.  If <write_once>
      // is true, then <size> also indicates the total size of the stream,
      // and no buffering is necessary (the stream is "blocked")
      stream(int size, VM_NODE_TYPE_MEM location, int address, boolean write_once = false);

      // same as above, for stream held in registers of a processor
      stream(int size, VM_NODE_TYPE_PROC location, boolean write_once = false);

      // returns total number of elements that have been pushed to this
      // stream (this is independent of the size and write_once properties)
      int length();
    }  
\end{verbatim}}

Each {\tt stream} is constructed with a {\tt size}.  The {\tt size}
refers only to the buffering strategy of the stream--that is, how many
elements should be kept live at once before overwriting old items with
new ones.  It is the responsibility of the low-level compiler to
ensure that no items are overwritten before they are consumed.  If the
{\tt write\_once} property is true, then no overwriting is allowed,
and the {\tt size} is also an upper bound on the stream's length at
runtime.

Each {\tt stream} is also constructed with a {\tt location}, which
indicates the memory bank in which the stream is held for its entire
lifetime.  The {\tt address} indicates where the stream begins in the
memory node.  Using the second constructor, a stream can also be
assigned to a processor, which means that it is buffered in the
processor's registers.

The {\tt length} method returns how many items have been written to
the stream from the time it was constructed.  Note that this is
unrelated to the buffer size of the stream.  It is also unrelated to
how many items will be written to the stream in the future; if this
quantity is predictable, then it can be calculated as a function of
the {\tt length} of inactive streams.

\subsubsection{Stream Graphs}
\label{sec:streamgraph}

A stream graph represents a static unit of streaming computation.  It
supports the following interface:
{\small
\begin{verbatim}
    class Graph {
      // construct a graph out of any number of kernels
      Graph(void* kernel1, void* kernel2, ...);

      // Start executing the graph, with the kernels executing on
      // processor <proc>.  It will run as long as possible -- until
      // an input stream is empty.
      void start(VM_NODE_TYPE_PROC proc);

      // Blocking call to wait until the graph is finished executing,
      // or an error occurs.  A status/error code is returned.  (Can 
      // only be called after "start").
      int wait();

      // Non-blocking call to check the status of the execution.  Returns
      // a status/error code.
      int check();
    }
\end{verbatim}}

A {\tt Graph} is constructed as a set of kernels.  The connectivity of
these kernels is implicit in the stream objects that are shared
between the input of one kernel and the output of another.  The graph
does not need to be structured (as in StreamIt).  However, no two
kernels in a graph may read from (or write to) the same stream object.

Graphs are executed using the {\tt start} method, which substitutes for
the KERNELLOAD and KERNELSTART functions in the previous proposal.
The argument to {\tt start} specifies which stream processor should
execute the arithmetic operations of the graph; note that this can be
the same processor that is running the control code.

The {\tt start} method runs each kernel ``as long as possible''.  For
kernels with static input rates, this means that the kernel is fired
atomically until there are fewer items on an input channel than is
required by the kernel.  For kernels with a dynamic input rate, this
means that the kernel is run until either 1) it attempts to peek or
pop an item beyond the end-of-stream marker, or 2) it executes once
without consuming or producing any items.

The {\tt wait} and {\tt check} methods replace the VM\_DONE\_SYNC and
VM\_DONEQUERY functions of the previous proposal.  Since the {\tt
start} function is asynchronous, these methods are needed to inspect
the status of graph execution.  The {\tt wait} method blocks until the
graph has finished, while {\tt check} returns the status immediately.

\sss{Transitioning Between Graphs}

A {\tt Graph} represents only the static sections of the stream graph.
As in the example of Section~\ref{sec:streamover}, dynamic control
flow can surround the construction of graphs and can predicate their
execution.  In addition, streams from one graph can be reused in
subsequent graphs.  This is essential for carrying over results from
one streaming computation to another.  In the example, streams {\tt
s3} and {\tt s4} are used in multiple graphs.  Streams can also be
declared in a global namespace if they need to be shared between
multiple processors.

It is also possible for the processor API to inspect the public
fields of a kernel following a streaming computation.  This could be
useful in passing parameters to subsequent stream graphs, or for
retrieving a reduction value from inside a kernel.  For instance:
{\small
\begin{verbatim}
    // --- kernel code ---
    class SumKernel : Kernel_1_0 <int> {
      public int sum;

      SumKernel(istream<int> in) : Kernel_1_0 (in) {
        sum = 0;
      }

      void work(istream<int> in) {
        sum += in.pop();
      }

      void work_info(istream<int> in) {
        KernelInfo.setPop(in, 1);
      }
    }

    // --- processor code ---
    stream<int> s1(100, SRF);

    SumKernel sk(s1);
    Graph g(new FileReader(s1, "input.dat"), sk);
    g.start(PROC1);
    g.wait();

    int final_sum = sk.sum;
\end{verbatim}}

There seems to be nothing fundamental to prevent the stream processor
API to mutate kernel fields as well as inspect them, but at this point
we disallow this since (for performance reasons) most mutation should
be done from within the kernel itself.

\sss{Restrictions on Stream Control}

We refer to a toplevel procedure that controls the execution of stream
kernels as a {\it stream control function}.  Generally a stream
control function will be spawned as a thread and assigned to a control
processor from the main thread of an application.

In order to facilitate static analysis in the low-level compiler,
there are a number of restrictions on the code that can appear in a
stream control function.  These are as follows:

\begin{enumerate}

\item The restrictions of stream kernels (see
Figure~\ref{fig:restrict}) also apply to stream control functions,
with the following caveats:

\begin{itemize}

\item Global memory is not directly addressable from stream control
functions, but results can be returned via arguments passed by
reference.

\item Recursive functions are allowed if they don't contain any
reference to a stream, kernel, or graph object.

\end{itemize}

\item No methods of class {\tt istream} or {\tt ostream} can be
invoked from a stream control function.

\item All stream, kernel, and graph variables have a static single
assignment in the program.  This exposes exactly which constructor is
invoked for a given variable name.

\item All streams must be declared with a size that is constant ({\it
i.e.,} the size argument to the stream constructor must be an {\tt
int} literal).

\item All assignments of kernels to processors must be constant ({\it
i.e.,} the processor argument to the {\tt graph.start} method must be a
constant literal).

\item $(*)$ A kernel can appear in only one graph.  That is, on every
execution path, a given kernel object is used as an argument to at
most one {\tt Graph} constructor.

\item $(*)$ Each graph can only be run once.  That is, on every
execution path, a given graph object is the target of at most one {\tt
start} call.

\item $(*)$ Only one graph can be run at a given time on a given
processor.  That is, for each processor $P$, there can be at most one
unfinished graph object that was run with a call of {\tt
start(}$P${\tt )}.

\end{enumerate}

The $(*)$ items represent dynamic properties that are not statically
verifiable by the low-level compiler.  Thus, they represent a contract
that the high-level compiler must respect for the sake of correctness.

\subsubsection{Pre-Defined Kernels}
\label{sec:predef}

This proposal integrates all memory management and network support for
streams into the graph model of the previous section.  This is done
using pre-defined kernels that can connect streams in different
memories, or route streams to network channels.

\subsubsection*{Memory Management Kernels}

The memory management kernels correspond directly to one or more of
the functions that the previous proposal included in the stream
processor API.

\ssss{Copy} The {\tt Copy} kernel substitutes for the load, store, and move
functions of the previous proposal.
{\small
\begin{verbatim}
    template <class I1, class O1>
    class Copy : Kernel_1_1 <I1, O1> {
      Copy(istream<I1> src_str, ostream<O1> dest_str, int length = ENTIRE_STREAM,
           int src_offset = 0, int dest_offset = 0, int record_size = 1, int stride = 1);
    }
\end{verbatim}}

The primary use of this kernel is for copying items between streams in
different memory banks, though it can also be used for copying between
streams in a single memory.  However, it can only copy across one
connection--in the architectural graph, there must be an edge from the
location of the input stream to the location of the output stream.
(When the source stream is located in main memory, this corresponds to
the previous notion of a load; when the source stream is located in
the SRF, this corresponds to the previous notion of a store.)

As in the previous proposal, the {\tt record\_size} indicates the
number of words in each data record, and the {\tt stride} indicates
the separation between records in the source stream.  The {\tt length}
indicates how many words should be transferred; by default, all
elements in the {\tt src\_str} are copied.  The {\tt src\_offset} and
{\tt dest\_offset} arguments can be used for shifting all accesses by
an offset in the source or destination streams.  

We omit the optional {\tt \_op} allowed by the previous proposal (for
in-place memory updates), because we believe that the new structure of
the stream graph will clearly expose these optimization opportunities.

\ssss{Scatter/Gather} The {\tt Scatter} and {\tt Gather} kernels are
very similar to the functions in the previous proposal.
{\small
\begin{verbatim}
    template <class I1, class I2, class O1>
    class Scatter : Kernel_2_1 <I1, I2, O2> {
      Scatter(istream<I1> src_str, istream<I2> index_str, ostream<O1> dest_str, 
              int src_offset=0, int dest_offset=0, int record_size = 1);
    }

    template <class I1, class I2, class O1>
    class Gather : Kernel_2_1 <I1, I2, O2> {
      Gather(istream<I1> src_str, istream<I2> index_str, ostream<O1> dest_str, 
             int src_offset=0, int dest_offset=0, int record_size = 1);
    }  
\end{verbatim}}

These kernels copy items from a source stream to a destination stream,
in chunks of {\tt record\_size} words.  In the {\tt Scatter} kernel,
the {\tt index\_str} indicates the positions in the output stream at
which the records should be written; in the {\tt Gather} kernel, the
{\tt index\_str} indicates the positions in the input stream at which
the records should be read.  The {\tt src\_offset} and {\tt
dest\_offset} arguments can be used for shifting all accesses by an
offset in the source or destination streams.

Like the {\tt Copy} kernel, these kernels assume that the
architectural graph contains an edge from the location of the input
stream to the location of the output stream.  The location of the
index stream can be on either of the two nodes.

\subsubsection*{Network Kernels}

The network kernels are for processor-processor communication.  They
serve as a substitute for the previous proposal's network management
of streams.

\ssss{Send} The {\tt Send} kernel sends a stream from one processor to
another, subject to the connection protocol described below.
{\small
\begin{verbatim}
    template <class I1>
    class Send : Kernel_1_0 <I1> {
      // send <src_str> on <channel> of <connection>
      Send(istream<I1> src_str,  VM_EDGE connection, int channel);
    }
\end{verbatim}}

Given that the kernel is executing on processor $P$, we require that
{\tt src\_str} is located in a memory connected to $P$ (or located
within $P$ itself), and that {\tt connection} is an edge from $P$ to a
neighboring processor.

\ssss{Receive} The {\tt Receive} kernel receives a stream from a
neighboring processor, subject to the connection protocol described
below.  
{\small
\begin{verbatim}
    template <class O1>
    class Receive : Kernel_0_1 <I1> {
      // receive <dest_str> from <channel> of <connection>.
      Receive(ostream<O1> dest_str,  VM_EDGE connection, int channel);
    }  
\end{verbatim}}

Given that the kernel is executing on processor $P$, we require that
{\tt dest\_str} is located in a memory connected to $P$ (or located
within $P$ itself), and that {\tt connection} is an edge into $P$ from
a a neighboring processor.

\ssss{Send/Receive Protocol} We refer to channel number $n$ of
connection $c$ as the pair $(c, n)$.  Note that $n$ is a virtual
channel identifier; $n$ does not need to fall within $[0,
\mbox{VM\_PROP\_CHAN\_NUM}]$ for connection $c$.  Rather, the
communication protocol will ensure that there are less than
VM\_PROP\_CHAN\_NUM active channels at a time.

The protocol maintains a queue of {\tt Send} and {\tt Receive} kernels
that are waiting to communicate across each $(c, n)$; let them be
$\mt{SendQ}(c, n)$ and $\mt{ReceiveQ}(c, n)$, respectively.  Kernels
are pushed onto these queues in the same order that their containing
graphs are executed from the stream processor API.  We disallow the
case where multiple kernels in a given graph are targeting the same
queue.  Thus, the order of the kernels in the queues is
well-defined\footnote{Unless there are multiple threads executing on
the control processor, in which case synchronization should be used to
ensure a deterministic ordering of the send/receive kernels across
threads.}.

To open a new session of data transfer across $(c, n)$, the following
conditions must be met:
\begin{enumerate}

\item Channel $n$ is {\it free} on connection $c$.  That is, no other
session is open on $(c, n)$, and $c$ has room for another active
channel.

\item $\mt{SendQ}(n,c)$ and $\mt{ReceiveQ}(c, n)$ are non-empty.

\end{enumerate}
If these conditions are satisfied, then a new session is opened
between the kernels at the front of $\mt{SendQ}(c, n)$ and
$\mt{ReceiveQ}(c, n)$.  Items are transmitted across the channel until
the graph containing the {\tt Send} kernel finishes its execution.  At
this point, an end-of-stream marker is inserted into the {\tt
dest\_str} of the {\tt Receive} kernel, the session is terminated, and
both the {\tt Send} and {\tt Receive} kernels are removed from the
respective queues for $(c, n)$.

Note that until a session is opened, all pending kernels are blocked.
The graphs that contain these kernels could possibly execute other
nodes, but the {\tt Send} or {\tt Receive} nodes must wait until the
channel is ready.

\subsubsection*{Example}

We consider one more example to illustrate the use of the above
kernels.  In this example, there are two processors that each contain
their own memory:

\begin{figure}[h]
\begin{center}
\psfig{figure=ex1.eps,width=2in}
\end{center}
\vspace{-12pt}
\end{figure}

The application does audio segmentation on a series of 10 input files
and plays a summary of each file on a speaker.  The first processor
does the segmentation itself, while the second processor filters the
extracted segments to provide a smooth transition between them.

{\small
\begin{verbatim}
    // --- code for PROC1 ---

    for (int i=0; i<10; i++) {

      stream<float> raw_data(10000, MEM1, true), spectrum(100, MEM1),
                    sim(10, PROC1), seg_indices(10, PROC1), sum_data(10, PROC1);

      Graph g(new FileReader(raw_data, filename[i]),       // load file
              new FFT(raw_data, spectrum, N),              // extract spectrum
              new SimilarityMatrix(spectrum, sim),         // detect local similarity
              new ExtractSegments(sim, seg_indices),       // make indices of summary segments
              new Gather(raw_data, seg_indices, sum_data), // gather summary audio in sum_data
              new Send(sum_data, c1, 3));                  // send over connection c1, channel 3

      g.start(PROC1);                                      // run for whole length of file
      g.wait();
    }

    // --- code for PROC2 ---

    for (int i=0; i<10; i++) {

      stream<float> sum_data(100, MEM2), smooth_data(100, MEM2);

      Graph g(new Receive(sum_data, c1, 3),                // receive summaries over channel
              new FIRFilter(sum_data, smooth_data),        // filter summaries
              new Speaker(smooth_data));                   // send to speaker

      g.start(PROC2);
      g.wait();

    }   
\end{verbatim}}
In processor 1, a {\tt Gather} kernel is used to load the audio file
at the indices where the summary segments appear.  The {\tt Gather}
kernel is directly connected to a {\tt Send} kernel which sends the
summary segments across virtual channel 3 of connection {\tt c1}.
Processor 2 uses a {\tt Receive} kernel to receive the summary
segments before filtering them and sending them to a speaker.  Note
that there are 10 sessions of data transfer between the processors,
and the amount of data transferred during each session depends on the
length of the audio file; a session is terminated when processor 1
finishes executing its stream graph.

\section{Memory}

With regards to the threaded VM, there are no changes to the memory
API.  As described above, this proposal allows streaming memory access
only via stream objects.  The layout of stream objects to specific
memory addresses is not done by the high-level compiler.

\section{Network}

With regards to the threaded VM, there are no changes to the network
API. As described above, this proposal allows streaming network access
only through a pre-defined set of kernels.

%  \section{Initialization for Peeking}

This section will develop a simple algorithm for constructing an
initialization schedule.  The algorithm developed here will use
hierarchy in a similar way section \ref{sec:calc-min-steady} used
hierarchy to compute steady schedules.  Schedules computed here
are not minimal, but remain reasonably small.

The amount of data required for initialization by stream $s$ will
be denoted by $init^{pop}_s$. The amount of data produced by
initialization schedule will be denoted by $init^{push}_{s}$. This
hierarchical technique will simplify calculations, but may be
unable to compute an initialization schedule for {\feedbackloops}
that would otherwise be possible to schedule.  An algorithm for
computing a valid initialization schedule for any stream that can
be initialized will be presented in Section \ref{sec:min-latency}.

\subsubsection{Notation for Initialization Schedules}

An initialization schedule for a stream $s$ is a set $I_s$ with
elements $I_s = \{c, u\}$.  The set includes a vector $c$ which
holds values $[e_s^{init}, o_s^{init}, u_s^{init}]$ for the
initialization schedule of the stream and a vector $u$ which
stores how many times each of the direct children of $s$ will
execute their steady state.  The elements are denoted $I_{s,S}$,
$I_{s,c}$ and $I_{s,u}$.

\subsubsection{\filter}

Since {\filters} do not buffer any data, they do not require
initialization schedules.  They may, however, require some data
for their initialization.  For a {\filter} $f$, this amount of data
is $e_f - o_f$, as explained above.  {\filters} will not produce any
data during initialization.

An initialization schedule of a {\filter} is thus $I_f = \{[e_f-o_f,
0, 0], \{\}\}$.

\subsubsection{\pipeline}

\begin{algorithm}
\caption{Single Appearance Initialization Counts for a {\pipeline}}
{\bf }($p$). Given a {\pipeline} $p$, calculate how many times each
child needs to execute its steady state to initialize the
{\pipeline} for peeking.
\begin{algorithmic}
\STATE compute $S_{p_{n-1}}$; $u_{n-1} = 0$ \FOR{$i=n-2$ downto 0}
\STATE compute $S_{p_i}$ \STATE $u_i = {I_{p_{i+1}, c_0} + u_{i+1}
* S_{p_{i+1}, c_1} - I_{p_i, c_2} \over I_{p_i, c_2}}$ \ENDFOR
\STATE $I_p = \{[I_{p_0, c_0} + (u_0 - 1) * S_{p_0, c_0}, u_0 *
S_{p_0, c_0}, I_{p_{n-1}, c_2}] ,u\}$
\end{algorithmic}
\end{algorithm}

Let $s_i$ denote $i$th child stream of the {\pipeline}.  Also, let
$m_i$ denote the number of times the $i$th child will execute its
steady state during execution of {\pipeline}'s initialization
schedule.

The $i$th stream needs to provide at least $init^{pop}_{s_{i+1}} +
m_{i+1} * pop_{s_{i+1}}$ data for {\pipeline}'s next child.  For the
last child $m_{n-1} = 0$, thus it only needs
$init^{pop}_{s_{n-1}}$ data to initialize.  Knowing the amount of
data required by the last child, we can compute how much data all
other children require, and thus compute how many times all other
children need to execute their steady state schedules.

In order to provide enough data for $s_{i+1}$, $s_{i}$ is going to
execute its initialization schedule, producing
$init^{push}_{s_{i}}$ data items, and then it is going to execute
its steady state schedule $m_i$ times to provide any additional
required data.  The $i$th child will need to execute its steady
state schedule $m_i = \left\lceil { init^{pop}_{s_{i+1}} + m_{i+1}
* pop_{s_{i+1}} - init^{push}_{s_i} \over{push_{s_i}}}
\right\rceil$ times.

Finally, since the first child of the {\pipeline} is directly
receiving the data that is meant to enter the {\pipeline}, it
follows that $init^{pop}_{p} = init^{pop}_{s_0} + m_0 *
pop_{s_0}$. Similarly, $init^{push}_{pipeline} =
init^{push}_{s_{i-1}}$.

Once all $m_i$s are known, the initialization schedule is
constructed according to the following algorithm:

\begin{singlespace}
\begin{verbatim}
initialization schedule (p) = empty for i = 0 .. n-1
    initialization schedule (p) += initialization schedule (s_i)
    for j = 1 .. m_i
        initialization schedule (p) += steady schedule (s_i)
    end for
end for
\end{verbatim}
\end{singlespace}

Notice, that since the children of the {\pipeline} use their steady
schedules in order to push extra data into the buffers, they may
be pushing more data than required, thus causing the pipeline to
consume more data for its initialization schedule than absolutely
required.  This over-estimation will propagate up with each
{\StreamIt} stream that contains this {\pipeline}.

We again use the example from Figure \ref{fig:steady-state} (a).
Since all children of the {\pipeline} are {\filters}, they do not have
any initialization schedules, and their $init^{push}_{s_i} = 0$.
By inspection we obtain

\begin{displaymath}
\begin{array}{rl}
init^{pop}_A = & 1-1=0 \\
init^{pop}_B = & 3-2 = 1 \\
init^{pop}_C = & 2 - 2 = 0 \\
init^{pop}_D = & 5-3=2 \\
\\
m_3 = & 0 \\
m_2 = & \left \lceil 2 + 3 * 0 - 0 \over 1 \right \rceil = 2\\
m_1 = & \left \lceil 0 + 2 * 2 - 0 \over 3 \right \rceil = 2\\
m_0 = & \left \lceil 1 + 2 * 2 - 0 \over 3 \right \rceil = 2\\
\end{array}
\end{displaymath}

Thus the initialization schedule is $(AABBCC)$.  Also,
$init^{pop}_p = 0 + 2 * 1 = 2$ and $init^{push}_p = 0$.

\subsubsection{\splitjoin}

\begin{algorithm}
\caption{Single Appearance Initialization Counts for a {\pipeline}}
{\bf }($p$). Given a {\pipeline} $p$, calculate how many times each
child needs to execute its steady state to initialize the
{\pipeline} for peeking.

\begin{algorithmic}

\STATE compute $S_{p_{n-1}}$; $u_{n-1} = 0$

\FOR{$i=n-2$ downto 0}

\STATE compute $S_{p_i}$

\STATE $u_i = {I_{p_{i+1}, c_0} + u_{i+1} * S_{p_{i+1}, c_1} -
I_{p_i, c_2} \over I_{p_i, c_2}}$

\ENDFOR

\STATE $I_p = \{[I_{p_0, c_0} + (u_0 - 1) * S_{p_0, c_0}, u_0 *
S_{p_0, c_0}, I_{p_{n-1}, c_2}] ,u\}$

\end{algorithmic}
\end{algorithm}

Initializing a {\splitjoin} is done by executing the {\splitter}
enough times to provide enough data for all the children to
initialize.  The {\joiner} is never run, so there may be some data
buffered between the children streams and the {\joiner}.

Let $s_i$ denote the $i$th child stream of the {\splitjoin},
$w_{s,i}$ denote the amount of data pushed by the {\splitter} of the
{\splitjoin} towards the $i$th child during an execution of the
{\splitter}, and $w_{j,i}$ denote the amount of data popped by the
{\joiner} from the $i$th child during its execution.

The {\splitter} must execute enough times to provide every child of
the {\splitjoin} with enough data to initialize.  For $i$th child,
the required number of executions is $\left\lceil init^{pop}_{s_i}
\over w_{s,i} \right\rceil$.  The {\splitter} needs to execute the
maximum amount of times required by any the children, that is
$m_{split} = \max \left\lceil init^{pop}_{s_i} \over w_{s,i}
\right\rceil, \forall i \in \{0,\dots,n-1\}$.  Thus the amount of
data required for initialization of a {\splitjoin} is
$init^{pop}_{sj} = pop_{split} * m_{split}$. Since the joiner will
never get executed, $init^{push}_{sj} = 0$.

Once $m_{\splitter}$ is has been calculated, the initialization
schedule is constructed according to the following algorithm:

\begin{singlespace}
\begin{verbatim}
initialization schedule (sj) = empty

for i = 1 .. m_{splitter}
  initialization schedule (sj) += {\splitter} execution
end for

for i = 0 .. n-1
    initialization schedule (sj) += initialization schedule (s_i)
end for
\end{verbatim}
\end{singlespace}

The following computes an initialization schedule for the example
{\splitjoin}  from Figure \ref{fig:steady-state} (b). Both children
of the {\splitjoin} are {\filters}, thus they do not have
initialization schedules and their $init^{push}_{s_i} = 0$. By
inspection, we obtain

\begin{displaymath}
\begin{array}{rl}
init^{pop}_A = & 2-2=0 \\
init^{pop}_B = & 3-2 = 1 \\
\\
m_{split} = & \max({0 \over 2}, {1 \over 1}) = \max (0, 1) = 1
\end{array}
\end{displaymath}

Thus the initialization schedule for this {\splitjoin} is simply
$({\splitter})$.  We also get $init^{pop}_{sj} = 1
* 3 = 3$ and $init^{push}_{sj} = 0$.

\subsubsection{\feedbackloop}

The final {\StreamIt} component left to initialize is the
{\feedbackloop}. Let $s_b$ be the body stream of the {\feedbackloop},
and $s_l$ be the feedback path stream of the {\feedbackloop} (thus
$push_{s_b}$ is the amount of data pushed by $s_b$ per steady
state execution, etc).  Let $m_{s_b}$ be the number of times $s_b$
needs to be executed in order to properly initialize the
{\feedbackloop} and $m_{s_l}$ be the number of times $s_l$ needs to
be executed to initialize the {\feedbackloop}.

Initialization for the {\feedbackloop} is calculated in a similar
way to initialization of a {\pipeline}.  Since the initial data is
inserted into the buffer between the loop stream and the {\joiner},
it follows that calculation of initialization requirements should
start from the loop stream as a "last" element - it will be
execute last in the initialization schedule. That means that the
loop stream will execute its initialization schedule, but it will
not execute its steady schedule, namely $m_{s_l} = 0$.  Thus we
have

\begin{displaymath}
\begin{array}{rl}
m_{split} & = \left\lceil {init^{pop}_{s_l} \over w_{s,1}} \right\rceil \\
m_{s_b} & = \left\lceil pop_{split} * m_{split} -
init^{push}_{s_b} \over push_{s_b} \right\rceil \\
m_{join} & = \left\lceil init^{pop}_{s_b} + pop_{s_b} * m_{s_b}
\over { push_{s_b}} \right\rceil
\end{array}
\end{displaymath}

We also calculate the overall consumption and production of data
during initialization of the {\splitjoin}

\begin{displaymath}
\begin{array}{rl}
init^{pop}_{fl} & = w_{j,0} * m_{join} \\
init^{push}_{fl} & = w_{s, 0} * m_{split}
\end{array}
\end{displaymath}

Now the schedule can be constructed by using the following
algorithm:

\begin{singlespace}
\begin{verbatim}
initialization schedule (fl) = empty for i = 1 .. m_{join}
  initialization schedule (fl) += {\joiner} execution
end for initialization schedule (fl) += initialization schedule
(s_b) for i = 0 .. m_{s_b}
    initialization schedule (fl) += steady state schedule (s_b)
end for for i = 1 .. m_{split}
  initialization schedule (fl) += {\splitter} execution
end for initialization schedule (fl) += initialization schedule
(s_l)
\end{verbatim}
\end{singlespace}

It is important to note, that this initialization schedule is not
legal if there isn't enough data in the buffer between the {\joiner}
and the loop stream, that is if $delay_{fl} < w_{j,1} *
m_{joiner}$.  This condition being true, does not mean, however,
that no initialization schedule exists for the particular
{\feedbackloop}.  The reason for this is that executing entire
steady state schedules of the body stream may consume more data
than is actually necessary to provide enough data for the loop
stream to receive $init^{pop}_{s_l}$ data.  Also the
$init^{pop}_{s_b}$ and $init^{pop}_{s_l}$ values may be larger
than necessary, because they also may use their children's steady
schedules in initialization.

An algorithm for initializing any legal stream structure is
presented in Section \ref{sec:min-latency}

The following computes an initialization schedule for the example
{\feedbackloop} from Figure \ref{fig:steady-state} (c). Both
children of the {\feedbackloop} are {\filters}, thus they do not have
initialization schedules and their $init^{push}_{s} = 0$. By
inspection, we obtain

\begin{displaymath}
\begin{array}{rl}
init^{pop}_B = & 3-2 = 1 \\
init^{pop}_L = & 7-5 = 2 \\
\\
m_{split} = & \left\lceil 2 \over 3 \right\rceil = 1 \\
m_{B} = & \left\lceil 3 * 1 - 0 \over 1 \right\rceil = 3 \\
m_{join} = & \left\lceil 1 + 2 * 3 \over 5 \right\rceil = 2 \\
\end{array}
\end{displaymath}

Thus the initialization schedule for this {\feedbackloop} is
$({\joiner}\ {\joiner}\ BBB\ {\splitter})$.  We also get
$init^{pop}_{fl} = 2 * 2 = 4$ and $init^{push}_{fl} = 3 * 1 = 3$.

\section{Steady Schedules}

\subsection{Filter}

The scheduling of a {\filter} is very simple.  Since a {\filter} has
no sub-components (it is an atomic unit), a steady schedule for a
{\filter} is simply an execution of the {\filter}.  Thus, for a
{\filter} $f$, $P_f = \{f, \{f\}, \{[e_f,o_f,u_f]\}\}$

\subsection{Pipeline}

Scheduling a {\pipeline} $p$ first requires calculating the steady
state $S_p$ and phasing schedules for all the children of $p$,
$P_{p_i}$. Once the steady state has been calculated,
multiplicities of execution of each child are known, and the
children are simply scheduled to execute an appropriate number of
times in a row, starting from first child.

\begin{algorithm}
\label{alg:sa-pipeline} \caption{Single Appearance Schedule for a
{\pipeline}} {\bf SASPipeline}($p$).  Given a {\pipeline} $p$,
calculate a phasing Single Appearance Schedule for $p$.
\begin{algorithmic}
\STATE compute $S_p$; $phase = \{\}$ \FOR{$i=0$ to $n_p$}
\FOR{$j=0$ to $S_{p,u,i}$} \FOR{$k=0$ to $|P_{p_i, P}|$} \STATE
$phase = phase + P_{p_i, P, k}$ \ENDFOR \ENDFOR \ENDFOR \STATE
$P_p = \{p, \{phase\}, \{S_{p,c}\}\}$
\end{algorithmic}
\end{algorithm}

This technique works here, because {\pipelines} do not have any
cycles between their children (though their children may have
cycles, ie. {\feedbackloops}), and because a correct initialization
schedule is assumed to have been executed.  Notice, that it is not
necessary to know the amount of data buffered between children of
the {\pipeline} in order to use this algorithm, because once the
{\pipeline} has been initialized, all the needed data will be
provided by the steady state schedule.

\subsection{SplitJoin}

Scheduling a {\splitjoin} is essentially identical to scheduling a
{\pipeline}.  Once steady schedule multiplicities are computed, the
{\splitter} is executed the appropriate number of times, followed by
all the immediate children, and finally the {\joiner}.

\begin{algorithm}
\label{alg:sa-pipeline} \caption{Single Appearance Schedule for a
{\splitjoin}} {\bf SASSplitJoin}($sj$).  Given a {\splitjoin} $sj$,
calculate a phasing Single Appearance Schedule for $sj$.
\begin{algorithmic}
\STATE compute $S_{sj}$; $phase = \{\}$ \FOR{$j=0$ to
$S_{sj,u,n_{sj}}$} \STATE $phase = phase + {\splitter}$ \ENDFOR
\FOR{$i=0$ to $n_p$} \FOR{$j=0$ to $S_{p,u,i}$} \FOR{$k=0$ to
$|P_{p_i, P}|$} \STATE $phase = phase + P_{p_i, P, k}$ \ENDFOR
\ENDFOR \ENDFOR \FOR{$j=0$ to $S_{sj,u,n_{sj} + 1}$} \STATE $phase
= phase + {\joiner}$ \ENDFOR \STATE $P_p = \{p, \{phase\},
\{S_{p,c}\}\}$
\end{algorithmic}
\end{algorithm}

Similarly to {\pipelines}, this technique works because {\splitjoins}
have no cycles, and because the {\splitjoin} is assumed to have been
initialized properly.

\section{Min Latency}

\subsection{Pipeline}

In order to create a steady state schedule for a {\pipeline}, an
initialization schedule must already exist (or at least its
results must have been computed).  This was not the case with
single appearance schedule, because just the assumption that the
{\pipeline} was properly initialized allowed for execution of the
all the child streams from top to bottom.  With minimal latency
scheduling, the amount of data buffered up in the {\pipeline} (or
any other stream, for that matter) makes a big difference in which
child streams need to produce (and thus possibly consume) data,
and which do not.

One consequence of this interaction between the initialization and
steady schedules is that the initialization schedule may affect
the size of the steady schedule.  The difference comes from the
fact, that a steady schedule doesn't necessarily process all the
data buffered up and ready for processing.  This means that it is
possible that the steady schedule will have an additional phase at
the end, that will simply push some data around internally to the
{\pipeline}.  The phase will not produce any data (otherwise, the
phase would be necessary anyway), but it may consume some data (to
complete the amounts required to execute a steady schedule) and/or
push the data in internal buffers lower, in order to return to the
buffering state from the beginning of the steady schedule.

To simplify the calculation of how many times each child needs to
get executed during initialization, initialization is computed
exactly the same way as steady schedule.  Namely, the bottom most
stream is executed enough (minimal number of) times to produce
some data, the stream above is executed just enough times to
provide enough data to the stream below, and so on, until either a
child stream does not need to fire, or the top most child is
reached. If the amount of additional data needed by $n+1$ child is
$m$, and the $n$th child is about to execute $p$th phase of
$phases_n$, then the number of executions of the $n$th child is
computed using following algorithm:

\begin{singlespace}
\begin{verbatim}
while (m > 0)
  fired_n = fired_n + 1
  m = m - push^p_{p_n}
  p = (p + 1) \% phases_n
\end{verbatim}
\end{singlespace}

The result of running this algorithm is that while the
initialization schedule grows a little larger than necessary, the
steady schedule remains as small as possible.  The reason for this
is that all the streams will automatically execute the right
number of times to provide enough data for the bottom most child
to fire enough times.

\subsubsection{\splitjoin}

{\splitjoin} schedule is computed in essentially the same way as
{\pipeline}, except that number of firings of child streams depends
on the {\joiner} and the number of firings of the {\splitter} depends
on the child streams.  Algorithm is omitted here for brevity.

\subsubsection{\feedbackloop}

{\feedbackloop} schedule is also scheduled in a way very similar to
the {\pipeline} schedule.  Since the data is output from the
{\splitter}, the calculation begins by setting number of executions
of the {\splitter} to 1.  Number of firings of the $body$ is
computed, then the {\joiner} and finally the $loop$.

Computing of the schedule for the {\feedbackloop} is guaranteed to
succeed, if the {\feedbackloop} has a valid schedule.  This is
because each component in the {\feedbackloop} is scheduled in a way
that requires minimal amount of data input in order to produce
some output.  Thus, if a deadlock is detected in the
{\feedbackloop}, there genuinly is not enough data in the
{\feedbackloop} to execute the {\splitter}, thus output data.

\section{Notation and equations}

Steady State $T_s = \{s, N, m, c, u\}$.
\begin{itemize}
\myitem $s$ - the stream itself

\myitem $L$ - children of the stream

\myitem $m$ - multiples of execution of the children

\myitem $c$ - peek/push/pop for $s$
\end{itemize}

\noindent Phasing Schedule $P^p_s = \{S, I\}$.
\begin{itemize}
\myitem $S$ - steady state phasing schedule phases

\myitem $I$ - phasing init schedule phases
\end{itemize}

\noindent a set of phases (steady state or init) $S = \{H, c\}$
\begin{itemize}
\myitem $H$ - the actual phases \myitem $c$ - peek/pop/push for
the set
\end{itemize}

\noindent Phase $H_s = \{s, A, c, u\}$.
\begin{itemize}
\myitem $s$ - the stream itself

\myitem $A$ - sub-phases to be executed for this phase

\myitem $c$ - peek/push/pop for the phase

\myitem $u$ - {\bf don't need - so don't use!} multiples for
direct children, {\splitter} and {\joiner}.  {\splitter} and {\joiner} are
the $n$th and the $n+1$st element.
\end{itemize}

\noindent Single Appearance $S_s = \{H, I\}$
\begin{itemize}
\myitem $H$ - steady state phase (single one!) \myitem $I$ -
initialization phase (single one!)
\end{itemize}

\subsection{Steady State}

The steady state is easy to determine.  It's already described,
and I don't want to go into it here.

$c$ vector describes the TRUE peek.

\subsection{Scheduling}

All scheduling is actually done within the framework of phased
scheduling.  Initialization has multiple phases (doh!), so I'll
have to be careful writing this up.


\subsection{Initialization Schedule}

Initialization schedules are done in a very simple way - the child
that is meant to output data is executed at least once.  Because
of feedbackloops, there may be more than one phase of
initialization (for min-latency).

\subsection{Single Appearance}

\subsubsection{Filter}

for {\filter} $f$:

\begin{displaymath}
\begin{array}{rl}
H = & \{f, [f], [1], [e_f, o_f, u_f], [1]\} \\
I = & \{f, [], [], [e_f - o_f], [1]\} \\
S_f = & \{H, I\}
\end{array}
\end{displaymath}

\subsubsection{Pipeline}

for {\pipeline} $p$, children $p_i$, $0 \le i < n_p$, $n_p$ children

\begin{algorithm}
\label{alg:sas-pipeline} \caption{Create a Single Appearance
Schedule for a {\pipeline}} {\bf SASPipeline}($T_p$). Given a steady
state for a {\pipeline}, create a single appearance schedule for it.
\begin{algorithmic}
\STATE no clue how to do this
\end{algorithmic}
\end{algorithm}

\subsection{Min-Latency}

\subsubsection{Peek/Pop/Push from phases}

\begin{algorithm}
\label{alg:ph-consumption} \caption{Computing Consumption of a Set
of Phases} {\bf ConsPhases}($H$).  Given a set of phases $H$,
compute the amount of data peeked/popped/pushed by this set of
phases.
\begin{algorithmic}
\STATE $e = 0, o = 0, u = 0$
\FOR{$i=0$ to $|H|-1$}
\STATE $e = \max(e, o + H_{i, c, e})$
\STATE $o = o + H_{i, c, o}$
\STATE $u = u + H_{i, c, u}$
\ENDFOR
\STATE return $[e, o, u]$
\end{algorithmic}
\end{algorithm}

\begin{algorithm}
\label{alg:init-peek} \caption{Create a Phasing schedule out of
set of phases for Steady State and Initialization Schedules} {\bf
MakePhases}($H^S, H^I$). Given sets of phases to execute for
steady state and initialization schedules, construct a proper
phasing schedule.
\begin{algorithmic}
\STATE $c_S = {\bf ConsPhases} (H^S)$
\STATE $c_I = {\bf ConsPhases} (H^I)$
\STATE $c_{I,e} = \max(c_{I,e}, c_{I,o} + c_{S,e} - c_{S,o})$
\STATE $S = \{H^S, c_S\}$
\STATE $I = \{H^I, c_I\}$
\STATE return $\{S, I\}$
\end{algorithmic}
\end{algorithm}

\begin{algorithm}
\label{alg:get-phase} \caption{Return an appropriate phase of a
schedule.  Phase 0 is the first initialization stage of the
schedule.  Once all initialization stages have been exhausted, all
steady state phases are returned, with wrap-around.} {\bf
GetPhase}(n, $P$)
\begin{algorithmic}
\IF{$n < |P_{I, A}|$}
\STATE return $P_{I, A, n}$
\ELSE
\STATE return $P_{S, A, (n - |P_{I,A}|) {\bf\ mod\ } |P_{S, A}|}$
\ENDIF
\end{algorithmic}
\end{algorithm}

%\subsubsection{\filter}

\begin{algorithm}
\label{alg:min-lat-filter} \caption{Return a min-latency phasing
schedule for a {\filter}} {\bf MLFilter}().
\begin{algorithmic}
\STATE $H = \{f, \{f\}, [e_f, o_f,u_f]\}$
\STATE return ${\bf MakePhases} (\{H\}, \emptyset)$
\end{algorithmic}
\end{algorithm}

%\subsubsection{\pipeline}

\begin{algorithm}
\label{alg:min-lat-init-pipeline} \caption{Return a set of phases
that execute a Minimum-Latency initialization schedule for a
{\pipeline}, the amount of data buffered as a result of
initialization, and how many phases each child has executed.} {\bf
MLInitStagesPipeline} $()$
\begin{algorithmic}
\STATE Let $b$ vector represent amount of data stored between children of
$p$
\STATE Let $d$ vector represent phase/stage being currently executed; 0
represents first initialization stage.


\STATE $b = 0$, $d = 0$

\STATE {\bf Compute how many times each child will get
executed at minimum}
\STATE Let $f$ represent amount of data needed by the $i$th child
to be initialized.  $f$ has $n_p+1$ elements, last child being
fictional and not requiring any initialization.
\STATE $f = 0$

\FOR{$i=n_p-1$ downto $0$}
\WHILE{$d_i < |P_{p_i, I, A}|$ {\bf or} $b_{i+1} < f_{i+1}$}
\STATE $H' = {\bf GetPhase}(P_{p_i}, d_i)$, $d_i = d_i + 1$
\STATE $f_i = \max(f_i, b_i + H'_{c, e})$, $b_i = b_i - H'_{c,
o}$, $b_{i+1} = b_{i+1} + H'_{c, u}$
\ENDWHILE
\ENDFOR

\STATE {\bf Compute initialization stages:}
\STATE Let $u$ represent minimum number of executions of each
child
\STATE $u = d$, $d = 0$, $b = 0$
\STATE $stages = \emptyset$
\WHILE{$\neg \forall i, u_{i} \le 0$}
\STATE{\bf Compute a stage}
\STATE Let $H$ be the stage I'm about to construct
\STATE Let $m$ be the number of phase executions that each child
will perform in this stage
\STATE $H = \{p, \emptyset, [0,0,0]\}$, $b_0 = 0$, $b_{n_p} = 0$, $m = 0$
\STATE $need = 1$
\FOR{$i = n_p-1$ downto $0$}
\STATE $nextNeed = 0$
\WHILE{$need > 0$}
\STATE $H' = {\bf GetPhase}(P_{p_i}, d_i + m_i)$, $m_i = m_i + 1$, $u_i = u_i - 1$
\STATE $need = need - H'_{c, u}$, $nextNeed = \max(nextNeed, H'_{c, e} - b_i)$
\STATE $b_i = b_i - H'_{c, o}$, $b_{i+1} = b_{i+1} + H'_{c, u}$
\ENDWHILE
\STATE $need = nextNeed$
\ENDFOR
\STATE $H_{c,e} = need$
\FOR{$i=0$ to $n_p-1$}
\WHILE{\bf true}
\STATE $H' = {\bf GetPhase}(P_{p_i}, d_i + m_i)$
\IF{$H'_{c, e} \le b_i$}
\STATE $m_i = m_i + 1$, $u_i = u_1 - 1$
\STATE $b_i = b_i - H'_{c, o}$, $b_{i+1} = b_{i+1} + H'_{c, u}$
\ELSE
\STATE {\bf break}
\ENDIF
\ENDWHILE
\ENDFOR
\STATE $H_{c,o} = -b_0$, $H_{c,u} = b_{n_p}$
\STATE{\bf Create the stage:}
\FOR{$i=0$ to $n_p-1$}
\WHILE{$m_i \ne 0$}
\STATE $H_A = H_A \circ {\bf GetPhase}(P_{p_i}, d_i)$
\STATE $d_i = d_i + 1$, $m_i = m_i - 1$
\ENDWHILE
\ENDFOR
\STATE $phases = phases \circ H$
\ENDWHILE
\STATE return $\{phases, b, d\}$
\end{algorithmic}
\end{algorithm}

\begin{algorithm}
\label{alg:min-lat-steady-pipeline} \caption{Return a set of
phases that execute a Minimum-Latency steady state schedule for a
{\pipeline}} {\bf MLSteadyStagesPipeline} $(T_p, b^{init}_p,
d^{init}_p)$. $T_p$ is the steady state for the {\pipeline}.
$b^{init}_p$ is the amount of data stored between children of the
{\pipeline} after initialization, $d^{init}_p$ is the number of
phases and stages executed by the initialization schedule.
\begin{algorithmic}
\STATE Let $b$ vector represent amount of data stored between children of
$p$
\STATE Let $d$ vector represent phase/stage being currently executed; 0
represents first initialization stage.
\STATE Let $u$ be number of phase executions for each child for a full steady state execution of the {\pipeline}.

\STATE $b = b^{init}_p$, $d = d^{init}_p$, $\forall i, u_i = T_{p, m, i} * |P_{p, S, A}|$
\STATE {\bf Compute steady state:}
\STATE $phases = \emptyset$
\WHILE{$u_{n_p -1} \ne 0$}
\STATE{\bf Compute a phase}
\STATE Let $H$ be the phase I'm about to construct
\STATE Let $m$ be the number of phase executions that each child
will perform in this phase
\STATE $H = \{p, \emptyset, [0,0,0]\}$, $b_0 = 0$, $b_{n_p} = 0$, $m = 0$
\STATE $need = 1$
\FOR{$i = n_p-1$ downto $0$}
\STATE $nextNeed = 0$
\WHILE{$need > 0$ {\bf and} $u_i \ne 0$}
\STATE $H' = {\bf GetPhase}(P_{p_i}, d_i + m_i)$, $m_i = m_i + 1$, $u_i = u_i - 1$
\STATE $need = need - H'_{c, u}$, $nextNeed = \max(nextNeed, H'_{c, e} - b_i)$
\STATE $b_i = b_i - H'_{c, o}$, $b_{i+1} = b_{i+1} + H'_{c, u}$
\ENDWHILE
\STATE $need = nextNeed$
\ENDFOR
\STATE $H_{c,e} = need$
\FOR{$i=0$ to $n_p-1$}
\WHILE{$u_i \ne 0$}
\STATE $H' = {\bf GetPhase}(P_{p_i}, d_i + m_i)$
\IF{$H'_{c, e} \le b_i$}
\STATE $m_i = m_i + 1$, $u_i = u_1 - 1$
\STATE $b_i = b_i - H'_{c, o}$, $b_{i+1} = b_{i+1} + H'_{c, u}$
\ELSE
\STATE {\bf break}
\ENDIF
\ENDWHILE
\ENDFOR
\STATE $H_{c,o} = -b_0$, $H_{c,u} = b_{n_p}$
\STATE{\bf Create a phase:}
\FOR{$i=0$ to $n_p-1$}
\WHILE{$m_i \ne 0$}
\STATE $H_A = H_A \circ {\bf GetPhase}(P_{p_i}, d_i)$
\STATE $d_i = d_i + 1$, $m_i = m_i - 1$
\ENDWHILE
\ENDFOR
\STATE $phases = phases \circ H$
\ENDWHILE
\STATE return $phases$
\end{algorithmic}
\end{algorithm}

\begin{algorithm}
\label{alg:min-lat-pipeline} \caption{Return a min-latency phasing
schedule for a {\pipeline}} {\bf MLPipeline}().
\begin{algorithmic}
\STATE $T = {\bf SSPipeline} ()$
\STATE $\{I, b^{init}, d^{init}\} = {\bf MLInitStagesPipeline}()$
\STATE $H = {\bf MLSteadyPhasesPipeline}(T, b^{init}, d^{init})$
\STATE return ${\bf MakePhases} (H, I)$
\end{algorithmic}
\end{algorithm}

\section{Old Latency}

\subsubsection{\filters}

The transfer function for amount of information carried by an item
of data before and after a {\filter} is quite simple.  The amount of
information consumed and produced by the {\filter} needs to remain
constant.  Using notation for a {\pipeline} $p$, the transfer
function for a {\filter} is $info_{p_n} = info_{p_{n-1}} * {o_{p_n}
\over u_{p_n}}$.

\subsubsection{{\splitters} and {\joiners}}

The transfer function for the amount of information carried by an
item of data through a {\splitter} or a {\joiner} is also very simple,
however it is not very intuitive, compared to the {\filter} transfer
function. An intuitive function would preserve the amount of
information carried across a {\splitter} or a {\joiner}, and split or
merge the information appropriately across all branches.  Such an
approach, however, can result in a situation where a {\joiner} is
joining data items with different amount of information carried in
different branches.  This type of a situation is difficult to
handle, because for simplicity all data items in a single buffer
should carry exactly the same amount of information.

The approach that solves the problem above is to not preserve the
amount of information across all branches when data is being split
or joined. Instead, every branch (including the input to a
{\splitter} or output of a {\joiner}) consumes or produces the same
amount of information on every iteration.

Using the {\splitjoin} notation, amount information per data item
transferred to $n$th branch of a {\splitter} is $info_{split_n} =
info_{input} * {o_{split} \over u_{split,n}}$, and the amount of
information per data item transferred from $n$th branch of a
{\joiner} is $info_{output} = info_{join_n} * {o_{join,n} \over
u_{join}}$.  The same equations will work for {\feedbackloops}, with
the property that going through a loop will preserve the amount of
information carried by a data item.

\subsubsection{checking for messages}

 by a {\filter} $f_0$ to {\filter} $f_1$. that consumes
(and thus produces) $info_{f_0}$ amount of information, and
delivered to {\filter} $f_1$ that consumes $info_{f_1}$ amount of
information on every execution of its {\work} function, then the
destination {\filter} must check for delivery of new at least
messages every $\left \lfloor {(l_1-l_0)
* info_{f_0}} \over info_{f_1} \right \rfloor$ firings of its
{\work} function. If this equation yields 0, {\filter} $f_1$ must
check for new messages before every firing of its {\work} function.

%  -- this is minor, and I don't know how to explain it well
%Note, that this frequency of checking for messages to deliver may
%not be sufficient to satisfy the latency requirements.  If the
%schedule enforces very tight information buffering (very close to
%absolute minimums or maximums, as explained below), it is possible
%that the destination {\filter} needs to check for message delivery
%more often.  This effect is schedule specific, and needs to be
%computed on a case-by-case basis.
%

\subsubsection{information and latency}

Next, the amount of information buffered between the source and
the destination of a message needs to be known.  As explained
earlier, there are three types of messages.  The easiest type of
message to handle in terms of buffering is the downstream positive
delay message.  Those messages impose no information buffering
requirements on the schedule.  This is because the information
wavefront that the message needs to be delivered with cannot
possibly have passed the destination filter - it cannot even have
passed the source filter.

The next type of message to consider is the downstream negative
delay message.  A downstream negative delay message will be
delivered to the destination \emph{before} the current source
{\filter} information wavefront reaches the destination {\filter}.
 This type of a message generates a minimal buffering requirement
on the schedule.  There is no maximal requirement, because if the
amount of information buffered up   between the source and
destination is large, the message will not be delivered on the
first iteration of checking for messages to be delivered, but on
some subsequent iteration.  At the minimum, the message must be
delivered just before the minimum latency wavefront reaches the
destination {\filter}.  Just after the destination {\filter} is
executed, the amount of data buffered up is reduced by
$info_{dest}$.  This is the situation when the minimal amount of
information is stored between the source and destination.  Thus,
the amount of information stored between the two {\filters} should
be at least $info_{src} * (-l_0) + info_{dest} {e_{dest} -
o_{dest} \over o_{dest}} - \min (info_{src}, info_{dest})$.

The case of upstream positive delay message is very similar to
downstream negative delay message.  The destination {\filter} must
receive the message before it produces the minimal allowed
information wavefront.  Thus there must be less than $info_{src}
* (l_1 + {e_{src} - o_{src} \over o_{src}})$ information
between the destination and source {\filters}.

The last step required for a complete framework relating latency
and information is to compute the amount of information between
two {\filters}.  This task is actually quite simple.  Given two
{\filters}, $f_0$ and $f_1$, the algorithm selects the upstream
{\filter}, and follows any non-cyclic downstream path towards the
other {\filter}.  Amount of information stored along the path
followed is summed up, and represents the information stored
between the two {\filters}.  Note, that this algorithm will follow
the body path of {\feedbackloops}, and select any branch of a
{\splitjoin}.  The reason this algorithm works is because {\filters}
do not destroy or create information when their {\work} functions
execute, and because amount of data stored along any branch of a
{\splitjoin} is the same.


%%   \section{Compiling for PCA Architecture}
%%   Copy from UT/IBM proposal

  %\section{Open Issues}

\begin{enumerate}

\item Is the hardware model for virtual channels realistic?

\item Need to specify the exact function calls for retrieving graph
structure of metadata description.

\end{enumerate}

% \bibliographystyle{abbrv}
% \bibliography{references}
  
\end{document}
