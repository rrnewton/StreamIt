\newcommand{\taba}[1]{\parbox{2.4in}{ ~ \vspace{-0pt} \\ #1 \vspace{-6pt} \\ }}
\newcommand{\tabb}[1]{\parbox{3.6in}{ ~ \vspace{-0pt} \\ #1 \vspace{-6pt} \\ }}
\newcommand{\mytable}[2]{
  {\small
    \begin{tabular}{|l|l|} \hline
      Attribute Name & #1 \\
      \hline \hline
      #2
    \end{tabular}
  }
}
\newcommand{\enttable}[1]{\mytable{Type - Units / Interpretation}{#1}}
\newcommand{\sumtable}[1]{\mytable{Summary}{#1}}
\newcommand{\justtable}[1]{\mytable{Summary and Justification}{#1}}
\newcommand{\entry}[3]{\taba{#1} & \tabb{#2 \\ #3} \\ \hline}
\newcommand{\summary}[2]{\taba{#1} & \tabb{#2} \\ \hline}
\newcommand{\just}[3]{\taba{#1} & \tabb{#2 \\ {\it #3}} \\ \hline}

\section{Virtual Machine Metadata Description and Interfaces}

An architecture is represented as a directed graph, where each node
represents a processor or memory unit and each edge represents a
uni-directional communication channel.  In the following, we describe
the essential properties of a processor node, a memory node, and a
communication channel.  An outline of the changes from the previous
proposal appears in Figure~\ref{fig:archdiff}.

\begin{figure}[t]
\begin{center}
\framebox[6.5in]{
\begin{minipage}{6in}

\begin{itemize}

\item {\bf Architecture description is a graph rather than a tree.}

\item {\bf Network properties are expressed pairwise,} as edges
between nodes, instead of as an interface at the nodes themselves.

\item {\bf No distinction is made between threaded processors and
streaming processors,} except for a new MASTER property to specify
control processors.

\end{itemize}
\caption{Outline of changes to the metadata description.
  \protect\label{fig:archdiff}}
\end{minipage}}
\end{center}
\end{figure}

\subsection{Processor Nodes}

A processor has a single program counter, as well as its own
instruction memory and address space.  Its work capacity can be
modeled as a superscalar.

\sss{Processor Node Properties (New)}

We introduce the following properties that were not in the previous
proposal: \\

\enttable{

  \entry{VM\_PROP\_PROC\_MASTER} {VM\_NODE\_TYPE\_PROC} {The master
    node that can control this processor as a slave in a batch
    dispatch mode.  If no such node exists, then this processor is its
    own master.  (In the terminology of the previous proposal,
    ``streaming'' processors should have a ``threaded'' master, while
    all ``threaded'' processors should be their own master.)}

  \entry{VM\_PROP\_PROC\_SIMD} {int32 - SIMD/Vector speedup factor.}
    {The amount by which the processor will speed up (over that
    specified by the ILP property) if the computation is data parallel.}

  \entry{VM\_PROP\_PROC\_REGS} {int32 - Number of registers} {The
    number of registers on the processor (possibly to include
    scratchpad or cache space?)}

  \entry {VM\_PROP\_PROC\_DMEMBANK} {VM\_NODE\_TYPE\_MEM} {The default
    memory node in which to hold data for this processor.  Though data
    could possibly be held in several memory nodes, this gives the
    compiler a clue as to the lowest-cost memory node that was
    intended for this processor.}

  \entry{VM\_PROP\_PROC\_IMEMBANK} {VM\_NODE\_TYPE\_MEM} {The default
  memory node in which to hold instructions for this processor.}

  \entry{VM\_PROP\_PROC\_ISIZE} {int32 - Number of words} {The number
    of words required to store a RISC instruction for this processor.}
}

\newpage
\sss{Processor Node Properties (Unmodified)}

We retain the following properties exactly as they appear in the
previous proposal: \\

\sumtable{

  \summary{VM\_PROP\_PROC\_ILP} {ILP capability (e.g., number of ALUs
    on the processor).}

  \summary{VM\_PROP\_PROC\_FREQ} {Processor frequency.}

  \summary{VM\_PROP\_OP\_LATENCY} {Instruction latency for a ``typical''
    instruction in cycles.}

  \summary{VM\_PROP\_PROC\_CONTEXT\_SWITCH} {Expected number of cycles for a
    context switch.}
}

\sss{Processor Node Properties (Removed)}

We remove the following properties that appeared in the previous
proposal: \\

\justtable{

  \just{VM\_PROP\_PROC\_TYPE} {Configurations/instruction sets this
    processor resource supports.} {Removed in favor of the MASTER
    property above.}
  
  \just{VM\_PROP\_PROC\_SPEED} {Speed of the processor (e.g., expected
    MIPS or SPEC for a ``typical'' workload).} {Removed for
    simplification.}
  
  \just{VM\_PROP\_PROC\_DMA\_CHANNELS} {DMA channels on this
    processor.} {Removed because network is encapsulated in
    connections.}

  \just{VM\_PROP\_PROC\_MEMBANKS} {Dedicated memory structures for the
    processor.} {Removed in favor of DMEMBANK and IMEMBANK properties.}

  \just {VM\_PROP\_PROC\_PROC\_NETIFS} {Dedicated network interfaces
    for this processor.}  {Removed because network is encapsulated in
    connections.}

}

\sss{Processor Node Configuration Settings}

As before (for properties that still exist), plus new ``Get''
configurations for the new properties.

\sss{Processor Node Statistics}

As before.

\subsection{Memory Nodes}

As before, except without the SRF node type.  SRF's can be modeled as
``regular'' data memory.

\subsection{Edges}

Each edge of the graph represents a ``connection'' for uni-directional
communication between two nodes.  A connection represents a set of
physical or virtual channels that share a set of properties.

Note that all types of nodes can be connected (i.e., there are
memory-memory, processor-memory, memory-processor, and
processor-processor connections.)  Also note that there can be
multiple connections between two nodes (or between a node and itself)
if there are communication channels with different characteristics.

\sss{Connection Properties}

Each edge in the architectural graph has the following properties: \\

\enttable{

  \entry{VM\_PROP\_CHAN\_TYPE} {VM\_CHAN\_TYPE} {The type of channels
  in this connection.  Must be one of the following:
  VM\_CHAN\_TYPE\_VIRTUAL, meaning that bandwidth is split between
  channels that are active at a given time, or
  VM\_CHAN\_TYPE\_PHYSICAL, meaning that each channel has the same
  bandwidth, independent of how many are currently active.}
  
  \entry{VM\_PROP\_CHAN\_NUM} {int32 - Number of channels} {Maximum \#
  of active channels.  We assume all of these channels are uniform (if
  not, then create another connection between the two nodes.)  }

  \entry{VM\_PROP\_CHAN\_BANDWIDTH} {float - Bandwidth bytes/sec}
  {Bandwidth for one active channel; scaling to multiple channels
  depends on type, as described above.}

  \entry{VM\_PROP\_CHAN\_SETUP\_COST} {float - nanoseconds} {The time
  required to setup a new connection between two nodes.}

  \entry{VM\_PROP\_CHAN\_BLOCKING\_FACTOR} {int32 - Number of words}
  {The minimum number of words that must be sent at once to fully
  utilize the bandwidth.}

  \entry{VM\_PROP\_CHAN\_OUTPUT\_WINDOW} {int32 - Number of words}
  {The maximum number of words that can be in flight before the sender
  blocks.}

  \entry{VM\_PROP\_CHAN\_INPUT\_WINDOW} {int32 - Number of words}
  {The maximum number of words that the receiver can see for
  out-of-order access (i.e., the maximum peek amount at receiver.)}

  \entry{VM\_PROP\_CHAN\_CONTROL} {set of VM\_NODE\_TYPE\_PROC} {A
  list of processors that can control the transfer of data across this
  connection.  This property is needed for the case of memory-memory
  connections where such a processor is not clear from the graph.  For
  processor-memory connections, this property contains only the
  processor connected to this edge; for processor-processor
  connections, this property contains only the source of the edge.}

}
