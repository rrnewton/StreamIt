\section{VM Description}

\begin{verbatim}

- use streamit filters as basis for connected kernels.  Say that the
  ``stream parameters'' are constant *per connectivity of the graph*
  -- that is, for a given run statement, it will not change.  This is
  in contrast to the fields of the filters, which might change as they
  are running.  (Have fields so that the compiler can separate
  dependence information.)

 - setting and reading of fields can only occur within the filter that
   owns the field, as well as from the processor that owns the filter.

- based on streamc / kernelc

- a "stream" is a new type of variable.  Represents a persistent
  finite piece of memory allocated to a stream / the values in a 
  piece of a the stream.

  - declared statically with their size and type.  Types are:

    - Blocked stream.  Means that each element in buffer is only
      written once; "size" represents total number of elements in
      stream.  

      e.g. "blocked_stream<int>[10000] str;"

    - Buffered stream.  Means that the buffer is shorter than the
      stream itself, and locations in the actual memory store will
      be written to multiple times.  The stream will start blocking
      when all of the locations have been written (that is, before
      wrapping around.)  The buffer management for the stream is
      left to the low-level compiler.

      e.g. "buffered_stream<char>[10000] str;"

- about kernels:

  - within kernels:
    - have push/pop/peek for communicating channel values

  - annotations for each kernel:

    - whether or not it's fit for SIMD execution (stateless,
      data parallel)
    - whether or not it contains dynamic control flow
      (if it does, then it can't be fit for SIMD)
    - should it be annotated with static rates?  (for streamit)

- obligatory details:

  - allow _structured_ control flow everywhere
  - can't push/pop/peek outside of kernel
  - support C opcodes as common operations
    e.g. logical, boolean, arithmetic
    float, 32, 16, 8 - bit numbers, boolean
  - functions allowed, but not recursive; inline semantics
  - no pointers
  - no dynamic memory allocation
  - scoping allowed w/ braces

- communicate dependence info from high to low
  - scalar value communication - _just scalars_

main() {

  kernel (s1, s2  --> bfs(up1), 10) {

  }

  kernel (s1, s2  --> bfs(up1), 10) {
  }

  kernel (s1, s2  --> bfs(up1), 10) {
  }

}

\end{verbatim}
